% Version control information:
% $HeadURL: https://ejerciciosestadistica.googlecode.com/svn/trunk/contrastes.tex $
% $LastChangedDate: 2011-02-17 12:09:26 +0100 (jue, 17 feb 2011) $
% $LastChangedRevision: 7 $
% $LastChangedBy: asalber $
% $Id: contrastes.tex 7 2011-02-17 11:09:26Z asalber $

\newproblem{hip-1}{far}{}
% ENUNCIADO
{Se sabe que una vacuna que se está utilizando al cabo de dos años sólo protege al 60\% de las personas a las que se
administró.
Se desarrolla una nueva vacuna, y se quiere saber si al cabo de dos años protege a más personas que la primera.
Para ello se seleccionan 10 personas al azar y se les inyecta la nueva vacuna.
Establecemos que si más de 8 de los vacunados conservan la protección al cabo de dos años, entonces consideraremos la
nueva vacuna mejor que la antigua.
Se pide:
\begin{enumerate}
\item Calcular la probabilidad de cometer un error de tipo I.
\item Si la nueva vacuna protegiera a un 80\% de las personas vacunadas al cabo de 2 años, ¿Cuál será la probabilidad
de cometer un error de tipo II?
\end{enumerate}

Repetir los cálculos si se toma una muestra de 100 personas y se establece que la vacuna es mejor si más de 75 de los
vacunados conservan la protección al cabo de 2 años.

\noindent \textbf{Nota}: Aproximar la distribución binomial mediante una distribución normal.
}
%SOLUCIÓN
{
Contraste: $H_0:p=0.6$, $H_1:p>0.6$. Muestra de tamaño 10:
\begin{itemize}
\item $P(\text{Rechazar }H_0/H_0)=0.0464$.
\item $P(\text{Aceptar }H_0/H_1)= 0.6242$.
\end{itemize}
Muestra de tamaño 100:
\begin{itemize}
\item $P(\text{Rechazar }H_0/H_0)=0.0011$.
\item $P(\text{Aceptar }H_0/H_1)=0.1056$.
\end{itemize}
}
%RESOLUCIÓN
{}


\newproblem{hip-2}{fisio}{}
% ENUNCIADO
{Un fisioterapeuta afirma que con un nuevo procedimiento de rehabilitación que él aplica, determinada lesión tiene un
tiempo de recuperación medio no mayor de 15 días.
Se seleccionan al azar 36 personas que sufren dicho tipo de lesión para verificar su afirmación, y se obtiene un tiempo
medio de recuperación de 13 días y una cuasivarianza de 9.
¿Contradice lo observado en la muestra la afirmación del fisioterapeuta para un $\alpha =0.05$? }
%SOLUCIÓN
{
Contraste para la media: $H_0:\mu=15$, $H_1:\mu<15$.\\
Región de aceptación para $\alpha=0.05$: $-1.6444<z$.\\
Estadístico del contraste: $z=-4$. Como cae fuera de la región de aceptación se rechaza la hipótesis nula y se
confirma confirma la afirmación del fisioterapeuta.
}
%RESOLUCIÓN
{}


\newproblem{hip-3}{far}{}
% ENUNCIADO
{Se decide retirar una cierta vacuna si produce más de un 10\% de reacciones alérgicas. Se consideran 100 pacientes sometidos a la vacuna y
se observan 15 reacciones alérgicas. ¿Debe retirarse la vacuna? (Utilizar un $\alpha =0.01$).
}
%SOLUCIÓN
{Contraste para la proporción: $H_0: p=0.1$, $H_1: p>0.1$.\\
Región de aceptación para $\alpha=0.01$: $z>2.3263$.\\
Estadístico del contraste: $z=1.6667$. Como cae dentro de la región de aceptación, se acepta la hipótesis nula y se
concluye que no hay pruebas suficientes para retirar la vacuna.
}
%RESOLUCIÓN
{}


\newproblem{hip-4}{med}{}
% ENUNCIADO
{Se utiliza un grupo de 150 pacientes para comprobar la teoría de que la vitamina C tiene alguna influencia en el
tratamiento del cáncer.
Los 150 pacientes fueron divididos en dos grupos de 75. Un grupo recibió 10 gramos de vitamina C y el otro un placebo
cada día, además de la medicación habitual.
De los que recibieron la vitamina C, 47 presentaban alguna mejoría al cabo de cuatro semanas, mientras que de los que
recibieron el placebo, 43 experimentaron mejoría.
Contrastar esta hipótesis.
}
%SOLUCIÓN
{
Contraste para la comparación de proporciones: $H_0:p_1=p_2$, $H_1:p_1\neq p_2$.\\
Región de aceptación para $\alpha=0.05$: $-1.96<z<1.96$.\\
Estadístico del contraste $z=0.6677$. Como cae dentro de la región de aceptación, se acepta la hipótesis nula
y no se pude concluir que la vitamina C tenga influencia en el tratamiento del cáncer.
}
%RESOLUCIÓN
{}


\newproblem{hip-5}{med}{}
% ENUNCIADO
{Se realizó en dos hospitales una encuesta entre los pacientes sobre la satisfacción con la atención recibida,
calificándola de 0 a 100.
En el hospital A rellenaron la encuesta 12 pacientes, obteniéndose una media de 85 y una cuasivarianza de 16, mientras
que en el hospital B rellenaron la encuesta 10 pacientes, obteniéndose una media de 81 y una cuasivarianza de 25.
¿Puede concluirse que el nivel de satisfacción en el hospital A es mayor que en el B?

\noindent \textbf{Nota}: Hacer previamente un contraste de igualdad de varianzas.
}
%SOLUCIÓN
{Contraste de comparación de varianzas: $H_0:\sigma_A=\sigma_B$, $H_1:\sigma_A\neq \sigma_B$.\\
Región de aceptación para $\alpha=0.05$: $0.2787<f<3.9121$.\\
Estadístico del contraste: $f=0.64$. Como cae dentro de la región de aceptación, se acepta la
hipótesis de que las varianzas son iguales.\\
Contraste de comparación de medias: $H_0:\mu_A=\mu_B$, $H_1:\mu_A>\mu_B$.\\
Región de aceptación para $\alpha=0.05$: $t>1.7247$
Estadístico del contraste: $t=2.0863$. Como cae fuera de la región de aceptación, se rechaza la hipótesis nula y se
concluye que hay pruebas significativas de que el nivel de satisfación en el hospital A es mayor que en el B.
}
%RESOLUCIÓN
{}


\newproblem{hip-6}{med}{*}
% ENUNCIADO
{Se compararon los niveles de ácido ascórbico en plasma de mujeres embarazadas fumadoras y no fumadoras,
obteniéndose los siguientes resultados expresados en gramos de ácido ascórbico por mililitro de plasma:
\begin{itemize}
\item[] Mujeres fumadoras: $0.97$--$0.72$--$1.00$--$0.81$--$0.62$--$1.32$--$1.24$.
\item[] Mujeres no fumadoras: $0.48$--$0.71$--$0.98$--$0.68$--$0.5$.
\end{itemize}
Realizar un contraste de hipótesis para comprobar si el nivel de ácido ascórbico en la sangre de mujeres fumadoras es
mayor que el de mujeres no fumadoras.
}
%SOLUCIÓN
{Contraste de comparación de varianzas: $H_0:\sigma_1=\sigma_2$, $H_1:\sigma_1\neq \sigma_2$.\\
Región de aceptación para $\alpha=0.05$: $0.1606<f<9.1973$.\\
Estadístico del contraste: $f=1.6592$. Como cae dentro de la región de aceptación, se acepta la hipótesis de que las
varianzas son iguales.\\
Contraste de comparación de medias: $H_0:\mu_1=\mu_2$, $H_2:\mu_1>\mu_2$.\\
Región de aceptación para $\alpha=0.05$: $t<1.8125$.\\
Estadístico del contraste: $t=2.0372$. Como cae fuera de la región de aceptación, se rechaza la hipótesis nula y se
concluye que las mujeres fumadoras tienen mayor nivel de ácido ascórbico. 
}
%RESOLUCIÓN
{}


\newproblem*{hip-7}{qui}{}
% ENUNCIADO
{Verificar la hipótesis de que el contenido medio de unos recipientes de ácido sulfúrico es de 10 litros, si los
contenidos de una muestra aleatoria de 10 recipientes son $10.2$, $9.7$, $10.1$, $10.3$, $10.1$, $10.1$, $9.8$, $9.9$,
$10.4$, $10.3$ y $9.8$ litros.
Utilizar un nivel de significación de $0.01$ y suponer que la distribución de los contenidos es normal.
}
%SOLUCIÓN
{}
%RESOLUCIÓN
{}


\newproblem{hip-8}{gen}{}
% ENUNCIADO
{Un fabricante de equipos de medida afirma que sus equipos pueden realizar al menos 12 mediciones más que los de la
competencia sin necesidad de un nuevo ajuste.
Para probar esta afirmación se realizan mediciones con 50 equipos de este fabricante y 50 de la competencia.
En los suyos el número de mediciones hasta necesitar un nuevo ajuste tuvo de media $86.7$ y cuasidesviación típica
$6.28$, mientras que en los de la competencia estos valores fueron $77.8$ y $5.61$ respectivamente.
Verificar la afirmación del fabricante con $\alpha=0.05$.
}
%SOLUCIÓN
{
Contraste de comparación de medias: $H_0:\mu_1<\mu_2+12$, $H_1:\mu_1\geq\mu_2 12$.\\
Intervalo de confianza para la diferencia de medias: $\mu_1-\mu_2\in (6.5659,\,11.2341)$ con un 95\% de confianza,
luego hay diferencias significativas entre el número medio de mediciones, pero no se puede afirmar que sean mayores de
12 mediciones.
}
%RESOLUCIÓN
{}


\newproblem*{hip-9}{med}{}
% ENUNCIADO
{Para determinar si un nuevo suero detiene la leucemia, se seleccionan 9 ratones con leucemia en una fase avanzada. 
Cinco reciben el tratamiento y cuatro no.
Los tiempos de supervivencia, en años, desde el momento que comenzó el experimento son los siguientes:
\begin{itemize}
\item[] Con tratamiento: $2.1$ -- $5.3$ -- $1.4$ -- $4.6$ -- $0.9$.
\item[] Sin tratamiento: $1.9$ -- $0.5$ -- $2.8$ -- $3.1$.
\end{itemize}
¿Puede afirmarse con un $\alpha=0.05$ que el suero es eficaz?
Suponer que ambas distribuciones son normales con varianzas iguales.
}
%SOLUCIÓN
{}
%RESOLUCIÓN
{}


\newproblem{hip-10}{gen}{}
% ENUNCIADO
{Un estudio afirma que el 70\% de los habitantes de la capital lee diariamente algún periódico.
¿Estaríamos de acuerdo con las conclusiones de dicho estudio si al preguntar a 15 personas elegidas aleatoriamente, 8
leen diariamente algún periódico? }
%SOLUCIÓN
{Contraste para la proporción: $H_0:p=0.7$, $H_1:p\neq 0.7$.\\
Región de aceptación para $\alpha=0.05$: $-1.96<z<1.96$.\\
Estadístico del contraste: $z=-1.408590$. Como cae dentro de la región de aceptación, se acepta la hipótesis nula
y se estaría de acuerdo con las afirmación del estudio.
}
%RESOLUCIÓN
{}


\newproblem*{hip-11}{gen}{}
% ENUNCIADO
{Un distribuidor de tabaco asegura que el 20\% de los fumadores de su ciudad prefiere los cigarrillos de marca $A$.
Se selecciona al azar una muestra de 20 fumadores, y 6 de ellos prefieren la marca $A$.
¿Qué conclusión se obtiene con $\alpha=0.05$?
}
%SOLUCIÓN
{}
%RESOLUCIÓN
{}


\newproblem{hip-12}{psi}{}
% ENUNCIADO
{En un estudio sobre el consumo de alcohol entre los jóvenes durante los fines de semana, se preguntó a 100 chicos y a
125 chicas, de los que 63 chicos y 59 chicas contestaron que consumían.
En vista de estos datos, ¿existe alguna diferencia significativa entre las respuestas de chicos y chicas?
Utilizar $\alpha=0.10$.
} 
%SOLUCIÓN
{
Contraste de comparación de proporciones: $H_0:p_1=p_2$, $H_1:p_1\neq p_2$.\\
Región de aceptación para $\alpha=0.01$: $-1.6449<z<1.6449$.\\
Estadístico del contraste: $z=2.4026$. Como cae fuera de la región de aceptación, se rechaza la hipótesis nula y se
concluye que hay diferencias significativas entre el consumo de alcohol de chicos y chicas.
}
%RESOLUCIÓN
{}


\newproblem{hip-13}{gen}{}
% ENUNCIADO
{Un fabricante de baterías para automóvil asegura que la duración de sus baterías tiene una distribución
aproximadamente normal con desviación típica no superior a $0.9$ años.
Si una muestra aleatoria de 10 de estas baterías tiene una cuasidesviación típica de $0.7$ años, ¿qué se puede concluir
sobre la afirmación del fabricante?
}
%SOLUCIÓN
{Contraste para la desviación típica: $H_0:\sigma=0.9$, $H_1:\sigma<0.9$.\\
Región de aceptación para $\alpha=0.05$: $3.3251<j$.\\
Estadístico del contraste: $j=4$. Como cade dentro de la región de aceptación, no se puede rechazar la hipótesis
nula y se concluye que no hay pruebas significativas de que sea cierta la afirmación del fabricante.
}
%RESOLUCIÓN
{}


\newproblem*{hip-14}{amb}{}
% ENUNCIADO
{En un estudio sobre el contenido de ortofósforo de las aguas de un río, se realizaron medidas en dos estaciones distintas.
Se sacaron 15 muestras de la estación 1 y 12 de la estación 2.
Las muestras de la estación 1 presentaron un contenido medio de ortofósforo de $3.84$ mg/l y una cuasidesviación típica
de $3.07$ mg/l, mientras que las de la estación 2 tuvieron media $1.49$ mg/l y una cuasidesviación típica $0.8$ mg/l.

Se pide:
\begin{enumerate}
\item Calcular el intervalo de confianza para el cociente de varianzas.
\item Realizar el contraste de hipótesis de igualdad de varianzas.
\end{enumerate}
Utilizar un $\alpha=2$.
}
%SOLUCIÓN
{}
%RESOLUCIÓN
{}


\newproblem*{hip-15}{amb}{}
% ENUNCIADO
{Se ha desarrollado un aditivo para gasolina que reduce la emisión de CO$_2$ en la combustión.
Para comprobar la efectividad del aditivo, se realiza un estudio en el que se mide en una muestra de 10 coches la
cantidad de CO$_2$ emitida (en Kg/l), tanto con gasolina con aditivo, como con gasolina sin aditivo, obteniendo los
siguientes resultados:
\[
\begin{array}{rcccccccccc}
\mbox{Sin aditivo:}  & 0.38 & 0.42 & 0.41 & 0.39 & 0.45 & 0.47 & 0.44 & 0.38 & 0.40 & 0.50  \\
\mbox{Con  aditivo:} & 0.38 & 0.36 & 0.38 & 0.32 & 0.39 & 0.45 & 0.39 & 0.39 & 0.35 & 0.48 \\
\end{array}
\]
Se pide:
\begin{enumerate}
\item Realizar un contraste de hipótesis para comprobar la efectividad del aditivo.
\item ¿Qué potencia tiene el contraste para detectar una reducción en la emisión de CO$_2$ de $0.2$ Kg/l?
\end{enumerate}
}
%SOLUCIÓN
{}
%RESOLUCIÓN
{}


\newproblem{hip-16}{psi}{}
% ENUNCIADO
{Un psicólogo está estudiando la concentración de una encima en la saliba como un posible indicador de la ansiedad crónica.
En un experimento se tomó una muestra de 12 neuróticos por ansiedad y otra de 10 personas con bajos niveles de ansiedad.
En ambas muestras se midió la concentración de la encima, obteniendo los siguientes resultados:
\[
\begin{array}{rcccccccccccc}
\hline
\mbox{Con ansiedad:} & 2.60 & 2.90 & 2.60 & 2.70 & 3.91 & 3.15 & 3.94 & 2.46 & 2.91 & 3.88 & 3.55 & 3.96\\
\mbox{Sin ansiedad:} & 2.37 & 1.10 & 2.55 & 2.64 & 2.20 & 2.12 & 2.47 & 2.90 & 1.66 & 2.72 \\
\hline
\end{array}
\]
¿Se puede concluir a partir de estos datos que la población de neuróticos con ansiedad y la población de personas sin
ansiedad son diferentes en el nivel medio de concentración de encimas?
Justificar la respuesta.
}
%SOLUCIÓN
{
Contraste de comparación de varianzas: $H_0:\sigma_1=\sigma_2$, $H_1:\sigma_1\neq\sigma_2$.\\
Región de aceptación para $\alpha=0.05$: $0.2787<f<3.9121$.\\
Estadístico del contraste: $f=1.2138$. Como cade dentro de la región de aceptación, se acepta la hipótesis nula y se
concluye que las varianzas son iguales.\\
Contraste de comparación de medias: $H_0:\mu_1=\mu_2$, $H_1:\mu_1\neq\mu_2$.\\
Región de aceptación para $\alpha=0.05$: $-2.0860<t<2.0860$.\\
Estadístico del contraste: $t=3.8407$. Como cae fuera de la región de aceptación, se rechaza la hipótesis nula y
se puede concluir que hay diferencia entre la concentración media de encimas para neuróticos con ansiedad y sin ansiedad.
}
%RESOLUCIÓN
{}


\newproblem{hip-17}{psi}{}
% ENUNCIADO
{En una investigación para ver la efectividad de una nueva droga antidepresiva, se ha tomado un muestra de 15
pacientes depresivos que han completado un cuestionario para detectar el nivel depresivo, antes y después de recibir
la droga.
En la puntuación del cuestionario los valores menores indican una mayor depresión.
Los resultados obtenidos han sido:
\[
\begin{array}{rccccccccccccccc}
\hline
\mbox{Antes:}  & 18 & 21 & 16 & 19 & 14 & 23 & 16 & 14 & 21 & 18 & 17 & 14 & 16 & 14 & 20 \\
\mbox{Después:}& 23 & 20 & 17 & 20 & 16 & 22 & 18 & 18 & 21 & 16 & 19 & 20 & 15 & 15 & 21 \\
\hline 
\end{array}
\]
Realizar un constraste para averiguar si la droga tiene un efecto positivo sobre la depresión.
¿Qué tamaño muestral sería necesario para detectar una diferencia en la puntuación como la que hay entre las medias de las muestras?
}
%SOLUCIÓN
{
Contraste para la media de la diferencia entre antes y después: $H_0:\mu=0$, $H_1:\mu<0$.\\
Región de aceptación par $\alpha=0.05$: $-1.7613<t$.\\
Estadístico del contraste: $t=-2.2563$. Como cae fuera de la región de aceptación, se rechaza la hipótesis nula y
se puede afirmar que la droga reduce la depresión.\\
El tamaño muestral para detectar una diferencia de $\delta=\bar x_1-\bar x_2 = 17.4-18.7333=-1.3333$ con una potencia
del 90\% es $n=26$ individuos.
}
%RESOLUCIÓN
{}


\newproblem{hip-18}{psi}{}
% ENUNCIADO
{Un experimento pretende contrastar la teoría de que la memoria a corto plazo se ve afectada por la similitud entre los
estímulos. El experimento consiste en leer en voz alta una secuencia de letras a un sujeto, quien después de una breve
pausa debe repetir la secuencia.
Si la teoría es correcta, habrá más errores en la lista que contenga letras que suenan de forma similar que si
contiene letras que se parecen.
A cada sujeto se le presentan dos tipos de secuencias, una con letras que suenan de forma similar y otra con letras
que se escriben de forma parecida.
Los errores producidos en cada secuencia son:
\[
\begin{array}{rccccccccc}
\hline
\mbox{Errores de letras que suenan parecidas:}  & 7 & 5 & 6 & 11 & 3 & 8 & 4 & 10 & 9\\
\mbox{Errores de letras con similiar escritura:}& 8 & 2 & 5 &  9 & 5 & 4 & 4 &  7 & 4\\
\hline 
\end{array} 
\]
¿Se puede validar la teoría?
}
%SOLUCIÓN
{
Contraste para la media de la diferencia entre los errores de letras que suenan parecidas y los errores en letras con
similar escritura: $H_0:\mu=0$, $H_1:\mu>0$.\\
Región de aceptación para $\alpha=0.05$: $t>1.8595$.\\
Estadístico del contraste: $t=2.1320$. Como cae fuera de la región de aceptación, se rechaza la hipótesis nula y
se concluye que la teoría es cierta.
}
%RESOLUCIÓN
{}


\newproblem{hip-19}{psi}{}
% ENUNCIADO
{Se sabe que el tiempo de reacción ante un estímulo sigue una distribución normal de media $30$ ms y desviación típica
$10$ ms.
Se cree que la alcoholemia aumenta el tiempo de reacción de los sujetos, y para comprobar esta hipótesis se ha tomado
una muestra aleatoria de 40 individuos a los que se les ha inducido una alcoholemia de $0.8$ g/l y en los que se ha
apreciado un tiempo medio de respuesta de $35$ ms y una desviación típica de $12$ ms.
¿Se puede afirmar que una alcoholemia de $0.8$ gm/l influye en el tiempo medio de respuesta con un riesgo
$\alpha=0.05$? ¿Y con un riesgo $\alpha=0.01$?

¿Cuál será la potencia del contraste para detectar una difererencia en la media del tiempo de reacción de 4 ms? ¿Cuál
debería ser el tamaño muestral para aumentar la potencia hasta un 90\%?
}
%SOLUCIÓN
{Contraste para la media: $H_0: \mu=30$, $H_1:\mu>30$.\\
Región de aceptación para $\alpha=0.01$: $z<2.3263$.\\
Estadístico del contraste: $z=2.6020$. Como cae fuera de la región de aceptación, se rechaza la hipótesis nula
tanto para $\alpha=0.01$ y con mayor motivo para $\alpha=0.05$, de manera que se concluye que la alcoholemia influye en
el tiempo de respuesta.\\
Potencia del contraste  para $\delta=4$: $1-\beta=1-0.5966=0.4034$.\\
El tamaño muestral para, $\alpha=0.05$, $\delta=4$ y una potencia del 90\% es $n=80$.  
}
%RESOLUCIÓN
{}


\newproblem{hip-20}{med}{}
%ENUNCIADO
{Se cree que el nivel medio de protrombina en plasma de una población normal tiene una media de 19mg/100ml y una
desviación típica de 4mg/100ml.
Para contrastar estas hipótesis se tomó una muestra de 8 individuos en los que se obtuvieron los siguientes niveles de
protrombina en plasma:
\[
16.3 - 18.4 - 20.0 - 17.6 - 15.4 - 23.7 - 17.8 - 19.5
\]
¿Se pueden aceptar ambas hipótesis con un riesgo $\alpha=0.1$?
}
%SOLUCIÓN
{Contraste para la media: $H_0:\mu=19$, $H_1:\mu\neq 19$.\\
Región de aceptación para $\alpha=0.1$: $-1.8946<t<1.8946$.\\
Estadístico del contraste: $t=-0.4552$. Como cae dentro de la región de aceptación, se mantiene la hipótesis de
que la media es 19mg/100ml.\\
Contraste para la varianza: $H_0:\sigma=4$, $H_1:\sigma\neq 4$.\\
Región de aceptación para $\alpha=0.1$: $2.1673<j<14.0671$.\\
Estadístico del contraste: $j=2.8743$. Como cae dentro de la región de aceptación, también se mantiene la hipótesis
de que la desviación típica es de 4mg/100ml.
}
%RESOLUCIÓN
{}


\newproblem{hip-21}{gen}{*}
%ENUNCIADO
{Para ver si la ley antitabaco está influyendo en el número de cigarros que se fuman mientras se está en los bares se
seleccionó una muestra en la que se midió el número de cigarros fumados por hora mientras se estaba en un bar antes de la entrada en vigor de la ley y otra
muestra distinta en la que también se midió el número de cigarros fumados por hora después de la entrada en vigor de la ley (se entiende
que con la ley en vigor los cigarros se fuman en el exterior de los bares). Los resultados aparecen en la siguientes tablas:
\begin{center}
\begin{tabular}{cc}
\multicolumn{2}{c}{Antes}\\
\hline
Cigarros & Personas \\
\hline
0-1 & 12\\
1-2 & 21\\
2-3 & 20\\
3-4 & 8\\
\hline
\end{tabular}
\qquad
\begin{tabular}{cc}
\multicolumn{2}{c}{Después}\\
\hline
Cigarros & Personas \\
\hline
0-1 & 22\\
1-2 & 18\\
2-3 & 8\\
3-4 & 4\\
\hline
\end{tabular}
\end{center}
Se pide:
\begin{enumerate}
\item Calcular el intervalo de confianza del 99\% para el número medio de cigarros fumados por hora en los bares antes de la entrada en
vigor de la ley. ¿Cuántos individuos serían necesarios para poder estimar dicha media con un margen de error no mayor de $\pm 0.1$ cigarros
por hora?
\item Contrastar si la nueva ley ha reducido significativamente el consumo medio de tabaco en los bares. ¿Cuánto vale el $p$-valor del
contraste?
\end{enumerate}
}
%SOLUCIÓN
{\begin{enumerate}
\item El intervalo de confianza del 99\% para el número medio de cigarros fumados por hora en los bares antes de la entrada en vigor de la
ley es $(1.5789,\,2.2079)$. El tamaño muestral necesario para estimar la media con un margend e error no mayor de $\pm 0.1$ cigarros es
$n=603$ individuos.
\item Contraste para de comparación de medias: $H_0:\mu_x=\mu_y$, $H_1:\mu_x> \mu_y$.\\
Región de aceptación para $\alpha=0.05$: $Z\leq z_\alpha=1.6449$.\\
Estadístico del contraste: $z= 2.8447$. Como cae dentro fuera de la región de aceptación, se rechaza la hipótesis nula y se concluye que la
ley ha reducido el consumo medio de cigarros por hora.\\
El $p$-valor del contraste vale $0.0022$.
\end{enumerate}
}
%RESOLUCIÓN
{Sean $X$ e $Y$ las variables que miden el número medio de cigarros fumados por hora antes y después de la entrada envigor de la ley respectivamente.
\begin{enumerate}  
\item Puesto que se trata de una muestra grande de $n_x=61$, la fórmula del intervalo de confianza para la media es
\[
\bar x \pm z_{\alpha/2}\frac{\hat s}{\sqrt n}.
\]
A partir de la tabla de frecuencias se calculan los estadísticos necesarios:
\begin{align*}
\bar x &= \frac{\sum x_in_i}{n_x} = \frac{0.5\cdot12+\cdots+3.5\cdot8}{61} = \frac{115.5}{61} = 1.8934,\\
s_x^2 & = \frac{\sum x_i^2n_i}{n_x}-\bar x^2 = \frac{0.5^2\cdot12+\cdots+3.5^2\cdot8}{61} -1.8934^2= \frac{273.25}{61}-3.585 = 0.8944,\\
\hat s_x^2 &= \frac{n_x}{n_x-1}s_x^2 = \frac{61}{60}0.8944 = 0.9093,\\
\hat s_x & = \sqrt{0.9093} = 0.9536.
\end{align*}
Como se pide un nivel de confianza del $99\%$ se tiene que $\alpha=0.01$ y $\alpha/2=0.005$, de modo que buscando en la tabla de la función de distribución de la normal estándar se tiene que $z_{\alpha/2} =2.5758$. Así pues, sustituyendo en la fórmula del intervalo se obtiene
\[
\bar x \pm z_{\alpha/2}\frac{\hat s}{\sqrt n} = 1.8934 \pm 2.5758\frac{0.9536}{\sqrt{61}} = 1.8934\pm 0.3145 = (1.5789,\,2.2079).
\]

Por otro lado, el número de individuos necesario para estimar la media con un margen de error no mayor de $\pm 0.1$ cigarros por hora y una confianza del $99\%$ es
\[
n = \frac{4 z_{\alpha/2}^2 \hat s^2}{A^2} = \frac{4\cdot 2.5758^2 \cdot 0.9093}{(2\cdot 0.1)^2} = 603.2974,
\]
es decir, se necesitarían como mínimo $603$ individuos. 

\item Para contrastar si la nueva ley ha reducido significativamente el consumo medio de tabaco en los bares hay que realizar un contraste unilateral de comparación de medias
\begin{align*}
H_0 &: \mu_x = \mu_y\\
H_1 &: \mu_x > \mu_y
\end{align*}
Como los tamaños muestrales son grandes, $n_x=61$ y $n_y=52$, y no se conocen las varianzas poblacionales, el estadístico de contraste es 
\[
Z = \frac{\bar x -\bar y}{\sqrt{\frac{\hat s_x^2}{n_x}+\frac{\hat s_y^2}{n_y}}},
\]
que sigue una distribución normal estándar.
Para calcularlo se necesitan, además de los estadísticos de $X$ calculados en el apartado anterior, los estadísticos de $Y$, que son:
\begin{align*}
\bar y &= \frac{\sum y_jn_j}{n_y} = \frac{0.5\cdot22+\cdots+3.5\cdot4}{52} = \frac{72}{52} = 1.3846,\\
s_y^2 & = \frac{\sum y_j^2n_j}{n_y}-\bar y^2 = \frac{0.5^2\cdot22+\cdots+3.5^2\cdot4}{52} -1.3846^2= \frac{145}{52}-1.9171 = 0.8713,\\
\hat s_y^2 &= \frac{n_y}{n_y-1}s_y^2 = \frac{52}{51}0.8713 = 0.8884,\\
\end{align*}
de manera que sustituyendo en la fórmula del estadístico de contraste se obtiene
\[
Z = \frac{\bar x -\bar y}{\sqrt{\frac{\hat s_x^2}{n_x}+\frac{\hat s_y^2}{n_y}}} 
= \frac{1.8934-1.3846}{\sqrt{\frac{0.9093}{61}+\frac{0.8884}{52}}} = 2.8447.
\]
Como el estadístico sigue una distribucion normal estándar, la región de aceptación para un nivel de signficación $\alpha=0.05$ es $Z\leq z_\alpha=1.6449$, y como el estadístico cae fuera de esta región se rechaza la hipótesis nula y se concluye que hay pruebas significativas de que la nueva ley ha reducido el consumo medio de tabaco en los bares.

Finalmente, el $p$-valor del contraste es $P(Z>2.8447)= 1- P(Z\leq 2.8447) = 1 - F(2.8447) = 0.0022$.
\end{enumerate}
}


\newproblem{hip-22}{gen}{*}
%ENUNCIADO
{En un estudio sobre el reparto de género de las tareas domésticas se ha medido el número medio de horas diarias que se destinan a las
tareas domésticas en un grupo de personas, obteniendo los siguientes resultados:
\[
\begin{array}{lrrrrrrrr}
\text{Mujeres:} & 3.2 & 3.1 & 2.7 & 4.4 & 3.7 & 3.9 & 2.4 & 3.6 \\
\text{Hombres:} & 3.3 & 2.1 & 1.7 & 2.4 & 1.6 & 1.8 & 2.7 
\end{array}
\]
Contrastar si las mujeres dedican más tiempo que los hombres a las tareas domésticas. 
¿Entre qué valores estará la diferencia del tiempo medio destinado a tareas domésticas entre mujeres y hombres para un 95\% de confianza?
}
%SOLUCIÓN
{Estadísticos muestrales:\\
$\bar x_M=3.375$, $s_M^2 = 0.3744$, $\hat s_M^2 = 0.4279$, $\hat s_M = 0.654$,\\ 
$\bar x_H=2.2286$, $s_H^2 = 0.3249$, $\hat s_H^2 = 0.379$, $\hat s_H = 0.616$.

Contraste de comparación de varianzas: $H_0:\ \sigma_M^2=\sigma_H^2$, \quad $H_1:\ \sigma_M^2\neq \sigma_H^2$,\\
Estadístico del contraste: $F=1.129$,\\
Región de aceptación para $\alpha=0.05$: $(0.1954,\,5.6955)$.\\
Por tanto, se acepta la hipótesis nula y se supone que las varianzas son iguales.

Contraste de comparación de medias: $H_0:\ \mu_M=\mu_H$, \quad $H_1:\ \mu_M> \mu_H$,\\
Estadístico del contraste: $T=3.479$.\\
Región de aceptación para $\alpha=0.05$: $(1.7709,\infty)$.\\
Por tanto, se rechaza la hipótesis nula y se concluye que las mujeres dedican más tiempo que los hombres a tareas domésticas.

Intervalo de confianza del $0.95$\% para $\mu_M-\mu_H$: $(0.4345,\,1.8583)$.
}
%RESOLUCIÓN
{
}


\newproblem{hip-23}{psi}{*}
%ENUNCIADO
{Un estudio sobre la adicción a los videojuegos ha medido el número medio de horas diárias que un grupo de jóvenes pasa
jugando a los videojuegos, obteniendo los siguientes resultados:
\[
2.8 - 4.1 - 1.8 - 2.2 - 0.5 - 3.2 - 1.6 - 1.1 - 2.5 - 0.7 - 4.5 - 3.3 - 1.4 - 1.4 - 1.2 - 1.7 
\] 
Se pide:
\begin{enumerate}
\item Calcular el intervalo de confianza para el número medio de horas diarias de juego.
\item Si un número de horas de juego superior a 2 horas puede suponer un alto riesgo de adicción, estimar mediante un intervalo de confianza
del 90\% el porcentaje de jóvenes que estarán en riesgo de adicción. ¿Qué tamaño muestral sería necesario para poder estimar dicho
porcentaje con un margen de error no mayor de $\pm 5\%$?
\end{enumerate}
}
%SOLUCIÓN
{\begin{enumerate}
\item Estadísticos muestrales: $\bar x=2.125$, $\hat s^2 = 1.3913$, $\hat s = 1.18$. Intervalo de confianza del $0.95$\% para $\mu$:
$(1.4962,\,2.7538)$.

\item Proporción muestral: $\hat p = 0.4375$. Intervalo de confianza del $0.9$\% para $p$: $(0.2335,\,0.6415)$.\\
Tamaño muestral necesario para estimar $p$ con una confianza del $0.9$ y un error $A=\pm 0.05 = 0.1$: $n=266.342$. 
\end{enumerate}
}
%RESOLUCIÓN
{
}
