\documentclass[a4paper,titlepage]{article}
%%%%%%%%%%%%%%%%%%%%%%%%%%%%
\usepackage{svn-multi}
% Version control information:
\svnidlong
{$HeadURL: http://172.20.100.115/svn/maths/ejercicios_estadistica/ejercicios_estadistica.tex $}
{$LastChangedDate: 2007-10-01 16:04:26 +0200 (lun, 01 oct 2007) $}
{$LastChangedRevision: 2 $}
{$LastChangedBy: alf $}
%\svnid{$Id: ejercicios_estadistica.tex 2 2007-10-01 14:04:26Z alf $}

\usepackage[spanish]{babel}
\usepackage[utf8x]{inputenc}
\usepackage{amsmath}
\usepackage{amssymb}
\usepackage{macros}
\usepackage{graphicx}
\usepackage{eurosym}
\usepackage{multicol}
\usepackage{fancybox}
\usepackage{enumitem}
\usepackage{probsoln-alf}
%\usepackage{times}
\usepackage[colorlinks=true]{hyperref}
\hypersetup{pdfauthor={Alfredo S\'anchez Alberca (asalber@ceu.es)}, pdftitle={Ejercicios de Estad\'istica} } 
\usepackage{url}
\usepackage[top=3cm, bottom=3cm, left=2.54cm, right=2.54cm, marginparwidth=2mm]{geometry}
\usepackage{fancyhdr}
\pagestyle{fancy}

\lhead{\textsc{\textcolor[rgb]{0.00,0.00,0.50}{Universidad San Pablo CEU}}} \rhead{\textsl{\textsf{\textcolor[rgb]{0.00,0.00,0.50}{Departamento de Métodos Cuantitativos}}}}
\renewcommand{\headrulewidth}{0pt}
\renewcommand{\floatpagefraction}{.8}
\renewcommand{\textfraction}{.1}

\pdfinfo{/CreationDate (D:\svnpdfdate)}
\svnRegisterAuthor{alf}{Alfredo Sánchez Alberca}

\makeatletter
\let\savees@listquot\es@listquot
\def\es@listquot{\protect\savees@listquot}
\makeatletter

\reversemarginpar
\showshortanswers
%\showanswers
%\PSNrandseed{2007}


\begin{document}
\sloppy

\title{\vskip 2cm
\shadowbox{\Huge \textbf{\textsf{\quad \textcolor[rgb]{0.00,0.00,0.50}{EJERCICIOS DE ESTADÍSTICA}\quad}}}\\
   \vskip 1cm
   {\Large \textsf{\textcolor[rgb]{0.50,0.00,0.25}{Asignatura: Estadística Aplicada a la Psicología }}}\\
   {\Large \textsf{\textcolor[rgb]{0.50,0.00,0.25}{Curso: 1º de Psicología}}}
   }
\author{
   Santiago Angulo Díaz-Parreño (\url{sangulo@ceu.es})
   \and
   José Miguel Cárdenas Rebollo (\url{cardenas@ceu.es})
   \and
   Anselmo Romero Limón (\url{arlimon@ceu.es})
   \and
   Alfredo Sánchez Alberca (\url{asalber@ceu.es})
}
\date{Curso 2009-2010\\[1cm]
\includegraphics[scale=0.3]{img/logo_uspceu_01}}

\maketitle
\newpage
\tableofcontents
\newpage

\input{descriptiva}
% Author Alfredo Sánchez Alberca (asalber@ceu.es)

\newproblem{reg-1}{gen}{}
% ENUNCIADO
{Dada la siguiente tabla de correlación:
\begin{center}
\begin{tabular}{|c||c|c|c|}
\hline
$X\setminus Y$ & 1 & 2 & 3 \\ \hline\hline
$\left[ -2,2\right) $ & 3 & 6 & 1 \\ \hline
$\left[ 2,6\right) $ & 4 & 7 & 3 \\ \hline
$\left[ 6,10\right) $ & 5 & 3 & 0 \\ \hline
\end{tabular}
\end{center}

Determinar:
\begin{enumerate}
\item  Las distribuciones marginales. Media, Moda y Mediana.
\item  Rectas de Regresión.
\item  Coeficiente de correlación lineal. Interpretar el resultado.
\end{enumerate}
}
%SOLUCIÓN
{}
%RESOLUCIÓN
{}


\newproblem{reg-2}{med}{}
%ENUNCIADO
{Una compañía de asistencia sanitaria hace un estudio del número de veces que, durante el último trimestre, han acudido sus asegurados a consultas de especialistas, en función de su edad. En la siguiente tabla se reflejan los resultados obtenidos:
\begin{center}
\begin{tabular}{|c||c|c|c|c|c|}
\hline
$\mbox{Edad}\setminus \mbox{Cons.}$ & 0 & 1 & 2 & 3 & 4 \\ \hline\hline
$\left[ 30,40\right) $ & 6 & 2 & 2 & 0 & 0  \\ \hline
$\left[ 40,50\right) $ & 4 & 3 & 6 & 4 & 1 \\ \hline
$\left[ 50,60\right) $ & 0 & 2 & 4 & 5 & 3 \\ \hline
$\left[ 60,70\right) $ & 0 & 0 & 3 & 4 & 5 \\ \hline
$\left[ 70,80\right) $ & 0 & 0 & 0 & 4 & 6 \\ \hline
\end{tabular}
\end{center}

Se pide:
\begin{enumerate}
\item Recta de regresión del número de consultas sobre la edad.
\item Coeficiente de correlación e interpretarlo.
\item ¿Cuántas consultas se espera que realice una persona de 52 años?¿Es fiable esta predicción?
\end{enumerate}
}
%SOLUCIÓN
{}
%RESOLUCIÓN
{}


\newproblem{reg-3}{far}{}
%ENUNCIADO
{Se determina la pérdida de actividad que experimenta un medicamento desde el momento de su fabricación a lo largo del
tiempo, obteniéndose el siguiente resultado: 

\begin{center}
\begin{tabular}{|c|c|c|c|c|c|}
\hline
Tiempo (en años) & 1 & 2 & 3 & 4 & 5 \\ \hline
Actividad restante (\%) & 96 & 84 & 70 & 58 & 52 \\ \hline
\end{tabular}
\end{center}

Se pide:
\begin{enumerate}
\item Calcular la recta de regresión de la actividad sobre el tiempo transcurrido.
\item Según el modelo lineal, ¿cuánto tiempo debe pasar para que la actividad del fármaco sea del 80\%? 
¿Cuándo será nula la actividad?
\end{enumerate}
}
%SOLUCIÓN
{Llamando $T$ al tiempo y $A$ a la actividad del fármaco:
\begin{enumerate}
\item $\bar t=3$ años, $\bar a=72\%$, $s_t^2=2$ años$^2$, $s_a^2=264\%^2$, $s_{ta}=-22.8$ años$\cdot\%$.\\
Recta de regresión de actividad sobre tiempo: $a=-11.4t+106.2$.
\item Recta de regresión de tiempo sobre actividad: $t=-0.086a+9.2182$.\\
$t(80)=2.3091$ años y $t(0)=9.2182$ años.
\end{enumerate}
}
%RESOLUCIÓN
{}


\newproblem{reg-4}{amb}{}
%ENUNCIADO
{Las temperaturas medias mensuales (en $^\circ$C) y las precipitaciones totales mensuales (en mm) durante el año 2001 en Madrid fueron:
\begin{center}
\begin{tabular}{|l|r|r|r|r|r|r|r|r|r|r|r|r|}
\cline{2-13}
\multicolumn{1}{c|}{} &    Ene &    Feb &    Mar &    Abr &    May &    Jun &    Jul &    Ago &    Sep &    Oct &    Nov &    Dic \\
\hline
Temp.               &  $7.2$ &  $8.4$ & $12.2$ & $13.7$ & $16.7$ & $23.3$ & $24.2$ & $25.5$ & $20.4$ & $16.2$ &  $8.1$ &  $4.2$ \\
\hline
Prec.                & $73.6$ & $31.7$ & $72.1$ & $20.7$ & $37.1$ & $3.8$ & $3.3$ & $1.5$ & $23.1$ & $67.0$ & $12.4$ & $18.0$ \\
\hline
\end{tabular}
\end{center}
¿Existe relación lineal entre las precipitaciones y la temperatura?
De acuerdo a esta relación, ¿qué cantidad de precipitaciones se espera que haya un mes con una temperatura media de 15$^\circ$C?¿Es fiable esta predicción?
}
%SOLUCIÓN
{}
%RESOLUCIÓN
{}


\newproblem{reg-5}{gen}{}
% ENUNCIADO
{Se ha realizado un estudio comparativo de las puntuaciones obtenidas por los alumnos en un test de ingreso en la
universidad ($X$), y el número de asignaturas aprobadas en el primer curso ($Y$). Los resultados obtenidos se expresan en
la siguiente tabla:

\begin{center}
\begin{tabular}{|c||c|c|c|c|c|}
\hline
$X\setminus Y$ & 0 & 1 & 2 & 3 & 4 \\ \hline\hline
$\left[ 0,10\right) $ & 2 & 2 & 1 & 0 & 0 \\ \hline
$\left[ 10,20\right) $ & 1 & 1 & 2 & 2 & 0 \\ \hline
$\left[ 20,30\right) $ & 0 & 1 & 3 & 4 & 1 \\ \hline
$\left[ 30,40\right) $ & 0 & 0 & 2 & 2 & 6 \\ \hline
\end{tabular}
\end{center}

Se desea calcular:
\begin{enumerate}
\item Recta de regresión de $X$ sobre $Y.$
\item Coeficiente de correlación e interpretación del mismo.
\item Si la universidad en cuestión sólo contara con alumnos que al menos logren aprobar dos asignaturas, ¿qué número
de preguntas respondidas correctamente exigirá en el test?
\end{enumerate}
}
%SOLUCIÓN
{
\begin{enumerate}
\item $\bar x=23$ puntos, $\bar y=2.4$ asignaturas, $s_x^2=116$ puntos$^2$, $s_y^2=1.5733$ asignaturas$^2$,
$s_x=10.7703$ puntos, $s_y=1.2453$ asignaturas y $s_{xy}=9.8$ puntos$\cdot$asignaturas.\\
Recta de regresión de $X$ sobre $Y$: $x=6.2288y+8.0508$.
\item $r=0.73$, lo que quiere decir que hay buena relación lineal entre las puntuaciones y las asignaturas aprobadas y
además es creciente (a mayor puntuación en el test, más asignaturas aprobadas).
\end{enumerate}
}
%RESOLUCIÓN
{}


\newproblem{reg-6}{nut}{*}
%ENUNCIADO
{La tabla siguiente representa la distribución bidimensional de frecuencias de una muestra de 80 personas en un estudio sobre la relación entre el nivel de colesterol en sangre ($X$) en mg/dl y la tensión arterial máxima ($Y$) en mmHg.
\[
\begin{array}{|c||c|c|c||c|}
\hline
X\setminus Y & [110,130) & [130,150) & [150,170) & n_x \\
\hline\hline
[170,190)   &           &     4     &           & 12\\
\hline
[190,210)   &    10     &    12     &     4     &   \\
\hline
[210,230)   &     7     &           &     8     &   \\
\hline
[230,250)   &     1     &           &           & 18\\
\hline\hline
n_y          &           &    30     &    24    &    \\
\hline
\end{array}
\]
Se pide:
\begin{enumerate}
\item Completar la tabla.
\item Calcular la recta de regresión del nivel de colesterol sobre la tensión.
\item Usar el modelo lineal para predecir el colesterol esperado para una persona con una tensión arterial de 160 mmHg.
\item Según el modelo lineal, ¿cuál es la tensión arterial máxima esperada para una persona cuyo nivel de colesterol es 270 mg/dl.
\end{enumerate}

Usar las siguientes sumas:
$\sum x_i=16960$ mg/dl, $\sum y_j=11160$ mmHg, $\sum x_i^2=3627200$ (mg/dl)$^2$, $\sum y_j^2=1576800$ mmHg$^2$ y
$\sum x_iy_j=2378800$ mg/dl$\cdot$mmHg.
}
%SOLUCIÓN
{
\begin{enumerate}
\item Tabla de frecuencias
\[
\begin{array}{|c||c|c|c||c|}
\hline
X\setminus Y & [110,130) & [130,150) & [150,170) & n_x \\
\hline\hline
[170,190)   &     8     &     4     &     0     & 12 \\
\hline
[190,210)   &    10     &    12     &     4     & 26 \\
\hline
[210,230)   &     7     &     9     &     8     & 24 \\
\hline
[230,250)   &     1     &     5     &    12     & 18 \\
\hline\hline
n_y          &   26     &    30     &    24     & 80 \\
\hline
\end{array}
\]
\item $\bar x=212$ mg/dl, $\bar y=139.5$ mmHg, $s_x^2=396$ (mg/dl)$^2$, $s_y^2=249.75$ mmHg$^2$ y $s_{xy}=161$
mg/dl$\cdot$mmHg. Recta de regresión del nivel de colesterol sobre la tensión arterial: $x=122.0721+0.6446y$. 
\item  $x(160)=225.2152$ mg/dl.
\item Recta de regresión de la tensión arterial sobre el colesterol: $y=0.4066x+53.3081$.\\
$y(270)=163.0808$ mmHg.
\end{enumerate}
}
%RESOLUCIÓN
{}


\newproblem{reg-7}{nut}{*}
%ENUNCIADO
{En un centro dietético se está probando una nueva dieta de adelgazamiento en una muestra de 12 individuos. Para cada
uno de ellos se ha medido el número de días que lleva con la dieta y el número de kilos perdidos desde entonces,
obteniéndose los siguientes resultados:
\begin{center}
(33 , 3.9), (51 , 5.9), (30 , 3.2), (55 , 6.0), (38 , 4.9), (62 , 6.2),\\
(35 , 4.5), (60 , 6.1), (44 , 5.6), (69 , 6.2), (47 , 5.8), (40 , 5.3)
\end{center}
Se pide:
\begin{enumerate}
\item Dibujar el diagrama de dispersión. Según la nube de puntos, ¿qué tipo de modelo explicaría mejor la relación
entre los días de dieta y los kilos perdidos? 
\item Calcular el modelo lineal y el logarítmico de los kilos perdidos con respecto a los días de dieta.
\item Utilizar el mejor de los modelos anteriores para predecir en número de kilos perdidos tras 40 días de dieta y tras 100 días.
¿Son fiables estas predicciones?
\end{enumerate}
Usar las siguientes sumas ($X$=Días de dieta e $Y$=Peso perdido): $\sum x_i=564$ días, $\sum \log(x_i)=45.8086$
$\log(\mbox{días})$, $\sum y_j=63.6$ kg, $\sum x_i^2=28234$ días$^2$, $\sum \log(x_i)^2=175.6603$ $\log(\mbox{días})^2$, $\sum y_j^2=347.7$ kg$^2$, $\sum x_iy_j=3108.5$ días$\cdot$kg, $\sum \log(x_i)y_j=245.4738$ $\log(\mbox{días})\cdot$kg.
}
%SOLUCIÓN
{Llamando $X$ a los días de dieta, $Y$ a los kg perdidos y $Z=\log X$.
\begin{enumerate}[start=2]
\item $\bar x=47$ días, $\bar y=5.3$ kg, $s_x^2=143.833$ días$^2$, $s_y^2=0.885$ kg$^2$, $s_{xy}=9.942$ días$\cdot$kg.
Modelo lineal: $y=0.069x+2.051$.\\ 
$\bar z=3.82$ $\log(\mbox{días})$, $s_z^2=0.07$ $\log^2(\mbox{días})$, $s_{yz}=0.22$ $\log(\mbox{días})\cdot\mbox{kg}$.\\
Modelo logarítmico: $y=3.4\log y-7.67$. 
\item Modelo lineal: $r^2=0.78$, modelo logarítmico: $r^2=0.86$.\\
Predicciones con el modelo logarítmico: $y(40)=4.86$ kg y $y(100)=7.98$ kg. 
Las predicciones son fiables ya que el coeficiente de determinación es alto, aunque la de 100 días no lo es tanto por estar fuera del rango de valores observados en la muestra. 
\end{enumerate}
}
%RESOLUCIÓN
{}


\newproblem{reg-8a}{far}{*}
%ENUNCIADO
{Al realizar un estudio sobre la dosificación de un cierto medicamento, se trataron 6 pacientes con dosis diarias de 2
mg, 7 pacientes con 3 mg y otros 7 pacientes con 4 mg. De los pacientes tratados con 2 mg, 2 curaron al cabo de 5 días,
y 4 al cabo de 6 días. De los pacientes tratados con 3 mg diarios, 2 curaron al cabo de 3 días, 4 al cabo de 5 días y 1
al cabo de 6 días. Y de los pacientes tratados con 4 mg diarios, 5 curaron al cabo de 3 días y 2 al cabo de 5 días.

Se pide:
\begin{enumerate}
\item Construir la tabla de la distribución conjunta de frecuencias.
\item Obtener las distribuciones de frecuencias marginales y calcular los principales estadísticos para cada variable. 
\item Calcular la covarianza e interpretarla. 
\end{enumerate}
}
%SOLUCIÓN
{Llamando $X$ a la dosis e $Y$ al tiempo de curación:
\begin{enumerate}[start=3]
\item $\bar x=3.05$ mg, $\bar y=4.55$ días, $s_x^2=0.648$ mg$^2$, $s_y^2=1.448$ días$^2$, $s_x=0.805$ mg, $s_y=1.203$
días y $s_{xy}=-0.678$ mg$\cdot$días, lo que indica que hay una relación lineal decreciente. 
\end{enumerate}
}
%RESOLUCIÓN
{}


\newproblem{reg-8b}{far}{*}
%ENUNCIADO
{Al realizar un estudio sobre la dosificación de un cierto medicamento, se trataron 6 pacientes con dosis diarias de 2
mg, 7 pacientes con 3 mg y otros 7 pacientes con 4 mg. De los pacientes tratados con 2 mg, 2 curaron al cabo de 5 días,
y 4 al cabo de 6 días. De los pacientes tratados con 3 mg diarios, 2 curaron al cabo de 3 días, 4 al cabo de 5 días y 1
al cabo de 6 días. Y de los pacientes tratados con 4 mg diarios, 5 curaron al cabo de 3 días y 2 al cabo de 5 días.

Se pide:
\begin{enumerate}
\item Dar el coeficiente de correlación e interpretación.
\item Determinar el tiempo esperado de curación para una dosis de 5 mg diarios.
\end{enumerate}
}
%SOLUCIÓN
{Llamando $X$ a la dosis e $Y$ al tiempo de curación:
\begin{enumerate}
\item $\bar x=3.05$ mg, $\bar y=4.55$ días, $s_x^2=0.648$ mg$^2$, $s_y^2=1.448$ días$^2$, $s_x=0.805$ mg, $s_y=1.203$
días y $s_{xy}=-0.678$ mg$\cdot$días.\\
$r=-0.7$, que quiere decir que hay buena relación lineal entre la dosis y el tiempo de curación, y además es
decreciente (a mayor dosis, menor tiempo de curación).
\item Recta de regresión del tiempo de curación sobre la dosis: $y=-1.046x+7.741$.\\
$y(5)=2.511$ días.
\end{enumerate}
}
%RESOLUCIÓN
{}


\newproblem{reg-9}{nut}{*}
%ENUNCIADO
{Después de tomar un litro de vino se ha medido la concentración de alcohol en la sangre en distintos instantes,
obteniendo:
\[
\begin{tabular}{|c|c|c|c|c|c|c|}
\hline
Tiempo después (minutos) & 30 & 60 & 90 & 120 & 150 & 180 \\ \hline
Concentración (gramos/litro) & 1.6 & 1.7 & 1.5 & 1.1 & 0.7 & 0.2 \\
\hline
\end{tabular}
\]

Se pide:
\begin{enumerate}
\item Calcular la recta de regresión de la concentración en función del tiempo.
\item ¿Qué concentración de alcohol habrá a los 100 minutos?
\item Si la concentración máxima de alcohol en la sangre que permite la ley para poder conducir es 0.8 g/l, ¿cuánto tiempo habrá que esperar después de tomarse un litro de vino para poder conducir sin infringir la ley?
\end{enumerate}
}
%SOLUCIÓN
{}
%RESOLUCIÓN
{}


\newproblem{reg-10}{gen}{}
%ENUNCIADO
{Se consideran dos variables aleatorias $X$ e $Y$ tales que:
\begin{itemize}
\item[--] La recta de regresión de $Y$ sobre $X$ viene dada por la ecuación: $y-x-2=0$.
\item[--] La recta de regresión de $X$ sobre $Y$ viene dada por la ecuación: $y-4x+22=0$.
\end{itemize}
Calcular:
\begin{enumerate}
\item  Valores de $\bar x$ e $\bar y$.
\item  Coeficiente de correlación lineal.
\end{enumerate}
}
%SOLUCIÓN
{
\begin{enumerate}
\item $\bar x=8$ y $\bar y=10$.
\item $r=0.5$.
\end{enumerate}
}
%RESOLUCIÓN
{}


\newproblem{reg-11}{gen}{}
%ENUNCIADO
{En el ajuste rectilíneo a una distribución bidimensional se sabe que $\bar x=2$, $\bar y=1$, y el coeficiente de correlación lineal es 0 ($r=0$).
\begin{enumerate}
\item  Si $x=10$, ¿cuál será el valor interpolado para $y$?.
\item  Si $y=5$, ¿cuál será el valor interpolado para $x$?.
\item  Dibuja las rectas de regresión de $Y$ sobre$X$, y la de $X$ sobre $Y$.
\end{enumerate}
}
%SOLUCIÓN
{
\begin{enumerate}
\item $y(10)=1$.
\item $x(5)=2$.
\end{enumerate}
}
%RESOLUCIÓN
{}


\newproblem{reg-12}{gen}{*}
%ENUNCIADO
{En un estudio para relacionar la longitud de la línea de la vida de la mano izquierda y la duración de la vida de una
persona se han obtenido datos de 50 personas con los siguientes resultados ($X$=longitud de la línea en cm, $Y$=edad al
morir en años):
\[ 
\sum y=3333 \quad \sum y^2=231933 \quad \sum x=459.9 \quad \sum x^2=4308.57 \quad \sum xy=30949. 
\]
A la vista de estos resultados, ¿cuanto vivirá, por termino medio, una persona con una línea de longitud 7.5 cm?
¿Es fiable esta estimación?  }
%SOLUCIÓN
{$\bar x=9.198$ cm, $\bar y=66.66$ años, $s_x^2=1.568$ cm$^2$, $s_y^2=195.104$ años$^2$ y $s_{xy}=6.393$
cm$\cdot$años.\\
Recta de regresión de la edad al morir sobre la longitud de la línea de la vida: $y=4.077x+29.158$.\\
$y(7.5)=59.736$ años.\\
$r^2=0.13$, lo que quiere decir que casi no hay relación lineal entre las variables y la predicción anterior no es
fiable.}
%RESOLUCIÓN
{}


\newproblem{reg-13}{gen}{*}
%ENUNCIADO
{En el estudio de regresión lineal con dos variables $X$ e $Y$ se sabe que $\overline{x}=30$, $\overline{y}=70$ y el
coeficiente de correlación lineal es $0.8$.
También se sabe que para $x=42$ el valor que predice la recta de regresión para $y$ es 78.

Se pide:
\begin{enumerate}
\item Calcular el valor de $x$ que se predice cuando $y=74$.
\item Explicar razonadamente en cuál de las dos variables es más representativa la media.
\end{enumerate}
}
%SOLUCIÓN
{
\begin{enumerate}
\item Recta de regresión de $X$ sobre $Y$: $x=0.96x-37.2$.\\
$x(74)=33.84$.
\item $cv_x=0.0408\sqrt{s_{xy}}>cv_y=0.0146\sqrt{s_{xy}}$ y por tanto es más representativa la media de $Y$ pues tiene
menor dispersión relativa.
\end{enumerate}
}
%RESOLUCIÓNl
{}


\newproblem{reg-14}{gen}{*}
%ENUNCIADO
{Se han medido dos variables $S$ y $T$ en 10 individuos, obteniéndose los siguientes resultados:
\begin{center}
(-1.5 , 2.25), (0.8 , 0.64), (-0.2 , 0.04), (-0.8 , 0.64), (0.4 , 0.16),\\
(0.2 , 0.04), (-2.1 , 4.41), (-0.4 , 0.16), (1.5 , 2.25), (2.1 ,
4.41).
\end{center}
Se pide:
\begin{enumerate}
\item Calcular la covarianza de $S$ y $T$.
\item ¿Se puede afirmar que $S$ y $T$ son independientes?
Justificar la respuesta.
\item ¿Qué valor predice la correspondiente recta de regresión para $t=2$?
\end{enumerate}
}
%SOLUCIÓN
{
\begin{enumerate}
\item $\bar s=0$, $\bar t=1.5$ y $s_{st}=0$.
\item No podemos afirmar que $S$ y $T$ son independientes, sólo se puede afirmar que no hay relación lineal.
\item $s(2)=0$.
\end{enumerate}
}
%RESOLUCIÓN
{}


\newproblem{reg-15}{amb}{}
%ENUNCIADO
{En un experimento se ha medido el número de bacterias por unidad de volumen en un cultivo, cada hora transcurrida,
obteniendo los siguientes resultados: 
\begin{center}
\begin{tabular}{c|ccccccccc}
Horas & 0 & 1 & 2 & 3 & 4 & 5 & 6 & 7 & 8  \\
\hline
Nº Bacterias & 25 & 28 & 47 & 65 & 86 & 121 & 190 & 290 & 362
\end{tabular}
\end{center}

Se pide:
\begin{enumerate}
\item Dibujar el diagrama de dispersión.
Según este diagrama, ¿qué tipo de modelo explicaría mejor la relación entre le número de bacterias y las horas
transcurridas?
\item Dibujar el diagrama de dispersión tomando una escala logarítmica para el número de bacterias.
\item Según el modelo anterior, ¿Cuántas bacterias tendríamos al cabo de 3 horas y media?
¿Y al cabo de 10 horas?
¿Son fiables estas predicciones?
\item ¿Cuánto tiempo tendría que transcurrir para que en el cultivo hubiese 100 bacterias?
\end{enumerate}
}
%SOLUCIÓN
{Llamando $X$ a las horas, $Y$ a las bacterias y $Z$ al logaritmo neperiano de las bacterias:
\begin{enumerate}[start=3]
\item $\bar x=4$ horas, $\bar z=4.5149$ log(bacterias), $s_x^2=6.6667$ horas$^2$, $s_z^2=0.8361$ log$^2$(bacterias) y
$s_{xz}=2.3466$ horas$\cdot$log(bacterias).\\
Modelo lineal del logaritmo de las bacterias sobre las horas: $z=0.3520x+3.1070$.\\
Modelo exponencial de las bacterias sobre las horas: $y=e^{0.3520x+3.1070}$.\\
$y(3.5)=76.6254$ bacterias y $y(10)=755.0986$ bacterias.
\item Modelo lineal de las horas sobre el logaritmo de las bacterias: $x=2.8218z-8.7403$.\\
Modelo logarítmico de las horas sobre las bacterias: $x=2.8218\log y-8.7403$.\\
$x(100)=4.25$ horas.
\end{enumerate}
}
%RESOLUCIÓN
{}


\newproblem{reg-16}{amb}{}
%ENUNCIADO
{Para evaluar la percepción de los ciudadanos sobre la contaminación atmosférica, se ha realizado un estudio en el que se ha medido en 12 ciudades la concentración media de CO (en mg/m$^3$ diarios), y la percepción mediana en la calidad el aire (en una muestra de individuos de tamaño fijo), medida en la escala MM=Muy Mala, M=Mala, A=Aceptable, B=Buena y MB=Muy Buena.
Los resultados obtenidos fueron:
\begin{center}
($12.8$ , A), ($11.6$ , A), ($9.8$ , B), ($10.3$ , MB), ($15.7$ , MM), ($18.2$ , M),\\
($11.8$ , B), ($16.7$ , M), ($14.5$ , M), ($12.1$ , A), ($19.4$ , MM), ($7.9$ , MB)
\end{center}
¿Existe relación entre la percepción de los habitantes de estas ciudades y la concentración de monóxido de carbono en la atmósfera de las mismas?
}
%SOLUCIÓN
{}
%RESOLUCIÓN
{}


\newproblem{reg-17}{med}{}
%ENUNCIADO
{En un estudio sobre la influencia del tabaco en los embarazos se ha medido en una muestra de 20 madres el número medio de cigarrillos diarios que fumaban las madres y el peso del recién nacido, obteniendo los siguientes resultados
\begin{center}
\begin{tabular}{|l|c|c|c|c|c|c|c|c|c|c|c|c|c|c|c|}
\hline
Cigarrillos &  2  &  3  & 10  &  8  & 12  &  6  &  6  &  5  &  4  &  9  & 14  &  3  &  7  & 8 &  2  \\
\hline
Peso (kg)  & 3.1 & 3.3 & 2.5 & 3.3 & 2.6 & 3.1 & 3.0 & 3.4 & 3.4 & 2.7 & 2.5 & 3.7 & 3.1 & 3 & 3.6 \\
\hline
\end{tabular}
\end{center}

Se pide:
\begin{enumerate}
\item Construir el modelo de regresión logarítmico del peso sobre el número de cigarrillos.
\item Según este modelo, ¿cuanto pesará el recién nacido si la madre fumaba 15 cigarrillos diários?
Es fiable esta predicción.
\item ¿Es mejor el modelo lineal a la hora de hacer predicciones?
\end{enumerate}
}
%SOLUCIÓN
{}
%RESOLUCIÓN
{}


\newproblem{reg-18}{med}{*}
%ENUNCIADO
{La tabla siguiente contiene los datos de las presiones sistólicas de 15 individuos en función de la edad de estos.
\[
\begin{array}{|c|c|c|c|c|c|}
\hline
\text{Edad} (x) & 20 & 30 & 40 & 50 & 60 \\
\hline
& 121 & 131 & 132 & 136 & 134\\
\text{Sistólica} (y) & 130 & 125 & 129 & 128 & 142 \\
& 125 & 128 & 131 & 134 & 137\\
\hline
\end{array}
\]

\begin{enumerate}
\item ¿Qué porcentaje de la varianza de la presión sistólica se explica, mediante un modelo de regresión lineal, por la
varianza de la edad? 
\item ¿Qué edad le correspondería a un individuo que presenta una presión sistólica de 133?
¿Es fiable esta predicción?
Razona la respuesta. 
\end{enumerate}
}
%SOLUCIÓN
{
\begin{enumerate}
\item $\bar x=40$ años, $\bar y=130.867$ mmHg, $s_x^2=200$ años$^2$, $s_y^2=26.295$ mmHg$^2$ y $s_{xy}=58.667$
años$\cdot$mmHg.\\
$r^2=0.654$, luego el modelo lineal explica el $65.4\%$ de la varianza de la presión sistólica.
\item Recta de regresión de la edad sobre la presión sistólica: $x=2.231y-251.978$.\\
$x(133)=44.745$ años. La predicción es bastante fiable pues el coeficiente de determinación es alto.
\end{enumerate}
}
%RESOLUCIÓN
{}


\newproblem{reg-19}{qui}{*}
%ENUNCIADO
{Se ha realizado un estudio de regresión para ver la relación que existe entre la velocidad de transformación de una determinada sustancia química en una reacción y la temperatura a la que se realiza dicha reacción (manteniendo las cantidades de reactivos constantes).
Según una recta de regresión, a 10 ºC le correspondería una velocidad de 5 gr/min, y a 30 ºC le correspondería una velocidad de 15 gr/min.
Y según la otra recta, a una velocidad de 8 gr/min le correspondería una temperatura de 17 ºC, y a una velocidad de 16 gr/min le correspondería una temperatura de \mbox{31 ºC}. Se pide:
\begin{enumerate}
\item Calcular las ecuaciones de las rectas de regresión.
\item Calcular las medias de ambas variables.
\item Calcular el coeficiente de determinación. ¿Podemos decir que las predicciones del enunciado son fiables? Justificar la respuesta.
\end{enumerate}
}
%SOLUCIÓN
{}
%RESOLUCIÓN
{}


\newproblem{reg-20}{amb}{*}
%ENUNCIADO
{En un estudio ambiental de una comunidad autónoma se afirma que el número de hectáreas quemadas en los últimos 6 años está relacionado con la cantidad de precipitación media caída en la comunidad, en litros por metro cuadrado. Los datos que han manejado son:
\begin{center}
\begin{tabular}{|l|l|l|}
\hline
\multicolumn{1}{|c|}{Año} & \multicolumn{1}{c|}{Hectáreas quemadas} & \multicolumn{1}{c|}{Precipitación (l/m$^2$)} \\
\hline
\multicolumn{1}{|c|}{2000} & \multicolumn{1}{c|}{1250} & \multicolumn{1}{c|}{420} \\
\hline
\multicolumn{1}{|c|}{2001} & \multicolumn{1}{c|}{1400} & \multicolumn{1}{c|}{380} \\
\hline
\multicolumn{1}{|c|}{2002} & \multicolumn{1}{c|}{850} & \multicolumn{1}{c|}{460} \\
\hline
\multicolumn{1}{|c|}{2003} & \multicolumn{1}{c|}{1650} & \multicolumn{1}{c|}{370} \\
\hline
\multicolumn{1}{|c|}{2004} & \multicolumn{1}{c|}{900} & \multicolumn{1}{c|}{410} \\
\hline
\multicolumn{1}{|c|}{2005} & \multicolumn{1}{c|}{1700} & \multicolumn{1}{c|}{310} \\
\hline
\end{tabular}
\end{center}

\begin{enumerate}
\item Calcular la recta de regresión del número de hectáreas quemadas en función de la precipitación media anual.
\item ¿Es el modelo lineal un buen modelo de ajuste para la nube de puntos? Justificar la respuesta.
\end{enumerate}
}
%SOLUCIÓN
{}
%RESOLUCIÓN
{}


\newproblem{reg-21}{far}{}
%ENUNCIADO
{Se desea comprobar si el número de ventas de un fármaco depende del descuento que se aplique sobre él.
Para ello se ha medido el número de ventas en farmacias que aplican distintos descuentos obteniendo la siguiente muestra:
\begin{center}
\begin{tabular}{|c||c|c|c|c|c|c|c|c|c|c|c|c|c|c|}
\hline Descuento (\%) & 20 & 16 & 15 & 10 & 12 & 11 & 16 & 8 & 18 & 12 & 12 & 10 & 15 & 14 \\
\hline Ventas & 98 & 46 & 40 & 15 & 21 & 19 & 50 & 8 & 71 & 24 & 21 & 16 & 39 & 32 \\
\hline
\end{tabular}
\end{center}
Se pide:
\begin{enumerate}
\item Construir los modelos exponencial y logarítmico.
\item ¿Cuál de ellos expresa mejor la relación entre el descuento y las ventas?
\item ¿Qué descuento tendremos que aplicar si queremos vender al menos 50 fármacos?
\end{enumerate}
}
%SOLUCIÓN
{}
%RESOLUCIÓN
{}


\newproblem{reg-22}{amb}{*}
% ENUNCIADO
{Para ver si un aditivo para la gasolina mejora la combustión aumentando la emisión de dióxido de carbono, se ha hecho un
estudio en el que se ha medido la cantidad de aditivo añadida a cada litro de gasolina y el porcentaje de CO$_2$ emitido
por un mismo motor, obteniendo la siguiente muestra: \[
\begin{array}{|l|rrrrrrrr|}
\hline
\mbox{Aditivo (cl/l)} &  0.2 &  0.4 &  0.6 &  0.8 &  1.0 &  1.2 &  1.4 &  1.6 \\
\hline
\mbox{CO$_2$ (\%)}         & 11.2 & 12.0 & 12.7 & 13.3 & 13.5 & 13.7 & 13.8 & 13.9 \\
\hline
\end{array}
\]
Se pide:
\begin{enumerate}
\item Calcular el modelo de regresión lineal y logarítmico del CO$_2$ sobre el aditivo.
¿Cuál de los dos modelos es mejor?
\item Según el mejor de los modelos anteriores, ¿cuánto CO$_2$ se producirá para $0.5$ cl de aditivo? ¿y para 2 cl?
¿Son fiables estas predicciones?
\item La normativa sobre emisión de gases exige que el porcentaje mínimo de CO$_2$ en la combustión debe superar al menos el $12.5\%$.
¿Cuánto aditivo es necesario para garantizar esto?
\end{enumerate}
}
%SOLUCIÓN
{}
%RESOLUCIÓN
{}


\newproblem{reg-23}{amb}{*}
%ENUNCIADO
{La siguiente tabla muestra los datos de emisiones de CO$_2$ y CH$_4$ (en kg/hab) y el producto interior bruto per cápita (en miles US\$) de varios países en el último año:
\[
\begin{array}{|l|r|r|r|}
\hline
\mbox{País} & \mbox{CO}_2 & \mbox{CH}_4 & \mbox{PIB}\\
\hline\hline
\mbox{Austria}     & 7.60 & 0.97 & 38.40\\ \hline
\mbox{España}      & 6.73 & 0.81	& 30.12\\ \hline
\mbox{Francia}     & 5.71 & 0.94	& 33.19\\ \hline
\mbox{EEUU}        &19.40 & 1.72	&	45.84\\ \hline
\mbox{Alemania}    & 9.80 & 0.83	& 34.18\\ \hline
\mbox{Canadá}      &15.60 & 3.08	& 38.43\\ \hline
\mbox{Italia}      & 7.29 & 0.58	& 30.44\\ \hline
\mbox{Japón}       &	9.44 & 0.16	& 33.58\\ \hline
\mbox{Australia}   &17.48 & 6.36	& 36.26\\ \hline
\mbox{Reino Unido} & 8.99 & 0.76	& 35.13\\ \hline
\end{array}
\qquad
\begin{array}{|l|r|r|r|}
\hline
\mbox{País} & \mbox{CO}_2 & \mbox{CH}_4 & \mbox{PIB}\\
\hline\hline
\mbox{Bolivia}     & 1.05 & 3.44	& 40.13\\ \hline
\mbox{Niger}       &	0.1	 & 0.12	&	 0.67\\ \hline
\mbox{Senegal}     &	0.35 & 0.76 &  1.69\\ \hline
\mbox{Pakistán}    & 0.65 & 0.59	&  2.59\\ \hline
\mbox{Filipinas}   &	0.83 & 0.46	&  3.38\\ \hline
\mbox{Perú}        & 0.94 & 0.75	&  7.80\\ \hline
\mbox{Túnez}      & 2.17 & 0.48	&  7.47\\ \hline
\mbox{Nepal}       & 0.13 & 0.90	&  1.21\\ \hline
\mbox{Nicaragua}   & 0.7	 & 0.32	&  2.62\\ \hline
\mbox{Mauritania}  & 0.97 & 0.85	&  2.01\\ \hline
\end{array}
\]
Utilizando los datos sin agrupar, calcular el modelo de regresión logarítmico que explique las emisiones de CO$_2$ en función del PIB y utilizarlo para predecir las emisiones de un país con 10 mil US\$ de PIB.
¿Es fiable la predicción?
}
%SOLUCIÓN
{}
%RESOLUCIÓN
{}


\newproblem{reg-24}{med}{*}
%ENUNCIADO
{En un banco de sangre se mantiene el plasma a 0ºF.
Cuando se necesita para una transfusión se calienta en un horno a una temperatura constante de 120ºF.
En un experimento se ha medido la temperatura del plasma a distintos instantes desde el comienzo del calentamiento.
Los resultados son: 
\begin{center}
\begin{tabular}{|lrrrrrrrr|}
\hline
Tiempo (min)	& 5 & 8 & 15 & 25 & 30 & 37 & 45 & 60\\
Temperatura (ºF) & 25 & 50 & 86 & 102 & 110 & 114 & 118 & 120\\
\hline
\end{tabular}
\end{center}
Se pide:
\begin{enumerate}
\item Dibujar el diagrama de dispersión. ¿Qué modelo expliaría la relación entre la temperatura y el tiempo?
\item ¿Qué transformación de escala tendríamos que realizar en las variables para tener una nube de puntos con una
tendencia lineal?
Hacer la representación gráfica. 
\item Construir el modelo de regresión logarítmico de la temperatura sobre el tiempo.
\item Según el modelo, ¿qué temperatura habrá a los 15 minutos?
¿Es fiable la predicción?
Justificar la respuesta.
\end{enumerate}
}
%SOLUCIÓN
{

}
%RESOLUCIÓN
{}


\newproblem{reg-25}{far}{*}
%ENUNCIADO
{La concentración de un fármaco en sangre, $C$ en mg/dl, es función del tiempo, $t$ en horas, y viene dada por la siguiente tabla: 
\begin{center}
\begin{tabular}{lrrrrrrr}
\toprule
Concentración de fármaco en sangre & 2 & 3 & 4 & 5 & 6 & 7 & 8\\
Horas & 25 & 36 & 48 & 64 & 86 & 114 & 168\\
\bottomrule
\end{tabular}
\end{center}
Se pide:
\begin{enumerate}
\item Construir el modelo de regresión lineal de la concentración del fármaco sobre el tiempo.
\item Construir el modelo de regresión exponencial de la concentración del fármaco sobre el tiempo.
\item Usar el mejor de los dos modelos anteriores para predecir la concentración de fármaco en sangre que habrá a las $4.8$ horas.
¿Es fiable la predicción?
\end{enumerate}

Usar las siguientes sumas ($C$=Concentración del fármaco y $T$=tiempo): $\sum c_i=35$ mg/dl, $\sum \log(c_i)=10.6046$
$\log(\mbox{mg/dl})$, $\sum t_j=541$ horas, $\sum \log(t_j)= 29.147$ $\log(\mbox{horas})$, $\sum c_i^2=203$ (mg/dl)$^2$,
$\sum \log(c_i)^2=17.5206$ $\log(\mbox{mg/dl})^2$, $\sum t_j^2=56937$ horas$^2$, $\sum \log(t_j)^2=124.0131$
$\log(\mbox{horas})^2$, $\sum c_it_j=3328$ mg/dl$\cdot$horas, $\sum c_i\log(t_j)=154.3387$
mg/dl$\cdot\log(\mbox{horas})$, $\sum \log(c_i)t_j=951.6961$ $\log(\mbox{mg/dl})\cdot$horas, $\sum
\log(c_i)\log(t_j)=46.08046$ $\log(\mbox{mg/dl})\cdot\log(\mbox{horas})$.
}
%SOLUCIÓN
{Llamando $T$ al tiempo, $C$ a la concentración y $Z$ al logaritmo de la concentración:
\begin{enumerate}
\item $\bar t=5$ horas, $\bar c=77.2857$ mg/dl, $s_t^2=4$ horas$^2$,  $x_𝑦^2=2160.7755$ (mg/dl)$^2$, $s_{tc}=89$ horas(mg/dl).\\
Modelo lineal de $C$ sobre $T$: $c=−33.9643+22.25t$.\\
$r^2=0.9165$.
\item  $\bar z=4.1639$ $\log$(mg/dl), $s_z^2=0.3785$ $\log^2$(mg/dl),
$s_{tz}=1.2291$ horas$\cdot\log$(mg/dl).\\
Modelo exponencial de $C$ sobre $T$: $c=e^{0.3073x+2.6275}$.\\
$r^2=0.9979$.
\item $c(4.8)= 60.498$ mg/dl y es bastante fiable ya que el coeficiente de determinación es muy alto.

\end{enumerate}
}
%RESOLUCIÓN
{En el primer apartado de este problema debemos trabajar con el modelo exponencial de la concentración en función del
tiempo, por lo que vamos a tener que calcular la recta de regresión de $z=\ln C$ en función de $t$. Además, en el
segundo apartado debemos trabajar con el modelo lineal de $t$ en función de $C$. Por lo tanto, la tabla con los
sumatorios precisos es:
\[
\begin{array}{|l|r|r|r|r|r|r|r|}
\hline
t_i & c_i & t_i^2 & c_i^2 & t_i \cdot c_i & z_i=\ln_i & z_i^2 & t_i \cdot z_i \\
\hline
2 & 25 & 4 & 625 & 50 & 3.219 & 10.362 & 6.438 \\
\hline
3 & 36 & 9 & 1296 & 108 & 3.584 & 12.845 & 10.752 \\
\hline
4 & 48 & 16 & 2304 & 192 & 3.871 & 14.985 & 15.484 \\
\hline
5 & 64 & 25 & 4096 & 320 & 4.159 & 17.297 & 20.795 \\
\hline
6 & 86 & 36 & 7396 & 516 & 4.454 & 19.838 & 26.724 \\
\hline
7 & 114 & 49 & 12996 & 798 & 4.736 & 22.430 & 33.152 \\
\hline
8 & 168 & 64 & 28224 & 1344 & 5.124 & 26.255 & 40.992 \\
\hline
\sum= 35 & 541 & 203 & 56937 & 3328 & 29.147 & 124.012 & 154.337 \\
\hline
\end{array}
\]

\begin{enumerate}
\item Para el modelo exponencial de la concentración en función del tiempo tenemos en cuenta que:
\[
C = a \cdot e^{bt}  \Leftrightarrow \ln C = \ln \left( {a \cdot e^{bt} } \right) = \ln a + bt
\]
Por lo tanto, si $z=\ln C$, entonces:
\[
z=\ln a +bt
\]
Y el modelo exponencial se transforma en un modelo lineal de $z$ en función de $t$.

Por otra parte, sabemos que la recta de regresión de $z$ en función de $t$ viene dada por:
\[
z-\bar z = \frac{s_{tz}}{s_t^2}(t-\bar t)
\]
Y teniendo en cuenta los sumatorios obtenidos:
\begin{align*}
\bar t &= \frac{\sum t_i}{n} = \frac{35}{7} = 5,\\
\bar z &= \frac{\sum z_i}{n} = \frac{29.147}{7} = 4.164,\\
s_t ^2  &= \frac{\sum t_i^2}{n}-\bar t^2 = \frac{203}{7}-5^2 = 4,\\
s_z ^2  &= \frac{\sum z_i^2}{n}-\bar z^2 = \frac{124.012}{7}-4.164^2 = 0.38,\\
s_{tz}  &= \frac{\sum t_i z_i}{n}-\bar t \cdot \bar z = \frac{154.337}{7}-5 \cdot 4.164 = 1.228.
\end{align*}

Donde la media de $t$ viene dada en horas, su varianza en horas al cuadrado, la media de $z$ no tiene unidades ($z$ es
un logaritmo neperiano), tampoco las tiene su varianza, y la covarianza tiene las unidades de $t$, es decir horas. 

Con todo ello, la ecuación de la recta de regresión de $z$ en función $t$ vale:
\[
z-4.164 = \frac{1.228}{4}(t-5)\Leftrightarrow z=2.629+0.307\cdot t
\]

Por lo tanto, teniendo en cuenta que:
\[
z=\ln a +b \cdot t=2.629+0.307 \cdot t
\]
obtenemos fácilmente que $b= 0.307$, y para $a$ despejamos tomando exponenciales:
\[
\ln a= 2.629\Leftrightarrow a=e^{2.629} =13.860.
\]
Con todo ello, cuando $t_0=4.8$ horas, el valor obtenido para $C_0$ (en mg/dl) vale:
\[
C(4.8)=13.860 e^{0.307 \cdot 4.8}=60.498 \text{ mg/dl}.
\]

Para ver si es fiable o no la predicción, calculamos el coeficiente de determinación (o el coeficiente de correlación):
\[
r^2  = \frac{{s_{tz} ^2 }}{{s_t ^2 s_z ^2 }} = \frac{{1.228^2 }}{{4 \cdot 0.377}} = 0.999,
\]
luego, mediante el modelo exponencial estamos explicando un $99.9\%$ de la variabilidad de la nube de puntos, y el
modelo exponencial es muy bueno. Por lo tanto, si el modelo es muy bueno y además la predicción la realizamos en
$t_0=4.8$, que está dentro del rango en el que hemos calculado el modelo, sin duda la predicción también será muy
fiable.

\item Para este nuevo apartado debemos predecir el tiempo que debe transcurrir para que la concentración sea de 100
mg/dl mediante un modelo lineal. Por lo tanto necesitamos la recta de regresión del tiempo en función de la concentración:
\[
t-\bar t = \frac{s_{tC}}{s_C^2}(C-\bar C)
\]
Mediante los sumatorios obtenidos en la tabla del comienzo, calculamos:
\begin{align*}
\bar C &= \frac{\sum C_i}{n} = \frac{541}{7} = 77.286,\\
s_C ^2 &= \frac{\sum z_i ^2}{n}-\bar z^2 = \frac{56937}{7}-77.286^2 = 2160.731,\\
s_{tC} &= \frac{\sum t_i C_i}{n}-\bar t \cdot \bar C = \frac{3328}{7}-5 \cdot 77.286 = 90.000.
\end{align*}
Donde la media de $C$ viene dada en mg/dl, su varianza en (mg/dl)$^2$, y la covarianza en horas$\cdot$(mg/dl).

Sustituyendo todo en la ecuación de la recta obtenemos:
\[
t-5 = \frac{90.000}{2160.731}(C-77.286)\Leftrightarrow t=0.0417 \cdot C+ 1.781
\]

Por lo tanto, si $C_0= 100$ entonces $t_0=5.951$ horas.
Para ver si la predicción es adecuada, de nuevo calculamos el coeficiente de determinación:
\[
r^2 = \frac{s_{tC}^2}{s_t ^2 s_C ^2} = \frac{90.000^2}{4 \cdot 2160.731} = 0.937.
\]
Lo cual nos confirma que el modelo lineal, aunque peor que el exponencial, sigue siendo un muy buen modelo.
Si a eso unimos que estamos realizando la predicción dentro del rango de concentraciones en las que lo hemos calculado,
concluimos que sí que será fiable. 
\end{enumerate}
}


\newproblem{reg-26}{fis}{}
%ENUNCIADO
{La actividad de una sustancia radiactiva en función del tiempo (en número de desintegraciones por segundo)
viene dada por la siguiente tabla:
\[
\begin{array}{|l|r|r|r|r|r|r|r|r|}
\hline
t\text{ (horas)} & 0 & 10 & 20 & 30 & 40 & 50 & 60 & 70 \\
\hline
A\text{ ($10^7$ desintegraciones/s)} & 25.9 & 8.16 & 2.57 & 0.81 & 0.25 & 0.08 & 0.03 & 0.01\\
\hline
\end{array}
\]

\begin{enumerate}
\item Representar los datos de la actividad en función del tiempo.
A la vista de la representación, ¿qué modelo de regresión explicaría mejor la relación entre la actividad y el tiempo
transcurrido?
\item Representar los datos de la actividad en función del tiempo en papel semilogarítmico (con escala logarítmica en
el eje de ordenadas).
\item Representar el logaritmo neperiano de la actividad en función del tiempo. ¿Qué modelo de regresión se utilizaría
para ajustar la nube de puntos obtenida?
\item Calcular la ecuación de la recta de regresión del logarítmo neperiaro de la actividad en función del tiempo.
\item Teniendo en cuenta que, en teoría, la actividad de una sustancia radiactiva en función del tiempo viene dada por
la ecuación:
\[
A(t) = A_0 e^{ - \lambda t}
\]
donde $A_0$ es la actividad inicial y $\lambda$ es la llamada constante de desintegración, propia de cada sustancia
radiactiva, utilizar la pendiente de la ecuación de la recta obtenida en el apartado anterior para calcular la
constante de desintegración radiactiva de la sustancia con la que se han generado los datos.
\end{enumerate}
}
%SOLUCIÓN
{Llamando $X$ al tiempo e $Y$ al logaritmo de la actividad
\begin{enumerate}[start=4]
\item $\bar x=35$, $\bar y=-0.7421$, $s_x^2=525$, $s_y^2=6.6664$ y $s_{xy}=-59.1434$.\\
Recta de regresión de $Y$ sobre $X$: $y=-0.1127x+3.2008$.
\item $\lambda=0.1127$.
\end{enumerate}
}
%RESOLUCIÓN
{}


\newproblem{reg-27}{fis}{}
%ENUNCIADO
{Para oscilaciones de pequeña amplitud, el periodo $T$ de oscilación de un péndulo simple viene dado por:
\[
T = 2\pi \sqrt {\frac{L}{g}}
\]
donde $L$ es la longitud del péndulo y $g$ la aceleración de la gravedad.
Para comprobar que dicha ley es cierta, se mide $T$ para varias longitudes del péndulo, obteniéndose la siguiente tabla:
\[
\begin{array}{|l|r|r|r|r|r|}
\hline
L\text{ (cm)} & 52.5 & 68.0 & 99.0 & 116.0 & 146.0 \\
\hline
T\text{ (seg)} & 1.449 & 1.639 & 1.999 & 2.153 & 2.408\\
\hline
\end{array}
\]

Se pide:
\begin{enumerate}
\item Representar los datos del periodo de oscilación frente a la longitud del péndulo.
¿Sería adecuado un modelo lineal para ajustar la nube de puntos?
\item Representar los datos del periodo de oscilación frente a la longitud en papel logarítmico (con escala logarítmica
tanto en el eje de abcisas como en el de ordenadas).
¿Qué modelo de regresión sería adecuado para ajustar la nube de puntos obtenida?
\item Tomar logaritmos neperianos tanto del periodo de oscilación como de la longitud y representar en una gráfica los
logaritmos obtenidos.
¿Qué modelo de regresión sería adecuado para ajustar la nube de puntos obtenida?
\item Calcular la ecuación de la recta de regresión que mejor ajusta la nube de puntos del apartado anterior.
\item Teniendo en cuenta el valor del término independiente de la recta obtenida en el apartado anterior, calcular el
valor de $g$.
\end{enumerate}
}
%SOLUCIÓN
{Llamando $X$ al logaritmo del la longitud e $Y$ al logaritmo del periodo:
\begin{enumerate}[start=4]
\item $\bar x=4.5025$ $\log$cm, $\bar y=0.6407$ $\log$s, $s_x^2=0.1353$ $\log^2$cm, $s_y^2=0.0339$ $\log^2$s, $s_{xy}=0.0677$ $\log$cm\cdot$\log$s.\\
Recta de regresión de $Y$ sobre $X$: $y=0.5006x-1.6132$.
\item $g=994,4145$ cm/s$^2$.
\end{enumerate}
}
%RESOLUCIÓN
{}


\newproblem{reg-28}{far}{}
%ENUNCIADO
{En análisis colorimétrico, es frecuente utilizar la fracción de luz que absorbe una determinada sustancia disuelta
como una medida de la concentración con la que dicha sustancia está presente en la disolución, siempre y cuando se
utilice luz monocromática y la misma longitud recorrida por la luz en cada una de las mediciones.
Si llamamos $I_0$ a la intensidad de luz incidente, $I$ a la intensidad de luz transmitida y $C$ a la concentración de
la sustancia analizada, en un experimento de análisis colorimétrico realizado con Mn y una longitud de onda de 525 nm,
se han obtenido los siguientes datos, donde la concentración de Mn viene dada en mg por cada 100 ml de disolución:
\[
\begin{array}{|l|r|r|r|r|}
\hline
C & 1.00 & 2.00 & 3.00 & 4.00\\
\hline
I/I_0 & 0.418 & 0.149 & 0.058 & 0.026\\
\hline
\end{array}
\]

Se pide:
\begin{enumerate}
\item Representar los datos considerando $I/I_0$ en función de función de $C$.
A la vista de la nube de puntos, ¿qué modelo de regresión sería el más adecuado para expresar la relación entre las
variables?
\item Representar los datos pero en papel semilogarítmico.
\item Calcular la ecuación de la recta de regresión del logaritmo neperiano de $I/I_0$ frente a $C$.
\end{enumerate}
}
%SOLUCIÓN
{Llamando $C$ a la concentración e $Z$ al logaritmo neperiano de $I/I_0$.
\begin{enumerate}[start=3]
\item $\bar c=2.5$ mg/100ml, $\bar z=-2.3183$, $s_c^2=1.25$ (mg/100ml)$^2$, $s_z^2=1.0788$ y $s_{cz}=-1.1595$.\\
Recta de regresión de $Z$ sobre $C$: $z=-0.9276c+0.0007$.
\end{enumerate}
}
%RESOLUCIÓN
{}


\newproblem{reg-29}{psi}{}
%ENUNCIADO
{Se han recogido por medio de unos cuestionarios los niveles de estrés y energía de 14 mujeres durante un año. A partir
de las respuestas del cuestionario se han asignado puntuaciones a cada una de ellas de manera que a mayor puntuación
mayor grado de estrés y energía. Los datos recogidos son:
\[
\begin{array}{rcccccccccccccc}
\hline
\mbox{Edad}   & 21 & 31 & 19 & 21 & 30 & 20 & 22 & 23 & 45 & 24 & 26 & 19 & 25 & 21\\
\mbox{Estrés} & 25 & 19 & 20 & 19 & 24 &  6 & 29 & 25 & 49 &  0 & 10 & 25 & 13 & 23\\
\mbox{Energía}& 25 & 20 & 45 & 60 & 50 & 50 & 10 & 60 & 40 & 60 & 50 & 60 & 85 & 50\\
\hline
\end{array}
\]

Se pide:
\begin{enumerate}
\item Dibujar un diagrama de dispersión que refleje la relación entre el estrés y la energía.
\item ¿Existe relación lineal entre el estrés y la energía?
¿Y entre el estrés y la edad?
Justificar la respuesta.
\item ¿Qué efecto tendría sobre el coeficiente de correlación lineal de la edad y el estrés la eliminación del individuo
de 45 años? Justificar la respuesta.
\item Calcular el coeficiente de correlación de Spearman entre estrés y energía e interpretarlo.
¿Coinciden las conclusiones con las que se deducen del coeficiente de correlación lineal? 
\end{enumerate}
}
%SOLUCIÓN
{Llamando $X$ a la edad, $Y$ al estrés y $Z$ a la energía:
\begin{enumerate}[start=2]
\item $r^2_{yz}=0.14$, lo que indica que casi no hay relación entre el estrés y la energía y $r^2_{xy}=0.31$ lo que
indica que hay una ligera relación entre el estrés y la edad.
\item El coeficiente de correlación lineal disminuye hasta valer casi 0, lo que indica que la relación lineal entre el
estrés y la edad del apartado anterior se debe a este dato atípico, así que, realmente no hay relación entre estrés y
edad.
\item $r_s=-0.41$ lo que indica que hay una ligera relación decreciente entre energía y estrés. 
\end{enumerate}
}
%RESOLUCIÓN
{}

\newproblem{reg-30}{psi}{}
%ENUNCIADO
{Para comprobar el efecto de la herencia genética sobre la inteligencia se desarrolló un estudio en el que se midió el
coeficiente intelectual de varias parejas de gemelos, obteniendo los siguientes resultados: 
\[
(128, 132)\ (116, 112)\ (86, 98)\ (65, 81)\ (104,96)\ (111,111)\ (101, 105)\ (72,75)
\]
Calcular el coeficiente de determinación lineal e interpretarlo.
¿Tiene sentido calcular el coeficiente de correlación?
}
%SOLUCIÓN
{Llamando $X$ al coeficiente intelectual del primer hermano e $Y$ al del segundo: $\bar x=97.875$, $\bar y=101.25$,
$s_x^2=418.3594$, $s_y^2=288.4375$, $s_{xy}=326.5313$ y $r^2=0.8836$, lo que indica que existe bastante relación
lineal entre el coeficiente intelectual de los gemelos. No tiene sentido el coeficiente de correlación lineal porque es
indiferente el orden en que tomemos a los gemelos.
}
%RESOLUCIÓN
{}


\newproblem{reg-31}{psi}{}
%ENUNCIADO
{En un estudio sobre la búsqueda visual se realiza un prueba que consiste en presentarle a un sujeto una matriz de $n$
símbolos y pedirle que pulse rápidamente un botón si entre los símbolos se encuentra uno concreto, u otro botón
diferente si no aparece dicho símbolo.
El tiempo de respuesta de cada participante (en centésimas de segundo) y el número de símbolos de cada matriz aparecen
en la siguiente tabla:
\[
\begin{array}{|l|c|ccccccccc|}
\hline
\mbox{Matrices con} & n & 4 & 5 & 6 & 7 & 8 & 9 & 10 & 11 & 12\\
\cline{2-11}
\mbox{el símbolo} & T & 22 & 24 & 23 & 31 & 33 & 45 & 42 & 46 & 50\\
\hline
\mbox{Matrices sin} & n & 4 & 5 & 6 & 7 & 8 & 9 & 10 & 11 & \\
\cline{2-11}
\mbox{el símbolo} & T & 25 & 24 & 32 & 35 & 43 & 49 & 52 & 56 &\\
\hline
\end{array}
\]

Se pide:
\begin{enumerate}
\item Construir la recta de regresión del tiempo de respuesta sobre el número de símbolos para las matrices con el
símbolo y también para las matrices sin el símbolo.
\item ¿En qué matrices, las que tienen el símbolo o las que no, explica mejor el tiempo de respuesta el número de símbolos?
Justificar la respuesta.
\item Según los modelos anteriores, ¿cuánto tiempo tardará en responder una persona elegida al azar en una matriz de 20
símbolos que contenga al símbolo?
¿Y si no lo contuviese? 
\end{enumerate}
}
%SOLUCIÓN
{Llamando $X$ al número de símbolos e $Y$ al tiempo de respuesta:
\begin{enumerate}
\item Matrices con el símbolo: $\bar x=8$ símbolos, $\bar y=35.1111$ seg, $s_x^2=6.6667$ símbolos$^2$, $s_y^2=104.4321$
seg$^2$, $s_{xy}=25.4446$ símbolos$\cdot$seg.\\
Recta de regresión del tiempo sobre el número de símbolos: $y=3.8333x+4.4444$.
Matrices sin el símbolo: $\bar x=7.5$ símbolos, $\bar y=39.5$ seg, $s_x^2=5.25$ símbolos$^2$, $s_y^2=132.25$
seg$^2$, $s_{xy}=26$ símbolos$\cdot$seg.\\
Recta de regresión del tiempo sobre el número de símbolos: $y=4.9525x+2.3571$.
\item $r^2=0.9292$ en las matrices con el símbolo y $r^2=0.9736$ en las matrices sin el símbolo, así que el número de
símbolos explica un poco mejor el tiempo de respuesta en las matrices sin el símbolo.
\item $y(20)=81.11$ seg si la matriz contiene el símbolo y $y(20)=101.4$ seg si la matriz no contiene el símbolo.
\end{enumerate}
}
%RESOLUCIÓN
{}


\newproblem{reg-32}{gen}{}
%ENUNCIADO
{Se ha realizado un estudio para averiguar la relación entre la edad y la fuerza física. Para ello se ha medido la edad
de 16 participantes y el máximo peso (en kg) que eran capaces de levantar. Los resultados obtenidos fueron: 
\begin{center}
\resizebox{0.7\textwidth}{!}{\input{img/regresion/diagrama_dispersion_fuerza_edad}}
\end{center}
Se pide: 
\begin{enumerate}
\item Calcular el coeficiente de determinación lineal de la muestra completa.
\item Calcular el coeficiente de determinación lineal de la muestra de personas menores de 25 años.
\item Calcular el coeficiente de determinación lineal de la muestra de personas mayores de 25 años.
\item ¿Para qué grupo de edades es más fuerte la relación lineal entre la fuerza física y la edad?
\end{enumerate}

Usar las siguientes sumas ($X$=Edad e $Y=$Peso máximo levantado).
\begin{itemize}[label=--]
\item Muestra total: $\sum x_i=431$ años, $\sum y_j=769$ kg, $\sum x_i^2=13173$ años$^2$, $\sum y_j^2=39675$
kg$^2$ y $\sum x_iy_j=21792$ años$\cdot$kg.
\item Muestra menores de 25 años: $\sum x_i=123$ años, $\sum y_j=294$ kg, $\sum x_i^2=2339$ años$^2$, $\sum y_j^2=14418$
kg$^2$ and $\sum x_iy_j=5766$ años$\cdot$kg.
\item Muestra mayores de 25 años: $\sum x_i=308$ años, $\sum y_j=475$ kg, $\sum x_i^2=10834$ años$^2$, $\sum y_j^2=25257$
kg$^2$ y $\sum x_iy_j=16026$ años$\cdot$kg.
\end{itemize}
}
%SOLUCIÓN
{Llamando $X$ a la edad e $Y$ al peso levantado.
\begin{enumerate}
\item  $\bar{x}=26.9375 años, $s_x^2=97.6836$ años$^2$, $\bar{y}=48.0625$ kg, $s_y^2=169.6836$ kg$^2$, $s_{xy}=67.3164$ años$\cdot$⋅kg y $r^2=0.2734$. 
\item $\bar x=15.5714$ años, $\bar y=42$ kg, $s_x^2=25.3878$ años$^2$, $s_y^2=295.7143$ kg$^2$, $s_{xy}=85.7143$ años$\cdot$kg y $r^2=0.9786$.
\item $\bar x=35.2222$ años, $\bar y=52.7778$ kg, $s_x^2=32.6173$ años$^2$, $s_y^2=20.8395$ kg$^2$,
$s_{xy}=-25.5062$ años$\cdot$kg y $r^2=0.9571$.
\item La relación lineal entre la edad y la fuerza física es un poco más fuerte en los menores de 25 años.
\end{enumerate}
}
%RESOLUCIÓN
{}


\newproblem{reg-33}{psi}{}
%ENUNCIADO
{Para evaluar la capacidad de aprendizaje en la realización de una tarea, se ha medido el tiempo que tarda en
realizarse una tarea en sucesivas repeticiones de la misma. Los resultados obtenidos son:
\[
\begin{array}{lcccccccccc}
\hline
\mbox{Repetición} & 1 & 2 & 3 & 4 & 5 & 6 & 7 & 8 & 9 & 10\\
\mbox{Tiempo (min)} & 80 & 65 & 56 & 50 & 48 & 43 & 41 & 38 & 37 & 35 \\
\hline
\end{array}
\]

Se pide:
\begin{enumerate}
\item Dibujar el diagrama de dispersión.
\item En vista del diagrama de dispersión, construir el modelo de regresión más adecuado del tiempo en función de las
repeticiones.
\item ¿Qué porcentaje de la variabilidad del tiempo explican las repeticiones?
\item ¿Cuanto tiempo tardará por término medio en la 5 repetición de la tarea?
\end{enumerate}
}
%SOLUCIÓN
{Llamando $X$ a las repeticiones, $Y$ al tiempo y $Z$ al logaritmo neperiano del tiempo, se tien:
\begin{enumerate}[start=2]
\item $\bar x=5.5$ repeticiones, $\bar z= 3.8644$ ln(min), $s_x^2=8.25$ repeticiones$^2$, $s_z^2=0.0637$ ln$^2$(min) y
$s_{xz}=-0.7014$ repeticiones$\cdot$ln(min).\\
Recta de regresión del logaritmo del tiempo sobre las repeticiones: $z=-0.085x+4.3320$.\\
Modelo exponencial del tiempo sobre las repeticiones: $y=e^{-0.085x+4.3320}$.
\item $R^2=0.9364$, es decir, un $93.64\%$.
\item $y(5)=49.74$ min.
\end{enumerate}
}
%RESOLUCIÓN
{}


\newproblem{reg-34}{psi}{}
%ENUNCIADO
{En un estudio se ha preguntado a un grupo de personas sobre su ideología política $X$ (izquierda, centro o derecha) y
su opinión sobre la subida o bajada de impuestos $Y$, obteniendo la siguiente tabla de frecuencias:
\begin{center}
\begin{tabular}{|l|c|c|c|}
\hline
$X\backslash Y$ & Bajada & Mantenimiento & Subida \\
\hline
Izquierda & 2 & 6 & 8 \\
\hline
Centro & 3 & 4 & 3 \\
\hline
Derecha & 6 & 5 & 3 \\
\hline
\end{tabular}
\end{center}
¿Se puede concluir que existe relación entre la ideología y la opinión sobre la subida o bajada de impuestos?
Justificar la respuesta.
}
%SOLUCIÓN
{$\chi^2=4.4$ y $C=0.49$ lo que indica que existe bastante relación entre las variables.}
%RESOLUCIÓN
{}


\newproblem{reg-35}{psi}{}
%ENUNCIADO
{Un estudio sobre 100 personas concluye que 26 personas son fumadores y bebedores habituales, 12 son bebedores pero no
fumadores, 18 son fumadores pero no bebedores y 44 no beben ni fuman habitualmente. Según estos datos, ¿podemos decir
que existe relación entre el tabaco y la bebida? Justificar la respuesta. 
}
%SOLUCIÓN
{$\chi^2=14.83$ y $C=0.36$ lo que indica que hay una relación moderada entre los hábitos de fumar y beber.}
%RESOLUCIÓN
{}


\newproblem{reg-36}{psi}{}
%ENUNCIADO
{En un estudio en el que participaron las 8 universidades de una región se ha valorado la excelencia docente e
investigadora, estableciendo los siguientes rankings (de mejor a peor):
\begin{center}
\begin{tabular}{lcccccccc}
Ranking docencia & 3 & 4 & 8 & 5 & 2 & 1 & 6 & 7\\
Ranking investigación & 6 & 5 & 4 & 3 & 7 & 8 & 1 & 2\\
\end{tabular}
\end{center}
¿Se puede decir que existe relación entre la excelencia docente e investigadora? Justificar la respuesta.
}
%SOLUCIÓN
{$r_s=-0.83$, lo que indica una fuerte relación inversa entre la excelencia docente y la excelencia investigadora.}
%RESOLUCIÓN
{}


\newproblem{reg-37}{fis}{*}
% ENUNCIADO
{En un grupo de pacientes se analiza el efecto de una sustancia dopante en el tiempo de respuesta a un determinado estímulo. Para ello, se
suministra en sucesivas dosis, de 0 a 70 mg, la misma cantidad de dopante a todos los miembros del grupo, y se anota el tiempo medio de
respuesta al estímulo, expresado en centésimas de segundo.
\[
\begin{array}{l|r|r|r|r|r|r|r|r}
X \text{ (mg)} & 0 & 10 & 20 & 30 & 40 & 50 & 60 & 70 \\
\hline
Y\ (10^{-2}\text{s}) & 28 & 46 & 62 & 81 & 100 & 132 & 195 & 302 \\
\end{array}
\]

\begin{enumerate}
\item Representar la nube de puntos. A la vista de la representación, ¿crees que el modelo lineal es el que mejor explica el tipo de
relación entre las variables?

\item Calcular la recta de regresión del tiempo en función de la cantidad de dopante, y utilizarla para predecir el tiempo de reacción medio
para una cantidad de dopante de 25 mg.

\item Hacer la misma predicción del apartado anterior con el modelo exponencial. ¿Qué predicción es mejor?

\item Si para el estímulo estudiado los tiempos de reacción superiores a un segundo se consideran peligrosos para la salud, ¿a partir de qué
nivel debería regularse, e incluso prohibirse, la administración de la sustancia dopante?

\end{enumerate}
}
%SOLUCIÓN
{\begin{enumerate}[start=2]
\item Recta de regresión de $Y$ sobre $X$: $y=3.44x-2.25$. $y(25)=83.82$ centésimas de segundo.
\item Recta de regresión de $X$ sobre $Y$: $x=0.25y+5.57$. $x(100)=30.46$ mg.
\end{enumerate}
}
%RESOLUCIÓN
{\begin{enumerate}
\item El diagrama de dispersión de $Y$ sobre $X$ es el siguiente
\[
\includegraphics[scale=0.5]{dispersion}\qquad
\]
A la vista del diagrama se puede decir que el modelo lineal no es el que mejor se ajustaría a la nube de puntos, sino posiblemente el exponencial.

\item La recta de regresión de $Y$ sobre $X$, tiene ecuación
\[
y=\bar y+\frac{s_{xy}}{s_{x}^2}(x-\bar x).
\]
Calculamos primero los estadísticos que necesitamos en la ecuación:
\begin{align*}
\bar x & = \frac{\sum x_{i}}{n}=\frac{0+10+\cdots+70}{8}=\frac{280}{8}=35,  \\
s_{x}^2 & = \frac{\sum x_{i}^2}{n}-\bar x^2 = \frac{0^2+10^2+\cdots+70^2}{8}-35^2=\frac{14000}{8}-35^2=7261.6875,  \\
s_{x} & = \sqrt{7261.6875}=22.91,  \\
\bar y & = \frac{\sum y_{j}}{n}=\frac{28+46+\cdots+302}{8}= \frac{946}{8}=118.25,  \\
s_{y}^2 & = \frac{\sum y_{j}^2}{n}-\bar y^2 = \frac{28^2+16^2+\cdots+302^2}{8}-118.25^2=\frac{169958}{8}-13983.0625=7261.6875,  \\
s_{y} & = \sqrt{7261.6875}=85.22,  \\
s_{xy} & = \frac{\sum x_{i}y_{j}}{N}-\bar x\bar y = \frac{0\cdot 28+10\cdot 46+\cdots +70\cdot 302}{8}-35\cdot 118.25 =\\
& = \frac{47570}{8}-4138.75=1807.5.  \\
\end{align*}
Sustituyendo en la ecuación anterior estos estadísticos calculados obtenemos la recta de regresión de $Y$ sobre $X$.
\[
y=118.25+\frac{1807.5}{525}(x-35)=3.44x-2.25.
\]
Según esta recta, el tiempo de reacción medio para una cantidad dopante de 25 mg sería
\[
y(25)=3.44\cdot 25-2.25=83.82 \textrm{ centésimas de segundo}.
\]

\item Para ver la cantidad dopante que le corresponde 1 segundo de tiempo de reacción, necesitamos utilizar la recta de regresión de $X$ sobre $Y$. La ecuación de esta recta es
\[
x=\bar x+\frac{s_{xy}}{s_{y}^2}(y-\bar y)= 35+\frac{1807.5}{7261.6875}(y-118.25)=0.25y+5.57.
\]
Y ana vez que tenemos la recta de regresión, para estimar la dosis correspondiente a 1 segundo, hacemos la predicción para $y=100$ centésimas de segundo (ya que las unidades de $X$ son centésimas de segundo $10^{-2}$ s). Sustituyendo en la ecuación anterior tenemos
\[ x(100)=0.25\cdot 100+5.57=30.46 \textrm{ mg}. \]
Como la relación entre $X$ e $Y$ es creciente ($s_{xy}>0$), a partir de $30.46$ mg el tiempo de reacción sería superior a un segundo, y por tanto, peligroso para la salud.
\end{enumerate}
}


\newproblem{reg-38}{fis}{*}
% ENUNCIADO
{La artrosis reumatoide es una enfermedad reumática que aparece con frecuencia en las personas mayores. Uno de los índices más utilizados
para ver el grado de actividad de la enfermedad es el RADAI (Rheumatoid Arthritis Disease Activity Index), que mide el grado de actividad en
una escala de 0 (mínima actividad) a 3 (máxima actividad). Para ver de qué manera influye la edad en el grado de actividad de la enfermedad
se ha seleccionado un grupo de personas mayores y se ha medido el índice RADAI en ellos, obteniendo la siguiente tabla de frecuencias:
\begin{center}
\begin{tabular}{|c|c|c|c|c|}
\hline
 RADAI$\backslash$Edad & 40-50 & 50-60 & 60-70 & 70-80 \\
\hline
         0-1          &   8   &   6   &   2   &   1   \\
\hline
         1-2          &   4   &   7   &   5   &   2   \\
\hline
         2-3          &   0   &   2   &   6   &   7   \\
\hline
\end{tabular}
\end{center}
Se pide:
\begin{enumerate}
\item Estudiar si existe relación lineal entre la edad y el RADAI.
\item Calcular la recta de regresión del RADAI sobre la edad. Según la recta, ¿cuánto aumentaría el grado de actividad de la enfermedad por
cada año que pasa?
\item Si se considera que los pacientes don un RADAI de 2 o superior necesitan ayuda en sus actividades diarias, ¿a qué edad se empezaría a
necesitar esta ayuda?
\end{enumerate}
}
%SOLUCIÓN
{Llamando $X$ a la variable que mide la edad e $Y$ a la que mide el RADAI.
\begin{enumerate}
\item $r=0.59$ que indica una relación lineal moderada. 
\item Recta de regresión de $Y$ sobre $X$: $y=0.0442x-1.1575$. Cada año que pase la actividad de la enfermedad aumentará $0.0442$ puntos en
el RADAI.
\item A los $63.4$ años.
\end{enumerate}
}
%RESOLUCIÓN
{Llamemos $X$ a la variable que mide la edad e $Y$ a la que mide el RADAI.
\begin{enumerate}
\item Para ver si existe relación lineal entre $X$ e $Y$, podemos calcular el
coeficiente de correlación lineal, pero para ello necesitamos la media y
desviación típica de cada variable y la covarianza. Antes de calcular estos
estadísticos, obtenemos las distribuciones marginales de cada variable a
partir de la tabla:
\[
\begin{array}{|c|c|c|c|c|c|}
\hline
 Y\backslash X & 40-50 & 50-60 & 60-70 & 70-80 & n_y \\
\hline
         0-1          &   8   &   6   &   2   &   1  & 17 \\
\hline
         1-2          &   4   &   7   &   5   &   2  & 18 \\
\hline
         2-3          &   0   &   2   &   6   &   7  & 15 \\
\hline
         n_x        &  12   &  15   &  13   &  10  & 50 \\
\hline
\end{array}
\]

A partir de aquí calculamos los estadísticos anteriores:
\begin{align*}
\overline{x} &=
\frac{\sum_{i}^{}x_{i}n_{i}}{n} = \frac{45\cdot 12+55\cdot 15+65\cdot
13+75\cdot 10}{50} = \frac{2960}{50} = 59.2,  \\
\overline{y} &=
\frac{\sum_{j}^{}y_{j}n_{j}}{n} = \frac{0.5\cdot 17+1.5\cdot 18+2.5\cdot
15}{50} = \frac{73}{50} = 1.46,  \\
s_{x}^{2} &= \frac{\sum_{i}^{}x_{i}^2n_{i}}{n}-\overline{x}^2 =
\frac{45^2\cdot 12+55^2\cdot 15+65^2\cdot
13+75\cdot 10}{50}-59.2^2 = \\
&= \frac{180850}{50}-3504.64 = 112.36,  \\
s_{x} &= \sqrt{112.36} = 10.6,  \\
s_{y}^{2} &= \frac{\sum_{j}^{}y_{j}^2n_{j}}{n}-\overline{y}^2 =
\frac{0.5^2\cdot 17+1.5^2\cdot 18+2.5^2\cdot
15}{50}-1.46^2 =\\
&= \frac{138.5}{50}-2.1316 = 0.6384,  \\
s_{y} &= \sqrt{0.6384} = 0.8,  \\
s_{xy} &=
\frac{\sum_{ij}^{}x_{i}y_{j}n_{ij}}{n}-\overline{x}\overline{y} =
\frac{45\cdot 0.5\cdot 8+55\cdot 1.5\cdot 4+ \cdots +75\cdot 2.5\cdot
7}{50}-59.2\cdot 1.46 =\\
&= \frac{4570}{50}-86.432 = 4.968.
\end{align*}

Con estos datos, el coeficiente de correlación lineal es
\[
r=\frac{s_{xy}}{s_xs_y}=\frac{4.968}{10.6\cdot 0.8}=0.59,
\]
que indica que existe relación lineal aunque no demasiado fuerte, sino más bien
moderada.

\item La recta de regresión de $Y$ sobre $X$ es
\[
 y=\overline{y}+\frac{s_{xy}}{s_{x}^{2}}(x-\overline{x})=1.46+\frac{4.968}{112.36}(x-59.2)=
 0.0442x-1.1575.
\]
El aumento del grado de actividad del RADAI por cada año que pasa nos lo da el
coeficiente de regresión de $Y$ sobre $X$, que es la pendiente de la recta de
regresión que hemos calculado, es decir, 0.0442 por cada año.

\item Para predecir a qué edad se empezaría a necesitar ayuda, necesitamos
calcular la recta de regresión de $X$ sobre $Y$, que tiene ecuación
\[
 x=\overline{x}+\frac{s_{xy}}{s_{y}^{2}}(y-\overline{y})=5.92+\frac{4.968}{0.6384}(y-1.46)=
 7.782y+47.83.
\]
Sustituyendo $y$ por 2 en esta ecuación tenemos
\[
x(2)=7.782\cdot 2+47.83=63.4 \textrm{ años}.
\]
\end{enumerate}}




\newproblem{reg-39}{fis}{*}
% ENUNCIADO
{En un equipo de baloncesto se ha introducido un programa de estiramientos para ver si se consigue reducir el número de lesiones. Durante
toda una temporada cada jugador realizó ejercicios de estiramiento durante un número fijo de minutos en cada entrenamiento. Al finalizar la
temporada se midió el número de lesiones y se obtuvieron los resultados de la siguiente tabla:
\begin{center}
\begin{tabular}{lrrrrrrrr}
\toprule
Minutos de estiramiento & 0 & 30 & 10 & 15 & 5 & 25 & 35 & 40\\
Lesiones & 4 & 1 & 2 & 2 & 3 & 1 & 0 & 1\\
\bottomrule
\end{tabular}
\end{center}
Se pide:
\begin{enumerate}
\item Calcular la recta de regresión del número de lesiones con respecto al tiempo de estiramiento. 
\item ¿Cual es la disminución de lesiones esperada por cada minuto de estiramiento?
\item ¿Cuántos minutos de estiramiento debe realizar un jugador para no tener ninguna lesión?
¿Es fiable esta predicción?
\end{enumerate}

Usar las siguientes sumas ($X$=Número de minutos de estiramiento, e $Y$=Número de lesiones):
$\sum x_i = 160$ minutos, $\sum y_j=14$ lesiones, $\sum x_i^2= 4700$ minutos$^2$, $\sum y_j^2=36$ lesiones$^2$ y $\sum
x_iy_j=160$  minutos$\cdot$lesiones.
}
%SOLUCIÓN
{Llamando $X$ a la variable que mide el tiempo de estiramiento, e $Y$ a la que mide el número de lesiones en cada jugador:
\begin{enumerate}
\item Recta de regresión de $Y$ sobre $X$: $y-0.08x+3.35$. 
\item Por cada minuto más de estiramiento habrá $0.08$ lesiones menos.
\item Para no tener ninguna lesión habrá que estirar al menos $38.26$ minutos. $r=-0.91$, luego la predicción es bastante fiable.
\end{enumerate}
}
%RESOLUCIÓN
{Llamemos $X$ a la variable que mide el tiempo de estiramiento, e $Y$ a la que
mide el número de lesiones en cada jugador.
\begin{enumerate}
\item La recta de regresión de $Y$ sobre $X$, tiene ecuación
\[
y=\overline{y}+\frac{s_{xy}}{s_{x}^2}(x-\overline{x}).
\]
Calculamos primero los estadísticos que necesitamos en la ecuación:
\begin{align*}
\overline{x} & = \frac{\sum x_{i}}{N}=\frac{0+30+\cdots+40}{8}=\frac{160}{8}=20,  \\
s_{x}^2 & = \frac{\sum x_{i}^2}{N}-\overline{x}^2 =
\frac{0^2+30^2+\cdots+40^2}{8}-20^2=\frac{4700}{8}-20^2=187.5,  \\
s_{x} & = \sqrt{187.5}=13.69,  \\
\overline{y} & = \frac{\sum y_{j}}{N}=\frac{4+1+\cdots+1}{8}=
\frac{14}{8}=1.75,  \\
s_{y}^2 & = \frac{\sum y_{j}^2}{N}-\overline{y}^2 =
\frac{4^2+1^2+\cdots+1^2}{8}-1.75^2=\frac{36}{8}-1.75^2=1.4375,  \\
s_{y} & = \sqrt{1.4375}=1.2,  \\
s_{xy} & = \frac{\sum x_{i}y_{j}}{N}-\overline{x}\overline{y} =
\frac{0\cdot 4+30\cdot 1+\cdots +40\cdot 1}{8}
-20\cdot 1.75 =\\
& = \frac{160}{8}-20\cdot 1.75=-15.  \\
\end{align*}
Sustituyendo en la ecuación anterior estos estadísticos calculados obtenemos la
recta de regresión de $Y$ sobre $X$.
\[
y=1.75-\frac{15}{187.5}(x-20)=-0.08x+3.35.
\]

El incremento que experimenta la variable $Y$ por cada unidad que se
incrementa la variable $X$ según la recta de regresión, es su pendiente o
coeficiente de regresión de $Y$ sobre $X$, que en este caso es $-0.08$. Así,
pues, por cada minuto más de estiramiento se espera tener $0.08$ lesiones
menos.

\item Para predecir el número de minutos que debería estirar un jugador que
quiere tener 0 lesiones, debemos calcular antes la recta de regresión de
tiempo de estiramiento sobre número de lesiones. La ecuación de esta recta
es
\[
x=\overline{x}+\frac{s_{xy}}{s_{y}^2}(y-\overline{y})=
20-\frac{15}{1.4375}(y-1.75)=-10.43y+38.26.
\]

Una vez que tenemos la recta de regresión, para estimar el valor de $X$ para $y=0$, basta con sustituir $y$ por $0$ en esta ecuación y
obtenemos \[ x(0)=-10.43\cdot 0+38.26=38.26. \]

Por último, para ver si esta estimación es fiable, calculamos el coeficiente de
correlación lineal
\[
r=\frac{s_{xy}}{s_{x}s_{y}}=\frac{-15}{13.69\cdot 1.2}=-0.91.
\]
Como el coeficiente de correlación lineal está próximo a -1, el modelo lineal es un
buen modelo y por tanto sus predicciones serán fiables.
\end{enumerate}
}


\newproblem{reg-40}{gen}{*}
% ENUNCIADO
{Un profesor está interesado en analizar la relación existente entre la nota que esperan obtener los alumnos en los exámenes de su
asignatura con la que de verdad obtienen una vez corregidos dichos exámenes. La tabla muestra la nota esperada y la obtenida para 10 alumnos
diferentes: 
\[
\begin{array}{ccc}
\hline
\text{Alumno} & \text{Nota esperada} & \text{Nota obtenida} \\
\hline \hline
   1    &      3.0      &      5.1      \\
\hline
   2    &      6.0      &      4.8      \\
\hline
   3    &      7.0      &      6.0      \\
\hline
   4    &      8.0      &      4.2      \\
\hline
   5    &      3.0      &      5.2      \\
\hline
   6    &      9.0      &      7.5      \\
\hline
   7    &      2.0      &      3.6      \\
\hline
   8    &      5.0      &      3.0      \\
\hline
   9    &      8.0      &      6.5      \\
\hline
   10   &      2.0      &      0.8      \\
\hline
\end{array}
\]

\begin{enumerate}
\item Calcular la recta de regresión de la nota obtenida en función de la nota esperada.
\item Calcular el coeficiente de correlación lineal e interpretarlo.
\item ¿Cuál es la nota que esperaba obtener un alumno que en realidad saca un 4.0?
\end{enumerate}
}
%SOLUCIÓN
{Llamando $X$ a la nota esperada e $Y$ a la nota real obtenida:
\begin{enumerate}
\item Recta de regresión de $Y$ sobre $X$: $y=0.485\,x+2.0994$.
\item $r=0.6786,$ lo que indica una relación creciente moderada.
\item $x(4)=4.6639$ puntos.
\end{enumerate}
}
%RESOLUCIÓN
{Llamemos $X$ a la nota esperada e $Y$ a la nota real obtenida.
\begin{enumerate}
\item La ecuación de la recta de regresión de $Y$ sobre $X$ es
\[
y=\bar y+\frac{s_{xy}}{s_{x}^2}(x-\bar x).
\]
Calculamos primero los estadísticos que necesitamos en la ecuación:
\begin{align*}
\bar x &= \frac{\sum x_{i}}{n}=\frac{53}{10}=5.3,  \\
s_{x}^2 &= \frac{\sum x_{i}^2}{n}-\bar x^2 = \frac{345}{10}-5.3^2=6.41,  \\
s_{x} &=\sqrt{6.41}=2.5318,\\
\bar y &= \frac{\sum y_{j}}{n}=\frac{46.7}{10}=4.67,  \\
s_{y}^2 &= \frac{\sum y_{j}^2}{n}-\bar y^2 =
\frac{250.83}{10}-4.67^2=3.2741,  \\
s_{y}&=\sqrt{3.2741}=1.8094,\\
s_{xy} &= \frac{\sum x_{i}y_{j}}{n}-\bar x\bar y = \frac{278.6}{10}-5.3\cdot 4.67 =3.109.  \\
\end{align*}
Sustituyendo en la ecuación anterior estos estadísticos calculados obtenemos la recta de regresión de $Y$ sobre $X$.
\[
y=4.67+\frac{3.109}{6.41}(x-5.3)=0.485\,x+2.0994.
\]

\item El coeficiente de correlación lineal es
\[
r=\frac{s_{xy}}{s_{x}s_{y}}=\frac{3.109}{2.5318\cdot 1.8094}=0.6786,
\]
y según este valor, podemos decir que existe una dependencia creciente moderada.

\item Para predecir la nota esperada por un alumno que saca un 4, necesitamos la recta de regresión de $X$ sobre $Y$. La ecuación de esta
recta es
\[
x=\bar x+\frac{s_{xy}}{s_{y}^2}(y-\bar y)= 5.3+\frac{3.109}{3.2741}(y-4.67)=0.9496\,y+0.8655.
\]
Sustituyendo $y$ por 4 en esta ecuación obtenemos la predicción deseada
\[
x(4)=0.9496\cdot 4+0.8655=4.6639.
\]
\end{enumerate}
}


\newproblem{reg-41}{fis}{*}
% ENUNCIADO
{Los investigadores están estudiando la correlación entre obesidad y la respuesta individual al dolor. 
La obesidad se mide como porcentaje sobre el peso ideal ($x$).
 La respuesta al dolor se mide utilizando el umbral de reflejo de flexión nociceptiva ($y$), que es una medida de sensación de punzada. Se obtuvieron los siguientes resultados:
\[
\begin{array}{lrrrrrrrrrr}
\toprule
x & 89 & 90 & 75 & 30 & 51 & 75 & 62 & 45 & 90 & 20\\
y & 10 & 12 & 4 & 4.5 & 5.5 & 7 & 9 & 8 & 15 & 3\\
\bottomrule
\end{array}
\]
\begin{enumerate}
\item Dibujar el diagrama de dispersión. Según la nube de puntos, ¿qué modelo explicaría mejor la relación entre el umbral de reflejo y el porcentaje sobre el peso ideal?
\item Según el modelo de regresión más adecuado, ¿qué respuesta al dolor se espera que tenga una persona con una obesidad del 50\%? ¿Es esta predicción fiable?
\item Según el modelo de regresión más adecuado, ¿cuál es la obesidad esperada de una persona con un umbral reflejo de flexión nociceptiva de 10? ¿Es esta predicción fiable?
\end{enumerate}

Usar las siguientes sumas: $\sum x_i=627$, $\sum \log(x_i)=40.3858$, $\sum y_j=78$,
$\sum \log(y_j)=19.4119$, $\sum x_i^2=45141$, $\sum \log(x_i)^2=165.4516$, $\sum y_j^2=738.5$, $\sum
\log(y_j)^2=40.0458$, $\sum x_iy_j=5538.5$, $\sum x_i\log(y_j)=1306.051$, $\sum \log(x_i)y_j=327.3887$, $\sum
\log(x_i)\log(y_j)=80.1831$.
}
%SOLUCIÓN
{
\begin{enumerate}[start=2]
\item $\bar{x}=62.9$, $s_x^2=588.09$, $\bar{y}=9.22$, $s_y^2=11.0056$ y $s_{xy}=82.0356$.\\
Recta de regresión de la respuesta al dolor sobre la obesidad: $y=1.3232+0.1255x$. $r^2=0.8422$.\\
$\overline{\log⁡(x)}=4.0412$, $s_{\log⁡(x)}^2=0.2366$ y $s_{\log⁡(x)y}=1.4973$.
Modelo logarítmico de la respuesta al dolor sobre la obesidad: $y=−16.3578+6.3293\log⁡(x)$.
$r^2=0.8611$.\\
$y(50)=8.4023$.
\item Modelo exponencial de la obesidad sobre la respuesta al dolor: $x=e^{2.7868+0.1361y}$.\\
$x(10)=63.2648$.
\end{enumerate}
}
%RESOLUCIÓN
{
}


\newproblem{reg-42}{med}{*}
%ENUNCIADO
{Se realiza un estudio para establecer una ecuación mediante la cual se pueda utilizar la concentración de estrona en saliva para predecir
la concentración del esteroide en plasma libre. Se extrajeron los siguientes datos de 10 varones sanos:
\[
\begin{array}{|lrrrrrrrrrr|}
\hline
\text{Estrona} & 1.4 & 7.5 & 8.5 & 9.0 & 9.0 & 11 & 13 & 14 & 14.5 & 16\\
\text{Esteroide} & 30.0 & 25.0 & 31.5 & 27.5 & 39.5 & 38.0 & 43.0 & 49.0 & 55.0 & 48.5\\
\hline
\end{array}
\]

\begin{enumerate}
\item Comprobar la idoneidad del modelo lineal de regresión. Si el modelo es apropiado, hallar la recta de regresión de la concentración de
estrona en función de la concentración de esteroide.
\item Si un individuo presenta una concentración de estrona en saliva de 10, ¿qué concentración de esteroide en plasma libre
predeciría el modelo de regresión lineal?
\item Para los dos primeros individuos, calcular los errores que se comenten al utilizar el modelo de regresión lineal para
predecir la concentración de estrona. Razonar a que se deben estos errores.
\end{enumerate}
}
%SOLUCIÓN
{
}
%RESOLUCIÓN
{
}


\newproblem{reg-43}{med}{*}
%ENUNCIADO
{En una análisis de niños sanos se deseaba establecer si existía relación lineal entre la edad (en años) del niño y el ángulo de Clarke (en
grados), obteniéndose en una muestra de 7 niños los valores que aparecen a continuación:
\[
\begin{array}{|l|r|r|r|r|r|r|r|}
\hline
\text{Edad} & 3 & 4 & 5 & 6 & 7 & 8 & 9 \\
\hline
\text{Ángulo de Clarke} & 24 & 26 & 30 & 31 & 34 & 32 & 33\\
\hline
\end{array}
\]
\begin{enumerate}
\item Calcular la ecuación de la recta de regresión del Ángulo de Clarke en función de la edad.
\item ¿Qué tanto por ciento de la variabilidad de la nube de puntos explicamos con el modelo lineal? ¿Se puede considerar un modelo
bueno?
\item Calcular el coeficiente de correlación de Spearman e interpretarlo. ¿Está en consonancia con el coeficiente de correlación lineal?
\end{enumerate}
}
%SOLUCIÓN
{
}
%RESOLUCIÓN
{
}


\newproblem{reg-44}{far}{*}
%ENUNCIADO
{Se quiere estudiar la relación entre las concentraciones de dos sustancias $X$ e $Y$ en la sangre.
Para ello se han medido las concentraciones de estas sustancias en siete individuos, ambas en microgramos por
decilitro de sangre, obteniendo los siguientes resultados
\[
\begin{array}{rrrrrrrr}
   \hline
X & 2.1 & 4.9 & 9.8 & 11.7 & 5.9 & 8.4 & 9.2 \\ 
  Y & 1.3 & 1.5 & 1.7 & 1.8 & 1.5 & 1.7 & 1.7 \\ 
   \hline
\end{array}
\]

Se pide:
\begin{enumerate}
\item ¿Existe relación lineal entre $Y$ y $X$?
\item ¿Existe relación potencial entre $Y$ y $X$?
\item Utilizar el mejor de los modelos anteriores para predecir la concentración de $Y$ para $x=8$ $\mu$gr/dl. ¿Es fiable la predicción?
Justificar la respuesta.	
\end{enumerate}
}
%SOLUCIÓN
{\begin{enumerate}
\item Modelo lineal: $r^2=0.9696$, luego existe una relación lineal muy fuerte.
\item Modelo potencial: $r^2=0.9688$, luego también existe una relación potencial muy fuerte pero un poco menor que la lineal.
\item $y(8)=1.6296$ $\mu$gr/dl.
\end{enumerate}
}
%RESOLUCIÓN
{Para el modelo lineal se tiene
\begin{enumerate}
\item Para ver si existe relación lineal entre $Y$ y $X$ se calcula el coeficiente de determinación lineal:
\begin{align*}
\bar x &= \frac{\sum x_i}{n} = \frac{2.1+\cdots+9.2}{7} = \frac{52}{7} = 7.4286 \text{ $\mu$gr/dl},\\
s_x^2 &= \frac{\sum x_i^2}{n}-\bar x^2 = \frac{2.1^2+\cdots+9.2^2}{7} -7.4286^2= \frac{451.36}{7}-55.1841 = 9.2963 \text{ $(\mu$gr/dl)$^2$},\\
\bar y &= \frac{\sum y_j}{n} = \frac{1.3+\cdots+1.7}{7} = \frac{11.2}{7} = 1.6 \text{ $\mu$gr/dl},\\
s_y^2 &= \frac{\sum y_j^2}{n}-\bar y^2 = \frac{1.3^2+\cdots+1.7^2}{7} -1.6^2= \frac{18.1}{7}-2.56 = 0.0257 \text{ $(\mu$gr/dl)$^2$},\\
s_{xy} &= \frac{\sum x_iy_j}{n}-\bar x\bar y = \frac{2.1\cdot1.3+\cdots+9.2\cdot1.7}{7}-7.4286\cdot1.6 = \frac{86.57}{7}-11.8858 = 0.4814 \text{ $(\mu$gr/dl)$^2$},\\
r^2 &= \frac{s_{xy}^2}{s_x^2 s_y^2} = \frac{0.4814^2}{9.2963\cdot 0.0257} = 0.9696.
\end{align*}
Como el coeficiente de determinación lineal está muy próximo a 1, podemos concluir que existe una relación lineal muy fuerte entre $X$ e $Y$.

\item Del mismo modo, para ver si existe relación potencial entre $Y$ y $X$ se calcula el coeficiente de determinación potencial.
Teniendo en cuenta que la ecuación del modelo potencial $y=ax^b$ se puede convertir en lineal aplicando el logarítmo tanto a $X$ como a $Y$, 
$\ln y = \ln a + b\ln x$, el coeficiente de determinación potencial entre $X$ y $Y$ es el mismo que el coeficiente de determinación lineal entre $\ln(X)$ y $\ln(Y)$.
Así pues, para calcularlo primero construimos las variables $U=\ln(X)$ y $V=\ln(Y)$:
\[
\begin{array}{rrrrrrrr}
   \hline
U = \ln X & 0.7419 & 1.5892 & 2.2824 & 2.4596 & 1.7750 & 2.1282 & 2.2192 \\ 
  V = \ln Y & 0.2624 & 0.4055 & 0.5306 & 0.5878 & 0.4055 & 0.5306 & 0.5306 \\ 
   \hline
\end{array}\]

Y el coeficiente de determinación potencial vale:
\begin{align*}
\bar u &= \frac{\sum u_i}{n} = \frac{0.7419+\cdots+2.2192}{7} = \frac{13.1955}{7} = 1.8851 \text{ $\ln(\mu$gr/dl)},\\
s_u^2 &= \frac{\sum u_i^2}{n}-\bar u^2 = \frac{0.7419^2+\cdots+2.2192^2}{7} -1.8851^2= \frac{26.9397}{7}-3.5536 = 0.295 \text{ $\ln^2(\mu$gr/dl)},\\
\bar v &= \frac{\sum v_j}{n} = \frac{0.2624+\cdots+0.5306}{7} = \frac{3.253}{7} = 0.4647 \text{ $\ln(\mu$gr/dl)},\\
s_v^2 &= \frac{\sum v_j^2}{n}-\bar v^2 = \frac{0.2624^2+\cdots+0.5306^2}{7} -0.4647^2= \frac{1.5878}{7}-0.2159 = 0.0109 \text{ $\ln^2(\mu$gr/dl)},\\
s_{uv} &= \frac{\sum u_iv_j}{n}-\bar u\bar v = \frac{0.7419\cdot0.2624+\cdots+2.2192\cdot0.5306}{7}-1.8851\cdot0.4647 =\\
&= \frac{6.5224}{7}-0.876 = 0.0557 \text{ $\ln^2(\mu$gr/dl)},\\
r^2 &= \frac{s_{uv}^2}{s_u^2 s_v^2} = \frac{0.0557^2}{0.295\cdot 0.0109} = 0.9688.
\end{align*}

Así pues, el modelo potencial también es muy buen modelo para explicar la relación entre $Y$ y $X$ aunque es un poco mejor el lineal.

\item Como el modelo lineal es un poco mejor que el potencial, hay que hacer la predicción con el modelo lineal. Para ello se calcula la recta
de regresión de $Y$ sobre $X$, que tiene ecuación
\[
y = \bar y +\frac{s_{xy}}{s_x^2}(x-\bar x) = 1.6 + \frac{0.4814}{9.2963}(x-7.4286) = 0.0518x+1.2153.
\]

Finalmente, la concentración de $Y$ para $x=8$ $\mu$gr/dl será
\[
y = 0.0518\cdot 8+1.2153 = 1.6296 \text{ $\mu$gr/dl}.
\]
\end{enumerate}
}


\newproblem*{reg-45}{far}{*}
%ENUNCIADO
{Dar ejemplos de:
\begin{enumerate}
\item Dos variables no relacionadas. 
\item Dos variables relacionadas de manera creciente. 
\item Dos variables relacionadas de manera decreciente. 
\end{enumerate}
}



\input{probabilidad}
% Author Alfredo Sánchez Alberca (asalber@ceu.es)

\newproblem{vad-1}{gen}{}
%ENUNCIADO
{Sea $X$ una variable aleatoria discreta cuya ley de probabilidad es
\[
\begin{array}{|c|c|c|c|c|c|}
\hline
X & 4 & 5 & 6 & 7 & 8 \\ 
\hline
f(x) & 0.15 & 0.35 & 0.10 & 0.25 & 0.15 \\ 
\hline
\end{array}
\]
\begin{enumerate}
\item  Calcular y representar gráficamente la función de distribución.
\item  Obtener:
\begin{enumerate}
\item  $P(X<7.5)$.
\item  $P(X>8)$.
\item  $P(4\leq X\leq 6.5)$.
\item  $P(5<X<6)$.
\end{enumerate}
\end{enumerate}
}
%SOLUCIÓN
{
\begin{enumerate}
\item \[
F(x)=
\begin{cases}
0 & \text{si $x<4$,}\\
0.15 & \text{si $4\leq x<5$,}\\
0.5 & \text{si $5\leq x<6$,}\\
0.6 & \text{si $6\leq x<7$,}\\
0.85 & \text{si $7\leq x<8$,}\\
1 & \text{si $8\leq x$.}
\end{cases}
\]
\item $P(X<7.5)=0.85$, $P(X>8)=0$, $P(4\leq x\leq 6.5)=0.6$ y $P(5<X<6)=0$.
\end{enumerate}
}
%RESOLUCIÓN
{}


\newproblem{vad-2}{gen}{}
%ENUNCIADO
{Sea la variable aleatoria X con la siguiente función de distribución:
\[
F(x)=
\begin{cases}
0 & \text{si $x<1$,} \\
1/5 & \text{si $1\leq x< 4$,} \\
3/4 & \text{si $4\leq x<6$,} \\
1 & \text{si $6\leq x$.}
\end{cases}
\]
Se pide:
\begin{enumerate}
\item Distribución de probabilidad.
\item Calcular la siguientes probabilidades:
\begin{enumerate}
\item $P(X=6)$.
\item $P(X=5)$.
\item $P(2<X<5.5)$.
\item $P(0\leq X<4)$.
\end{enumerate}
\item Calcular la media.
\item Calcular la desviación típica. 
\end{enumerate}
}
%SOLUCIÓN
{
\begin{enumerate}
\item \[
\begin{array}{|c|c|c|c|}
\hline
X & 1 & 4 & 6 \\
\hline
f(x) & 1/5 & 11/20 & 1/4\\
\hline
\end{array}
\]
\item $P(X=6)=1/4$, $P(X=5)=0$, $P(2<X<5.5)=11/20$ y $P(0\leq X<4)=1/5$.
\item $\mu=3.9$.
\item $\sigma=1.6703$.
\end{enumerate}
}
%RESOLUCIÓN
{}


\newproblem{vad-3}{med}{*}
%ENUNCIADO
{Se realiza un experimento aleatorio consistente en inyectar un virus a tres tipos de ratas y observar si sobreviven o
no.
Se sabe que la probabilidad de que viva el primer tipo de rata es $0.5$, la de que viva el segundo es $0.4$ y la de
que viva el tercero $0.3$.
Se pide:
\begin{enumerate}
\item Construir la variable aleatoria que mida el número de ratas vivas y su función de probabilidad.
\item Calcular la función de distribución.
\item Calcular $P(X\leq 1)$, $P(X\geq 2)$ y $P(X=1.5)$.
\item Calcular la media y la desviación típica. ¿Es representativa la media?
\end{enumerate}
}
%SOLUCIÓN
{
\begin{enumerate}
\item \[
\begin{array}{|c|c|c|c|c|}
\hline
X & 0 & 1 & 2 & 3 \\
\hline
f(x) & 0.21 & 0.44 & 0.29 & 0.06\\
\hline
\end{array}
\]
\item \[
F(x)=
\begin{cases}
0 & \text{si $x<0$,}\\
0.21 & \text{si $0\leq x<1$,}\\
0.65 & \text{si $1\leq x<2$,}\\
0.94 & \text{si $2\leq x<3$,}\\
1 & \text{si $3\leq x$.}
\end{cases}
\]
\item $P(X\leq 1)=0.65$, $P(X\geq 2)=0.35$ y $P(X=1.5)=0$.
\item $\mu=1.2$ ratas, $\sigma^2=0.7$ ratas$^2$ y $\sigma=0.84$ ratas.
\end{enumerate}
}
%RESOLUCIÓN
{}


\newproblem*{vad-4}{gen}{}
%ENUNCIADO
{Una tómbola asegura que en cada 1000 boletos hay 500 con ``siga intentándolo'', 100 con un premio de 1\euro, 60 con un premio de 2\euro, 30 con un premio de 3\euro, y 10 con un premio de 10\euro. Un individuo decide comprar un boleto que cuesta 1\euro. Se pide:

\begin{enumerate}
\item Construir una variable aleatoria que mida la ganancia (o pérdida) con la compra de un boleto.
\item ¿Cual es la probabilidad de que pierda dinero?
\item ¿Qué ganancia espera obtener?
\end{enumerate}
}
%SOLUCIÓN
{}
%RESOLUCIÓN
{}


\newproblem{vad-5}{med}{}
%ENUNCIADO
{La probabilidad de curación de un paciente al ser sometido a un determinado tratamiento es $0.85$.
Calcular la probabilidad de que en un grupo de $6$ enfermos sometidos a tratamiento:
\begin{enumerate}
\item se curen la mitad.
\item se curen al menos $4$.
\end{enumerate}
}
%SOLUCIÓN
{Llamando $X$ al número de pacientes curados de los 6 sometidos al tratamiento, se tiene que $X\sim B(6,\,0.85)$.
\begin{enumerate}
\item $P(X=3)=0.041$.
\item $P(X\geq 4)=0.9526$.
\end{enumerate}
}
%RESOLUCIÓN
{}


\newproblem*{vad-6}{gen}{*}
%ENUNCIADO
{En una mesa de juego, en 1654, Meré propuso a Pascal la siguiente afirmación: ``es más probable obtener al menos un as con cuatro dados, que al menos un doble as en veinticuatro tiradas de dos dados''.

Demostrar que Meré tenía razón.
}
%SOLUCIÓN
{}
%RESOLUCIÓN
{}


\newproblem{vad-7}{med}{}
%ENUNCIADO
{Se sabe que la probabilidad de que aparezca una bacteria en un mm$^3$ de cierta disolución es de $0.002$.
Si en cada mm$^3$ a los sumo puede aparecer una bacteria, determinar la probabilidad de que en un cm$^3$ haya como
máximo $5$ bacterias.}
%SOLUCIÓN
{Llamando $X$ al número de bacterias en 1 cm$^3$ de disolución, se tiene $X\sim B(1000,\,0.002)\approx P(2)$.\\
$P(X\leq 5)=0.9834$.
}
%RESOLUCIÓN
{}


\newproblem{vad-8}{med}{}
%ENUNCIADO
{Diez individuos entran en contacto con un portador de tuberculosis. La probabilidad de que la enfermedad se contagie del portador a un sujeto cualquiera es $0.10$.
\begin{enumerate}
\item ¿Qué probabilidad hay de que ninguno se contagie?
\item ¿Qué probabilidad hay de que al menos dos se contagien?
\item ¿Cuántos se espera que contraigan la enfermedad?
\end{enumerate}
}
%SOLUCIÓN
{Llamando $X$ al número de personas contagiadas. 
\begin{enumerate}
\item $P(X=0)=0.3487$. 
\item $P(X\geq 2)=0.2639$.
\item $\mu=1$.
\end{enumerate}
}
%RESOLUCIÓN
{}


\newproblem*{vad-9}{gen}{*}
%ENUNCIADO
{Las matrículas de los coches constan de una parte numérica formada por cuatro cifras, y una parte literal. Se pide:

\begin{enumerate}
\item  Hallar la probabilidad de que al pasar 30 coches haya menos de dos cuya parte numérica sea capicúa.
\item  ¿Cuántos coches deben pasar para que la probabilidad de que alguno tenga la parte numérica capicúa sea mayor
que 0.1?
\end{enumerate}
}
%SOLUCIÓN
{}
%RESOLUCIÓN
{}


\newproblem{vad-10}{med}{}
%ENUNCIADO
{La probabilidad de que al administrar una vacuna dé una determinada reacción es $0.001$. Si se vacunan 2000 personas, ¿Cuál es la probabilidad de que aparezca alguna reacción adversa?
}
%SOLUCIÓN
{Llamando $X$ al número de reacciones adversas, $P(X\geq 1)=0.8648$.
}
%RESOLUCIÓN
{}


\newproblem{vad-11}{gen}{*}
%ENUNCIADO
{El número medio de llamadas por minuto que llegan a una centralita telefónica es igual a 120.
Hallar las probabilidades de los sucesos siguientes: 

\begin{enumerate}
\item  $A=\{\text{durante 2 segundos lleguen a la centralita menos de 4 llamadas}\}$.
\item  $B=\{\text{durante 3 segundos lleguen a la centralita 3 llamadas como mínimo}\}$.
\end{enumerate}
}
%SOLUCIÓN
{
\begin{enumerate}
\item Si $X$ es el número de llamadas en 2 segundos, entonces $X\sim P(4)$ y $P(X<4)=0.4335$.
\item Si $Y$ es el número de llamadas en 3 segundos, entonces $Y\sim P(6)$ y $P(Y\geq 3)=0.938$.
\end{enumerate}
}
%RESOLUCIÓN
{}


\newproblem*{vad-12}{nut}{}
%ENUNCIADO
{En un laboratorio de análisis de alimentos se sabe, de experiencias anteriores, que la probabilidad de que una muestra de café contenga plomo en cantidades superiores a las permitidas por la legislación vigente es $0.2$. Si se reciben $50$ muestras de café, ¿cuál es la probabilidad de rechazar al menos $5$?
¿Cuál será el número medio de muestras que rechazaremos?
}
%SOLUCIÓN
{}
%RESOLUCIÓN
{}


\newproblem{vad-13}{far}{}
%ENUNCIADO
{Un proceso de fabricación de un fármaco produce por término medio 6 fármacos defectuosos por hora.
¿Cuál es probabilidad de que en un hora se produzcan menos de 3 fármacos defectuosos?
¿Y la de que en la próxima media hora se produzcan más de un fármaco defectuoso?}
%SOLUCIÓN
{Llamando $X$ al número de fármacos defectuosos en 1 hora, se tiene $X\sim P(6)$ y $P(X<3)=0.062$.\\
Llamando $Y$ al número de fármacos defectuosos en $1/2$ hora, se tiene $Y\sim P(3)$ y $P(Y>1)=0.8009$.}
%RESOLUCIÓN
{}


\newproblem{vad-14}{gen}{}
%ENUNCIADO
{Una mecanógrafa comete, en promedio, una errata cada 2000 caracteres que escribe.
Suponiendo que escribe un folio con treinta líneas y setenta caracteres por línea, ¿cuál es la probabilidad de que
cometa más de un error en dicho folio?}
%SOLUCIÓN
{Llamando $X$ al número de errores en un folio, se tiene que $X\sim B(2100,\,1/2000)\approx P(1.05)$, y $P(X>1)=0.2826$.
}
%RESOLUCIÓN
{}


\newproblem{vad-15}{gen}{}
%ENUNCIADO
{Un examen de tipo test consta de 10 preguntas con tres respuestas posibles para cada una de ellas.
Se obtiene un punto por cada respuesta acertada y se pierde medio punto por cada pregunta fallada.
Un alumno sabe tres de las preguntas del test y las contesta correctamente, pero no sabe las otras siete y las contesta
al azar.
¿Qué probabilidad tiene de aprobar el examen?}
%SOLUCIÓN
{Llamando $X$ al número de preguntas acertadas de las 7 contestadas al azar, se tiene $X\sim B(7,\,1/3)$ y $P(X\geq
4)=0.1733$.}
%RESOLUCIÓN
{}


\newproblem*{vad-16}{gen}{*}
%ENUNCIADO
{Un equipo de fútbol tiene 7 delanteros. Se sabe que por término medio, cada delantero se pierde 5 partidos por lesión en una temporada de 40 partidos. Suponiendo que todos los delanteros tienen la misma probabilidad de lesionarse, se pide:
\begin{enumerate}
\item  ¿Cuál es la probabilidad de que en un partido determinado tenga menos de 5 delanteros en condiciones de jugar?
\item  ¿Cuál es la probabilidad de que a lo largo de la temporada haya más de un partido en que tenga menos de 5  delanteros en condiciones de jugar?
\end{enumerate}
}
%SOLUCIÓN
{}
%RESOLUCIÓN
{}


\newproblem*{vad-17}{amb}{}
%ENUNCIADO
{Para reforestar un bosque se compran árboles a un vivero en el que el 4\% de los árboles suele morir debido a una enfermedad. Si la repoblación se efectúa por parcelas en las que se ponen 10 árboles, se pide:
\begin{enumerate}
\item Calcular la probabilidad de que no muera ningún árbol en una parcela.
\item Calcular la probabilidad de no mueran más de 2 árboles en una parcela.
\item Si en total se reforestan 3000 parcelas, ¿cuál es la probabilidad de que haya alguna en la que mueran más de dos árboles?
\end{enumerate}
}
%SOLUCIÓN
{}
%RESOLUCIÓN
{}


\newproblem{vad-18}{med}{*}
%ENUNCIADO
{En un estudio sobre un determinado tipo de parásito que ataca el riñón de las ratas, se sabe que el número medio de
parásitos en cada riñón es 3.
Se pide:
\begin{enumerate}
\item  Calcular la probabilidad de que una rata tenga más de 8 parásitos.\\
(Nota: se supone que una rata normal tiene dos riñones).
\item  Si se tienen 10 ratas, ¿cuál es la probabilidad de que haya al menos 9 con parásitos?
\end{enumerate}
}
%SOLUCIÓN
{
\begin{enumerate}
\item Si $X$ es el número de parásitos en una rata, $X\sim P(6)$ y $P(X>8)=0.1528$.
\item Si $Y$ es el número de ratas con parásitos en un grupo de 10 ratas, entonces $Y\sim B(10,\,0.9975)$ y $P(Y\geq
9)=0.9997$.
\end{enumerate}
}
%RESOLUCIÓN
{}


\newproblem{vad-19}{med}{*}
% ENUNCIADO
{Se ha comprobado experimentalmente que una de cada 20 billones de células expuestas a un determinado tipo de radiación muta volviéndose
cancerígena. Sabiendo que el cuerpo humano tiene aproximadamente 1 billón de células por kilogramo de tejido, calcular la probabilidad de
que una persona de 60 kg expuesta a dicha radiación desarrolle cáncer. Si la radiación ha afectado a 3 personas de 60 kg, ¿cuál es la
probabilidad de que desarrolle el cáncer más de una? }
% SOLUCIÓN
{Llamando $X$ al número de mutaciones, $P(X>0)=0.9502$.\\ 
Llamando $Y$ al número de personas que desarrollan el cáncer, $P(Y\geq 1)=0.9999$.
}
% RESOLUCIÓN
{}


\newproblem{vad-20}{med}{*}
%ENUNCIADO
{En un servicio de urgencias de cierto hospital se sabe que, en media, llegan 2 pacientes a la hora.
Calcular:
\begin{enumerate}
\item Si los turnos en urgencias son de 8 horas, ¿cuál será la probabilidad de que en un turno lleguen más de 5 pacientes?
\item Si el servicio de urgencias tiene capacidad para atender adecuadamente como mucho a 4 pacientes a la hora, ¿cuál
es la probabilidad de que a lo largo de un turno de 8 horas el servicio de urgencias se vea desbordado en alguna de las
horas del turno?
\end{enumerate}
}
%SOLUCIÓN
{
\begin{enumerate}
\item Llamando $X$ al número de pacientes en un turno, se tiene que $X\sim P(16)$ y $P(X>5)=0.9986$.
\item Llamando $Y$ al número de horas en el que el servicio se vea desbordado porque lleguen más de 4 pacientes, se
tiene que $Y\sim B(8,\,0.0527)$ y $P(Y\geq 1)=0.3515$.
\end{enumerate}
}
%RESOLUCIÓN
{}


\newproblem*{vad-21}{amb}{}
%ENUNCIADO
{En un Parque Nacional se contabilizan 15 linces. Si sabemos, por estudios previos, que mueren en promedio 1 de cada 10 individuos a lo largo de un año (ya sea por accidentes, caza de furtivos o por causas naturales):
\begin{enumerate}
\item ¿Cuál es la probabilidad de que en el Parque Nacional se contabilicen más de 2 muertes de linces en un año?
\item Suponiendo un periodo de 12 años, ¿cuál es la probabilidad de que en el Parque Nacional haya algún año
en el que mueran 2 linces?
\end{enumerate}
}
%SOLUCIÓN
{}
%RESOLUCIÓN
{}


\newproblem{vad-22}{med}{*}
%ENUNCIADO
{El síndrome de Turner es una anomalía genética que se caracteriza porque las mujeres tienen sólo un cromosoma $X$.
Afecta aproximadamente a 1 de cada 2000 mujeres.
Además, aproximadamente 1 de cada 10 mujeres con síndrome de Turner, como consecuencia, también sufren un
estrechamiento anormal de la aorta.
Se pide:
\begin{enumerate}
\item En un grupo de 4000 mujeres, ¿cuál es la probabilidad de que haya más de 3 afectadas por el síndrome de Turner?
¿Y de que haya alguna con estrechamiento de aorta como consecuencia de padecer el síndrome de Turner?
\item En un grupo de 20 chicas afectadas por el síndrome de Turner, ¿cuál es la probabilidad de que menos de 3 sufran
un estrechamiento anormal de la aorta?
\end{enumerate}
}
%SOLUCIÓN
{
\begin{enumerate}
\item Si $X$ es el número de mujeres afectadas por el síndrome de Turner en el grupo de 4000 mujeres, entonces $X\sim
B(4000,\,1/2000)\approx P(2)$ y $P(X>3)=0.1429$.\\
Si $Y$ es el número de mujeres con estrechamiento de la aorta en el grupo de 4000 mujeres, entonces $Y\sim
B(4000,\,1/20000)\approx P(0.2)$ y $P(Y>0)=0.1813$.
\item Si $Z$ es el número de mujeres con estrechamiento de la aorta en el grupo de 20 mujeres con el síndrome de
Turner, entonces $Z\sim B(20,\,1/10)$ y $P(Z<3)=0.6769$.
\end{enumerate}
}
%RESOLUCIÓN
{}


\newproblem{vad-23}{amb}{}
%ENUNCIADO
{Por estudios previos se sabe que, en una comarca, hay dos tipos de larvas que parasitan, de forma completamente
independiente, los chopos, y que producen su muerte.
Si la larva de tipo $A$ está parasitando un 15\% de los chopos, y la $B$ un 30\%, y en una zona concreta de la comarca
hay 10 chopos:
\begin{enumerate}
\item ¿Qué probabilidad hay que de estén siendo parasitados por $A$ más de dos?
\item ¿Qué probabilidad hay de que estén libres de $B$ más de 8?
\item ¿Qué probabilidad hay de que más de 1 tenga los dos tipos de larva?
\item ¿Qué probabilidad hay de que más de 3 tengan algún tipo de larva?
\end{enumerate}
}
%SOLUCIÓN
{
\begin{enumerate}
\item $X_A$ es el número de chopos parasitados por larvas del tipo $A$, entonces $X_A\sim B(10,\,0.15)$ y
$P(X_A>2)=0.1798$.
\item Si $X_{\overline{B}}$ es el número de chopos no parasitados por larvas del tipo $B$, entonces
$X_{\overline{B}}\sim B(10,\,0.7)$ y $P(X_{\overline{B}}>8)=0.1493$.
\item Si llamamos $X_{A\cap B}$ al número de chopos parasitados por larvas de ambos tipos, entonces $X_{A\cap B}\sim
B(10,\,0.045)$ y $P(X_{A\cap B}>1)=0.0717$.
\item Si llamamos $X_{A\cup B}$ al número de chopos parasitados por algún tipo de larva, entonces $X_{A\cup B}\sim
B(10,\,0.405)$ y $P(X_{A\cup B}>3)=0.6302$.
\end{enumerate}
}
%RESOLUCIÓN
{}


\newproblem*{vad-24}{amb}{*}
%ENUNCIADO
{En un Parque Regional se contabilizan 10 parejas de buitre leonado. Si sabemos que el 70\% de las parejas de esta especie logran que alguna de sus crías sobreviva:
\begin{enumerate}
\item ¿Cuál es la probabilidad de que 8 parejas de buitre leonado del Parque logren que alguna de sus crías sobreviva?
\item ¿Cuál es la probabilidad de que alguna pareja logre que alguna de sus crías sobreviva?
\item Si sabemos que en dicho Parque Regional nidifican, en promedio, 8 parejas al año, ¿cuál es la probabilidad de que en un año concreto nidifiquen más de 6?
\end{enumerate}
}
%SOLUCIÓN
{}
%RESOLUCIÓN
{}


\newproblem*{vad-25}{gen}{}
%ENUNCIADO
{Al lanzar 100 veces una moneda, ¿cuál es la probabilidad de obtener entre 40 y 60 caras?
}
%SOLUCIÓN
{}
%RESOLUCIÓN
{}


\newproblem{vad-26}{psi}{}
%ENUNCIADO
{El trastorno de pánico aparece en 1 de cada 75 personas.
¿Cuál es la probabilidad de que en un grupo de 100 personas aparezca alguna con trastorno de pánico?
¿Cuál es el número esperado de personas con trastorno de pánico en ese grupo?
}
%SOLUCIÓN
{Llamando $X$ al número de personas que sufren trastorno del pánico en el grupo de 100 personas, se tiene que $X\sim
B(100,\,1/75)$ y $P(X\geq 1)=0.7379$.}
%RESOLUCIÓN
{}


\newproblem{vad-27}{med}{}
%ENUNCIADO
{Se sabe que 2 de cada 1000 pacientes son alérgicos a un fármaco $A$, y que 6 de cada 1000 lo son a un fármaco $B$.
Además, el 30\% de los alérgicos a $B$, también lo son a $A$.
Si se aplican los dos fármacos a 500 personas,
\begin{enumerate}
\item ¿Cuál es la probabilidad de que no haya ninguna con alergia a $A$?
\item ¿Cuál es la probabilidad de que haya al menos 2 con alergia a $B$?
\item ¿Cuál es la probabilidad de que haya menos de 2 con las dos alergias?
\item ¿Cuál es la probabilidad de que haya alguna con alergia?
\end{enumerate}
}
%SOLUCIÓN
{
\begin{enumerate}
\item Llamando $X_A$ al número de personas alérgicas al fármaco $A$ en el grupo de 500 personas, se tiene que $X_A\sim
B(500,\,0.002)\approx P(1)$ y $P(X_A=0)=0.3678$.
\item Llamando $X_B$ al número de personas alérgicas al fármaco $B$ en el grupo de 500 personas, se tiene que $X_B\sim
B(500,\,0.006)\approx P(3)$ y $P(X_B\geq 2)=0.8009$.
\item Llamando $X_{A\cap B}$ al número de personas alérgicas a ambos fármacos $A\cap B$ en el grupo de 500 personas, se
tiene que $X_A\sim B(500,\,0.0018)\approx P(0.9)$ y $P(X_{A\cap B}<2)=0.7725$.
\item Llamando $X_{A\cup B}$ al número de personas alérgicas a alguno de los fármacos $A\cup B$ en el grupo de 500
personas, se tiene que $X_A\sim B(500,\,0.0062)\approx P(3.1)$ y $P(X_{A\cup B}\geq 1)=0.9550$.
\end{enumerate}
}
%RESOLUCIÓN
{}


\newproblem{vad-28}{gen}{}
%ENUNCIADO
{En una clase hay 40 alumnos de los cuales el 35\% son fumadores.
Si se toma una muestra aleatoria con reemplazamiento de 4 alumnos, ¿cuál es la probabilidad de que haya al menos 1
fumador?
¿Cuál sería dicha probabilidad si la muestra se hubiese tomado sin reemplazamiento?}
%SOLUCIÓN
{Si $X$ es el número de fumadores en una muestra aleatoria con reemplazamiento de tamaño $4$, entonces $X\sim
B(4,\,0.35)$ y $P(X\geq 1)=0.8215$.\\
Si la muestra es sin reemplazamiento $P(X\geq 1)= 0.8364$.}
%RESOLUCIÓN
{}


\newproblem{vad-29}{med}{*}
%ENUNCIADO
{Se sabe que por término medio 2 de cada 10000 niños que nacen son albinos.
\begin{enumerate}
\item Si en una región nacen cada año 22000 niños ¿cuál es la probabilidad de que un año nazcan al menos 4 albinos?
\item ¿Cuál es la probabilidad de que en esa región, en un periodo de 10 años no nazca ningún niño albino?
\end{enumerate}
}
%SOLUCIÓN
{
\begin{enumerate}
\item Llamando $X$ al número de niños albinos que nacen en un año, se tiene que $X\sim B(22000,\,2/10000)\approx
P(4.4)$ y $P(X\geq 4)=0.6406$.
\item Llamando $Y$ al número de niños albinos que nacen en 10 años, se tiene que $Y\sim B(220000,\,2/10000)\approx
P(44)$ y $P(Y=0)=0$.
\end{enumerate}
}
%RESOLUCIÓN
{}


\newproblem{vad-30}{med}{*}
%ENUNCIADO
{Suponiendo una facultad en la que hay un 60\% de chicas y un 40\% de chicos:
\begin{enumerate}
\item  Si un año van 6 alumnos a hacer prácticas en un hospital, ¿qué probabilidad hay de que vayan más chicos que chicas?
\item En un período de 5 años, ¿cuál es la probabilidad de que más de 1 año no haya ido ningún chico?
\end{enumerate}
}
%SOLUCIÓN
{\begin{enumerate}
\item Si $X$ es el número de chicos, $X\sim B(6,\,0.4)$  y $P(X\geq 4)=0.1792$.
\item Si $Y$ es el número de años que no ha ido ningún chico, $Y \sim B(5,\,0.0467)$ y $P(Y>1)=0.0199$.
\end{enumerate}
}
%RESOLUCIÓN
{\begin{enumerate}
\item Si consideramos un total de $n=6$ alumnos, con un $60\%$ de chicas y un $40\%$ de chicos, la variable $X$ que es el número de alumnos chicos que van a hacer las prácticas de un total de 6, sigue una distribución binomial de 6 intentos y probabilidad de éxito igual a $0.4$: $X\sim B(6\,,\,0.4)$. Como nos piden la probabilidad de que haya más chicos que chicas, eso se consigue si el número de chicos es 4 o más. Por lo tanto, nos piden la probabilidad de que $X$ sea mayor o igual que 4:
\[
P(X \ge 4) = P(X = 4) + P(X = 5) + P(X = 6)
\]
Calculando estas probabilidades con la función de probabilidad de la variable binomial, tenemos
\begin{align*}
P(X = 4) &= \binom{6}{4}\cdot 0.4^4  \cdot (1-0.4)^{6-4}  = 12\cdot 0.4^4\cdot 0.6^2 = 0.1382,\\
P(X = 5) &= \binom{6}{5}\cdot 0.4^5  \cdot (1-0.4)^{6-5}  = 6\cdot 0.4^5 \cdot 0.6 = 0.0369,\\
P(X = 6) &= \binom{6}{6}\cdot 0.4^6  \cdot (1-0.4)^{6-6}  = 1\cdot 0.4^6 \cdot 0.6^0 = 0.0041
\end{align*}
Y sumando los tres resultados obtenidos:
\[
P(X \ge 4) = P(X = 4) + P(X = 5) + P(X = 6)= 0.1382+0.0369+0.0041=0.1792.
\]

\item Si consideramos, en un total de 5 años, la probabilidad de que más de un año no haya ido ningún chico, la variable a tener en cuenta $Y$ será el número de años en el total de 5 sin ningún chico, y de nuevo esta variable sigue una distribución binomial, esta vez con 5 intentos, cuyo éxito viene dado por la probabilidad de que en un año concreto no haya ningún chico: $Y \sim B(5\,,\,p)$, donde $p=P(X=0)$.
\[
p = P(X = 0) = \binom{6}{0}\cdot 0.4^0  \cdot (1-0.4)^{6-0}  = 1\cdot 0.4^0\cdot 0.6^6 = 0.0467.
\]
Por lo tanto, $Y \sim B(5\,,\,0.0467)$.

Además, nos piden la probabilidad de más de una año en el total de 5; es decir: 
\[
P(Y>1)=1-P(Y\leq 1) = 1-P(Y=0)-P(Y=1).
\]
Aplicando, de nuevo, la fórmula de la función de probabilidad de la binomial, obtenemos:
\begin{align*}
P(Y = 0) &= \binom{5}{0}\cdot 0.0467^0  \cdot (1 - 0.0467)^{5-0} = 1\cdot 0.0467^0\cdot 0.9533^5  = 0.7873,\\
P(Y = 1) &= \binom{5}{1}\cdot 0.0467^1  \cdot (1 - 0.0467)^{5-1} = 5\cdot 0.0467^1\cdot 0.9533^4  = 0.1928.
\end{align*}
Teniendo lo anterior en cuenta, la probabilidad que nos piden vale:
\[
P(Y>1)=1-P(Y=0)-P(Y=1)=1-0.7873-0.1928=0.0199.
\]
\end{enumerate}
}


\newproblem*{vad-31}{med}{*}
%ENUNCIADO
{La probabilidad de que en un grupo de 5 individuos mayores de 70 años todos padezcan arterioesclerosis cerebral es de $12,5$ por mil.
\begin{enumerate}
\item ¿Cuál es la probabilidad de padecer la enfermedad entre los mayores de 70 años?
\item En un grupo de 1000 personas, ¿cuál es la probabilidad de que padezcan la enfermedad más de 450?
\end{enumerate}
}
%SOLUCIÓN
{}
%RESOLUCIÓN
{}


\newproblem{vad-32}{gen}{*}
%ENUNCIADO
{¿Cuánto habría que restar a cada pregunta errada en un examen de tipo test de 5 preguntas con cuatro opciones y sólo
una correcta, para que un individuo que responda al azar tenga una puntuación esperada de 0? 
}
%SOLUCIÓN
{$1/3$.}
%RESOLUCIÓN
{Si llamamos $X$ al número de preguntas acertadas, está claro que $X$ sigue una distribución binomial $B(5,\,1/4)$ ya
que el examen tiene 4 preguntas, y la probabilidad de acertar cualquiera de ellas al azar es $1/4$ ya que hay cuatro
opciones y sólo una es la correcta.
}


\newproblem{vad-33}{psi}{}
%ENUNCIADO
{Se sabe que el $6.8$\% las personas presentan a lo largo de su adolescencia un trastorno de hiperactividad, de los
cuales tres cuartas partes son mujeres.
Si en la población hay el mismo número de hombres y mujeres, se pide:
\begin{enumerate}
\item Calcular la probabilidad de que en una muestra de tres hombres, haya alguno que haya tenido hiperactividad en su
adolescencia. 
\item Calcular la probabilidad de que en una muestra de 2 hombres y 2 mujeres, haya alguno que haya tenido
hiperactividad en su adolescencia. 
\end{enumerate}
}
%SOLUCIÓN
{
\begin{enumerate}
\item Si llamamos $X$ al número de hombres que han tenido hiperactividad en su adolescencia en una muestra de 3
hombres, se tiene que $X\sim B(3,\,0.034)$ y $P(X\geq 1)=0.0986$.
\item Si llamamos $X_H$ al número de hombres que han tenido hiperactividad en su adolescencia en una muestra de 2
hombres y $X_M$ al número de mujeres que han tenido hiperactividad en su adolescencia en una muestra de 2 mujeres,
entonces $X_H\sim B(2,\,0.034)$ y $X_M(2,\,0.102)$. Entonces $P(X_H\geq 1\cup X_M\geq 1)=0.2475$.
\end{enumerate}
}
%RESOLUCIÓN
{}


\newproblem{vad-34}{gen}{*}
%ENUNCIADO
{A un hospital llegan pacientes por la mañana a efectuarse extracciones de sangre. Se ha medido la frecuencia de llegada de los mismos en
intervalos de 15 minutos. La distribución de probabilidad (medida de forma frecuentista) se  muestra en la siguiente tabla:
\[
\begin{array}{c|c|c|c|c|c|c|c|}
  X   &  0  &  1  &  2   &  3   &  4   &  5  &  6   \\
\hline
 P(x) & 0.1 & 0.2 & 0.25 & 0.15 & 0.15 & 0.1 & 0.05 \\
\end{array}
\]
Se pide:
\begin{enumerate}
\item Calcular la probabilidad de que en un intervalo de 15 minutos lleguen 2 o más personas, y probabilidad de que lleguen menos de 8
personas.
\item ¿Cuál es el número medio esperado de personas que llegarán a sacarse sangre cada 15 minutos?
\item Suponiendo que el número de personas que llegan a sacarse sangre en 15 minutos sigue una distribución de Poisson de media la  
calculada en el apartado anterior, ¿cuál es la probabilidad de que llegue alguna en 15 minutos? ¿Y de que llegue alguna en 5 minutos? 
\end{enumerate}
} 
%SOLUCIÓN
{Llamemos $J$ al suceso consitente en que una persona con la lesión sea joven, $C$ al suceso consistente en curarse, y $A$ y $B$ a los sucesos consistentes en aplicar las respectivas técnicas de
rehabilitación:
\begin{enumerate}
\item Llamando $X$ al número de personas que llegan en un intervalo de 15 minutos: $P(X\geq 2)=0.7$ y $P(X<8)=1$.
\item $\mu=2.55$ personas.
\item Suponiendo $X\sim P(2.55)$, $P(X\geq 1)=0.9219$.\\
Llamando $Y$ al número de personas que llegan en un intervalo de 5 minutos, $P(Y\geq 1)=0.5726$.
\end{enumerate}
}
%RESOLUCIÓN
{\begin{enumerate}
\item Teniendo en cuenta que nos dan la distribución de probabilidad de la variable aleatoria discreta $X$ que expresa el número de
pacientes que llegan en 15 minutos, nos están pidiendo $P(X\geq 2)$ y $P(X<8)$. Estas probabilidades son:
\begin{align*}
P(X\geq 2) & =1-P(X<2)=1-P(X=0)-P(X=1)=1-0.2-0.3=0.7,\\
P(X<8)& =1-P(X\geq 8)=1.
\end{align*}
\item El número esperado es la media de la variable aleatoria:
\[
\mu =\sum xf(x)=0\cdot 0.1+1\cdot 0.2+2\cdot 0.25+3\cdot 0.15+4\cdot 0.15+5\cdot 0.1+6\cdot 0.05=2.55 \text{ personas}.
\]

\item Suponiendo que $X$ sigue una distribución de Poisson con $\lambda =2.55$, $X\sim P(2.55),$ la probabilidad de que llegue alguna
persona en 15 minutos vendrá dada por:
\[
P(X\geq 1)=1-P(X=0)=1-e^{-2.55}\dfrac{2.55^{0}}{0!}=0.9219.
\]

Para la segunda pregunta, teniendo en cuenta que el número medio de personas que llegan cada 15 minutos es $2.55$, en 5 minutos llegarán en
media $2.55/3 = 0.85$ personas, y, por tanto, tenemos una nueva variable aleatoria $Y$, que seguirá una distribución de Poisson con $\lambda
^{\prime }=0.85.$
\[
P(Y\geq 1)=1-P(Y=0)=1-e^{-0.85}\dfrac{0.85^{0}}{0!}=0.5726.
\]
\end{enumerate}
}


\newproblem{vad-35}{gen}{*}
%ENUNCIADO
{En las siguientes tablas, indicar razonadamente, en los caso que sea posible,
los valores de $h$ que deben ponerse en cada tabla para que se tenga una
distribución de probabilidad:
\[
\begin{array}{c|c}
x & f(x) \\
\hline
-2 & 0.3  \\
5 & h  \\
8 & 0.1
\end{array}
\qquad
\begin{array}{c|c}
x & f(x) \\
\hline
 1 & -0.2 \\
 3 & 0.7 \\
 4 & h
\end{array}
 \qquad
\begin{array}{c|c}
x & f(x) \\
\hline
 2 & h \\
 3 & 0.5 \\
 4 & 0.6
\end{array}
\]

En las tablas que constituyan una distribución de probabilidad:
\begin{enumerate}
\item Representar gráficamente la función de distribución.
\item Calcular media y desviación típica.
\item Calcular la mediana.
\item Si a los valores de $x$ se multiplican por una constante $k<0$, ¿cómo se ve afectada la media? ¿Y la desviación típica?
\end{enumerate}
} 
%SOLUCIÓN
{La única tabla que puede ser una distribución de probabilidad es la primera para $h=0.6$.
\begin{enumerate}[start=2]
\item $\mu=3.2$ y $\sigma=3.516$.
\item $Me=5$.
\item $\mu_y=k\mu_x$ y $\sigma_y=|k|\sigma_x$.
\end{enumerate}
}
%RESOLUCIÓN
{La segunda tabla no puede ser una distribución de probabilidad pues $f(1)=-0.2$ y la función de probabilidad no puede tomar valores
negativos. Por otro lado, la tercera tabla tampoco puede ser una distribución de probabilidad ya que la suma de todas las probabilidades
debe ser 1, y para ello debería ser $h=-0.1$, lo cual no es posible al no poder tomar valores negativos. Así pues la única tabla que puede
ser una distribución de probabilidad es la primera, y como la suma de todas las probabilidades tiene que ser 1, $0.3+h+0.1=1$, se deduce que
$h=0.6$. Trabajaremos pues, con la distribución
\[
\begin{array}{r|r}
 x  & f(x) \\
\hline
 -2 & 0.3  \\
 5  & 0.6  \\
 8  & 0.1  \\
\end{array}
\]

\begin{enumerate}
\item La función de distribución se define como $F(x_0)=P(X\leq x_0)$, y por tanto,  mide probabilidades acumuladas. Acumulando las
probabilidades de la tabla anterior tenemos
\[
\begin{array}{r|r|r}
 x  & \multicolumn{1}{c|}{f(x)} & \multicolumn{1}{c}{F(x)}\\
\hline
 -2 & 0.3 & 0.3 \\
 5  & 0.6 & 0.9 \\
 8  & 0.1 & 1 \\
\end{array}
\]
O lo que es lo mismo, expresado como una función a trozos
\[
F(x)=
\left\{%
\begin{array}{ll}
   0, & \hbox{si $x<-2$;} \\
   0.3, & \hbox{si $-2\leq x<5$;} \\
   0.9, & \hbox{si $5\leq x<8$;} \\
   1, & \hbox{si $x\geq 8$.} \\
\end{array}%
\right.
\]

La gráfica de esta función es la siguiente
\begin{center}
\includegraphics[scale=0.3]{grafica1}\hspace*{1cm}
\end{center}

\item Calculamos los estadísticos que nos piden
\begin{align*}
\mu &= \sum x_if(x_i)=-2\cdot0.3+5\cdot 0.6+8\cdot 0.1=3.2,\\
\sigma^2 &= \sum x_i^2f(x_i)-\mu^2=(-2)^2\cdot 0.3+5^2\cdot 0.6+8^2\cdot 0.1-3.2^2=22.6-10.24=12.36,\\
\sigma &= \sqrt{12.36}=3.516.
\end{align*}

\item La mediana es el valor que deja acumulada una probabilidad $0.5$, es decir, $F(med)=0.5$, y mirando en la función de distribución, el
valor donde se consigue acumular esta probabilidad es el 5.

\item Sea $Y=kX$ donde $k<0$. Por las propiedades de las transformaciones lineales de variables aleatorias, tenemos que $\mu_y=k\mu_x$, y
por tanto la media quedará también multiplicada por la constante $k$. Para la desviación típica tenemos que $\sigma_y=|k|\sigma_x$ y la
desviación típica quedará multiplicada por el valor absoluto de $k$.
\end{enumerate}
}


\newproblem{vad-36}{gen}{*}
%ENUNCIADO
{En una empresa el número de días al año que los empleados están de baja es, por término medio, 5. Suponiendo que un año tiene 240 días
laborables y que cada mes tiene 20, se pide:
\begin{enumerate}
\item Calcular el porcentaje de empleados que no faltarían más de 5 días al año.
\item Calcular la probabilidad de que un empleado falte algún día en un mes.
\item ¿Cual es la probabilidad de que en un año haya más de 2 meses en los que haya faltado alguna vez?
\end{enumerate}
} 
%SOLUCIÓN
{
\begin{enumerate}
\item Llamando $X$ a la variable que mide el número de días de baja al año de cada empleado, $X\sim B(240,\,5/240)\approx P(5)$ y
$P(X\leq 5)=0.616$.
\item Llamando $Y$ a la variable que mide el número de días de baja al mes de cada empleado, $Y\sim B(20,\,5/240)$ y $P(Y\geq 1)=0.3437$.
\item Llamando $Z$ a la variable que mide el número de meses al año en que un empleado falta alguna vez, $Z\sim B(12\,,\,0.3437)$ y
$P(Z>2)=0.8379$.
\end{enumerate}
}
%RESOLUCIÓN
{
\begin{enumerate}
\item Sea $X$ la variable que mide el número de días de baja al año de cada empleado. Entonces \mbox{$X\sim B(240\,,\,5/240)$}, pero como
$n=240>30$ y $p=5/240<0.1$, podemos aproximarla como una distribución Poisson $P(5)$. La probabilidad de que un empleado no falte más de 5
días al año es
\begin{align*}
P(X\leq 5)&= P(X=0)+P(X=1)+P(X=2)+P(X=3)+P(X=4)+P(X=5)= \\
&= e^{-5}\frac{5^0}{0!}+
e^{-5}\frac{5^1}{1!}+e^{-5}\frac{5^2}{2!}+e^{-5}\frac{5^3}{3!}
+e^{-5}\frac{5^4}{4!}+e^{-5}\frac{5^5}{5!}= \\
&= 0.0067+0.0337+0.0842+0.1404+0.1755+0.1755=0.616,
\end{align*}
es decir, un $61.6\%$.

\item Sea $Y$ la variable que mide el número de días de baja al mes de cada empleado. Entonces \mbox{$Y\sim B(20,\,5/240)$}, y la
probabilidad de que algún empleado falte algún día en un mes es
\begin{align*}
P(Y\geq1)&=1-P(Y<1)=1-P(Y=0)=
1-\binom{20}{0}\left(\frac{5}{240}\right)^0\left(1-\frac{5}{240}\right)^{20}= \\
&=1-\left(\frac{235}{240}\right)^{20}=0.3437.
\end{align*}

\item Sea ahora $Z$ la variable que mide el número de meses al año en que un empleado falta alguna vez. Entonces, como la probabilidad de
que un empleado falte alguna vez en un mes, según el apartado anterior es $0.3437$, tenemos que $Z\sim B(12\,,\,0.3437)$. Así pues, la
probabilidad que nos piden  es
\begin{align*}
P(Z>2)&=1-P(Z\leq 2)=1-P(Z=0)-P(Z=1)-P(Z=2)=\\
&= 1-\binom{12}{0}0.3437^0 0.6563^{12}-\binom{12}{1}0.3437^1 0.6563^{11}-
\binom{12}{2}0.3437^2 0.6563^{10}=\\
&=1-0,0064-0,0401-0,1156=0.8379.
\end{align*}
\end{enumerate}
}


\newproblem{vad-37}{med}{*}
%ENUNCIADO
{Sabiendo que la prevalencia de la isquemia cardíaca es del 1\%, y que la aplicación de un test diagnóstico para detectar la isquemia
cardíaca tiene una sensibilidad del 90\%, y una especificidad del 95\%. Calcular:
\begin{enumerate}
\item Los valores predictivos, tanto el positivo como el negativo.
\item La probabilidad de diagnóstico acertado.
\item Si tenemos un grupo de 10 enfermos de isquemia cardíaca, ¿cuál es la probabilidad de que diagnostiquemos la enfermedad a
menos de 8?
\end{enumerate}
} 
%SOLUCIÓN
{
}
%RESOLUCIÓN
{
}


\newproblem{vad-38}{med}{*}
%ENUNCIADO
{Un test diagnóstico para una enfermedad devuelve un 1\% de resultados positivos, y sus valores predictivos positivo y negativo valen, respectivamente, $0.95$ y $0.98$. Se pide:
\begin{enumerate}
\item ¿Cuál es la prevalencia de la enfermedad?
\item ¿Cuánto valen la sensibilidad y la especificidad del test?
\item Si aplicamos el test a 12 individuos enfermos, ¿qué probabilidad hay de que se equivoque en alguno de ellos?
\item Si aplicamos el test a 12 individuos, ¿que probabilidad hay de que acierte en todos?
\end{enumerate}
} 
%SOLUCIÓN
{
\begin{enumerate}
\item $P(E)=0.0293$.
\item Sensibilidad $P(+|E)=0.3242$ y especificidad $P(-|\bar E)=0.9995$. 
\item Llamando $X$ al número de diagnósticos erróneos en 12 individuos enfermos, $P(X\geq 1)=1$. 
\item Llamando $Y$ al número de diagnósticos acertados en 12 individuos, $P(X=12)=0.7818$. 
\end{enumerate}
}
%RESOLUCIÓN
{
}


\newproblem{vad-39}{med}{*}
%ENUNCIADO
{La probabilidad de que en un grupo de 5 individuos mayores de 70 años todos padezcan arterioesclerosis cerebral es de $12.5$ por mil.
\begin{enumerate}
\item ¿Cuál es la probabilidad de padecer la enfermedad entre los mayores de 70 años?
\item En un grupo de 1000 personas, ¿cuál es la probabilidad de que padezcan la enfermedad más de 450?
\end{enumerate}
} 
%SOLUCIÓN
{
}
%RESOLUCIÓN
{
}


\newproblem{vad-40}{med}{*}
%ENUNCIADO
{Si sabemos, por estudios previos, que las cepas que provocarán la gripe del siguiente otoño-invierno afectarán a un 20\% de la
población:
\begin{enumerate}
\item ¿Cuál es la probabilidad de que en una población de 10000 habitantes queden infectados menos de 1900?
\item Suponiendo que se vacunan los 10000 habitantes y sabiendo, por estudios previos, que la vacuna inmuniza al 98\% de los vacunados,
¿Cuál es la probabilidad de que queden sin inmunizar menos de 180?
\item De nuevo, suponiendo que se han vacunado los 10000 habitantes y teniendo en cuenta que, por estudios previos, la vacuna produce
reacciones alérgicas en uno de cada 5000 casos, ¿cuál es la probabilidad de que se produzca alguna reacción alérgica en dicha población?
\end{enumerate}
} 
%SOLUCIÓN
{
}
%RESOLUCIÓN
{
}


\input{variables_aleatorias_continuas}
% Author Alfredo S�nchez Alberca (asalber@ceu.es)

\newproblem{ico-1}{gen}{}
% ENUNCIADO
{Una muestra aleatoria de tamaño $81$ extraída de una población normal con $\sigma ^2=64$, tiene una $\overline{x}=78$.
Calcular el intervalo de confianza del $95\%$ para $\mu$.
}
%SOLUCIÓN
{$\mu\in 79\pm 1.742 = (76.258,\,79.742)$.
}
%RESOLUCIÓN
{}


\newproblem{ico-2}{nut}{}
% ENUNCIADO
{Para determinar si un pescado es o no apto para el consumo por su contenido en Hg (mercurio), se realizan $15$
valoraciones obteniendo una media de $0.44$ ppm (partes por millón) de Hg, y una desviación típica de $0.08$ ppm.
Calcular los límites de confianza para la media, a un nivel de significación $\alpha =0.1.$ }
% SOLUCIÓN
{
$\mu\in 0.44\pm 0.0376$ ppm = $(0.4024\text{ppm},\,0.4776\text{ppm})$.
}
% RESOLUCIÓN
{}


\newproblem{ico-3}{qui}{}
% ENUNCIADO
{Se obtuvieron cinco determinaciones del pH de una solución con los siguientes resultados: 
\[7.90, 7.85, 7.89, 7.86, 7.87.\]
Hallar unos límites de confianza de la media de todas las determinaciones del pH de la misma solución, al nivel de
significación $\alpha =0.01.$
}
% SOLUCIÓN
{
$\mu \in 7.874\pm 0.0426 = (7.8314,\,7.9166)$.
}
% RESOLUCIÓN
{}


\newproblem{ico-4}{med}{}
% ENUNCIADO
{Se desea saber cuál debe ser el tamaño muestral mínimo de una muestra para poder realizar la estimación de la tasa media
de glucosa plasmática de una determinada población, con un nivel de confianza $0.95$ y pretendiendo una amplitud de $2.5$
mg.

\noindent NOTA: En una muestra previa de tamaño 10 se obtuvo una desviación típica de 10 mg.
}
%SOLUCIÓN
{
249 individuos.
}
%RESOLUCIÓN
{}


\newproblem{ico-5}{far}{}
%ENUNCIADO
{Para que un fármaco sea efectivo, la concentración de un determinado principio activo debe ser 20 mg/mm$^3$.
Se recibe un lote de dicho fármaco y se analizan 10 para medir la concentración del principio activo, obteniendo los
resultados siguientes:
\[
17.6 - 19.2 - 21.3 - 15.1 - 17.6 - 18.9 - 16.2 - 18.3 - 19 - 16.4.
\]
En vista de los resultados, ¿podremos rechazar el lote con una confianza $0.95$ de no equivocarnos?
}
%SOLUCIÓN
{
$17.96\pm 1.278$ mg/mm$^3$ = $(16.682\text{mg/mm}^3,\,19.238\text{mg/mm}^3)$.
}
%RESOLUCIÓN
{}


\newproblem*{ico-6}{nut}{*}
%ENUNCIADO
{En un estudio sobre el consumo anual de litros de cerveza entre la población de menores de 18 años de una ciudad se
obtuvo la siguiente muestra: 
\[ 42, 16, 60, 29, 7, 20, 30, 25, 38, 5.\]
Se pide:
\begin{enumerate}
\item Calcular el intervalo de confianza del 95\% para la media. Si se considera que un consumo medio por encima de 40
litros es peligroso, ¿existen pruebas significativas para afirmar que la población de partida no está en peligro?
\item ¿Qué tamaño muestral mínimo hubiese sido necesario para conseguir un intervalo de confianza de amplitud 5?
\end{enumerate}
}
%SOLUCIÓN
{}
%RESOLUCIÓN
{}


\newproblem*{ico-7}{amb}{}
%ENUNCIADO
{En una explotación minera se mide el contenido en mercurio de las rocas extraídas.
Tras analizar 20 rocas, se obtiene un contenido medio del $10.8\%$ y una desviación típica de $2.7\%$. Se pide: 
\begin{enumerate}
\item Si, para que la explotación sea rentable, el porcentaje medio de contenido de mercurio debe ser superior al
10\%, ¿existen pruebas para afirmar que la explotación será rentable?
\item ¿Y si para que la explotación sea rentable, el contenido de mineral debe tener cierta uniformidad ($\sigma<3$)?
\end{enumerate}
}
%SOLUCIÓN
{}
%RESOLUCIÓN
{}


\newproblem*{ico-8}{amb}{}
%ENUNCIADO
{Para ver si una población de aves se ha visto afectada por los vertidos tóxicos de una fábrica, se pretende estimar
la proporción de aves contaminadas con metales pesados.
Para ello se realiza un sondeo preliminar con 30 aves, de las que 5 resultan estar contaminadas.
Se pide:
\begin{enumerate}
\item Construir un intervalo de confianza del 95\% para la proporción de aves contaminadas.
\item Si se desea estimar la proporción con un error máximo de $\pm 2\%$, ¿qué tamaño muestral habría que tomar?
\end{enumerate}
}
%SOLUCIÓN
{}
%RESOLUCIÓN
{}


\newproblem{ico-9}{far}{}
%ENUNCIADO
{Se realizó un estudio sobre el contenido de principio activo de un determinado fármaco a partir de una muestra,
determinándose los siguientes resultados en mg/cm$^{3}$:
\[ 46.4-46.1-45.8-47.0-46.1-45.9-45.8-46.9-45.2-46.0. \]
Obtener un intervalo de confianza del 95\% para la varianza del contenido de principio activo de dicho fármaco,
suponiendo que sigue una distribución normal.
}
%SOLUCIÓN
{
$\sigma^2 \in (0.1356(\text{mg/cm}^3)^2,\,0.9541(\text{mg/cm}^3)^2)$.
}
%RESOLUCIÓN
{}


\newproblem{ico-10}{med}{}
%ENUNCIADO
{Se desea hacer un estudio estadístico sobre el número de hematíes en las mujeres.
Se selecciona una muestra de 25 mujeres y se obtiene un número medio de hematíes de 4300000 con una desviación típica
de 200000 (en cada milímetro cúbico de sangre).
Calcular el intervalo de confianza para la media y la varianza del número de hematíes de las mujeres en la población,
con un nivel de significación $0.1$.
}
%SOLUCIÓN
{
Intervalo de confianza del 95\% para la media: $\mu \in 4.3\cdot 10^6\pm 69850$ hematíes = $(4230150,\,4369850)$.\\
Intervalo de confianza del 95\% para la varianza: $\sigma^2\in (2747\cdot 10^10,\,7246\cdot 10^10)$ hematíes$^2$. 
}
%RESOLUCIÓN
{}


\newproblem*{ico-11}{med}{*}
%ENUNCIADO
{Para determinar el nivel medio de colesterol en la sangre de una población, se realizaron análisis sobre una muestra
de 8 personas, obteniéndose los siguientes resultados:
\begin{center}
196 -- 212 -- 188 -- 206 -- 203 -- 210 -- 201 -- 198
\end{center}
Hallar intervalos de confianza para la media y la varianza de nivel de colesterol con un nivel de significación 0.1, suponiendo que el nivel de colesterol en la población sigue una distribución normal.
}
%SOLUCIÓN
{}
%RESOLUCIÓN
{}


\newproblem{ico-12}{med}{}
%ENUNCIADO
{Para determinar la concentración media de albúmina en la sangre se realizaron mediciones sobre un grupo experimental
obteniéndose los siguientes resultados, expresados en g/l: 
\[38-42-46-37-49-42-40-36.\]
Obtener un intervalo de confianza para la varianza de la población con un nivel de significación $0.05$.
}
%SOLUCIÓN
{
$\sigma^2\in (8.844\text{g}^2/\text{l}^2,\,83.728\text{g}^2/\text{l}^2)$.
}
%RESOLUCIÓN
{}


\newproblem{ico-13}{med}{}
%ENUNCIADO
{El tiempo que tarda en hacer efecto un analgésico sigue una distribución aproximadamente normal.
En una muestra de 20 pacientes se obtuvo una media de $25.4$ minutos y una desviación típica de $5.8$ minutos.
Calcular el intervalo de confianza del 90\% e interpretarlo.
¿Cuántos pacientes habría que tomar para poder estimar la media con una precisión de $\pm 1$ minuto? }
%SOLUCIÓN
{Intervalo de confianza del 90\% para $\mu$: $(23.0992,\,27.7008)$.\\
Tamaño muestral para una precisión de $\pm 1$ minuto: $n=96$.}
%RESOLUCIÓN
{}


\newproblem{ico-14}{med}{}
%ENUNCIADO
{En un estudio para el estado de la salud oral de una ciudad, se tomó una muestra elegida al azar de 280 varones entre
35 y 44 años y se contó el número de piezas dentarias en la boca.
Tras la revisión pertinente, los dentistas informaron que había 70 individuos con 28 o más dientes.
Se desea realizar una estimación por intervalo de confianza de la proporción de individuos de esta ciudad con 28
dientes o más, con un nivel de confianza $0.95$.
}
%SOLUCIÓN
{
$p\in 0.25\pm 0.0507 = (0.1993,\,0.3007)$.
}
%RESOLUCIÓN
{}


\newproblem*{ico-15}{med}{}
%ENUNCIADO
{Leemos en una revista médica que la cuarta parte de los cancerosos de cierto tumor de estómago presentan vómitos, con
una precisión o tolerancia del 10\% y con una confianza del 99\%.
¿Con cuántos pacientes se ha realizado el estudio?
}
%SOLUCIÓN
{}
%RESOLUCIÓN
{}


\newproblem{ico-16}{med}{*}
%ENUNCIADO
{Un país está siendo afectado por una epidemia de un virus.
Para valorar la gravedad de la situación se tomaron 40 personas al azar y se comprobó que 12 de ellas tenían el virus.
Determinar el intervalo de confianza para el porcentaje de infectados con un nivel de significación $0.05$.
}
%SOLUCIÓN
{
$p\in (0.1580,\,0.4420)$ con un 95\% de confianza.
}
%RESOLUCIÓN
{}


\newproblem{ico-17}{gen}{}
%ENUNCIADO
{Se desea obtener un intervalo de confianza del 95\% para la diferencia de marcas obtenidas por chicos y chicas en una
prueba física.
Se toma una muestra de 50 chicas y 75 chicos, obteniendo las chicas una marca media de 76 y los chicos de 82.
Además, se conocen las desviaciones típicas de las marcas obtenidas en las poblaciones de chicas y chicos, que son 6 y
8 respectivamente.
}
%SOLUCIÓN
{
$\mu_1-\mu_2 \in 6\pm 2.458$ puntos = $(3.542\text{puntos},\,8.458)$.
}
%RESOLUCIÓN
{}


\newproblem*{ico-18}{amb}{}
%ENUNCIADO
{Las temperaturas medias mensuales (en $^\circ$C) durante el año 2001 en Madrid y Sevilla fueron:
\begin{center}
\begin{tabular}{|l|r|r|r|r|r|r|r|r|r|r|r|r|}
\hline
Ciudad &    Ene &    Feb &    Mar &    Abr &    May &    Jun &    Jul &    Ago &    Sep &    Oct &    Nov &    Dic \\
\hline
 Madrid                  &  $7.2$ &  $8.4$ & $12.2$ & $13.7$ & $16.7$ & $23.3$ & $24.2$ & $25.5$ & $20.4$ & $16.2$ &  $8.1$ &  $4.2$ \\
\hline
 Sevilla                 & $12.1$ & $13.6$ & $16.8$ & $19.0$ & $20.9$ & $27.0$ & $26.6$ & $28.2$ & $24.7$ & $21.3$ & $13.8$ & $11.5$ \\
\hline
\end{tabular}
\end{center}
¿Existen diferencias significativas en las temperaturas medias de ambas ciudades?
}
%SOLUCIÓN
{}
%RESOLUCIÓN
{}


\newproblem{ico-19}{fis}{}
%ENUNCIADO
{Se está ensayando un nuevo procedimiento de rehabilitación para una cierta lesión.
Para ello se trataron nueve pacientes con el procedimiento tradicional y otros nueve con el nuevo, y se midieron los
días que tardaron en recuperase, obteniéndose los siguientes resultados:
\begin{center}
\begin{tabular}{ll}
Método tradicional: & 32-37-35-28-41-44-35-31-34\\
Método nuevo: & 35-31-29-25-34-40-27-32-31
\end{tabular}
\end{center}
Se desea obtener un intervalo de confianza del 95\% para la diferencia de las medias del tiempo de recuperación
obtenido con ambos procedimientos.
Se supone que los tiempos de recuperación siguen una distribución normal, y que las varianzas son aproximadamente
iguales para los dos procedimientos.
}
%SOLUCIÓN
{
$\mu_1-\mu_2\in 3.667\pm 4.712$ días = $(-1.045\text{ días},\,8.379\text{ días})$.
}
%RESOLUCIÓN
{}


\newproblem{ico-20}{med}{}
%ENUNCIADO
{En un hospital pediátrico se comprobó que de 200 niños con un determinado síndrome, 48 murieron antes de cumplir un
año de edad, mientras que sólo 25 de 125 niñas con el mismo síndrome murieron.
¿Se puede afirmar con cierta seguridad que el síndrome es más letal en los niños que en las niñas?
}
%SOLUCIÓN
{
$p_1-p_2 \in 0.04\pm 0.077 = (-0.037,\,0.117)$ luego no se puede afirmar que el síndrome sea más letal en los niños que
en las niñas con un 95\% de confianza.}
%RESOLUCIÓN
{}


\newproblem{ico-21}{med}{}
%ENUNCIADO
{Se ha realizado un estudio para investigar el efecto del ejercicio físico en el nivel de colesterol en la sangre. En el estudio
participaron once personas, a las que se les midió el nivel de colesterol antes y después de desarrollar un programa de ejercicios. Los
resultados obtenidos fueron los siguientes
\begin{center}
\begin{tabular}{|c|c|c|}
\hline Persona & Nivel previo & Nivel posterior \\ 
\hline\hline 
1 & 182 & 198 \\
\hline 
2 & 232 & 210 \\ 
\hline 
3 & 191 & 194 \\ 
\hline 
4 & 200 & 220 \\ 
\hline 
5 & 148 & 138 \\ 
\hline 
6 & 249 & 220 \\ 
\hline 
7 & 276 & 219 \\ 
\hline 
8 & 213 & 161 \\
\hline 
9 & 241 & 210 \\ 
\hline 
10 & 280 & 213 \\ 
\hline 
11 & 262 & 226 \\ 
\hline
\end{tabular}
\end{center}
Hallar un intervalo de confianza del 90\% para la diferencia del nivel medio de colesterol antes y después del ejercicio.
}
%SOLUCIÓN
{
$\mu_{x_1-x_2}\in 24.0909\pm 19.4766$ mg/dl = $(4.6143\text{mg\dl},\, 43.5675\text{mg/dl})$.
}
%RESOLUCIÓN
{}


\newproblem{ico-22}{qui}{}
%ENUNCIADO
{Dos químicos $A$ y $B$ realizan 14 y 16 determinaciones, respectivamente, de plutonio.
Los resultados obtenidos se muestran en la siguiente tabla
\begin{center}
\begin{tabular}{cc|cc}
\multicolumn{2}{c|}{$A$} & \multicolumn{2}{c}{$B$} \\ \hline
263.36 & 254.68 & 286.53 & 254.54 \\
248.64 & 276.32 & 284.55 & 286.30 \\
243.64 & 256.42 & 272.52 & 282.90 \\
272.68 & 261.10 & 283.85 & 253.75 \\
287.33 & 268.41 & 252.01 & 245.26 \\
287.26 & 282.65 & 275.08 & 266.08 \\
250.97 & 284.27 & 267.53 & 252.05 \\
&  & 253.82 & 269.81
\end{tabular}
\end{center}
Se pide:

\begin{enumerate}
\item Calcular intervalos de confianza del 95\% de confianza para cada caso.
\item ¿Se puede decir que existen diferencias significativas en la media?
\end{enumerate}
}
%SOLUCIÓN
{
\begin{enumerate}
\item Intervalo de confianza del 95\% para la media de $A$: $\mu_A \in 266.98\pm 8.681 = (258.299,\,275.661)$.\\
Intervalo de confianza del 95\% para la media de $B$: $\mu_B \in 267.91\pm 7.677 = (260.233,\,275.587)$.
\item $\mu_A-\mu_B \in -0.93\pm 11.020 = (-11.95,\,10.92)$, luego no hay diferencias significativas en las medias con
un 95\% de confianza.
\end{enumerate}
}
%RESOLUCIÓN
{}


\newproblem{ico-23}{med}{*}
%ENUNCIADO
{Un equipo de investigación está interesado en ver si una droga reduce el colesterol en la sangre.
Con tal fin se toma una muestra de 10 pacientes y determina el contenido de colesterol antes y después del tratamiento.
Los resultados expresados en miligramos por cada 100 mililitros son los siguientes:
\[
\begin{tabular}{|c|c|c|c|c|c|c|c|c|c|c|}
\hline
Paciente & 1 & 2 & 3 & 4 & 5 & 6 & 7 & 8 & 9 & 10 \\ \hline
Antes & 217 & 252 & 229 & 200 & 209 & 213 & 215 & 260 & 232 & 216 \\ \hline
Después & 209 & 241 & 230 & 208 & 206 & 211 & 209 & 228 & 224 & 203 \\
\hline
\end{tabular}
\]
Se pide:

\begin{enumerate}
\item Construir la variable Diferencia que recoja la diferencia entre los niveles de colesterol antes y después del
tratamiento, y calcular el intervalo de confianza con $1-\alpha =0.95$ para la media de dicha variable.
\item A la vista del intervalo anterior, ¿hay pruebas significativas de que la droga disminuye el nivel de colesterol
en sangre?
\end{enumerate}
}
%SOLUCIÓN
{
\begin{enumerate}
\item $\mu_{x_1-x_2}\in 7.4\pm 7.572$ mg/100ml = $(-0.172\text{mg/100ml},\,14.972\text{mg/100ml})$.
\item No se puede afirmar que la droga disminuya el colesterol con un 95\% de confianza.
\end{enumerate}
}
%RESOLUCIÓN
{}


\newproblem*{ico-24}{fis}{*}
%ENUNCIADO
{Se está ensayando un nuevo procedimiento de rehabilitación para una cierta lesión.
Se sabe que de 80 deportistas tratados con el procedimiento tradicional, se recuperaron perfectamente 26, mientras que
de los 20 tratados con el nuevo procedimiento se han recuperado 11.
¿Se puede afirmar con una confianza del 95\% que el nuevo procedimiento es mejor que el tradicional?
}
%SOLUCIÓN
{}
%RESOLUCIÓN
{}


\newproblem{ico-25}{amb}{*}
%ENUNCIADO
{En una muestra aleatoria de 200 personas, 114 están a favor de la fluoración de las aguas.
Se pide:
\begin{enumerate}
\item Hallar el intervalo de confianza del 96\% para la fracción de la población que está a favor de la fluoración de
las aguas.
\item ¿Qué tamaño mínimo de muestras habría que tomar para tener una confianza del 96\% de que la proporción muestral
difiere menos de $0.02$ de la proporción real de la población?
\end{enumerate}
}
%SOLUCIÓN
{
\begin{enumerate}
\item $p\in 0.375\pm 0.197 = (0.178,\,0.572)$.
\item 2634 individuos.
\end{enumerate}
}
%RESOLUCIÓN
{}


\newproblem*{ico-26}{far}{*}
%ENUNCIADO
{Para ver si una campaña de publicidad sobre un fármaco ha influido en sus ventas, se tomó una muestra de 8 farmacias
y se midió el número de fármacos vendidos durante un mes, antes y después de la campaña, obteniéndose los siguientes
resultados:
\begin{center}
\begin{tabular}{|c||c|c|c|c|c|c|c|c|}
\hline
Antes & 147 & 163 & 121 & 205 & 132 & 190 & 176 & 147  \\
\hline
Después & 150 & 171 & 132 & 208 & 141 & 184 & 182 & 145  \\
\hline
\end{tabular}
\end{center}
Obtener la variable diferencia y construir un intervalo de confianza para la media de la diferencia con un nivel de
significación $0.05$.
¿Existen pruebas suficientes para afirmar con un 95 \% de confianza que la campaña de publicidad ha aumentado las
ventas?
}
%SOLUCIÓN
{}
%RESOLUCIÓN
{}


\newproblem{ico-27}{med}{*}
%ENUNCIADO
{Para comparar la eficacia de dos tratamientos $A$ y $B$ en la prevención de repeticiones de infarto de miocardio, se
aplicó el tratamiento $A$ a 80 pacientes y el $B$ a 60.
Al cabo de dos años se observó que habían sufrido un nuevo infarto 14 pacientes de los sometidos al tratamiento $A$ y
15 de los del $B$.
Se pide:
\begin{enumerate}
\item Construir un intervalo de confianza del $95\%$ para la diferencia entre las proporciones de personas sometidas a los tratamientos $A$ y $B$ que no vuelven a sufrir un infarto.
\item A la vista del resultado obtenido, razonar si con ese nivel de confianza puede afirmarse que uno de los tratamientos es más eficaz que el otro.
\end{enumerate}
}
%SOLUCIÓN
{
\begin{itemize}
\item $p_A-p_B\in (-0.2126,\,0.0626)$ con un nivel de confianza del 95\%.
\item No puede afirmarse que un tratamiento sea más eficaz que otro pues la diferencia de medias podría ser positiva,
negativa o cero. 
\end{itemize}
}
%RESOLUCIÓN
{}


\newproblem*{ico-28}{amb}{}
%ENUNCIADO
{La siguiente tabla muestra el porcentaje de industrias españolas y europeas según el consumo de energía primaria de
las mismas durante el año 2002 (se estudiaron 1000 industrias).
\[
\begin{tabular}{|l|c|c|}
\hline
Fuente energética  &  España  & Resto de Europa \\
\hline
Petróleo            & $52.2$\% &    $40.4$\%     \\
Carbón              & $15.2$\% &    $14.8$\%     \\
Nuclear             & $13.0$\% &    $15.2$\%     \\
Gas                 & $12.8$\% &    $23.5$\%     \\
Energías Renovables & $6.5$\%  &     $6.1$\%     \\
Saldo Energético    & $0.2$\%  &       0\%       \\
\hline
\end{tabular}
\]
Estudiar para qué energías el porcentaje de industrias de España es significativamente diferente del resto de Europa.
}
%SOLUCIÓN
{}
%RESOLUCIÓN
{}


\newproblem{ico-29}{nut}{*}
%ENUNCIADO
{En un análisis de obesidad dependiendo del hábitat en niños menores de 5 años, se obtienen los siguientes resultados:
\begin{center}
\begin{tabular}{|l|l|l|}
\cline{2-3}
\multicolumn{1}{l|}{} & Casos analizados & Casos con sobrepeso \\
\hline
Hábitat rural & 1150 & 480 \\
\hline
Hábitat urbano & 1460 & 660 \\
\hline
\end{tabular}
\end{center}
Se pide:

\begin{enumerate}
\item  Construir un intervalo de confianza, con un nivel de significación $0.01$, para la proporción de niños menores
de 5 años con sobrepeso en el hábitat rural. Igualmente para el hábitat urbano.
\item Construir un intervalo de confianza, con un nivel de confianza del $95\%$, para la diferencia de proporciones de
niños menores de 5 años con sobrepeso entre el hábitat rural y el urbano.
A la vista del resultado obtenido, ¿se puede concluir, con un $95\%$ de confianza, que la proporción de niños menores
de 5 años con sobrepeso depende del hábitat?
\end{enumerate}
}
%SOLUCIÓN
{
\begin{itemize}
\item $p_R \in (0.3799,\,0.4548)$ y $p_U\in (0.4185,\,0.4856)$ con un 99\% de confianza.
\item $p_R-p_U\in (-0.0729,\,0.0036)$ con un nivel de confianza del 95\%, luego no se puede afirmar que haya
diferencias en las proporciones de niños menores de 5 años con sobrepeso.
\end{itemize}
}
%RESOLUCIÓN
{}


\newproblem*{ico-30}{med}{}
%ENUNCIADO
{Se ha realizado un estudio con 1000 mujeres que han dado a luz recientemente, elegidas al azar entre los registros de
los diferentes hospitales de la comunidad de Madrid, para saber si un nuevo protocolo (visitas al médico y consumo de
ciertos fármacos) resulta más efectivo para prevenir las infecciones (ya sean pre, intra o postparto).
Del total, 750 han seguido el protocolo habitual, entre las cuales 35 han sufrido algún tipo de infección; mientras
que 250 han seguido el protocolo nuevo y 9 de ellas han padecido alguna infección.
¿Se puede afirmar, con un 95\% de confianza, que la proporción de mujeres que ha tenido algún tipo de infección ha
sido diferente según el protocolo utilizado?
}
%SOLUCIÓN
{}
%RESOLUCIÓN
{}


\newproblem*{ico-31}{amb}{*}
%ENUNCIADO
{La siguiente tabla muestra los datos de emisiones de CO$_2$ y CH$_4$ (en Kg/hab) y el producto interior bruto per
cápita (en miles US\$) de varios países en el último año:
\[
\begin{array}{|l|r|r|r|}
\hline
\mbox{País} & \mbox{CO}_2 & \mbox{CH}_4 & \mbox{PIB}\\
\hline\hline
\mbox{Austria}     & 7.60 & 0.97 & 38.40\\ \hline
\mbox{España}      & 6.73 & 0.81	& 30.12\\ \hline
\mbox{Francia}     & 5.71 & 0.94	& 33.19\\ \hline
\mbox{EEUU}        &19.40 & 1.72	&	45.84\\ \hline
\mbox{Alemania}    & 9.80 & 0.83	& 34.18\\ \hline
\mbox{Canadá}      &15.60 & 3.08	& 38.43\\ \hline
\mbox{Italia}      & 7.29 & 0.58	& 30.44\\ \hline
\mbox{Japón}       &	9.44 & 0.16	& 33.58\\ \hline
\mbox{Australia}   &17.48 & 6.36	& 36.26\\ \hline
\mbox{Reino Unido} & 8.99 & 0.76	& 35.13\\ \hline
\end{array}
\qquad
\begin{array}{|l|r|r|r|}
\hline
\mbox{País} & \mbox{CO}_2 & \mbox{CH}_4 & \mbox{PIB}\\
\hline\hline
\mbox{Bolivia}     & 1.05 & 3.44	& 40.13\\ \hline
\mbox{Niger}       &	0.1	 & 0.12	&	 0.67\\ \hline
\mbox{Senegal}     &	0.35 & 0.76 &  1.69\\ \hline
\mbox{Pakistán}    & 0.65 & 0.59	&  2.59\\ \hline
\mbox{Filipinas}   &	0.83 & 0.46	&  3.38\\ \hline
\mbox{Perú}        & 0.94 & 0.75	&  7.80\\ \hline
\mbox{Túnez}      & 2.17 & 0.48	&  7.47\\ \hline
\mbox{Nepal}       & 0.13 & 0.90	&  1.21\\ \hline
\mbox{Nicaragua}   & 0.7	 & 0.32	&  2.62\\ \hline
\mbox{Mauritania}  & 0.97 & 0.85	&  2.01\\ \hline
\end{array}
\]
¿Existen diferencias significativas en la emisión de CO$_2$ entre los países con un PIB superior a 10 mil US\$ y los
países con un PIB inferior? ¿Y en las emisiones de CH$_4$? Justificar la respuesta.
}
%SOLUCIÓN
{}
%RESOLUCIÓN
{}


\newproblem{ico-32}{gen}{}
%ENUNCIADO
{En una muestra de 250 estudiantes de una universidad, 146 hablaban inglés.
¿Entre qué valores estará el porcentaje de individuos de la universidad que hablan inglés, con un nivel de confianza
del 90\%?
}
%SOLUCIÓN
{
$p\in (0.5327,\,0.6353)$ con un 90\% de confianza.
}
%RESOLUCIÓN
{}


\newproblem{ico-33}{psi}{}
%ENUNCIADO
{Si el porcentaje de indivudos daltónicos de una muestra aleatoria es 18\%, ¿cuál será el mínimo tamaño muestral
necesario para conseguir una estimación del porcentaje de daltónicos con una confianza del 95\% y un error menor del 3\%?
}
%SOLUCIÓN
{
$n=2520$.
}
%RESOLUCIÓN
{}


\newproblem{ico-34}{psi}{}
% ENUNCIADO
{Un psicólogo está estudiando la concentración de una encima en la saliba como un posible indicador de la ansiedad crónica.
En un experimento se tomó una muestra de 12 neuróticos por ansiedad y otra de 10 personas con bajos niveles de ansiedad.
En ambas muestras se midió la concentración de la encima, obteniendo los siguientes resultados:
\[
\begin{array}{rcccccccccccc}
\hline
\mbox{Con ansiedad:} & 2.60 & 2.90 & 2.60 & 2.70 & 3.91 & 3.15 & 3.94 & 2.46 & 2.91 & 3.88 & 3.55 & 3.96\\
\mbox{Sin ansiedad:} & 2.37 & 1.10 & 2.55 & 2.64 & 2.20 & 2.12 & 2.47 & 2.90 & 1.66 & 2.72 \\
\hline
\end{array}
\]
¿Se puede concluir a partir de estos datos que la población de neuróticos con ansiedad y la población de personas sin
ansiedad son diferentes en el nivel medio de concentración de encimas?
Justificar la respuesta.
}
%SOLUCIÓN
{
$\frac{\sigma_1^2}{\sigma_2^2}\in (0.3383,\,4.7486)$ con un nivel de confianza del 95\%, luego se puede suponer que las
varianzas son iguales.\\
$\mu_1-\mu_2\in (0.4296,\,1.4510)$ con un 95\% de confianza, luego se puede concluir que hay diferencias entre las
medias.}
%RESOLUCIÓN
{}


\newproblem{ico-35}{gen}{}
%ENUNCIADO
{Las notas en Estadística de una muestra de 10 alumnos han sido:
\begin{center}
$6.3$, $5.4$, $4.1$, $5.0$, $8.2$, $7.6$, $6.4$, $5.6$, $4.3$, $5.2$
\end{center}
Dar una estimación puntual de la nota media, de la varianza y del porcentaje de aprobados en la clase.
}
%SOLUCIÓN
{$\bar x= 5.81$ puntos, $\hat s^2=1.7721$ puntos$^2$ y $\hat p=0.8$.}
%RESOLUCIÓN
{}


\newproblem{ico-36}{psi}{}
%ENUNCIADO
{Para estudiar si la estación del año influye en el estado de ánimo de la gente, se ha tomado una muestra 12 personas y
se ha medido su nivel de depresión en verano e invierno mediante un cuestionario con puntuaciones de 0 a 100 (a mayor
puntuación mayor depresión). Los resultados obtenidos fueron:
\begin{center}
\begin{tabular}{|l|r|r|r|r|r|r|r|r|r|r|r|r|}
\hline
Invierno & 65 & 72 & 84 & 31 & 80 & 61 & 75 & 52 & 73 & 79 & 85 & 71\\
\hline
Verano &   60 & 51 & 81 & 45 & 62 & 53 & 70 & 52 & 64 & 51 & 67 & 62\\
\hline
\end{tabular}
\end{center}
¿Se puede afirmar que la estación del año influye en el estado de ánimo de la gente con un 99\% de confianza?
¿Cómo influye?
}
%SOLUCIÓN
{
No se puede afirmar que la estación influya en el estado de ánimo ya que la media de la diferencia está en
$(-0.7498,\,19.0831)$ con un 99\% de confianza.}
%RESOLUCIÓN
{}


\newproblem{ico-37}{psi}{}
%ENUNCIADO
{Para realizar una determinada tarea se necesitan personas con un cierto nivel de pericia.
Una empresa está interesada en contratar a un grupo de personas que tengan la pericia necesaria para realizar la tarea,
pero que además sean bastante homogéneas en cuanto a la pericia, es decir, que su rendimiento sea parecido.
Para ver si un grupo de alumnos que se han formado en dicha tarea cumplen los requisitos, se ha tomado una muestra y se
les ha sometido una prueba para ver cuántas tareas son capaces de realizar con éxito en una hora.
Los resultados obtenidos fueron:
\begin{center}
16 - 12 - 14 - 21 - 11 - 15 - 17 - 15 - 18 - 24 - 10 - 14 - 17 - 14 
\end{center}
Se pide: 
\begin{enumerate}
\item Si la empresa busca un grupo de empleados capaces de realizar una media de al menos 12 tareas por hora, ¿se puede
afirmar con una confianza del 95\% que el grupo lo cumple?
\item Si la empresa busca que entre los trabajadores haya una dispersión media de $\sigma<3$ tareas, ¿se puede afirmar
con una confianza del 95\% que el grupo lo cumple? 
\end{enumerate} 
}
%SOLUCIÓN
{
\begin{itemize}
\item Lo cumple ya que $\mu\in (13.4026,\,17.7403)$ con un 95\% de confianza.
\item No lo cumple ya que $\sigma \in (2.7232,\,6,0516)$ con un 95\% de confianza.
\end{itemize}
}
%RESOLUCIÓN
{}


\newproblem*{ico-38}{psi}{}
%ENUNCIADO
{Se cree que el consumo de tabaco va ligado al consumo de alcohol y para corroborar esta hipótesis se ha realizado un
estudio en el que se han obtenido los siguientes datos
\begin{center}
\begin{tabular}{lcc}
 & Bebedores & No bebedores\\
\cline{2-3}
Fumadores & \multicolumn{1}{|c}{487} & \multicolumn{1}{c|}{137} \\
No fumadores & \multicolumn{1}{|c}{312} & \multicolumn{1}{c|}{365}\\
\cline{2-3}
\end{tabular}
\end{center}
¿Se puede afirmar que existe relación entre el consumo de tabaco y el de alcohol?
Justificar la respuesta.
}
%SOLUCIÓN
{}
%RESOLUCIÓN
{}


\newproblem{ico-39}{far}{*}
%ENUNCIADO
{Para analizar la eficacia de un determinado fármaco para aumentar las horas de sueño se realiza un estudio con 8 pacientes a los que se les
controla durante todo un mes, en el que no toman el medicamento y se cuantifica la media de las horas de sueño para cada uno de ellos. El
mismo proceso se lleva a cabo durante todo un mes en el que sí que toman el fármaco. Los resultados (en horas de sueño) aparecen en la
siguiente tabla:
\[
\begin{array}{ccc}
\hline
\text{Paciente} & \text{Sin el fármaco} & \text{Con el fármaco} \\
\hline
    1     &        8.3        &         9.0         \\
    2     &        6.9        &         8.1         \\
    3     &        7.2        &         9.2         \\
    4     &        9.1        &         9.5         \\
    5     &        8.2        &         8.5         \\
    6     &        7.5        &         9.0         \\
    7     &        6.1        &         8.0         \\
    8     &        7.4        &         7.9         \\
\hline
\end{array}
\]
¿Se puede concluir con un nivel de confianza del 99\% que el fármaco ha sido eficaz para aumentar las horas de sueño?
}
%SOLUCIÓN
{El intervalo de confianza de 99\% para la media de la diferencia entre las horas de sueño sin el fármaco y con el fármaco es
$(-1.9067\,,\,-0.2183)$, luego si se puede concluir que el fármaco ha sido eficaz para aumentar las horas de sueño.
}
%RESOLUCIÓN
{Para ver si el fármaco ha sido eficaz necesitamos calcular el intervalo de confianza para la media de la diferencia entre horas de sueño
sin y con el fármaco. Al tratarse de datos pareados, construimos una nueva variable $X$ que recoja esta diferencia.
\[
\begin{array}{cccc}
\hline
\text{Paciente} & \text{Sin el fármaco} & \text{Con el fármaco} & X=\text{Diferencia} \\
\hline 
    1     &        8.3        &         9.0         & -0.7\\
    2     &        6.9        &         8.1         & -1.2\\
    3     &        7.2        &         9.2         & -2.0\\
    4     &        9.1        &         9.5         & -0.4\\
    5     &        8.2        &         8.5         & -0.3\\
    6     &        7.5        &         9.0         & -1.5\\
    7     &        6.1        &         8.0         & -1.9\\
    8     &        7.4        &         7.9         & -0.5\\
\hline
\end{array}
\]

Puesto que tenemos una muestra pequeña (8 pacientes), debemos utilizar el intervalo de confianza de la t de Student, cuya fórmula es
\[
\bar x\pm t(n-1)_{\alpha/2}\frac{\hat s}{\sqrt{n}}.
\]
Calculamos primero los estadísticos necesarios para construir el intervalo
\begin{align*}
\bar x &= \frac{\sum x_{i}}{n}=\frac{-0.7-\cdots -0.5}{8}= \frac{-8.5}{8}=-1.0625,\\
s^2 &= \frac{\sum{x_{i}^2}}{n}-\bar x^2=\frac{(-0.7)^2+\cdots +(-0.5)^2}{8}-(-1.0625)^2= \frac{12.29}{8}-1.1289=0.4073,\\
\hat s^2 &= \frac{n}{n-1}\cdot s^2= \frac{8}{7}\cdot 0.4073 = 0.4655,\\
\hat s &= \sqrt{0.4655}=0.6823.
\end{align*}
Como nos piden una confianza del 99\%, $\alpha=0.01$ y el valor de $t(n-1)_{\alpha/2}=t(7)_{0.005}$ lo buscamos en la tabla de la función de
distribución de la t de Student con 7 grados de libertad. Dicho valor es $t(7)_{0.005}=3.499$.

Sustituyendo estos valores en la fórmula del intervalo de confianza, tenemos
\[
-1.0625\pm 3.499\frac{0.6823}{\sqrt{8}}= -1.0625\pm 0.8442,
\]
con lo que el intervalo de confianza resultante es
$(-1.9067\,,\,-0.2183)$.

Según este intervalo, al 99\% de confianza existen pruebas significativas de que la media de la diferencia de horas es negativa y por tanto
el número de horas de sueño después del fármaco aumenta por término medio. Podemos concluir, con un 99\% de confianza, que el fármaco es
eficaz.
}


\newproblem{ico-40}{med}{*}
% ENUNCIADO
{Se quiere probar si la cirrosis hepática hace variar el índice de colinesterasa en suero. Se eligen 2 muestras aleatorias e independientes,
una primera de 60 individuos normales, con media $1.6$ y desviación típica $0.3$, y la segunda de 50 individuos cirróticos, con media $1.1$
y desviación típica $0.4$. ¿Podemos concluir que existen diferencias significativas, con un 99\% de confianza, entre las medias de la
colinesterasa en individuos normales e individuos cirróticos?
}
% SOLUCIÓN
{
}
% RESOLUCIÓN
{
}


\newproblem{ico-41}{med}{*}
% ENUNCIADO
{Un grupo de investigadores obtuvo datos acerca de las concentraciones de amilasa en el suero de muestras de individuos sanos y de
individuos hospitalizados, con el objetivo de determinar si la concentración media es, o no, diferente en ambas poblaciones. Las
concentraciones, en unidades/ml, en 10 individuos sanos fueron:
\begin{center}
\begin{tabular}{llllllllll}
100 & 103 & 96 & 93 & 91 & 104 & 93 & 99 & 88 & 91 \\
\end{tabular}
\end{center}
Y en 12 individuos enfermos fueron:
\begin{center}
\begin{tabular}{llllllllllll}
118 & 115 & 101 & 104 & 116 & 114 & 112 & 113 & 117 & 123&119&121 \\
\end{tabular}
\end{center}

Suponiendo que la concentración de amilasa en suero sigue una distribución normal, tanto en individuos sanos como hospitalizados, y que las
varianzas son desconocidas pero iguales, se pide:
\begin{enumerate}
\item Calcular el intervalo de confianza para la diferencia de medias con un nivel de confianza del 95\%.
\item ¿A qué conclusión deben llegar los investigadores sobre la igualdad o no de la concentración de amilasa? Justificar la respuesta.
\end{enumerate}
}
% SOLUCIÓN
{
}
% RESOLUCIÓN
{
}


\newproblem{ico-42}{med}{*}
% ENUNCIADO
{En un estudio se pretende contrastar el efecto de una vacuna contra la alergia. Para ello, se dispone de un grupo experimental formado por
200 individuos y otro control formado por 300 individuos.
Teniendo en cuenta que en el grupo experimental sufrieron alergia 20 personas, con una duración media de 20 días y una desviación típica de
6, y en el control la sufrieron 42, con una duración media de 25 días y una desviación típica de 7.
Se pide:
\begin{enumerate}
\item Calcular el intervalo de confianza al 90\% para la diferencia de proporciones de los afectados por la alergia entre los que han sido
vacunados y los que no. ¿Se puede concluir que la vacuna ha disminuido significativamente la proporción de afectados por la alergia? Justificar adecuadamente la respuesta. 
\item Suponiendo varianzas poblacionales iguales y que la duración de la alergia sigue una distribución normal, tanto en el grupo
experimental como en el grupo control, calcular el intervalo de confianza del 95\% para la diferencia entre la duración media de la alergia
en el grupo experimental y en el grupo control. ¿Se puede concluir que la vacuna ha disminuido significativamente la duración de la alergia? Justificar adecuadamente la respuesta.  
\end{enumerate}
}
% SOLUCIÓN
{
}
% RESOLUCIÓN
{
}


% Version control information:
% $HeadURL: https://ejerciciosestadistica.googlecode.com/svn/trunk/contrastes.tex $
% $LastChangedDate: 2011-02-17 12:09:26 +0100 (jue, 17 feb 2011) $
% $LastChangedRevision: 7 $
% $LastChangedBy: asalber $
% $Id: contrastes.tex 7 2011-02-17 11:09:26Z asalber $

\newproblem{hip-1}{far}{}
% ENUNCIADO
{Se sabe que una vacuna que se está utilizando al cabo de dos años sólo protege al 60\% de las personas a las que se
administró.
Se desarrolla una nueva vacuna, y se quiere saber si al cabo de dos años protege a más personas que la primera.
Para ello se seleccionan 10 personas al azar y se les inyecta la nueva vacuna.
Establecemos que si más de 8 de los vacunados conservan la protección al cabo de dos años, entonces consideraremos la
nueva vacuna mejor que la antigua.
Se pide:
\begin{enumerate}
\item Calcular la probabilidad de cometer un error de tipo I.
\item Si la nueva vacuna protegiera a un 80\% de las personas vacunadas al cabo de 2 años, ¿Cuál será la probabilidad
de cometer un error de tipo II?
\end{enumerate}

Repetir los cálculos si se toma una muestra de 100 personas y se establece que la vacuna es mejor si más de 75 de los
vacunados conservan la protección al cabo de 2 años.

\noindent \textbf{Nota}: Aproximar la distribución binomial mediante una distribución normal.
}
%SOLUCIÓN
{
Contraste: $H_0:p=0.6$, $H_1:p>0.6$. Muestra de tamaño 10:
\begin{itemize}
\item $P(\text{Rechazar }H_0/H_0)=0.0464$.
\item $P(\text{Aceptar }H_0/H_1)= 0.6242$.
\end{itemize}
Muestra de tamaño 100:
\begin{itemize}
\item $P(\text{Rechazar }H_0/H_0)=0.0011$.
\item $P(\text{Aceptar }H_0/H_1)=0.1056$.
\end{itemize}
}
%RESOLUCIÓN
{}


\newproblem{hip-2}{fisio}{}
% ENUNCIADO
{Un fisioterapeuta afirma que con un nuevo procedimiento de rehabilitación que él aplica, determinada lesión tiene un
tiempo de recuperación medio no mayor de 15 días.
Se seleccionan al azar 36 personas que sufren dicho tipo de lesión para verificar su afirmación, y se obtiene un tiempo
medio de recuperación de 13 días y una cuasivarianza de 9.
¿Contradice lo observado en la muestra la afirmación del fisioterapeuta para un $\alpha =0.05$? }
%SOLUCIÓN
{
Contraste para la media: $H_0:\mu=15$, $H_1:\mu<15$.\\
Región de aceptación para $\alpha=0.05$: $-1.6444<z$.\\
Estadístico del contraste: $z=-4$. Como cae fuera de la región de aceptación se rechaza la hipótesis nula y se
confirma confirma la afirmación del fisioterapeuta.
}
%RESOLUCIÓN
{}


\newproblem{hip-3}{far}{}
% ENUNCIADO
{Se decide retirar una cierta vacuna si produce más de un 10\% de reacciones alérgicas. Se consideran 100 pacientes sometidos a la vacuna y
se observan 15 reacciones alérgicas. ¿Debe retirarse la vacuna? (Utilizar un $\alpha =0.01$).
}
%SOLUCIÓN
{Contraste para la proporción: $H_0: p=0.1$, $H_1: p>0.1$.\\
Región de aceptación para $\alpha=0.01$: $z>2.3263$.\\
Estadístico del contraste: $z=1.6667$. Como cae dentro de la región de aceptación, se acepta la hipótesis nula y se
concluye que no hay pruebas suficientes para retirar la vacuna.
}
%RESOLUCIÓN
{}


\newproblem{hip-4}{med}{}
% ENUNCIADO
{Se utiliza un grupo de 150 pacientes para comprobar la teoría de que la vitamina C tiene alguna influencia en el
tratamiento del cáncer.
Los 150 pacientes fueron divididos en dos grupos de 75. Un grupo recibió 10 gramos de vitamina C y el otro un placebo
cada día, además de la medicación habitual.
De los que recibieron la vitamina C, 47 presentaban alguna mejoría al cabo de cuatro semanas, mientras que de los que
recibieron el placebo, 43 experimentaron mejoría.
Contrastar esta hipótesis.
}
%SOLUCIÓN
{
Contraste para la comparación de proporciones: $H_0:p_1=p_2$, $H_1:p_1\neq p_2$.\\
Región de aceptación para $\alpha=0.05$: $-1.96<z<1.96$.\\
Estadístico del contraste $z=0.6677$. Como cae dentro de la región de aceptación, se acepta la hipótesis nula
y no se pude concluir que la vitamina C tenga influencia en el tratamiento del cáncer.
}
%RESOLUCIÓN
{}


\newproblem{hip-5}{med}{}
% ENUNCIADO
{Se realizó en dos hospitales una encuesta entre los pacientes sobre la satisfacción con la atención recibida,
calificándola de 0 a 100.
En el hospital A rellenaron la encuesta 12 pacientes, obteniéndose una media de 85 y una cuasivarianza de 16, mientras
que en el hospital B rellenaron la encuesta 10 pacientes, obteniéndose una media de 81 y una cuasivarianza de 25.
¿Puede concluirse que el nivel de satisfacción en el hospital A es mayor que en el B?

\noindent \textbf{Nota}: Hacer previamente un contraste de igualdad de varianzas.
}
%SOLUCIÓN
{Contraste de comparación de varianzas: $H_0:\sigma_A=\sigma_B$, $H_1:\sigma_A\neq \sigma_B$.\\
Región de aceptación para $\alpha=0.05$: $0.2787<f<3.9121$.\\
Estadístico del contraste: $f=0.64$. Como cae dentro de la región de aceptación, se acepta la
hipótesis de que las varianzas son iguales.\\
Contraste de comparación de medias: $H_0:\mu_A=\mu_B$, $H_1:\mu_A>\mu_B$.\\
Región de aceptación para $\alpha=0.05$: $t>1.7247$
Estadístico del contraste: $t=2.0863$. Como cae fuera de la región de aceptación, se rechaza la hipótesis nula y se
concluye que hay pruebas significativas de que el nivel de satisfación en el hospital A es mayor que en el B.
}
%RESOLUCIÓN
{}


\newproblem{hip-6}{med}{*}
% ENUNCIADO
{Se compararon los niveles de ácido ascórbico en plasma de mujeres embarazadas fumadoras y no fumadoras,
obteniéndose los siguientes resultados expresados en gramos de ácido ascórbico por mililitro de plasma:
\begin{itemize}
\item[] Mujeres fumadoras: $0.97$--$0.72$--$1.00$--$0.81$--$0.62$--$1.32$--$1.24$.
\item[] Mujeres no fumadoras: $0.48$--$0.71$--$0.98$--$0.68$--$0.5$.
\end{itemize}
Realizar un contraste de hipótesis para comprobar si el nivel de ácido ascórbico en la sangre de mujeres fumadoras es
mayor que el de mujeres no fumadoras.
}
%SOLUCIÓN
{Contraste de comparación de varianzas: $H_0:\sigma_1=\sigma_2$, $H_1:\sigma_1\neq \sigma_2$.\\
Región de aceptación para $\alpha=0.05$: $0.1606<f<9.1973$.\\
Estadístico del contraste: $f=1.6592$. Como cae dentro de la región de aceptación, se acepta la hipótesis de que las
varianzas son iguales.\\
Contraste de comparación de medias: $H_0:\mu_1=\mu_2$, $H_2:\mu_1>\mu_2$.\\
Región de aceptación para $\alpha=0.05$: $t<1.8125$.\\
Estadístico del contraste: $t=2.0372$. Como cae fuera de la región de aceptación, se rechaza la hipótesis nula y se
concluye que las mujeres fumadoras tienen mayor nivel de ácido ascórbico. 
}
%RESOLUCIÓN
{}


\newproblem*{hip-7}{qui}{}
% ENUNCIADO
{Verificar la hipótesis de que el contenido medio de unos recipientes de ácido sulfúrico es de 10 litros, si los
contenidos de una muestra aleatoria de 10 recipientes son $10.2$, $9.7$, $10.1$, $10.3$, $10.1$, $10.1$, $9.8$, $9.9$,
$10.4$, $10.3$ y $9.8$ litros.
Utilizar un nivel de significación de $0.01$ y suponer que la distribución de los contenidos es normal.
}
%SOLUCIÓN
{}
%RESOLUCIÓN
{}


\newproblem{hip-8}{gen}{}
% ENUNCIADO
{Un fabricante de equipos de medida afirma que sus equipos pueden realizar al menos 12 mediciones más que los de la
competencia sin necesidad de un nuevo ajuste.
Para probar esta afirmación se realizan mediciones con 50 equipos de este fabricante y 50 de la competencia.
En los suyos el número de mediciones hasta necesitar un nuevo ajuste tuvo de media $86.7$ y cuasidesviación típica
$6.28$, mientras que en los de la competencia estos valores fueron $77.8$ y $5.61$ respectivamente.
Verificar la afirmación del fabricante con $\alpha=0.05$.
}
%SOLUCIÓN
{
Contraste de comparación de medias: $H_0:\mu_1<\mu_2+12$, $H_1:\mu_1\geq\mu_2 12$.\\
Intervalo de confianza para la diferencia de medias: $\mu_1-\mu_2\in (6.5659,\,11.2341)$ con un 95\% de confianza,
luego hay diferencias significativas entre el número medio de mediciones, pero no se puede afirmar que sean mayores de
12 mediciones.
}
%RESOLUCIÓN
{}


\newproblem*{hip-9}{med}{}
% ENUNCIADO
{Para determinar si un nuevo suero detiene la leucemia, se seleccionan 9 ratones con leucemia en una fase avanzada. 
Cinco reciben el tratamiento y cuatro no.
Los tiempos de supervivencia, en años, desde el momento que comenzó el experimento son los siguientes:
\begin{itemize}
\item[] Con tratamiento: $2.1$ -- $5.3$ -- $1.4$ -- $4.6$ -- $0.9$.
\item[] Sin tratamiento: $1.9$ -- $0.5$ -- $2.8$ -- $3.1$.
\end{itemize}
¿Puede afirmarse con un $\alpha=0.05$ que el suero es eficaz?
Suponer que ambas distribuciones son normales con varianzas iguales.
}
%SOLUCIÓN
{}
%RESOLUCIÓN
{}


\newproblem{hip-10}{gen}{}
% ENUNCIADO
{Un estudio afirma que el 70\% de los habitantes de la capital lee diariamente algún periódico.
¿Estaríamos de acuerdo con las conclusiones de dicho estudio si al preguntar a 15 personas elegidas aleatoriamente, 8
leen diariamente algún periódico? }
%SOLUCIÓN
{Contraste para la proporción: $H_0:p=0.7$, $H_1:p\neq 0.7$.\\
Región de aceptación para $\alpha=0.05$: $-1.96<z<1.96$.\\
Estadístico del contraste: $z=-1.408590$. Como cae dentro de la región de aceptación, se acepta la hipótesis nula
y se estaría de acuerdo con las afirmación del estudio.
}
%RESOLUCIÓN
{}


\newproblem*{hip-11}{gen}{}
% ENUNCIADO
{Un distribuidor de tabaco asegura que el 20\% de los fumadores de su ciudad prefiere los cigarrillos de marca $A$.
Se selecciona al azar una muestra de 20 fumadores, y 6 de ellos prefieren la marca $A$.
¿Qué conclusión se obtiene con $\alpha=0.05$?
}
%SOLUCIÓN
{}
%RESOLUCIÓN
{}


\newproblem{hip-12}{psi}{}
% ENUNCIADO
{En un estudio sobre el consumo de alcohol entre los jóvenes durante los fines de semana, se preguntó a 100 chicos y a
125 chicas, de los que 63 chicos y 59 chicas contestaron que consumían.
En vista de estos datos, ¿existe alguna diferencia significativa entre las respuestas de chicos y chicas?
Utilizar $\alpha=0.10$.
} 
%SOLUCIÓN
{
Contraste de comparación de proporciones: $H_0:p_1=p_2$, $H_1:p_1\neq p_2$.\\
Región de aceptación para $\alpha=0.01$: $-1.6449<z<1.6449$.\\
Estadístico del contraste: $z=2.4026$. Como cae fuera de la región de aceptación, se rechaza la hipótesis nula y se
concluye que hay diferencias significativas entre el consumo de alcohol de chicos y chicas.
}
%RESOLUCIÓN
{}


\newproblem{hip-13}{gen}{}
% ENUNCIADO
{Un fabricante de baterías para automóvil asegura que la duración de sus baterías tiene una distribución
aproximadamente normal con desviación típica no superior a $0.9$ años.
Si una muestra aleatoria de 10 de estas baterías tiene una cuasidesviación típica de $0.7$ años, ¿qué se puede concluir
sobre la afirmación del fabricante?
}
%SOLUCIÓN
{Contraste para la desviación típica: $H_0:\sigma=0.9$, $H_1:\sigma<0.9$.\\
Región de aceptación para $\alpha=0.05$: $3.3251<j$.\\
Estadístico del contraste: $j=4$. Como cade dentro de la región de aceptación, no se puede rechazar la hipótesis
nula y se concluye que no hay pruebas significativas de que sea cierta la afirmación del fabricante.
}
%RESOLUCIÓN
{}


\newproblem*{hip-14}{amb}{}
% ENUNCIADO
{En un estudio sobre el contenido de ortofósforo de las aguas de un río, se realizaron medidas en dos estaciones distintas.
Se sacaron 15 muestras de la estación 1 y 12 de la estación 2.
Las muestras de la estación 1 presentaron un contenido medio de ortofósforo de $3.84$ mg/l y una cuasidesviación típica
de $3.07$ mg/l, mientras que las de la estación 2 tuvieron media $1.49$ mg/l y una cuasidesviación típica $0.8$ mg/l.

Se pide:
\begin{enumerate}
\item Calcular el intervalo de confianza para el cociente de varianzas.
\item Realizar el contraste de hipótesis de igualdad de varianzas.
\end{enumerate}
Utilizar un $\alpha=2$.
}
%SOLUCIÓN
{}
%RESOLUCIÓN
{}


\newproblem*{hip-15}{amb}{}
% ENUNCIADO
{Se ha desarrollado un aditivo para gasolina que reduce la emisión de CO$_2$ en la combustión.
Para comprobar la efectividad del aditivo, se realiza un estudio en el que se mide en una muestra de 10 coches la
cantidad de CO$_2$ emitida (en Kg/l), tanto con gasolina con aditivo, como con gasolina sin aditivo, obteniendo los
siguientes resultados:
\[
\begin{array}{rcccccccccc}
\mbox{Sin aditivo:}  & 0.38 & 0.42 & 0.41 & 0.39 & 0.45 & 0.47 & 0.44 & 0.38 & 0.40 & 0.50  \\
\mbox{Con  aditivo:} & 0.38 & 0.36 & 0.38 & 0.32 & 0.39 & 0.45 & 0.39 & 0.39 & 0.35 & 0.48 \\
\end{array}
\]
Se pide:
\begin{enumerate}
\item Realizar un contraste de hipótesis para comprobar la efectividad del aditivo.
\item ¿Qué potencia tiene el contraste para detectar una reducción en la emisión de CO$_2$ de $0.2$ Kg/l?
\end{enumerate}
}
%SOLUCIÓN
{}
%RESOLUCIÓN
{}


\newproblem{hip-16}{psi}{}
% ENUNCIADO
{Un psicólogo está estudiando la concentración de una encima en la saliba como un posible indicador de la ansiedad crónica.
En un experimento se tomó una muestra de 12 neuróticos por ansiedad y otra de 10 personas con bajos niveles de ansiedad.
En ambas muestras se midió la concentración de la encima, obteniendo los siguientes resultados:
\[
\begin{array}{rcccccccccccc}
\hline
\mbox{Con ansiedad:} & 2.60 & 2.90 & 2.60 & 2.70 & 3.91 & 3.15 & 3.94 & 2.46 & 2.91 & 3.88 & 3.55 & 3.96\\
\mbox{Sin ansiedad:} & 2.37 & 1.10 & 2.55 & 2.64 & 2.20 & 2.12 & 2.47 & 2.90 & 1.66 & 2.72 \\
\hline
\end{array}
\]
¿Se puede concluir a partir de estos datos que la población de neuróticos con ansiedad y la población de personas sin
ansiedad son diferentes en el nivel medio de concentración de encimas?
Justificar la respuesta.
}
%SOLUCIÓN
{
Contraste de comparación de varianzas: $H_0:\sigma_1=\sigma_2$, $H_1:\sigma_1\neq\sigma_2$.\\
Región de aceptación para $\alpha=0.05$: $0.2787<f<3.9121$.\\
Estadístico del contraste: $f=1.2138$. Como cade dentro de la región de aceptación, se acepta la hipótesis nula y se
concluye que las varianzas son iguales.\\
Contraste de comparación de medias: $H_0:\mu_1=\mu_2$, $H_1:\mu_1\neq\mu_2$.\\
Región de aceptación para $\alpha=0.05$: $-2.0860<t<2.0860$.\\
Estadístico del contraste: $t=3.8407$. Como cae fuera de la región de aceptación, se rechaza la hipótesis nula y
se puede concluir que hay diferencia entre la concentración media de encimas para neuróticos con ansiedad y sin ansiedad.
}
%RESOLUCIÓN
{}


\newproblem{hip-17}{psi}{}
% ENUNCIADO
{En una investigación para ver la efectividad de una nueva droga antidepresiva, se ha tomado un muestra de 15
pacientes depresivos que han completado un cuestionario para detectar el nivel depresivo, antes y después de recibir
la droga.
En la puntuación del cuestionario los valores menores indican una mayor depresión.
Los resultados obtenidos han sido:
\[
\begin{array}{rccccccccccccccc}
\hline
\mbox{Antes:}  & 18 & 21 & 16 & 19 & 14 & 23 & 16 & 14 & 21 & 18 & 17 & 14 & 16 & 14 & 20 \\
\mbox{Después:}& 23 & 20 & 17 & 20 & 16 & 22 & 18 & 18 & 21 & 16 & 19 & 20 & 15 & 15 & 21 \\
\hline 
\end{array}
\]
Realizar un constraste para averiguar si la droga tiene un efecto positivo sobre la depresión.
¿Qué tamaño muestral sería necesario para detectar una diferencia en la puntuación como la que hay entre las medias de las muestras?
}
%SOLUCIÓN
{
Contraste para la media de la diferencia entre antes y después: $H_0:\mu=0$, $H_1:\mu<0$.\\
Región de aceptación par $\alpha=0.05$: $-1.7613<t$.\\
Estadístico del contraste: $t=-2.2563$. Como cae fuera de la región de aceptación, se rechaza la hipótesis nula y
se puede afirmar que la droga reduce la depresión.\\
El tamaño muestral para detectar una diferencia de $\delta=\bar x_1-\bar x_2 = 17.4-18.7333=-1.3333$ con una potencia
del 90\% es $n=26$ individuos.
}
%RESOLUCIÓN
{}


\newproblem{hip-18}{psi}{}
% ENUNCIADO
{Un experimento pretende contrastar la teoría de que la memoria a corto plazo se ve afectada por la similitud entre los
estímulos. El experimento consiste en leer en voz alta una secuencia de letras a un sujeto, quien después de una breve
pausa debe repetir la secuencia.
Si la teoría es correcta, habrá más errores en la lista que contenga letras que suenan de forma similar que si
contiene letras que se parecen.
A cada sujeto se le presentan dos tipos de secuencias, una con letras que suenan de forma similar y otra con letras
que se escriben de forma parecida.
Los errores producidos en cada secuencia son:
\[
\begin{array}{rccccccccc}
\hline
\mbox{Errores de letras que suenan parecidas:}  & 7 & 5 & 6 & 11 & 3 & 8 & 4 & 10 & 9\\
\mbox{Errores de letras con similiar escritura:}& 8 & 2 & 5 &  9 & 5 & 4 & 4 &  7 & 4\\
\hline 
\end{array} 
\]
¿Se puede validar la teoría?
}
%SOLUCIÓN
{
Contraste para la media de la diferencia entre los errores de letras que suenan parecidas y los errores en letras con
similar escritura: $H_0:\mu=0$, $H_1:\mu>0$.\\
Región de aceptación para $\alpha=0.05$: $t>1.8595$.\\
Estadístico del contraste: $t=2.1320$. Como cae fuera de la región de aceptación, se rechaza la hipótesis nula y
se concluye que la teoría es cierta.
}
%RESOLUCIÓN
{}


\newproblem{hip-19}{psi}{}
% ENUNCIADO
{Se sabe que el tiempo de reacción ante un estímulo sigue una distribución normal de media $30$ ms y desviación típica
$10$ ms.
Se cree que la alcoholemia aumenta el tiempo de reacción de los sujetos, y para comprobar esta hipótesis se ha tomado
una muestra aleatoria de 40 individuos a los que se les ha inducido una alcoholemia de $0.8$ g/l y en los que se ha
apreciado un tiempo medio de respuesta de $35$ ms y una desviación típica de $12$ ms.
¿Se puede afirmar que una alcoholemia de $0.8$ gm/l influye en el tiempo medio de respuesta con un riesgo
$\alpha=0.05$? ¿Y con un riesgo $\alpha=0.01$?

¿Cuál será la potencia del contraste para detectar una difererencia en la media del tiempo de reacción de 4 ms? ¿Cuál
debería ser el tamaño muestral para aumentar la potencia hasta un 90\%?
}
%SOLUCIÓN
{Contraste para la media: $H_0: \mu=30$, $H_1:\mu>30$.\\
Región de aceptación para $\alpha=0.01$: $z<2.3263$.\\
Estadístico del contraste: $z=2.6020$. Como cae fuera de la región de aceptación, se rechaza la hipótesis nula
tanto para $\alpha=0.01$ y con mayor motivo para $\alpha=0.05$, de manera que se concluye que la alcoholemia influye en
el tiempo de respuesta.\\
Potencia del contraste  para $\delta=4$: $1-\beta=1-0.5966=0.4034$.\\
El tamaño muestral para, $\alpha=0.05$, $\delta=4$ y una potencia del 90\% es $n=80$.  
}
%RESOLUCIÓN
{}


\newproblem{hip-20}{med}{}
%ENUNCIADO
{Se cree que el nivel medio de protrombina en plasma de una población normal tiene una media de 19mg/100ml y una
desviación típica de 4mg/100ml.
Para contrastar estas hipótesis se tomó una muestra de 8 individuos en los que se obtuvieron los siguientes niveles de
protrombina en plasma:
\[
16.3 - 18.4 - 20.0 - 17.6 - 15.4 - 23.7 - 17.8 - 19.5
\]
¿Se pueden aceptar ambas hipótesis con un riesgo $\alpha=0.1$?
}
%SOLUCIÓN
{Contraste para la media: $H_0:\mu=19$, $H_1:\mu\neq 19$.\\
Región de aceptación para $\alpha=0.1$: $-1.8946<t<1.8946$.\\
Estadístico del contraste: $t=-0.4552$. Como cae dentro de la región de aceptación, se mantiene la hipótesis de
que la media es 19mg/100ml.\\
Contraste para la varianza: $H_0:\sigma=4$, $H_1:\sigma\neq 4$.\\
Región de aceptación para $\alpha=0.1$: $2.1673<j<14.0671$.\\
Estadístico del contraste: $j=2.8743$. Como cae dentro de la región de aceptación, también se mantiene la hipótesis
de que la desviación típica es de 4mg/100ml.
}
%RESOLUCIÓN
{}


\newproblem{hip-21}{gen}{*}
%ENUNCIADO
{Para ver si la ley antitabaco está influyendo en el número de cigarros que se fuman mientras se está en los bares se
seleccionó una muestra en la que se midió el número de cigarros fumados por hora mientras se estaba en un bar antes de la entrada en vigor de la ley y otra
muestra distinta en la que también se midió el número de cigarros fumados por hora después de la entrada en vigor de la ley (se entiende
que con la ley en vigor los cigarros se fuman en el exterior de los bares). Los resultados aparecen en la siguientes tablas:
\begin{center}
\begin{tabular}{cc}
\multicolumn{2}{c}{Antes}\\
\hline
Cigarros & Personas \\
\hline
0-1 & 12\\
1-2 & 21\\
2-3 & 20\\
3-4 & 8\\
\hline
\end{tabular}
\qquad
\begin{tabular}{cc}
\multicolumn{2}{c}{Después}\\
\hline
Cigarros & Personas \\
\hline
0-1 & 22\\
1-2 & 18\\
2-3 & 8\\
3-4 & 4\\
\hline
\end{tabular}
\end{center}
Se pide:
\begin{enumerate}
\item Calcular el intervalo de confianza del 99\% para el número medio de cigarros fumados por hora en los bares antes de la entrada en
vigor de la ley. ¿Cuántos individuos serían necesarios para poder estimar dicha media con un margen de error no mayor de $\pm 0.1$ cigarros
por hora?
\item Contrastar si la nueva ley ha reducido significativamente el consumo medio de tabaco en los bares. ¿Cuánto vale el $p$-valor del
contraste?
\end{enumerate}
}
%SOLUCIÓN
{\begin{enumerate}
\item El intervalo de confianza del 99\% para el número medio de cigarros fumados por hora en los bares antes de la entrada en vigor de la
ley es $(1.5789,\,2.2079)$. El tamaño muestral necesario para estimar la media con un margend e error no mayor de $\pm 0.1$ cigarros es
$n=603$ individuos.
\item Contraste para de comparación de medias: $H_0:\mu_x=\mu_y$, $H_1:\mu_x> \mu_y$.\\
Región de aceptación para $\alpha=0.05$: $Z\leq z_\alpha=1.6449$.\\
Estadístico del contraste: $z= 2.8447$. Como cae dentro fuera de la región de aceptación, se rechaza la hipótesis nula y se concluye que la
ley ha reducido el consumo medio de cigarros por hora.\\
El $p$-valor del contraste vale $0.0022$.
\end{enumerate}
}
%RESOLUCIÓN
{Sean $X$ e $Y$ las variables que miden el número medio de cigarros fumados por hora antes y después de la entrada envigor de la ley respectivamente.
\begin{enumerate}  
\item Puesto que se trata de una muestra grande de $n_x=61$, la fórmula del intervalo de confianza para la media es
\[
\bar x \pm z_{\alpha/2}\frac{\hat s}{\sqrt n}.
\]
A partir de la tabla de frecuencias se calculan los estadísticos necesarios:
\begin{align*}
\bar x &= \frac{\sum x_in_i}{n_x} = \frac{0.5\cdot12+\cdots+3.5\cdot8}{61} = \frac{115.5}{61} = 1.8934,\\
s_x^2 & = \frac{\sum x_i^2n_i}{n_x}-\bar x^2 = \frac{0.5^2\cdot12+\cdots+3.5^2\cdot8}{61} -1.8934^2= \frac{273.25}{61}-3.585 = 0.8944,\\
\hat s_x^2 &= \frac{n_x}{n_x-1}s_x^2 = \frac{61}{60}0.8944 = 0.9093,\\
\hat s_x & = \sqrt{0.9093} = 0.9536.
\end{align*}
Como se pide un nivel de confianza del $99\%$ se tiene que $\alpha=0.01$ y $\alpha/2=0.005$, de modo que buscando en la tabla de la función de distribución de la normal estándar se tiene que $z_{\alpha/2} =2.5758$. Así pues, sustituyendo en la fórmula del intervalo se obtiene
\[
\bar x \pm z_{\alpha/2}\frac{\hat s}{\sqrt n} = 1.8934 \pm 2.5758\frac{0.9536}{\sqrt{61}} = 1.8934\pm 0.3145 = (1.5789,\,2.2079).
\]

Por otro lado, el número de individuos necesario para estimar la media con un margen de error no mayor de $\pm 0.1$ cigarros por hora y una confianza del $99\%$ es
\[
n = \frac{4 z_{\alpha/2}^2 \hat s^2}{A^2} = \frac{4\cdot 2.5758^2 \cdot 0.9093}{(2\cdot 0.1)^2} = 603.2974,
\]
es decir, se necesitarían como mínimo $603$ individuos. 

\item Para contrastar si la nueva ley ha reducido significativamente el consumo medio de tabaco en los bares hay que realizar un contraste unilateral de comparación de medias
\begin{align*}
H_0 &: \mu_x = \mu_y\\
H_1 &: \mu_x > \mu_y
\end{align*}
Como los tamaños muestrales son grandes, $n_x=61$ y $n_y=52$, y no se conocen las varianzas poblacionales, el estadístico de contraste es 
\[
Z = \frac{\bar x -\bar y}{\sqrt{\frac{\hat s_x^2}{n_x}+\frac{\hat s_y^2}{n_y}}},
\]
que sigue una distribución normal estándar.
Para calcularlo se necesitan, además de los estadísticos de $X$ calculados en el apartado anterior, los estadísticos de $Y$, que son:
\begin{align*}
\bar y &= \frac{\sum y_jn_j}{n_y} = \frac{0.5\cdot22+\cdots+3.5\cdot4}{52} = \frac{72}{52} = 1.3846,\\
s_y^2 & = \frac{\sum y_j^2n_j}{n_y}-\bar y^2 = \frac{0.5^2\cdot22+\cdots+3.5^2\cdot4}{52} -1.3846^2= \frac{145}{52}-1.9171 = 0.8713,\\
\hat s_y^2 &= \frac{n_y}{n_y-1}s_y^2 = \frac{52}{51}0.8713 = 0.8884,\\
\end{align*}
de manera que sustituyendo en la fórmula del estadístico de contraste se obtiene
\[
Z = \frac{\bar x -\bar y}{\sqrt{\frac{\hat s_x^2}{n_x}+\frac{\hat s_y^2}{n_y}}} 
= \frac{1.8934-1.3846}{\sqrt{\frac{0.9093}{61}+\frac{0.8884}{52}}} = 2.8447.
\]
Como el estadístico sigue una distribucion normal estándar, la región de aceptación para un nivel de signficación $\alpha=0.05$ es $Z\leq z_\alpha=1.6449$, y como el estadístico cae fuera de esta región se rechaza la hipótesis nula y se concluye que hay pruebas significativas de que la nueva ley ha reducido el consumo medio de tabaco en los bares.

Finalmente, el $p$-valor del contraste es $P(Z>2.8447)= 1- P(Z\leq 2.8447) = 1 - F(2.8447) = 0.0022$.
\end{enumerate}
}


\newproblem{hip-22}{gen}{*}
%ENUNCIADO
{En un estudio sobre el reparto de género de las tareas domésticas se ha medido el número medio de horas diarias que se destinan a las
tareas domésticas en un grupo de personas, obteniendo los siguientes resultados:
\[
\begin{array}{lrrrrrrrr}
\text{Mujeres:} & 3.2 & 3.1 & 2.7 & 4.4 & 3.7 & 3.9 & 2.4 & 3.6 \\
\text{Hombres:} & 3.3 & 2.1 & 1.7 & 2.4 & 1.6 & 1.8 & 2.7 
\end{array}
\]
Contrastar si las mujeres dedican más tiempo que los hombres a las tareas domésticas. 
¿Entre qué valores estará la diferencia del tiempo medio destinado a tareas domésticas entre mujeres y hombres para un 95\% de confianza?
}
%SOLUCIÓN
{Estadísticos muestrales:\\
$\bar x_M=3.375$, $s_M^2 = 0.3744$, $\hat s_M^2 = 0.4279$, $\hat s_M = 0.654$,\\ 
$\bar x_H=2.2286$, $s_H^2 = 0.3249$, $\hat s_H^2 = 0.379$, $\hat s_H = 0.616$.

Contraste de comparación de varianzas: $H_0:\ \sigma_M^2=\sigma_H^2$, \quad $H_1:\ \sigma_M^2\neq \sigma_H^2$,\\
Estadístico del contraste: $F=1.129$,\\
Región de aceptación para $\alpha=0.05$: $(0.1954,\,5.6955)$.\\
Por tanto, se acepta la hipótesis nula y se supone que las varianzas son iguales.

Contraste de comparación de medias: $H_0:\ \mu_M=\mu_H$, \quad $H_1:\ \mu_M> \mu_H$,\\
Estadístico del contraste: $T=3.479$.\\
Región de aceptación para $\alpha=0.05$: $(1.7709,\infty)$.\\
Por tanto, se rechaza la hipótesis nula y se concluye que las mujeres dedican más tiempo que los hombres a tareas domésticas.

Intervalo de confianza del $0.95$\% para $\mu_M-\mu_H$: $(0.4345,\,1.8583)$.
}
%RESOLUCIÓN
{
}


\newproblem{hip-23}{psi}{*}
%ENUNCIADO
{Un estudio sobre la adicción a los videojuegos ha medido el número medio de horas diárias que un grupo de jóvenes pasa
jugando a los videojuegos, obteniendo los siguientes resultados:
\[
2.8 - 4.1 - 1.8 - 2.2 - 0.5 - 3.2 - 1.6 - 1.1 - 2.5 - 0.7 - 4.5 - 3.3 - 1.4 - 1.4 - 1.2 - 1.7 
\] 
Se pide:
\begin{enumerate}
\item Calcular el intervalo de confianza para el número medio de horas diarias de juego.
\item Si un número de horas de juego superior a 2 horas puede suponer un alto riesgo de adicción, estimar mediante un intervalo de confianza
del 90\% el porcentaje de jóvenes que estarán en riesgo de adicción. ¿Qué tamaño muestral sería necesario para poder estimar dicho
porcentaje con un margen de error no mayor de $\pm 5\%$?
\end{enumerate}
}
%SOLUCIÓN
{\begin{enumerate}
\item Estadísticos muestrales: $\bar x=2.125$, $\hat s^2 = 1.3913$, $\hat s = 1.18$. Intervalo de confianza del $0.95$\% para $\mu$:
$(1.4962,\,2.7538)$.

\item Proporción muestral: $\hat p = 0.4375$. Intervalo de confianza del $0.9$\% para $p$: $(0.2335,\,0.6415)$.\\
Tamaño muestral necesario para estimar $p$ con una confianza del $0.9$ y un error $A=\pm 0.05 = 0.1$: $n=266.342$. 
\end{enumerate}
}
%RESOLUCIÓN
{
}


\section{Estadística Descriptiva}
\begin{enumerate}[leftmargin=*]
\item \useproblem{des-3}
\item \useproblem{des-41}
\item \useproblem{des-51}
\item \useproblem{des-34}
\item \useproblem{des-7}
\item \useproblem{des-10}
\item \useproblem{des-12}
\item \useproblem{des-11}
\item \useproblem{des-23}
\item \useproblem{des-25}
\item \useproblem{des-24}
\item \useproblem{des-29}
\item \useproblem{des-49}
\item \useproblem{des-50}
\end{enumerate}

\section{Regresión y Correlación}
\begin{enumerate}[leftmargin=*,resume]
\item \useproblem{reg-5}
\item \useproblem{reg-8}
\item \useproblem{reg-3}
\item \useproblem{reg-30}
\item \useproblem{reg-10}
\item \useproblem{reg-11}
\item \useproblem{reg-7}
\item \useproblem{reg-14}
\item \useproblem{reg-15}
\item \useproblem{reg-25}
\item \useproblem{reg-29}
\item \useproblem{reg-31}
\item \useproblem{reg-32}
\item \useproblem{reg-33}
\item \useproblem{reg-34}
\item \useproblem{reg-35}
\item \useproblem{reg-36}
\end{enumerate}

\section{Probabilidad}
\begin{enumerate}[leftmargin=*,resume]
\item \useproblem{pro-31}
\item \useproblem{pro-33}
\item \useproblem{pro-1}
\item \useproblem{pro-2}
\item \useproblem{pro-3}
\item \useproblem{pro-5}
\item \useproblem{pro-8}
\item \useproblem{pro-10}
\item \useproblem{pro-9}
\item \useproblem{pro-12}
\item \useproblem{pro-16}
\item \useproblem{pro-17}
\item \useproblem{pro-19}
\item \useproblem{pro-20}
\item \useproblem{pro-25}
\item \useproblem{pro-38}
\item \useproblem{pro-39}
\item \useproblem{pro-40}
\end{enumerate}

\section{Variables Aleatorias}
\begin{enumerate}[leftmargin=*,resume]
\item \useproblem{vad-1}
\item \useproblem{vad-2}
\item \useproblem{vad-3}
\item \useproblem{vad-5}
\item \useproblem{vad-7}
\item \useproblem{vad-26}
\item \useproblem{vad-11}
\item \useproblem{vad-14}
\item \useproblem{vad-15}
\item \useproblem{vad-18}
\item \useproblem{vad-20}
\item \useproblem{vad-22}
\item \useproblem{vad-27}
\item \useproblem{vad-28}
\item \useproblem{vad-29}
\item \useproblem{vad-30}
\item \useproblem{vad-32}
\item \useproblem{vad-33}
\item \useproblem{vac-6}
\item \useproblem{vac-7}
\item \useproblem{vac-11}
\item \useproblem{vac-14}
\item \useproblem{vac-17}
\item \useproblem{vac-18}
\item \useproblem{vac-20}
\item \useproblem{vac-21}
\item \useproblem{vac-24}
\item \useproblem{vac-32}
\end{enumerate}

\section{Estimación de parámetros}
\begin{enumerate}[leftmargin=*,resume]
\item \useproblem{ico-35}
\item \useproblem{ico-1}
\item \useproblem{ico-3}
\item \useproblem{ico-13}
\item \useproblem{ico-37}
\item \useproblem{ico-14}
\item \useproblem{ico-32}
\item \useproblem{ico-33}
\item \useproblem{ico-16}
\item \useproblem{ico-21}
\item \useproblem{ico-19}
\item \useproblem{ico-36}
\item \useproblem{ico-34}
\item \useproblem{ico-27}
\item \useproblem{ico-29}
\item \useproblem{ico-38}
\end{enumerate}


\section{Contraste de hipótesis}
\begin{enumerate}[leftmargin=*,resume]
\item \useproblem{hip-1}
\item \useproblem{hip-19}
\item \useproblem{hip-2}
\item \useproblem{hip-20}
\item \useproblem{hip-3}
\item \useproblem{hip-13}
\item \useproblem{hip-10}
\item \useproblem{hip-5}
\item \useproblem{hip-8}
\item \useproblem{hip-16}
\item \useproblem{hip-17}
\item \useproblem{hip-18}
\item \useproblem{hip-4}
\item \useproblem{hip-12}
\end{enumerate}

\vspace{2cm}

\textsc{Nota}: Los problemas marcados con una estrella ($\bigstar$) son problemas de
exámenes de otros años.
\end{document}
