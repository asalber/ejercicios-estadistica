% Author Alfredo S�nchez Alberca (asalber@ceu.es)

\newproblem{ico-1}{gen}{}
% ENUNCIADO
{Una muestra aleatoria de tamaño $81$ extraída de una población normal con $\sigma ^2=64$, tiene una $\overline{x}=78$.
Calcular el intervalo de confianza del $95\%$ para $\mu$.
}
%SOLUCIÓN
{$\mu\in 79\pm 1.742 = (76.258,\,79.742)$.
}
%RESOLUCIÓN
{}


\newproblem{ico-2}{nut}{}
% ENUNCIADO
{Para determinar si un pescado es o no apto para el consumo por su contenido en Hg (mercurio), se realizan $15$
valoraciones obteniendo una media de $0.44$ ppm (partes por millón) de Hg, y una desviación típica de $0.08$ ppm.
Calcular los límites de confianza para la media, a un nivel de significación $\alpha =0.1.$ }
% SOLUCIÓN
{
$\mu\in 0.44\pm 0.0376$ ppm = $(0.4024\text{ppm},\,0.4776\text{ppm})$.
}
% RESOLUCIÓN
{}


\newproblem{ico-3}{qui}{}
% ENUNCIADO
{Se obtuvieron cinco determinaciones del pH de una solución con los siguientes resultados: 
\[7.90, 7.85, 7.89, 7.86, 7.87.\]
Hallar unos límites de confianza de la media de todas las determinaciones del pH de la misma solución, al nivel de
significación $\alpha =0.01.$
}
% SOLUCIÓN
{
$\mu \in 7.874\pm 0.0426 = (7.8314,\,7.9166)$.
}
% RESOLUCIÓN
{}


\newproblem{ico-4}{med}{}
% ENUNCIADO
{Se desea saber cuál debe ser el tamaño muestral mínimo de una muestra para poder realizar la estimación de la tasa media
de glucosa plasmática de una determinada población, con un nivel de confianza $0.95$ y pretendiendo una amplitud de $2.5$
mg.

\noindent NOTA: En una muestra previa de tamaño 10 se obtuvo una desviación típica de 10 mg.
}
%SOLUCIÓN
{
249 individuos.
}
%RESOLUCIÓN
{}


\newproblem{ico-5}{far}{}
%ENUNCIADO
{Para que un fármaco sea efectivo, la concentración de un determinado principio activo debe ser 20 mg/mm$^3$.
Se recibe un lote de dicho fármaco y se analizan 10 para medir la concentración del principio activo, obteniendo los
resultados siguientes:
\[
17.6 - 19.2 - 21.3 - 15.1 - 17.6 - 18.9 - 16.2 - 18.3 - 19 - 16.4.
\]
En vista de los resultados, ¿podremos rechazar el lote con una confianza $0.95$ de no equivocarnos?
}
%SOLUCIÓN
{
$17.96\pm 1.278$ mg/mm$^3$ = $(16.682\text{mg/mm}^3,\,19.238\text{mg/mm}^3)$.
}
%RESOLUCIÓN
{}


\newproblem*{ico-6}{nut}{*}
%ENUNCIADO
{En un estudio sobre el consumo anual de litros de cerveza entre la población de menores de 18 años de una ciudad se
obtuvo la siguiente muestra: 
\[ 42, 16, 60, 29, 7, 20, 30, 25, 38, 5.\]
Se pide:
\begin{enumerate}
\item Calcular el intervalo de confianza del 95\% para la media. Si se considera que un consumo medio por encima de 40
litros es peligroso, ¿existen pruebas significativas para afirmar que la población de partida no está en peligro?
\item ¿Qué tamaño muestral mínimo hubiese sido necesario para conseguir un intervalo de confianza de amplitud 5?
\end{enumerate}
}
%SOLUCIÓN
{}
%RESOLUCIÓN
{}


\newproblem*{ico-7}{amb}{}
%ENUNCIADO
{En una explotación minera se mide el contenido en mercurio de las rocas extraídas.
Tras analizar 20 rocas, se obtiene un contenido medio del $10.8\%$ y una desviación típica de $2.7\%$. Se pide: 
\begin{enumerate}
\item Si, para que la explotación sea rentable, el porcentaje medio de contenido de mercurio debe ser superior al
10\%, ¿existen pruebas para afirmar que la explotación será rentable?
\item ¿Y si para que la explotación sea rentable, el contenido de mineral debe tener cierta uniformidad ($\sigma<3$)?
\end{enumerate}
}
%SOLUCIÓN
{}
%RESOLUCIÓN
{}


\newproblem*{ico-8}{amb}{}
%ENUNCIADO
{Para ver si una población de aves se ha visto afectada por los vertidos tóxicos de una fábrica, se pretende estimar
la proporción de aves contaminadas con metales pesados.
Para ello se realiza un sondeo preliminar con 30 aves, de las que 5 resultan estar contaminadas.
Se pide:
\begin{enumerate}
\item Construir un intervalo de confianza del 95\% para la proporción de aves contaminadas.
\item Si se desea estimar la proporción con un error máximo de $\pm 2\%$, ¿qué tamaño muestral habría que tomar?
\end{enumerate}
}
%SOLUCIÓN
{}
%RESOLUCIÓN
{}


\newproblem{ico-9}{far}{}
%ENUNCIADO
{Se realizó un estudio sobre el contenido de principio activo de un determinado fármaco a partir de una muestra,
determinándose los siguientes resultados en mg/cm$^{3}$:
\[ 46.4-46.1-45.8-47.0-46.1-45.9-45.8-46.9-45.2-46.0. \]
Obtener un intervalo de confianza del 95\% para la varianza del contenido de principio activo de dicho fármaco,
suponiendo que sigue una distribución normal.
}
%SOLUCIÓN
{
$\sigma^2 \in (0.1356(\text{mg/cm}^3)^2,\,0.9541(\text{mg/cm}^3)^2)$.
}
%RESOLUCIÓN
{}


\newproblem{ico-10}{med}{}
%ENUNCIADO
{Se desea hacer un estudio estadístico sobre el número de hematíes en las mujeres.
Se selecciona una muestra de 25 mujeres y se obtiene un número medio de hematíes de 4300000 con una desviación típica
de 200000 (en cada milímetro cúbico de sangre).
Calcular el intervalo de confianza para la media y la varianza del número de hematíes de las mujeres en la población,
con un nivel de significación $0.1$.
}
%SOLUCIÓN
{
Intervalo de confianza del 95\% para la media: $\mu \in 4.3\cdot 10^6\pm 69850$ hematíes = $(4230150,\,4369850)$.\\
Intervalo de confianza del 95\% para la varianza: $\sigma^2\in (2747\cdot 10^10,\,7246\cdot 10^10)$ hematíes$^2$. 
}
%RESOLUCIÓN
{}


\newproblem*{ico-11}{med}{*}
%ENUNCIADO
{Para determinar el nivel medio de colesterol en la sangre de una población, se realizaron análisis sobre una muestra
de 8 personas, obteniéndose los siguientes resultados:
\begin{center}
196 -- 212 -- 188 -- 206 -- 203 -- 210 -- 201 -- 198
\end{center}
Hallar intervalos de confianza para la media y la varianza de nivel de colesterol con un nivel de significación 0.1, suponiendo que el nivel de colesterol en la población sigue una distribución normal.
}
%SOLUCIÓN
{}
%RESOLUCIÓN
{}


\newproblem{ico-12}{med}{}
%ENUNCIADO
{Para determinar la concentración media de albúmina en la sangre se realizaron mediciones sobre un grupo experimental
obteniéndose los siguientes resultados, expresados en g/l: 
\[38-42-46-37-49-42-40-36.\]
Obtener un intervalo de confianza para la varianza de la población con un nivel de significación $0.05$.
}
%SOLUCIÓN
{
$\sigma^2\in (8.844\text{g}^2/\text{l}^2,\,83.728\text{g}^2/\text{l}^2)$.
}
%RESOLUCIÓN
{}


\newproblem{ico-13}{med}{}
%ENUNCIADO
{El tiempo que tarda en hacer efecto un analgésico sigue una distribución aproximadamente normal.
En una muestra de 20 pacientes se obtuvo una media de $25.4$ minutos y una desviación típica de $5.8$ minutos.
Calcular el intervalo de confianza del 90\% e interpretarlo.
¿Cuántos pacientes habría que tomar para poder estimar la media con una precisión de $\pm 1$ minuto? }
%SOLUCIÓN
{Intervalo de confianza del 90\% para $\mu$: $(23.0992,\,27.7008)$.\\
Tamaño muestral para una precisión de $\pm 1$ minuto: $n=96$.}
%RESOLUCIÓN
{}


\newproblem{ico-14}{med}{}
%ENUNCIADO
{En un estudio para el estado de la salud oral de una ciudad, se tomó una muestra elegida al azar de 280 varones entre
35 y 44 años y se contó el número de piezas dentarias en la boca.
Tras la revisión pertinente, los dentistas informaron que había 70 individuos con 28 o más dientes.
Se desea realizar una estimación por intervalo de confianza de la proporción de individuos de esta ciudad con 28
dientes o más, con un nivel de confianza $0.95$.
}
%SOLUCIÓN
{
$p\in 0.25\pm 0.0507 = (0.1993,\,0.3007)$.
}
%RESOLUCIÓN
{}


\newproblem*{ico-15}{med}{}
%ENUNCIADO
{Leemos en una revista médica que la cuarta parte de los cancerosos de cierto tumor de estómago presentan vómitos, con
una precisión o tolerancia del 10\% y con una confianza del 99\%.
¿Con cuántos pacientes se ha realizado el estudio?
}
%SOLUCIÓN
{}
%RESOLUCIÓN
{}


\newproblem{ico-16}{med}{*}
%ENUNCIADO
{Un país está siendo afectado por una epidemia de un virus.
Para valorar la gravedad de la situación se tomaron 40 personas al azar y se comprobó que 12 de ellas tenían el virus.
Determinar el intervalo de confianza para el porcentaje de infectados con un nivel de significación $0.05$.
}
%SOLUCIÓN
{
$p\in (0.1580,\,0.4420)$ con un 95\% de confianza.
}
%RESOLUCIÓN
{}


\newproblem{ico-17}{gen}{}
%ENUNCIADO
{Se desea obtener un intervalo de confianza del 95\% para la diferencia de marcas obtenidas por chicos y chicas en una
prueba física.
Se toma una muestra de 50 chicas y 75 chicos, obteniendo las chicas una marca media de 76 y los chicos de 82.
Además, se conocen las desviaciones típicas de las marcas obtenidas en las poblaciones de chicas y chicos, que son 6 y
8 respectivamente.
}
%SOLUCIÓN
{
$\mu_1-\mu_2 \in 6\pm 2.458$ puntos = $(3.542\text{puntos},\,8.458)$.
}
%RESOLUCIÓN
{}


\newproblem*{ico-18}{amb}{}
%ENUNCIADO
{Las temperaturas medias mensuales (en $^\circ$C) durante el año 2001 en Madrid y Sevilla fueron:
\begin{center}
\begin{tabular}{|l|r|r|r|r|r|r|r|r|r|r|r|r|}
\hline
Ciudad &    Ene &    Feb &    Mar &    Abr &    May &    Jun &    Jul &    Ago &    Sep &    Oct &    Nov &    Dic \\
\hline
 Madrid                  &  $7.2$ &  $8.4$ & $12.2$ & $13.7$ & $16.7$ & $23.3$ & $24.2$ & $25.5$ & $20.4$ & $16.2$ &  $8.1$ &  $4.2$ \\
\hline
 Sevilla                 & $12.1$ & $13.6$ & $16.8$ & $19.0$ & $20.9$ & $27.0$ & $26.6$ & $28.2$ & $24.7$ & $21.3$ & $13.8$ & $11.5$ \\
\hline
\end{tabular}
\end{center}
¿Existen diferencias significativas en las temperaturas medias de ambas ciudades?
}
%SOLUCIÓN
{}
%RESOLUCIÓN
{}


\newproblem{ico-19}{fis}{}
%ENUNCIADO
{Se está ensayando un nuevo procedimiento de rehabilitación para una cierta lesión.
Para ello se trataron nueve pacientes con el procedimiento tradicional y otros nueve con el nuevo, y se midieron los
días que tardaron en recuperase, obteniéndose los siguientes resultados:
\begin{center}
\begin{tabular}{ll}
Método tradicional: & 32-37-35-28-41-44-35-31-34\\
Método nuevo: & 35-31-29-25-34-40-27-32-31
\end{tabular}
\end{center}
Se desea obtener un intervalo de confianza del 95\% para la diferencia de las medias del tiempo de recuperación
obtenido con ambos procedimientos.
Se supone que los tiempos de recuperación siguen una distribución normal, y que las varianzas son aproximadamente
iguales para los dos procedimientos.
}
%SOLUCIÓN
{
$\mu_1-\mu_2\in 3.667\pm 4.712$ días = $(-1.045\text{ días},\,8.379\text{ días})$.
}
%RESOLUCIÓN
{}


\newproblem{ico-20}{med}{}
%ENUNCIADO
{En un hospital pediátrico se comprobó que de 200 niños con un determinado síndrome, 48 murieron antes de cumplir un
año de edad, mientras que sólo 25 de 125 niñas con el mismo síndrome murieron.
¿Se puede afirmar con cierta seguridad que el síndrome es más letal en los niños que en las niñas?
}
%SOLUCIÓN
{
$p_1-p_2 \in 0.04\pm 0.077 = (-0.037,\,0.117)$ luego no se puede afirmar que el síndrome sea más letal en los niños que
en las niñas con un 95\% de confianza.}
%RESOLUCIÓN
{}


\newproblem{ico-21}{med}{}
%ENUNCIADO
{Se ha realizado un estudio para investigar el efecto del ejercicio físico en el nivel de colesterol en la sangre. En el estudio
participaron once personas, a las que se les midió el nivel de colesterol antes y después de desarrollar un programa de ejercicios. Los
resultados obtenidos fueron los siguientes
\begin{center}
\begin{tabular}{|c|c|c|}
\hline Persona & Nivel previo & Nivel posterior \\ 
\hline\hline 
1 & 182 & 198 \\
\hline 
2 & 232 & 210 \\ 
\hline 
3 & 191 & 194 \\ 
\hline 
4 & 200 & 220 \\ 
\hline 
5 & 148 & 138 \\ 
\hline 
6 & 249 & 220 \\ 
\hline 
7 & 276 & 219 \\ 
\hline 
8 & 213 & 161 \\
\hline 
9 & 241 & 210 \\ 
\hline 
10 & 280 & 213 \\ 
\hline 
11 & 262 & 226 \\ 
\hline
\end{tabular}
\end{center}
Hallar un intervalo de confianza del 90\% para la diferencia del nivel medio de colesterol antes y después del ejercicio.
}
%SOLUCIÓN
{
$\mu_{x_1-x_2}\in 24.0909\pm 19.4766$ mg/dl = $(4.6143\text{mg\dl},\, 43.5675\text{mg/dl})$.
}
%RESOLUCIÓN
{}


\newproblem{ico-22}{qui}{}
%ENUNCIADO
{Dos químicos $A$ y $B$ realizan 14 y 16 determinaciones, respectivamente, de plutonio.
Los resultados obtenidos se muestran en la siguiente tabla
\begin{center}
\begin{tabular}{cc|cc}
\multicolumn{2}{c|}{$A$} & \multicolumn{2}{c}{$B$} \\ \hline
263.36 & 254.68 & 286.53 & 254.54 \\
248.64 & 276.32 & 284.55 & 286.30 \\
243.64 & 256.42 & 272.52 & 282.90 \\
272.68 & 261.10 & 283.85 & 253.75 \\
287.33 & 268.41 & 252.01 & 245.26 \\
287.26 & 282.65 & 275.08 & 266.08 \\
250.97 & 284.27 & 267.53 & 252.05 \\
&  & 253.82 & 269.81
\end{tabular}
\end{center}
Se pide:

\begin{enumerate}
\item Calcular intervalos de confianza del 95\% de confianza para cada caso.
\item ¿Se puede decir que existen diferencias significativas en la media?
\end{enumerate}
}
%SOLUCIÓN
{
\begin{enumerate}
\item Intervalo de confianza del 95\% para la media de $A$: $\mu_A \in 266.98\pm 8.681 = (258.299,\,275.661)$.\\
Intervalo de confianza del 95\% para la media de $B$: $\mu_B \in 267.91\pm 7.677 = (260.233,\,275.587)$.
\item $\mu_A-\mu_B \in -0.93\pm 11.020 = (-11.95,\,10.92)$, luego no hay diferencias significativas en las medias con
un 95\% de confianza.
\end{enumerate}
}
%RESOLUCIÓN
{}


\newproblem{ico-23}{med}{*}
%ENUNCIADO
{Un equipo de investigación está interesado en ver si una droga reduce el colesterol en la sangre.
Con tal fin se toma una muestra de 10 pacientes y determina el contenido de colesterol antes y después del tratamiento.
Los resultados expresados en miligramos por cada 100 mililitros son los siguientes:
\[
\begin{tabular}{|c|c|c|c|c|c|c|c|c|c|c|}
\hline
Paciente & 1 & 2 & 3 & 4 & 5 & 6 & 7 & 8 & 9 & 10 \\ \hline
Antes & 217 & 252 & 229 & 200 & 209 & 213 & 215 & 260 & 232 & 216 \\ \hline
Después & 209 & 241 & 230 & 208 & 206 & 211 & 209 & 228 & 224 & 203 \\
\hline
\end{tabular}
\]
Se pide:

\begin{enumerate}
\item Construir la variable Diferencia que recoja la diferencia entre los niveles de colesterol antes y después del
tratamiento, y calcular el intervalo de confianza con $1-\alpha =0.95$ para la media de dicha variable.
\item A la vista del intervalo anterior, ¿hay pruebas significativas de que la droga disminuye el nivel de colesterol
en sangre?
\end{enumerate}
}
%SOLUCIÓN
{
\begin{enumerate}
\item $\mu_{x_1-x_2}\in 7.4\pm 7.572$ mg/100ml = $(-0.172\text{mg/100ml},\,14.972\text{mg/100ml})$.
\item No se puede afirmar que la droga disminuya el colesterol con un 95\% de confianza.
\end{enumerate}
}
%RESOLUCIÓN
{}


\newproblem*{ico-24}{fis}{*}
%ENUNCIADO
{Se está ensayando un nuevo procedimiento de rehabilitación para una cierta lesión.
Se sabe que de 80 deportistas tratados con el procedimiento tradicional, se recuperaron perfectamente 26, mientras que
de los 20 tratados con el nuevo procedimiento se han recuperado 11.
¿Se puede afirmar con una confianza del 95\% que el nuevo procedimiento es mejor que el tradicional?
}
%SOLUCIÓN
{}
%RESOLUCIÓN
{}


\newproblem{ico-25}{amb}{*}
%ENUNCIADO
{En una muestra aleatoria de 200 personas, 114 están a favor de la fluoración de las aguas.
Se pide:
\begin{enumerate}
\item Hallar el intervalo de confianza del 96\% para la fracción de la población que está a favor de la fluoración de
las aguas.
\item ¿Qué tamaño mínimo de muestras habría que tomar para tener una confianza del 96\% de que la proporción muestral
difiere menos de $0.02$ de la proporción real de la población?
\end{enumerate}
}
%SOLUCIÓN
{
\begin{enumerate}
\item $p\in 0.375\pm 0.197 = (0.178,\,0.572)$.
\item 2634 individuos.
\end{enumerate}
}
%RESOLUCIÓN
{}


\newproblem*{ico-26}{far}{*}
%ENUNCIADO
{Para ver si una campaña de publicidad sobre un fármaco ha influido en sus ventas, se tomó una muestra de 8 farmacias
y se midió el número de fármacos vendidos durante un mes, antes y después de la campaña, obteniéndose los siguientes
resultados:
\begin{center}
\begin{tabular}{|c||c|c|c|c|c|c|c|c|}
\hline
Antes & 147 & 163 & 121 & 205 & 132 & 190 & 176 & 147  \\
\hline
Después & 150 & 171 & 132 & 208 & 141 & 184 & 182 & 145  \\
\hline
\end{tabular}
\end{center}
Obtener la variable diferencia y construir un intervalo de confianza para la media de la diferencia con un nivel de
significación $0.05$.
¿Existen pruebas suficientes para afirmar con un 95 \% de confianza que la campaña de publicidad ha aumentado las
ventas?
}
%SOLUCIÓN
{}
%RESOLUCIÓN
{}


\newproblem{ico-27}{med}{*}
%ENUNCIADO
{Para comparar la eficacia de dos tratamientos $A$ y $B$ en la prevención de repeticiones de infarto de miocardio, se
aplicó el tratamiento $A$ a 80 pacientes y el $B$ a 60.
Al cabo de dos años se observó que habían sufrido un nuevo infarto 14 pacientes de los sometidos al tratamiento $A$ y
15 de los del $B$.
Se pide:
\begin{enumerate}
\item Construir un intervalo de confianza del $95\%$ para la diferencia entre las proporciones de personas sometidas a los tratamientos $A$ y $B$ que no vuelven a sufrir un infarto.
\item A la vista del resultado obtenido, razonar si con ese nivel de confianza puede afirmarse que uno de los tratamientos es más eficaz que el otro.
\end{enumerate}
}
%SOLUCIÓN
{
\begin{itemize}
\item $p_A-p_B\in (-0.2126,\,0.0626)$ con un nivel de confianza del 95\%.
\item No puede afirmarse que un tratamiento sea más eficaz que otro pues la diferencia de medias podría ser positiva,
negativa o cero. 
\end{itemize}
}
%RESOLUCIÓN
{}


\newproblem*{ico-28}{amb}{}
%ENUNCIADO
{La siguiente tabla muestra el porcentaje de industrias españolas y europeas según el consumo de energía primaria de
las mismas durante el año 2002 (se estudiaron 1000 industrias).
\[
\begin{tabular}{|l|c|c|}
\hline
Fuente energética  &  España  & Resto de Europa \\
\hline
Petróleo            & $52.2$\% &    $40.4$\%     \\
Carbón              & $15.2$\% &    $14.8$\%     \\
Nuclear             & $13.0$\% &    $15.2$\%     \\
Gas                 & $12.8$\% &    $23.5$\%     \\
Energías Renovables & $6.5$\%  &     $6.1$\%     \\
Saldo Energético    & $0.2$\%  &       0\%       \\
\hline
\end{tabular}
\]
Estudiar para qué energías el porcentaje de industrias de España es significativamente diferente del resto de Europa.
}
%SOLUCIÓN
{}
%RESOLUCIÓN
{}


\newproblem{ico-29}{nut}{*}
%ENUNCIADO
{En un análisis de obesidad dependiendo del hábitat en niños menores de 5 años, se obtienen los siguientes resultados:
\begin{center}
\begin{tabular}{|l|l|l|}
\cline{2-3}
\multicolumn{1}{l|}{} & Casos analizados & Casos con sobrepeso \\
\hline
Hábitat rural & 1150 & 480 \\
\hline
Hábitat urbano & 1460 & 660 \\
\hline
\end{tabular}
\end{center}
Se pide:

\begin{enumerate}
\item  Construir un intervalo de confianza, con un nivel de significación $0.01$, para la proporción de niños menores
de 5 años con sobrepeso en el hábitat rural. Igualmente para el hábitat urbano.
\item Construir un intervalo de confianza, con un nivel de confianza del $95\%$, para la diferencia de proporciones de
niños menores de 5 años con sobrepeso entre el hábitat rural y el urbano.
A la vista del resultado obtenido, ¿se puede concluir, con un $95\%$ de confianza, que la proporción de niños menores
de 5 años con sobrepeso depende del hábitat?
\end{enumerate}
}
%SOLUCIÓN
{
\begin{itemize}
\item $p_R \in (0.3799,\,0.4548)$ y $p_U\in (0.4185,\,0.4856)$ con un 99\% de confianza.
\item $p_R-p_U\in (-0.0729,\,0.0036)$ con un nivel de confianza del 95\%, luego no se puede afirmar que haya
diferencias en las proporciones de niños menores de 5 años con sobrepeso.
\end{itemize}
}
%RESOLUCIÓN
{}


\newproblem*{ico-30}{med}{}
%ENUNCIADO
{Se ha realizado un estudio con 1000 mujeres que han dado a luz recientemente, elegidas al azar entre los registros de
los diferentes hospitales de la comunidad de Madrid, para saber si un nuevo protocolo (visitas al médico y consumo de
ciertos fármacos) resulta más efectivo para prevenir las infecciones (ya sean pre, intra o postparto).
Del total, 750 han seguido el protocolo habitual, entre las cuales 35 han sufrido algún tipo de infección; mientras
que 250 han seguido el protocolo nuevo y 9 de ellas han padecido alguna infección.
¿Se puede afirmar, con un 95\% de confianza, que la proporción de mujeres que ha tenido algún tipo de infección ha
sido diferente según el protocolo utilizado?
}
%SOLUCIÓN
{}
%RESOLUCIÓN
{}


\newproblem*{ico-31}{amb}{*}
%ENUNCIADO
{La siguiente tabla muestra los datos de emisiones de CO$_2$ y CH$_4$ (en Kg/hab) y el producto interior bruto per
cápita (en miles US\$) de varios países en el último año:
\[
\begin{array}{|l|r|r|r|}
\hline
\mbox{País} & \mbox{CO}_2 & \mbox{CH}_4 & \mbox{PIB}\\
\hline\hline
\mbox{Austria}     & 7.60 & 0.97 & 38.40\\ \hline
\mbox{España}      & 6.73 & 0.81	& 30.12\\ \hline
\mbox{Francia}     & 5.71 & 0.94	& 33.19\\ \hline
\mbox{EEUU}        &19.40 & 1.72	&	45.84\\ \hline
\mbox{Alemania}    & 9.80 & 0.83	& 34.18\\ \hline
\mbox{Canadá}      &15.60 & 3.08	& 38.43\\ \hline
\mbox{Italia}      & 7.29 & 0.58	& 30.44\\ \hline
\mbox{Japón}       &	9.44 & 0.16	& 33.58\\ \hline
\mbox{Australia}   &17.48 & 6.36	& 36.26\\ \hline
\mbox{Reino Unido} & 8.99 & 0.76	& 35.13\\ \hline
\end{array}
\qquad
\begin{array}{|l|r|r|r|}
\hline
\mbox{País} & \mbox{CO}_2 & \mbox{CH}_4 & \mbox{PIB}\\
\hline\hline
\mbox{Bolivia}     & 1.05 & 3.44	& 40.13\\ \hline
\mbox{Niger}       &	0.1	 & 0.12	&	 0.67\\ \hline
\mbox{Senegal}     &	0.35 & 0.76 &  1.69\\ \hline
\mbox{Pakistán}    & 0.65 & 0.59	&  2.59\\ \hline
\mbox{Filipinas}   &	0.83 & 0.46	&  3.38\\ \hline
\mbox{Perú}        & 0.94 & 0.75	&  7.80\\ \hline
\mbox{Túnez}      & 2.17 & 0.48	&  7.47\\ \hline
\mbox{Nepal}       & 0.13 & 0.90	&  1.21\\ \hline
\mbox{Nicaragua}   & 0.7	 & 0.32	&  2.62\\ \hline
\mbox{Mauritania}  & 0.97 & 0.85	&  2.01\\ \hline
\end{array}
\]
¿Existen diferencias significativas en la emisión de CO$_2$ entre los países con un PIB superior a 10 mil US\$ y los
países con un PIB inferior? ¿Y en las emisiones de CH$_4$? Justificar la respuesta.
}
%SOLUCIÓN
{}
%RESOLUCIÓN
{}


\newproblem{ico-32}{gen}{}
%ENUNCIADO
{En una muestra de 250 estudiantes de una universidad, 146 hablaban inglés.
¿Entre qué valores estará el porcentaje de individuos de la universidad que hablan inglés, con un nivel de confianza
del 90\%?
}
%SOLUCIÓN
{
$p\in (0.5327,\,0.6353)$ con un 90\% de confianza.
}
%RESOLUCIÓN
{}


\newproblem{ico-33}{psi}{}
%ENUNCIADO
{Si el porcentaje de indivudos daltónicos de una muestra aleatoria es 18\%, ¿cuál será el mínimo tamaño muestral
necesario para conseguir una estimación del porcentaje de daltónicos con una confianza del 95\% y un error menor del 3\%?
}
%SOLUCIÓN
{
$n=2520$.
}
%RESOLUCIÓN
{}


\newproblem{ico-34}{psi}{}
% ENUNCIADO
{Un psicólogo está estudiando la concentración de una encima en la saliba como un posible indicador de la ansiedad crónica.
En un experimento se tomó una muestra de 12 neuróticos por ansiedad y otra de 10 personas con bajos niveles de ansiedad.
En ambas muestras se midió la concentración de la encima, obteniendo los siguientes resultados:
\[
\begin{array}{rcccccccccccc}
\hline
\mbox{Con ansiedad:} & 2.60 & 2.90 & 2.60 & 2.70 & 3.91 & 3.15 & 3.94 & 2.46 & 2.91 & 3.88 & 3.55 & 3.96\\
\mbox{Sin ansiedad:} & 2.37 & 1.10 & 2.55 & 2.64 & 2.20 & 2.12 & 2.47 & 2.90 & 1.66 & 2.72 \\
\hline
\end{array}
\]
¿Se puede concluir a partir de estos datos que la población de neuróticos con ansiedad y la población de personas sin
ansiedad son diferentes en el nivel medio de concentración de encimas?
Justificar la respuesta.
}
%SOLUCIÓN
{
$\frac{\sigma_1^2}{\sigma_2^2}\in (0.3383,\,4.7486)$ con un nivel de confianza del 95\%, luego se puede suponer que las
varianzas son iguales.\\
$\mu_1-\mu_2\in (0.4296,\,1.4510)$ con un 95\% de confianza, luego se puede concluir que hay diferencias entre las
medias.}
%RESOLUCIÓN
{}


\newproblem{ico-35}{gen}{}
%ENUNCIADO
{Las notas en Estadística de una muestra de 10 alumnos han sido:
\begin{center}
$6.3$, $5.4$, $4.1$, $5.0$, $8.2$, $7.6$, $6.4$, $5.6$, $4.3$, $5.2$
\end{center}
Dar una estimación puntual de la nota media, de la varianza y del porcentaje de aprobados en la clase.
}
%SOLUCIÓN
{$\bar x= 5.81$ puntos, $\hat s^2=1.7721$ puntos$^2$ y $\hat p=0.8$.}
%RESOLUCIÓN
{}


\newproblem{ico-36}{psi}{}
%ENUNCIADO
{Para estudiar si la estación del año influye en el estado de ánimo de la gente, se ha tomado una muestra 12 personas y
se ha medido su nivel de depresión en verano e invierno mediante un cuestionario con puntuaciones de 0 a 100 (a mayor
puntuación mayor depresión). Los resultados obtenidos fueron:
\begin{center}
\begin{tabular}{|l|r|r|r|r|r|r|r|r|r|r|r|r|}
\hline
Invierno & 65 & 72 & 84 & 31 & 80 & 61 & 75 & 52 & 73 & 79 & 85 & 71\\
\hline
Verano &   60 & 51 & 81 & 45 & 62 & 53 & 70 & 52 & 64 & 51 & 67 & 62\\
\hline
\end{tabular}
\end{center}
¿Se puede afirmar que la estación del año influye en el estado de ánimo de la gente con un 99\% de confianza?
¿Cómo influye?
}
%SOLUCIÓN
{
No se puede afirmar que la estación influya en el estado de ánimo ya que la media de la diferencia está en
$(-0.7498,\,19.0831)$ con un 99\% de confianza.}
%RESOLUCIÓN
{}


\newproblem{ico-37}{psi}{}
%ENUNCIADO
{Para realizar una determinada tarea se necesitan personas con un cierto nivel de pericia.
Una empresa está interesada en contratar a un grupo de personas que tengan la pericia necesaria para realizar la tarea,
pero que además sean bastante homogéneas en cuanto a la pericia, es decir, que su rendimiento sea parecido.
Para ver si un grupo de alumnos que se han formado en dicha tarea cumplen los requisitos, se ha tomado una muestra y se
les ha sometido una prueba para ver cuántas tareas son capaces de realizar con éxito en una hora.
Los resultados obtenidos fueron:
\begin{center}
16 - 12 - 14 - 21 - 11 - 15 - 17 - 15 - 18 - 24 - 10 - 14 - 17 - 14 
\end{center}
Se pide: 
\begin{enumerate}
\item Si la empresa busca un grupo de empleados capaces de realizar una media de al menos 12 tareas por hora, ¿se puede
afirmar con una confianza del 95\% que el grupo lo cumple?
\item Si la empresa busca que entre los trabajadores haya una dispersión media de $\sigma<3$ tareas, ¿se puede afirmar
con una confianza del 95\% que el grupo lo cumple? 
\end{enumerate} 
}
%SOLUCIÓN
{
\begin{itemize}
\item Lo cumple ya que $\mu\in (13.4026,\,17.7403)$ con un 95\% de confianza.
\item No lo cumple ya que $\sigma \in (2.7232,\,6,0516)$ con un 95\% de confianza.
\end{itemize}
}
%RESOLUCIÓN
{}


\newproblem*{ico-38}{psi}{}
%ENUNCIADO
{Se cree que el consumo de tabaco va ligado al consumo de alcohol y para corroborar esta hipótesis se ha realizado un
estudio en el que se han obtenido los siguientes datos
\begin{center}
\begin{tabular}{lcc}
 & Bebedores & No bebedores\\
\cline{2-3}
Fumadores & \multicolumn{1}{|c}{487} & \multicolumn{1}{c|}{137} \\
No fumadores & \multicolumn{1}{|c}{312} & \multicolumn{1}{c|}{365}\\
\cline{2-3}
\end{tabular}
\end{center}
¿Se puede afirmar que existe relación entre el consumo de tabaco y el de alcohol?
Justificar la respuesta.
}
%SOLUCIÓN
{}
%RESOLUCIÓN
{}


\newproblem{ico-39}{far}{*}
%ENUNCIADO
{Para analizar la eficacia de un determinado fármaco para aumentar las horas de sueño se realiza un estudio con 8 pacientes a los que se les
controla durante todo un mes, en el que no toman el medicamento y se cuantifica la media de las horas de sueño para cada uno de ellos. El
mismo proceso se lleva a cabo durante todo un mes en el que sí que toman el fármaco. Los resultados (en horas de sueño) aparecen en la
siguiente tabla:
\[
\begin{array}{ccc}
\hline
\text{Paciente} & \text{Sin el fármaco} & \text{Con el fármaco} \\
\hline
    1     &        8.3        &         9.0         \\
    2     &        6.9        &         8.1         \\
    3     &        7.2        &         9.2         \\
    4     &        9.1        &         9.5         \\
    5     &        8.2        &         8.5         \\
    6     &        7.5        &         9.0         \\
    7     &        6.1        &         8.0         \\
    8     &        7.4        &         7.9         \\
\hline
\end{array}
\]
¿Se puede concluir con un nivel de confianza del 99\% que el fármaco ha sido eficaz para aumentar las horas de sueño?
}
%SOLUCIÓN
{El intervalo de confianza de 99\% para la media de la diferencia entre las horas de sueño sin el fármaco y con el fármaco es
$(-1.9067\,,\,-0.2183)$, luego si se puede concluir que el fármaco ha sido eficaz para aumentar las horas de sueño.
}
%RESOLUCIÓN
{Para ver si el fármaco ha sido eficaz necesitamos calcular el intervalo de confianza para la media de la diferencia entre horas de sueño
sin y con el fármaco. Al tratarse de datos pareados, construimos una nueva variable $X$ que recoja esta diferencia.
\[
\begin{array}{cccc}
\hline
\text{Paciente} & \text{Sin el fármaco} & \text{Con el fármaco} & X=\text{Diferencia} \\
\hline 
    1     &        8.3        &         9.0         & -0.7\\
    2     &        6.9        &         8.1         & -1.2\\
    3     &        7.2        &         9.2         & -2.0\\
    4     &        9.1        &         9.5         & -0.4\\
    5     &        8.2        &         8.5         & -0.3\\
    6     &        7.5        &         9.0         & -1.5\\
    7     &        6.1        &         8.0         & -1.9\\
    8     &        7.4        &         7.9         & -0.5\\
\hline
\end{array}
\]

Puesto que tenemos una muestra pequeña (8 pacientes), debemos utilizar el intervalo de confianza de la t de Student, cuya fórmula es
\[
\bar x\pm t(n-1)_{\alpha/2}\frac{\hat s}{\sqrt{n}}.
\]
Calculamos primero los estadísticos necesarios para construir el intervalo
\begin{align*}
\bar x &= \frac{\sum x_{i}}{n}=\frac{-0.7-\cdots -0.5}{8}= \frac{-8.5}{8}=-1.0625,\\
s^2 &= \frac{\sum{x_{i}^2}}{n}-\bar x^2=\frac{(-0.7)^2+\cdots +(-0.5)^2}{8}-(-1.0625)^2= \frac{12.29}{8}-1.1289=0.4073,\\
\hat s^2 &= \frac{n}{n-1}\cdot s^2= \frac{8}{7}\cdot 0.4073 = 0.4655,\\
\hat s &= \sqrt{0.4655}=0.6823.
\end{align*}
Como nos piden una confianza del 99\%, $\alpha=0.01$ y el valor de $t(n-1)_{\alpha/2}=t(7)_{0.005}$ lo buscamos en la tabla de la función de
distribución de la t de Student con 7 grados de libertad. Dicho valor es $t(7)_{0.005}=3.499$.

Sustituyendo estos valores en la fórmula del intervalo de confianza, tenemos
\[
-1.0625\pm 3.499\frac{0.6823}{\sqrt{8}}= -1.0625\pm 0.8442,
\]
con lo que el intervalo de confianza resultante es
$(-1.9067\,,\,-0.2183)$.

Según este intervalo, al 99\% de confianza existen pruebas significativas de que la media de la diferencia de horas es negativa y por tanto
el número de horas de sueño después del fármaco aumenta por término medio. Podemos concluir, con un 99\% de confianza, que el fármaco es
eficaz.
}


\newproblem{ico-40}{med}{*}
% ENUNCIADO
{Se quiere probar si la cirrosis hepática hace variar el índice de colinesterasa en suero. Se eligen 2 muestras aleatorias e independientes,
una primera de 60 individuos normales, con media $1.6$ y desviación típica $0.3$, y la segunda de 50 individuos cirróticos, con media $1.1$
y desviación típica $0.4$. ¿Podemos concluir que existen diferencias significativas, con un 99\% de confianza, entre las medias de la
colinesterasa en individuos normales e individuos cirróticos?
}
% SOLUCIÓN
{
}
% RESOLUCIÓN
{
}


\newproblem{ico-41}{med}{*}
% ENUNCIADO
{Un grupo de investigadores obtuvo datos acerca de las concentraciones de amilasa en el suero de muestras de individuos sanos y de
individuos hospitalizados, con el objetivo de determinar si la concentración media es, o no, diferente en ambas poblaciones. Las
concentraciones, en unidades/ml, en 10 individuos sanos fueron:
\begin{center}
\begin{tabular}{llllllllll}
100 & 103 & 96 & 93 & 91 & 104 & 93 & 99 & 88 & 91 \\
\end{tabular}
\end{center}
Y en 12 individuos enfermos fueron:
\begin{center}
\begin{tabular}{llllllllllll}
118 & 115 & 101 & 104 & 116 & 114 & 112 & 113 & 117 & 123&119&121 \\
\end{tabular}
\end{center}

Suponiendo que la concentración de amilasa en suero sigue una distribución normal, tanto en individuos sanos como hospitalizados, y que las
varianzas son desconocidas pero iguales, se pide:
\begin{enumerate}
\item Calcular el intervalo de confianza para la diferencia de medias con un nivel de confianza del 95\%.
\item ¿A qué conclusión deben llegar los investigadores sobre la igualdad o no de la concentración de amilasa? Justificar la respuesta.
\end{enumerate}
}
% SOLUCIÓN
{
}
% RESOLUCIÓN
{
}


\newproblem{ico-42}{med}{*}
% ENUNCIADO
{En un estudio se pretende contrastar el efecto de una vacuna contra la alergia. Para ello, se dispone de un grupo experimental formado por
200 individuos y otro control formado por 300 individuos.
Teniendo en cuenta que en el grupo experimental sufrieron alergia 20 personas, con una duración media de 20 días y una desviación típica de
6, y en el control la sufrieron 42, con una duración media de 25 días y una desviación típica de 7.
Se pide:
\begin{enumerate}
\item Calcular el intervalo de confianza al 90\% para la diferencia de proporciones de los afectados por la alergia entre los que han sido
vacunados y los que no. ¿Se puede concluir que la vacuna ha disminuido significativamente la proporción de afectados por la alergia? Justificar adecuadamente la respuesta. 
\item Suponiendo varianzas poblacionales iguales y que la duración de la alergia sigue una distribución normal, tanto en el grupo
experimental como en el grupo control, calcular el intervalo de confianza del 95\% para la diferencia entre la duración media de la alergia
en el grupo experimental y en el grupo control. ¿Se puede concluir que la vacuna ha disminuido significativamente la duración de la alergia? Justificar adecuadamente la respuesta.  
\end{enumerate}
}
% SOLUCIÓN
{
}
% RESOLUCIÓN
{
}

