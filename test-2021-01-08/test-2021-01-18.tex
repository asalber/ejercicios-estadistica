% Author: Alfredo Sánchez Alberca (asalber@ceu.es}
% !TEX program = xelatex
\documentclass[aspectratio=169,10pt,t]{beamer}
%-------------------------------------------------------------------------------
% GENERAL PACKAGES
%-------------------------------------------------------------------------------
% Language
\usepackage{polyglossia}
\setmainlanguage{spanish}
% Maths
\usepackage{amsmath} % Math symbols and environments
\usepackage{amsfonts}
\usepackage{amssymb}
% Tables
\usepackage{array}
\usepackage{multirow}
\usepackage{booktabs}

% Graphics
\usepackage{graphicx}
\usepackage{tikz}
\usetikzlibrary{positioning}

% Colors
\definecolor{blueceu}{RGB}{0,164,227}
\definecolor{greenceu}{RGB}{194,205,24}
\definecolor{redceu}{RGB}{238,50,36}
\definecolor{purpleceu}{RGB}{169,78,145}
\definecolor{greyceu}{RGB}{117,117,97}
\definecolor{darkgrey}{RGB}{40,40,50}
\definecolor{softblueceu}{RGB}{193,225,246}
\setbeamercolor{structure}{fg=blueceu}
\setbeamercolor{normal text}{fg=darkgrey}
\hypersetup{colorlinks, urlcolor=purpleceu}

% Boxes
\usepackage[most]{tcolorbox}
\usepackage{setspace}
\newtcolorbox{datos}{
  enhanced,
  colback=blueceu!10, 
  colframe=blueceu, 
  fonttitle=\bfseries, 
  left=3pt, 
  right=3pt, 
  boxrule=0.5pt,
  code={\setstretch{1.2}},
  title={Datos},
}

%-------------------------------------------------------------------------------
% FONTS
%-------------------------------------------------------------------------------
\usepackage{fontspec}
\setmainfont[Ligatures=TeX]{TeX Gyre Pagella}
\usepackage{unicode-math}
\setmathfont[math-style=ISO, bold-style=ISO]{TeX Gyre Pagella Math}
% Creative common icons
\usepackage[
    type={CC},
    modifier={by-nc-sa},
    version={3.0},
    imagemodifier={-eu}
]{doclicense}

%-------------------------------------------------------------------------------
% CONFIGURATION
%-------------------------------------------------------------------------------
\setbeamersize{text margin left=.5cm, text margin right=.5cm} % Defines margin sizes
\beamertemplatenavigationsymbolsempty % Hide navitation bar
\usefonttheme[onlymath]{serif} % Math text in serif
\setbeamertemplate{blocks}[rounded] % Blocks with rounded corners
%\setbeamercolor{block title}{bg=RoyalBlue!10} % Color of block title
%\setbeamercolor{block body}{bg=RoyalBlue!10} % Color of block body

%-------------------------------------------------------------------------------
% COMMANDS
%-------------------------------------------------------------------------------
% \newcommand{\sen}{\operatorname{sen}}
% \newcommand{\tg}{\operatorname{tg}}
% \newcommand{\arcsen}{\operatorname{arcsen}}
% \newcommand{\arctg}{\operatorname{arctg}}

%-------------------------------------------------------------------------------
% DOCUMENT
%-------------------------------------------------------------------------------
\begin{document}
%---------------------------------------------------------------------SLIDE----
\begin{frame}[c]
\vspace{1.5cm}

\begin{center}
\structure{\LARGE {\textbf{Ejercicios de Cálculo}}}
\bigskip

\large
\begin{tabular}{rl}
Temas: & \structure{Ecuaciones diferenciales ordinarias}\\
Titulaciones: & \structure{Farmacia y Biotecnología}
\end{tabular}

\bigskip
Alfredo Sánchez Alberca\\
\url{asalber@ceu.es}\\
\url{https://aprendeconalf.es}\\

\includegraphics[scale=0.2]{../img/logo_uspceu}

\bigskip
\doclicenseIcon
\end{center}
\end{frame}

%---------------------------------------------------------------------SLIDE----
\begin{frame}[c]
\large
Un test para diagnosticar el cáncer de próstata produce un 1\% de falsos positivos y un 0.2\% de falsos negativos. Se sabe también que una población 1 cada 400 hombres sufre este tipo de cáncer.
\begin{enumerate}
\item Calcular la sensibilidad y la especificidad del test.
\item Si un hombre tiene un resultado positivo en el test, ¿cuál es la probabilidad de que tenga cáncer de próstata?
\item Calcular e interpretar el valor negativo predictivo del test.
\item ¿Es este test mejor para detectar o para descartar el cáncer de próstata?
\item Para ver si existe asociación entre el cáncer de próstata y la práctica del deporte, se tomó una muestra de 1000 hombres, de los cuales 700 practicaban deporte, y se observó que había 2 hombres con cáncer de próstata en el grupo de los que practicaban deporte y 3 hombres con cáncer de próstata en el grupo de los que no practicaban deporte. Calcular el riesgo relativo y el odds ratio de sufrir cáncer de próstata cuando se practica deporte. 
\end{enumerate}
\end{frame}

%------------------------------------------------------------------SLIDE----
\begin{frame}
\begin{columns}
\begin{column}[T]{0.70\textwidth}
Un test para diagnosticar el cáncer de próstata produce un 1\% de falsos positivos y un 0.2\% de falsos negativos. Se sabe también que una población 1 cada 400 hombres sufre este tipo de cáncer.
\begin{enumerate}
\item Calcular la sensibilidad y la especificidad del test.
\end{enumerate}
\end{column}
\quad
\begin{column}[T]{0.30\textwidth}
\begin{datos}
Prevalencia: $P(C)=1/400$\\
FP: 1\%\\
FN: 0.2\%
\end{datos}
\end{column}
\end{columns}
\end{frame}

%------------------------------------------------------------------SLIDE----
\begin{frame}
\begin{columns}
\begin{column}[T]{0.70\textwidth}
\begin{enumerate}
\item[2] Si un hombre tiene un resultado positivo en el test, ¿cuál es la probabilidad de que tenga cáncer de próstata?
\end{enumerate}
\end{column}
\begin{column}[T]{0.30\textwidth}
\begin{datos}
Prevalencia: $P(C)=1/400$\\
$P(FP) = 0.01$\\
$P(FN) = 0.002$\\
$P(VP) = 0.0005$\\
$P(VN) = 0.9875$
\end{datos}
\end{column}
\end{columns}
\end{frame}

%------------------------------------------------------------------SLIDE----
\begin{frame}
\begin{columns}
\begin{column}[T]{0.70\textwidth}
\begin{enumerate}
\item[3] Calcular e interpretar el valor negativo predictivo del test.
\end{enumerate}
\end{column}
\begin{column}[T]{0.30\textwidth}
\begin{datos}
Prevalencia: $P(C)=1/400$\\
$P(FP) = 0.01$\\
$P(FN) = 0.002$\\
$P(VP) = 0.0005$\\
$P(VN) = 0.9875$
\end{datos}
\end{column}
\end{columns}
\end{frame}

%------------------------------------------------------------------SLIDE----
\begin{frame}
\begin{columns}
\begin{column}[T]{0.70\textwidth}
\begin{enumerate}
\item[4] ¿Es este test mejor para detectar o para descartar el cáncer de próstata?
\end{enumerate}
\end{column}
\begin{column}[T]{0.30\textwidth}
\begin{datos}
VPP: $P(C|+) = 0.0476$\\
VPN: $P(\overline C|-) = 0.998$
\end{datos}
\end{column}
\end{columns}
\end{frame}

%------------------------------------------------------------------SLIDE----
\begin{frame}
\begin{columns}
\begin{column}[T]{0.70\textwidth}
\begin{enumerate}
\item[5] Para ver si existe asociación entre el cáncer de próstata y la práctica del deporte, se tomó una muestra de 1000 hombres, de los cuales 700 practicaban deporte, y se observó que había 2 hombres con cáncer de próstata en el grupo de los que practicaban deporte y 3 hombres con cáncer de próstata en el grupo de los que no practicaban deporte. Calcular el riesgo relativo y el odds ratio de sufrir cáncer de próstata cuando se practica deporte. 
\end{enumerate}
\end{column}
\begin{column}[T]{0.30\textwidth}
\begin{datos}
Grupo Tratamiento:\\
$n=700$ y $n(C)=2$\\
Grupo Control:\\
$n=300$ y $n(C)=3$
\end{datos}
\end{column}
\end{columns}
\end{frame}
\end{document}