% Author Alfredo Sánchez Alberca (asalber@ceu.es)

\newproblem{vad-1}{gen}{}
%ENUNCIADO
{Sea $X$ una variable aleatoria discreta cuya ley de probabilidad es
\[
\begin{array}{|c|c|c|c|c|c|}
\hline
X & 4 & 5 & 6 & 7 & 8 \\ 
\hline
f(x) & 0.15 & 0.35 & 0.10 & 0.25 & 0.15 \\ 
\hline
\end{array}
\]
\begin{enumerate}
\item  Calcular y representar gráficamente la función de distribución.
\item  Obtener:
\begin{enumerate}
\item  $P(X<7.5)$.
\item  $P(X>8)$.
\item  $P(4\leq X\leq 6.5)$.
\item  $P(5<X<6)$.
\end{enumerate}
\end{enumerate}
}
%SOLUCIÓN
{
\begin{enumerate}
\item \[
F(x)=
\begin{cases}
0 & \text{si $x<4$,}\\
0.15 & \text{si $4\leq x<5$,}\\
0.5 & \text{si $5\leq x<6$,}\\
0.6 & \text{si $6\leq x<7$,}\\
0.85 & \text{si $7\leq x<8$,}\\
1 & \text{si $8\leq x$.}
\end{cases}
\]
\item $P(X<7.5)=0.85$, $P(X>8)=0$, $P(4\leq x\leq 6.5)=0.6$ y $P(5<X<6)=0$.
\end{enumerate}
}
%RESOLUCIÓN
{}


\newproblem{vad-2}{gen}{}
%ENUNCIADO
{Sea la variable aleatoria X con la siguiente función de distribución:
\[
F(x)=
\begin{cases}
0 & \text{si $x<1$,} \\
1/5 & \text{si $1\leq x< 4$,} \\
3/4 & \text{si $4\leq x<6$,} \\
1 & \text{si $6\leq x$.}
\end{cases}
\]
Se pide:
\begin{enumerate}
\item Distribución de probabilidad.
\item Calcular la siguientes probabilidades:
\begin{enumerate}
\item $P(X=6)$.
\item $P(X=5)$.
\item $P(2<X<5.5)$.
\item $P(0\leq X<4)$.
\end{enumerate}
\item Calcular la media.
\item Calcular la desviación típica. 
\end{enumerate}
}
%SOLUCIÓN
{
\begin{enumerate}
\item \[
\begin{array}{|c|c|c|c|}
\hline
X & 1 & 4 & 6 \\
\hline
f(x) & 1/5 & 11/20 & 1/4\\
\hline
\end{array}
\]
\item $P(X=6)=1/4$, $P(X=5)=0$, $P(2<X<5.5)=11/20$ y $P(0\leq X<4)=1/5$.
\item $\mu=3.9$.
\item $\sigma=1.6703$.
\end{enumerate}
}
%RESOLUCIÓN
{}


\newproblem{vad-3}{med}{*}
%ENUNCIADO
{Se realiza un experimento aleatorio consistente en inyectar un virus a tres tipos de ratas y observar si sobreviven o
no.
Se sabe que la probabilidad de que viva el primer tipo de rata es $0.5$, la de que viva el segundo es $0.4$ y la de
que viva el tercero $0.3$.
Se pide:
\begin{enumerate}
\item Construir la variable aleatoria que mida el número de ratas vivas y su función de probabilidad.
\item Calcular la función de distribución.
\item Calcular $P(X\leq 1)$, $P(X\geq 2)$ y $P(X=1.5)$.
\item Calcular la media y la desviación típica. ¿Es representativa la media?
\end{enumerate}
}
%SOLUCIÓN
{
\begin{enumerate}
\item \[
\begin{array}{|c|c|c|c|c|}
\hline
X & 0 & 1 & 2 & 3 \\
\hline
f(x) & 0.21 & 0.44 & 0.29 & 0.06\\
\hline
\end{array}
\]
\item \[
F(x)=
\begin{cases}
0 & \text{si $x<0$,}\\
0.21 & \text{si $0\leq x<1$,}\\
0.65 & \text{si $1\leq x<2$,}\\
0.94 & \text{si $2\leq x<3$,}\\
1 & \text{si $3\leq x$.}
\end{cases}
\]
\item $P(X\leq 1)=0.65$, $P(X\geq 2)=0.35$ y $P(X=1.5)=0$.
\item $\mu=1.2$ ratas, $\sigma^2=0.7$ ratas$^2$ y $\sigma=0.84$ ratas.
\end{enumerate}
}
%RESOLUCIÓN
{}


\newproblem*{vad-4}{gen}{}
%ENUNCIADO
{Una tómbola asegura que en cada 1000 boletos hay 500 con ``siga intentándolo'', 100 con un premio de 1\euro, 60 con un premio de 2\euro, 30 con un premio de 3\euro, y 10 con un premio de 10\euro. Un individuo decide comprar un boleto que cuesta 1\euro. Se pide:

\begin{enumerate}
\item Construir una variable aleatoria que mida la ganancia (o pérdida) con la compra de un boleto.
\item ¿Cual es la probabilidad de que pierda dinero?
\item ¿Qué ganancia espera obtener?
\end{enumerate}
}
%SOLUCIÓN
{}
%RESOLUCIÓN
{}


\newproblem{vad-5}{med}{}
%ENUNCIADO
{La probabilidad de curación de un paciente al ser sometido a un determinado tratamiento es $0.85$.
Calcular la probabilidad de que en un grupo de $6$ enfermos sometidos a tratamiento:
\begin{enumerate}
\item se curen la mitad.
\item se curen al menos $4$.
\end{enumerate}
}
%SOLUCIÓN
{Llamando $X$ al número de pacientes curados de los 6 sometidos al tratamiento, se tiene que $X\sim B(6,\,0.85)$.
\begin{enumerate}
\item $P(X=3)=0.041$.
\item $P(X\geq 4)=0.9526$.
\end{enumerate}
}
%RESOLUCIÓN
{}


\newproblem*{vad-6}{gen}{*}
%ENUNCIADO
{En una mesa de juego, en 1654, Meré propuso a Pascal la siguiente afirmación: ``es más probable obtener al menos un as con cuatro dados, que al menos un doble as en veinticuatro tiradas de dos dados''.

Demostrar que Meré tenía razón.
}
%SOLUCIÓN
{}
%RESOLUCIÓN
{}


\newproblem{vad-7}{med}{}
%ENUNCIADO
{Se sabe que la probabilidad de que aparezca una bacteria en un mm$^3$ de cierta disolución es de $0.002$.
Si en cada mm$^3$ a los sumo puede aparecer una bacteria, determinar la probabilidad de que en un cm$^3$ haya como
máximo $5$ bacterias.}
%SOLUCIÓN
{Llamando $X$ al número de bacterias en 1 cm$^3$ de disolución, se tiene $X\sim B(1000,\,0.002)\approx P(2)$.\\
$P(X\leq 5)=0.9834$.
}
%RESOLUCIÓN
{}


\newproblem{vad-8}{med}{}
%ENUNCIADO
{Diez individuos entran en contacto con un portador de tuberculosis. La probabilidad de que la enfermedad se contagie del portador a un sujeto cualquiera es $0.10$.
\begin{enumerate}
\item ¿Qué probabilidad hay de que ninguno se contagie?
\item ¿Qué probabilidad hay de que al menos dos se contagien?
\item ¿Cuántos se espera que contraigan la enfermedad?
\end{enumerate}
}
%SOLUCIÓN
{Llamando $X$ al número de personas contagiadas. 
\begin{enumerate}
\item $P(X=0)=0.3487$. 
\item $P(X\geq 2)=0.2639$.
\item $\mu=1$.
\end{enumerate}
}
%RESOLUCIÓN
{}


\newproblem*{vad-9}{gen}{*}
%ENUNCIADO
{Las matrículas de los coches constan de una parte numérica formada por cuatro cifras, y una parte literal. Se pide:

\begin{enumerate}
\item  Hallar la probabilidad de que al pasar 30 coches haya menos de dos cuya parte numérica sea capicúa.
\item  ¿Cuántos coches deben pasar para que la probabilidad de que alguno tenga la parte numérica capicúa sea mayor
que 0.1?
\end{enumerate}
}
%SOLUCIÓN
{}
%RESOLUCIÓN
{}


\newproblem{vad-10}{med}{}
%ENUNCIADO
{La probabilidad de que al administrar una vacuna dé una determinada reacción es $0.001$. Si se vacunan 2000 personas, ¿Cuál es la probabilidad de que aparezca alguna reacción adversa?
}
%SOLUCIÓN
{Llamando $X$ al número de reacciones adversas, $P(X\geq 1)=0.8648$.
}
%RESOLUCIÓN
{}


\newproblem{vad-11}{gen}{*}
%ENUNCIADO
{El número medio de llamadas por minuto que llegan a una centralita telefónica es igual a 120.
Hallar las probabilidades de los sucesos siguientes: 

\begin{enumerate}
\item  $A=\{\text{durante 2 segundos lleguen a la centralita menos de 4 llamadas}\}$.
\item  $B=\{\text{durante 3 segundos lleguen a la centralita 3 llamadas como mínimo}\}$.
\end{enumerate}
}
%SOLUCIÓN
{
\begin{enumerate}
\item Si $X$ es el número de llamadas en 2 segundos, entonces $X\sim P(4)$ y $P(X<4)=0.4335$.
\item Si $Y$ es el número de llamadas en 3 segundos, entonces $Y\sim P(6)$ y $P(Y\geq 3)=0.938$.
\end{enumerate}
}
%RESOLUCIÓN
{}


\newproblem*{vad-12}{nut}{}
%ENUNCIADO
{En un laboratorio de análisis de alimentos se sabe, de experiencias anteriores, que la probabilidad de que una muestra de café contenga plomo en cantidades superiores a las permitidas por la legislación vigente es $0.2$. Si se reciben $50$ muestras de café, ¿cuál es la probabilidad de rechazar al menos $5$?
¿Cuál será el número medio de muestras que rechazaremos?
}
%SOLUCIÓN
{}
%RESOLUCIÓN
{}


\newproblem{vad-13}{far}{}
%ENUNCIADO
{Un proceso de fabricación de un fármaco produce por término medio 6 fármacos defectuosos por hora.
¿Cuál es probabilidad de que en un hora se produzcan menos de 3 fármacos defectuosos?
¿Y la de que en la próxima media hora se produzcan más de un fármaco defectuoso?}
%SOLUCIÓN
{Llamando $X$ al número de fármacos defectuosos en 1 hora, se tiene $X\sim P(6)$ y $P(X<3)=0.062$.\\
Llamando $Y$ al número de fármacos defectuosos en $1/2$ hora, se tiene $Y\sim P(3)$ y $P(Y>1)=0.8009$.}
%RESOLUCIÓN
{}


\newproblem{vad-14}{gen}{}
%ENUNCIADO
{Una mecanógrafa comete, en promedio, una errata cada 2000 caracteres que escribe.
Suponiendo que escribe un folio con treinta líneas y setenta caracteres por línea, ¿cuál es la probabilidad de que
cometa más de un error en dicho folio?}
%SOLUCIÓN
{Llamando $X$ al número de errores en un folio, se tiene que $X\sim B(2100,\,1/2000)\approx P(1.05)$, y $P(X>1)=0.2826$.
}
%RESOLUCIÓN
{}


\newproblem{vad-15}{gen}{}
%ENUNCIADO
{Un examen de tipo test consta de 10 preguntas con tres respuestas posibles para cada una de ellas.
Se obtiene un punto por cada respuesta acertada y se pierde medio punto por cada pregunta fallada.
Un alumno sabe tres de las preguntas del test y las contesta correctamente, pero no sabe las otras siete y las contesta
al azar.
¿Qué probabilidad tiene de aprobar el examen?}
%SOLUCIÓN
{Llamando $X$ al número de preguntas acertadas de las 7 contestadas al azar, se tiene $X\sim B(7,\,1/3)$ y $P(X\geq
4)=0.1733$.}
%RESOLUCIÓN
{}


\newproblem*{vad-16}{gen}{*}
%ENUNCIADO
{Un equipo de fútbol tiene 7 delanteros. Se sabe que por término medio, cada delantero se pierde 5 partidos por lesión en una temporada de 40 partidos. Suponiendo que todos los delanteros tienen la misma probabilidad de lesionarse, se pide:
\begin{enumerate}
\item  ¿Cuál es la probabilidad de que en un partido determinado tenga menos de 5 delanteros en condiciones de jugar?
\item  ¿Cuál es la probabilidad de que a lo largo de la temporada haya más de un partido en que tenga menos de 5  delanteros en condiciones de jugar?
\end{enumerate}
}
%SOLUCIÓN
{}
%RESOLUCIÓN
{}


\newproblem*{vad-17}{amb}{}
%ENUNCIADO
{Para reforestar un bosque se compran árboles a un vivero en el que el 4\% de los árboles suele morir debido a una enfermedad. Si la repoblación se efectúa por parcelas en las que se ponen 10 árboles, se pide:
\begin{enumerate}
\item Calcular la probabilidad de que no muera ningún árbol en una parcela.
\item Calcular la probabilidad de no mueran más de 2 árboles en una parcela.
\item Si en total se reforestan 3000 parcelas, ¿cuál es la probabilidad de que haya alguna en la que mueran más de dos árboles?
\end{enumerate}
}
%SOLUCIÓN
{}
%RESOLUCIÓN
{}


\newproblem{vad-18}{med}{*}
%ENUNCIADO
{En un estudio sobre un determinado tipo de parásito que ataca el riñón de las ratas, se sabe que el número medio de
parásitos en cada riñón es 3.
Se pide:
\begin{enumerate}
\item  Calcular la probabilidad de que una rata tenga más de 8 parásitos.\\
(Nota: se supone que una rata normal tiene dos riñones).
\item  Si se tienen 10 ratas, ¿cuál es la probabilidad de que haya al menos 9 con parásitos?
\end{enumerate}
}
%SOLUCIÓN
{
\begin{enumerate}
\item Si $X$ es el número de parásitos en una rata, $X\sim P(6)$ y $P(X>8)=0.1528$.
\item Si $Y$ es el número de ratas con parásitos en un grupo de 10 ratas, entonces $Y\sim B(10,\,0.9975)$ y $P(Y\geq
9)=0.9997$.
\end{enumerate}
}
%RESOLUCIÓN
{}


\newproblem{vad-19}{med}{*}
% ENUNCIADO
{Se ha comprobado experimentalmente que una de cada 20 billones de células expuestas a un determinado tipo de radiación muta volviéndose
cancerígena. Sabiendo que el cuerpo humano tiene aproximadamente 1 billón de células por kilogramo de tejido, calcular la probabilidad de
que una persona de 60 kg expuesta a dicha radiación desarrolle cáncer. Si la radiación ha afectado a 3 personas de 60 kg, ¿cuál es la
probabilidad de que desarrolle el cáncer más de una? }
% SOLUCIÓN
{Llamando $X$ al número de mutaciones, $P(X>0)=0.9502$.\\ 
Llamando $Y$ al número de personas que desarrollan el cáncer, $P(Y\geq 1)=0.9999$.
}
% RESOLUCIÓN
{}


\newproblem{vad-20}{med}{*}
%ENUNCIADO
{En un servicio de urgencias de cierto hospital se sabe que, en media, llegan 2 pacientes a la hora.
Calcular:
\begin{enumerate}
\item Si los turnos en urgencias son de 8 horas, ¿cuál será la probabilidad de que en un turno lleguen más de 5 pacientes?
\item Si el servicio de urgencias tiene capacidad para atender adecuadamente como mucho a 4 pacientes a la hora, ¿cuál
es la probabilidad de que a lo largo de un turno de 8 horas el servicio de urgencias se vea desbordado en alguna de las
horas del turno?
\end{enumerate}
}
%SOLUCIÓN
{
\begin{enumerate}
\item Llamando $X$ al número de pacientes en un turno, se tiene que $X\sim P(16)$ y $P(X>5)=0.9986$.
\item Llamando $Y$ al número de horas en el que el servicio se vea desbordado porque lleguen más de 4 pacientes, se
tiene que $Y\sim B(8,\,0.0527)$ y $P(Y\geq 1)=0.3515$.
\end{enumerate}
}
%RESOLUCIÓN
{}


\newproblem*{vad-21}{amb}{}
%ENUNCIADO
{En un Parque Nacional se contabilizan 15 linces. Si sabemos, por estudios previos, que mueren en promedio 1 de cada 10 individuos a lo largo de un año (ya sea por accidentes, caza de furtivos o por causas naturales):
\begin{enumerate}
\item ¿Cuál es la probabilidad de que en el Parque Nacional se contabilicen más de 2 muertes de linces en un año?
\item Suponiendo un periodo de 12 años, ¿cuál es la probabilidad de que en el Parque Nacional haya algún año
en el que mueran 2 linces?
\end{enumerate}
}
%SOLUCIÓN
{}
%RESOLUCIÓN
{}


\newproblem{vad-22}{med}{*}
%ENUNCIADO
{El síndrome de Turner es una anomalía genética que se caracteriza porque las mujeres tienen sólo un cromosoma $X$.
Afecta aproximadamente a 1 de cada 2000 mujeres.
Además, aproximadamente 1 de cada 10 mujeres con síndrome de Turner, como consecuencia, también sufren un
estrechamiento anormal de la aorta.
Se pide:
\begin{enumerate}
\item En un grupo de 4000 mujeres, ¿cuál es la probabilidad de que haya más de 3 afectadas por el síndrome de Turner?
¿Y de que haya alguna con estrechamiento de aorta como consecuencia de padecer el síndrome de Turner?
\item En un grupo de 20 chicas afectadas por el síndrome de Turner, ¿cuál es la probabilidad de que menos de 3 sufran
un estrechamiento anormal de la aorta?
\end{enumerate}
}
%SOLUCIÓN
{
\begin{enumerate}
\item Si $X$ es el número de mujeres afectadas por el síndrome de Turner en el grupo de 4000 mujeres, entonces $X\sim
B(4000,\,1/2000)\approx P(2)$ y $P(X>3)=0.1429$.\\
Si $Y$ es el número de mujeres con estrechamiento de la aorta en el grupo de 4000 mujeres, entonces $Y\sim
B(4000,\,1/20000)\approx P(0.2)$ y $P(Y>0)=0.1813$.
\item Si $Z$ es el número de mujeres con estrechamiento de la aorta en el grupo de 20 mujeres con el síndrome de
Turner, entonces $Z\sim B(20,\,1/10)$ y $P(Z<3)=0.6769$.
\end{enumerate}
}
%RESOLUCIÓN
{}


\newproblem{vad-23}{amb}{}
%ENUNCIADO
{Por estudios previos se sabe que, en una comarca, hay dos tipos de larvas que parasitan, de forma completamente
independiente, los chopos, y que producen su muerte.
Si la larva de tipo $A$ está parasitando un 15\% de los chopos, y la $B$ un 30\%, y en una zona concreta de la comarca
hay 10 chopos:
\begin{enumerate}
\item ¿Qué probabilidad hay que de estén siendo parasitados por $A$ más de dos?
\item ¿Qué probabilidad hay de que estén libres de $B$ más de 8?
\item ¿Qué probabilidad hay de que más de 1 tenga los dos tipos de larva?
\item ¿Qué probabilidad hay de que más de 3 tengan algún tipo de larva?
\end{enumerate}
}
%SOLUCIÓN
{
\begin{enumerate}
\item $X_A$ es el número de chopos parasitados por larvas del tipo $A$, entonces $X_A\sim B(10,\,0.15)$ y
$P(X_A>2)=0.1798$.
\item Si $X_{\overline{B}}$ es el número de chopos no parasitados por larvas del tipo $B$, entonces
$X_{\overline{B}}\sim B(10,\,0.7)$ y $P(X_{\overline{B}}>8)=0.1493$.
\item Si llamamos $X_{A\cap B}$ al número de chopos parasitados por larvas de ambos tipos, entonces $X_{A\cap B}\sim
B(10,\,0.045)$ y $P(X_{A\cap B}>1)=0.0717$.
\item Si llamamos $X_{A\cup B}$ al número de chopos parasitados por algún tipo de larva, entonces $X_{A\cup B}\sim
B(10,\,0.405)$ y $P(X_{A\cup B}>3)=0.6302$.
\end{enumerate}
}
%RESOLUCIÓN
{}


\newproblem*{vad-24}{amb}{*}
%ENUNCIADO
{En un Parque Regional se contabilizan 10 parejas de buitre leonado. Si sabemos que el 70\% de las parejas de esta especie logran que alguna de sus crías sobreviva:
\begin{enumerate}
\item ¿Cuál es la probabilidad de que 8 parejas de buitre leonado del Parque logren que alguna de sus crías sobreviva?
\item ¿Cuál es la probabilidad de que alguna pareja logre que alguna de sus crías sobreviva?
\item Si sabemos que en dicho Parque Regional nidifican, en promedio, 8 parejas al año, ¿cuál es la probabilidad de que en un año concreto nidifiquen más de 6?
\end{enumerate}
}
%SOLUCIÓN
{}
%RESOLUCIÓN
{}


\newproblem*{vad-25}{gen}{}
%ENUNCIADO
{Al lanzar 100 veces una moneda, ¿cuál es la probabilidad de obtener entre 40 y 60 caras?
}
%SOLUCIÓN
{}
%RESOLUCIÓN
{}


\newproblem{vad-26}{psi}{}
%ENUNCIADO
{El trastorno de pánico aparece en 1 de cada 75 personas.
¿Cuál es la probabilidad de que en un grupo de 100 personas aparezca alguna con trastorno de pánico?
¿Cuál es el número esperado de personas con trastorno de pánico en ese grupo?
}
%SOLUCIÓN
{Llamando $X$ al número de personas que sufren trastorno del pánico en el grupo de 100 personas, se tiene que $X\sim
B(100,\,1/75)$ y $P(X\geq 1)=0.7379$.}
%RESOLUCIÓN
{}


\newproblem{vad-27}{med}{}
%ENUNCIADO
{Se sabe que 2 de cada 1000 pacientes son alérgicos a un fármaco $A$, y que 6 de cada 1000 lo son a un fármaco $B$.
Además, el 30\% de los alérgicos a $B$, también lo son a $A$.
Si se aplican los dos fármacos a 500 personas,
\begin{enumerate}
\item ¿Cuál es la probabilidad de que no haya ninguna con alergia a $A$?
\item ¿Cuál es la probabilidad de que haya al menos 2 con alergia a $B$?
\item ¿Cuál es la probabilidad de que haya menos de 2 con las dos alergias?
\item ¿Cuál es la probabilidad de que haya alguna con alergia?
\end{enumerate}
}
%SOLUCIÓN
{
\begin{enumerate}
\item Llamando $X_A$ al número de personas alérgicas al fármaco $A$ en el grupo de 500 personas, se tiene que $X_A\sim
B(500,\,0.002)\approx P(1)$ y $P(X_A=0)=0.3678$.
\item Llamando $X_B$ al número de personas alérgicas al fármaco $B$ en el grupo de 500 personas, se tiene que $X_B\sim
B(500,\,0.006)\approx P(3)$ y $P(X_B\geq 2)=0.8009$.
\item Llamando $X_{A\cap B}$ al número de personas alérgicas a ambos fármacos $A\cap B$ en el grupo de 500 personas, se
tiene que $X_A\sim B(500,\,0.0018)\approx P(0.9)$ y $P(X_{A\cap B}<2)=0.7725$.
\item Llamando $X_{A\cup B}$ al número de personas alérgicas a alguno de los fármacos $A\cup B$ en el grupo de 500
personas, se tiene que $X_A\sim B(500,\,0.0062)\approx P(3.1)$ y $P(X_{A\cup B}\geq 1)=0.9550$.
\end{enumerate}
}
%RESOLUCIÓN
{}


\newproblem{vad-28}{gen}{}
%ENUNCIADO
{En una clase hay 40 alumnos de los cuales el 35\% son fumadores.
Si se toma una muestra aleatoria con reemplazamiento de 4 alumnos, ¿cuál es la probabilidad de que haya al menos 1
fumador?
¿Cuál sería dicha probabilidad si la muestra se hubiese tomado sin reemplazamiento?}
%SOLUCIÓN
{Si $X$ es el número de fumadores en una muestra aleatoria con reemplazamiento de tamaño $4$, entonces $X\sim
B(4,\,0.35)$ y $P(X\geq 1)=0.8215$.\\
Si la muestra es sin reemplazamiento $P(X\geq 1)= 0.8364$.}
%RESOLUCIÓN
{}


\newproblem{vad-29}{med}{*}
%ENUNCIADO
{Se sabe que por término medio 2 de cada 10000 niños que nacen son albinos.
\begin{enumerate}
\item Si en una región nacen cada año 22000 niños ¿cuál es la probabilidad de que un año nazcan al menos 4 albinos?
\item ¿Cuál es la probabilidad de que en esa región, en un periodo de 10 años no nazca ningún niño albino?
\end{enumerate}
}
%SOLUCIÓN
{
\begin{enumerate}
\item Llamando $X$ al número de niños albinos que nacen en un año, se tiene que $X\sim B(22000,\,2/10000)\approx
P(4.4)$ y $P(X\geq 4)=0.6406$.
\item Llamando $Y$ al número de niños albinos que nacen en 10 años, se tiene que $Y\sim B(220000,\,2/10000)\approx
P(44)$ y $P(Y=0)=0$.
\end{enumerate}
}
%RESOLUCIÓN
{}


\newproblem{vad-30}{med}{*}
%ENUNCIADO
{Suponiendo una facultad en la que hay un 60\% de chicas y un 40\% de chicos:
\begin{enumerate}
\item  Si un año van 6 alumnos a hacer prácticas en un hospital, ¿qué probabilidad hay de que vayan más chicos que chicas?
\item En un período de 5 años, ¿cuál es la probabilidad de que más de 1 año no haya ido ningún chico?
\end{enumerate}
}
%SOLUCIÓN
{\begin{enumerate}
\item Si $X$ es el número de chicos, $X\sim B(6,\,0.4)$  y $P(X\geq 4)=0.1792$.
\item Si $Y$ es el número de años que no ha ido ningún chico, $Y \sim B(5,\,0.0467)$ y $P(Y>1)=0.0199$.
\end{enumerate}
}
%RESOLUCIÓN
{\begin{enumerate}
\item Si consideramos un total de $n=6$ alumnos, con un $60\%$ de chicas y un $40\%$ de chicos, la variable $X$ que es el número de alumnos chicos que van a hacer las prácticas de un total de 6, sigue una distribución binomial de 6 intentos y probabilidad de éxito igual a $0.4$: $X\sim B(6\,,\,0.4)$. Como nos piden la probabilidad de que haya más chicos que chicas, eso se consigue si el número de chicos es 4 o más. Por lo tanto, nos piden la probabilidad de que $X$ sea mayor o igual que 4:
\[
P(X \ge 4) = P(X = 4) + P(X = 5) + P(X = 6)
\]
Calculando estas probabilidades con la función de probabilidad de la variable binomial, tenemos
\begin{align*}
P(X = 4) &= \binom{6}{4}\cdot 0.4^4  \cdot (1-0.4)^{6-4}  = 12\cdot 0.4^4\cdot 0.6^2 = 0.1382,\\
P(X = 5) &= \binom{6}{5}\cdot 0.4^5  \cdot (1-0.4)^{6-5}  = 6\cdot 0.4^5 \cdot 0.6 = 0.0369,\\
P(X = 6) &= \binom{6}{6}\cdot 0.4^6  \cdot (1-0.4)^{6-6}  = 1\cdot 0.4^6 \cdot 0.6^0 = 0.0041
\end{align*}
Y sumando los tres resultados obtenidos:
\[
P(X \ge 4) = P(X = 4) + P(X = 5) + P(X = 6)= 0.1382+0.0369+0.0041=0.1792.
\]

\item Si consideramos, en un total de 5 años, la probabilidad de que más de un año no haya ido ningún chico, la variable a tener en cuenta $Y$ será el número de años en el total de 5 sin ningún chico, y de nuevo esta variable sigue una distribución binomial, esta vez con 5 intentos, cuyo éxito viene dado por la probabilidad de que en un año concreto no haya ningún chico: $Y \sim B(5\,,\,p)$, donde $p=P(X=0)$.
\[
p = P(X = 0) = \binom{6}{0}\cdot 0.4^0  \cdot (1-0.4)^{6-0}  = 1\cdot 0.4^0\cdot 0.6^6 = 0.0467.
\]
Por lo tanto, $Y \sim B(5\,,\,0.0467)$.

Además, nos piden la probabilidad de más de una año en el total de 5; es decir: 
\[
P(Y>1)=1-P(Y\leq 1) = 1-P(Y=0)-P(Y=1).
\]
Aplicando, de nuevo, la fórmula de la función de probabilidad de la binomial, obtenemos:
\begin{align*}
P(Y = 0) &= \binom{5}{0}\cdot 0.0467^0  \cdot (1 - 0.0467)^{5-0} = 1\cdot 0.0467^0\cdot 0.9533^5  = 0.7873,\\
P(Y = 1) &= \binom{5}{1}\cdot 0.0467^1  \cdot (1 - 0.0467)^{5-1} = 5\cdot 0.0467^1\cdot 0.9533^4  = 0.1928.
\end{align*}
Teniendo lo anterior en cuenta, la probabilidad que nos piden vale:
\[
P(Y>1)=1-P(Y=0)-P(Y=1)=1-0.7873-0.1928=0.0199.
\]
\end{enumerate}
}


\newproblem*{vad-31}{med}{*}
%ENUNCIADO
{La probabilidad de que en un grupo de 5 individuos mayores de 70 años todos padezcan arterioesclerosis cerebral es de $12,5$ por mil.
\begin{enumerate}
\item ¿Cuál es la probabilidad de padecer la enfermedad entre los mayores de 70 años?
\item En un grupo de 1000 personas, ¿cuál es la probabilidad de que padezcan la enfermedad más de 450?
\end{enumerate}
}
%SOLUCIÓN
{}
%RESOLUCIÓN
{}


\newproblem{vad-32}{gen}{*}
%ENUNCIADO
{¿Cuánto habría que restar a cada pregunta errada en un examen de tipo test de 5 preguntas con cuatro opciones y sólo
una correcta, para que un individuo que responda al azar tenga una puntuación esperada de 0? 
}
%SOLUCIÓN
{$1/3$.}
%RESOLUCIÓN
{Si llamamos $X$ al número de preguntas acertadas, está claro que $X$ sigue una distribución binomial $B(5,\,1/4)$ ya
que el examen tiene 4 preguntas, y la probabilidad de acertar cualquiera de ellas al azar es $1/4$ ya que hay cuatro
opciones y sólo una es la correcta.
}


\newproblem{vad-33}{psi}{}
%ENUNCIADO
{Se sabe que el $6.8$\% las personas presentan a lo largo de su adolescencia un trastorno de hiperactividad, de los
cuales tres cuartas partes son mujeres.
Si en la población hay el mismo número de hombres y mujeres, se pide:
\begin{enumerate}
\item Calcular la probabilidad de que en una muestra de tres hombres, haya alguno que haya tenido hiperactividad en su
adolescencia. 
\item Calcular la probabilidad de que en una muestra de 2 hombres y 2 mujeres, haya alguno que haya tenido
hiperactividad en su adolescencia. 
\end{enumerate}
}
%SOLUCIÓN
{
\begin{enumerate}
\item Si llamamos $X$ al número de hombres que han tenido hiperactividad en su adolescencia en una muestra de 3
hombres, se tiene que $X\sim B(3,\,0.034)$ y $P(X\geq 1)=0.0986$.
\item Si llamamos $X_H$ al número de hombres que han tenido hiperactividad en su adolescencia en una muestra de 2
hombres y $X_M$ al número de mujeres que han tenido hiperactividad en su adolescencia en una muestra de 2 mujeres,
entonces $X_H\sim B(2,\,0.034)$ y $X_M(2,\,0.102)$. Entonces $P(X_H\geq 1\cup X_M\geq 1)=0.2475$.
\end{enumerate}
}
%RESOLUCIÓN
{}


\newproblem{vad-34}{gen}{*}
%ENUNCIADO
{A un hospital llegan pacientes por la mañana a efectuarse extracciones de sangre. Se ha medido la frecuencia de llegada de los mismos en
intervalos de 15 minutos. La distribución de probabilidad (medida de forma frecuentista) se  muestra en la siguiente tabla:
\[
\begin{array}{c|c|c|c|c|c|c|c|}
  X   &  0  &  1  &  2   &  3   &  4   &  5  &  6   \\
\hline
 P(x) & 0.1 & 0.2 & 0.25 & 0.15 & 0.15 & 0.1 & 0.05 \\
\end{array}
\]
Se pide:
\begin{enumerate}
\item Calcular la probabilidad de que en un intervalo de 15 minutos lleguen 2 o más personas, y probabilidad de que lleguen menos de 8
personas.
\item ¿Cuál es el número medio esperado de personas que llegarán a sacarse sangre cada 15 minutos?
\item Suponiendo que el número de personas que llegan a sacarse sangre en 15 minutos sigue una distribución de Poisson de media la  
calculada en el apartado anterior, ¿cuál es la probabilidad de que llegue alguna en 15 minutos? ¿Y de que llegue alguna en 5 minutos? 
\end{enumerate}
} 
%SOLUCIÓN
{Llamemos $J$ al suceso consitente en que una persona con la lesión sea joven, $C$ al suceso consistente en curarse, y $A$ y $B$ a los sucesos consistentes en aplicar las respectivas técnicas de
rehabilitación:
\begin{enumerate}
\item Llamando $X$ al número de personas que llegan en un intervalo de 15 minutos: $P(X\geq 2)=0.7$ y $P(X<8)=1$.
\item $\mu=2.55$ personas.
\item Suponiendo $X\sim P(2.55)$, $P(X\geq 1)=0.9219$.\\
Llamando $Y$ al número de personas que llegan en un intervalo de 5 minutos, $P(Y\geq 1)=0.5726$.
\end{enumerate}
}
%RESOLUCIÓN
{\begin{enumerate}
\item Teniendo en cuenta que nos dan la distribución de probabilidad de la variable aleatoria discreta $X$ que expresa el número de
pacientes que llegan en 15 minutos, nos están pidiendo $P(X\geq 2)$ y $P(X<8)$. Estas probabilidades son:
\begin{align*}
P(X\geq 2) & =1-P(X<2)=1-P(X=0)-P(X=1)=1-0.2-0.3=0.7,\\
P(X<8)& =1-P(X\geq 8)=1.
\end{align*}
\item El número esperado es la media de la variable aleatoria:
\[
\mu =\sum xf(x)=0\cdot 0.1+1\cdot 0.2+2\cdot 0.25+3\cdot 0.15+4\cdot 0.15+5\cdot 0.1+6\cdot 0.05=2.55 \text{ personas}.
\]

\item Suponiendo que $X$ sigue una distribución de Poisson con $\lambda =2.55$, $X\sim P(2.55),$ la probabilidad de que llegue alguna
persona en 15 minutos vendrá dada por:
\[
P(X\geq 1)=1-P(X=0)=1-e^{-2.55}\dfrac{2.55^{0}}{0!}=0.9219.
\]

Para la segunda pregunta, teniendo en cuenta que el número medio de personas que llegan cada 15 minutos es $2.55$, en 5 minutos llegarán en
media $2.55/3 = 0.85$ personas, y, por tanto, tenemos una nueva variable aleatoria $Y$, que seguirá una distribución de Poisson con $\lambda
^{\prime }=0.85.$
\[
P(Y\geq 1)=1-P(Y=0)=1-e^{-0.85}\dfrac{0.85^{0}}{0!}=0.5726.
\]
\end{enumerate}
}


\newproblem{vad-35}{gen}{*}
%ENUNCIADO
{En las siguientes tablas, indicar razonadamente, en los caso que sea posible,
los valores de $h$ que deben ponerse en cada tabla para que se tenga una
distribución de probabilidad:
\[
\begin{array}{c|c}
x & f(x) \\
\hline
-2 & 0.3  \\
5 & h  \\
8 & 0.1
\end{array}
\qquad
\begin{array}{c|c}
x & f(x) \\
\hline
 1 & -0.2 \\
 3 & 0.7 \\
 4 & h
\end{array}
 \qquad
\begin{array}{c|c}
x & f(x) \\
\hline
 2 & h \\
 3 & 0.5 \\
 4 & 0.6
\end{array}
\]

En las tablas que constituyan una distribución de probabilidad:
\begin{enumerate}
\item Representar gráficamente la función de distribución.
\item Calcular media y desviación típica.
\item Calcular la mediana.
\item Si a los valores de $x$ se multiplican por una constante $k<0$, ¿cómo se ve afectada la media? ¿Y la desviación típica?
\end{enumerate}
} 
%SOLUCIÓN
{La única tabla que puede ser una distribución de probabilidad es la primera para $h=0.6$.
\begin{enumerate}[start=2]
\item $\mu=3.2$ y $\sigma=3.516$.
\item $Me=5$.
\item $\mu_y=k\mu_x$ y $\sigma_y=|k|\sigma_x$.
\end{enumerate}
}
%RESOLUCIÓN
{La segunda tabla no puede ser una distribución de probabilidad pues $f(1)=-0.2$ y la función de probabilidad no puede tomar valores
negativos. Por otro lado, la tercera tabla tampoco puede ser una distribución de probabilidad ya que la suma de todas las probabilidades
debe ser 1, y para ello debería ser $h=-0.1$, lo cual no es posible al no poder tomar valores negativos. Así pues la única tabla que puede
ser una distribución de probabilidad es la primera, y como la suma de todas las probabilidades tiene que ser 1, $0.3+h+0.1=1$, se deduce que
$h=0.6$. Trabajaremos pues, con la distribución
\[
\begin{array}{r|r}
 x  & f(x) \\
\hline
 -2 & 0.3  \\
 5  & 0.6  \\
 8  & 0.1  \\
\end{array}
\]

\begin{enumerate}
\item La función de distribución se define como $F(x_0)=P(X\leq x_0)$, y por tanto,  mide probabilidades acumuladas. Acumulando las
probabilidades de la tabla anterior tenemos
\[
\begin{array}{r|r|r}
 x  & \multicolumn{1}{c|}{f(x)} & \multicolumn{1}{c}{F(x)}\\
\hline
 -2 & 0.3 & 0.3 \\
 5  & 0.6 & 0.9 \\
 8  & 0.1 & 1 \\
\end{array}
\]
O lo que es lo mismo, expresado como una función a trozos
\[
F(x)=
\left\{%
\begin{array}{ll}
   0, & \hbox{si $x<-2$;} \\
   0.3, & \hbox{si $-2\leq x<5$;} \\
   0.9, & \hbox{si $5\leq x<8$;} \\
   1, & \hbox{si $x\geq 8$.} \\
\end{array}%
\right.
\]

La gráfica de esta función es la siguiente
\begin{center}
\includegraphics[scale=0.3]{grafica1}\hspace*{1cm}
\end{center}

\item Calculamos los estadísticos que nos piden
\begin{align*}
\mu &= \sum x_if(x_i)=-2\cdot0.3+5\cdot 0.6+8\cdot 0.1=3.2,\\
\sigma^2 &= \sum x_i^2f(x_i)-\mu^2=(-2)^2\cdot 0.3+5^2\cdot 0.6+8^2\cdot 0.1-3.2^2=22.6-10.24=12.36,\\
\sigma &= \sqrt{12.36}=3.516.
\end{align*}

\item La mediana es el valor que deja acumulada una probabilidad $0.5$, es decir, $F(med)=0.5$, y mirando en la función de distribución, el
valor donde se consigue acumular esta probabilidad es el 5.

\item Sea $Y=kX$ donde $k<0$. Por las propiedades de las transformaciones lineales de variables aleatorias, tenemos que $\mu_y=k\mu_x$, y
por tanto la media quedará también multiplicada por la constante $k$. Para la desviación típica tenemos que $\sigma_y=|k|\sigma_x$ y la
desviación típica quedará multiplicada por el valor absoluto de $k$.
\end{enumerate}
}


\newproblem{vad-36}{gen}{*}
%ENUNCIADO
{En una empresa el número de días al año que los empleados están de baja es, por término medio, 5. Suponiendo que un año tiene 240 días
laborables y que cada mes tiene 20, se pide:
\begin{enumerate}
\item Calcular el porcentaje de empleados que no faltarían más de 5 días al año.
\item Calcular la probabilidad de que un empleado falte algún día en un mes.
\item ¿Cual es la probabilidad de que en un año haya más de 2 meses en los que haya faltado alguna vez?
\end{enumerate}
} 
%SOLUCIÓN
{
\begin{enumerate}
\item Llamando $X$ a la variable que mide el número de días de baja al año de cada empleado, $X\sim B(240,\,5/240)\approx P(5)$ y
$P(X\leq 5)=0.616$.
\item Llamando $Y$ a la variable que mide el número de días de baja al mes de cada empleado, $Y\sim B(20,\,5/240)$ y $P(Y\geq 1)=0.3437$.
\item Llamando $Z$ a la variable que mide el número de meses al año en que un empleado falta alguna vez, $Z\sim B(12\,,\,0.3437)$ y
$P(Z>2)=0.8379$.
\end{enumerate}
}
%RESOLUCIÓN
{
\begin{enumerate}
\item Sea $X$ la variable que mide el número de días de baja al año de cada empleado. Entonces \mbox{$X\sim B(240\,,\,5/240)$}, pero como
$n=240>30$ y $p=5/240<0.1$, podemos aproximarla como una distribución Poisson $P(5)$. La probabilidad de que un empleado no falte más de 5
días al año es
\begin{align*}
P(X\leq 5)&= P(X=0)+P(X=1)+P(X=2)+P(X=3)+P(X=4)+P(X=5)= \\
&= e^{-5}\frac{5^0}{0!}+
e^{-5}\frac{5^1}{1!}+e^{-5}\frac{5^2}{2!}+e^{-5}\frac{5^3}{3!}
+e^{-5}\frac{5^4}{4!}+e^{-5}\frac{5^5}{5!}= \\
&= 0.0067+0.0337+0.0842+0.1404+0.1755+0.1755=0.616,
\end{align*}
es decir, un $61.6\%$.

\item Sea $Y$ la variable que mide el número de días de baja al mes de cada empleado. Entonces \mbox{$Y\sim B(20,\,5/240)$}, y la
probabilidad de que algún empleado falte algún día en un mes es
\begin{align*}
P(Y\geq1)&=1-P(Y<1)=1-P(Y=0)=
1-\binom{20}{0}\left(\frac{5}{240}\right)^0\left(1-\frac{5}{240}\right)^{20}= \\
&=1-\left(\frac{235}{240}\right)^{20}=0.3437.
\end{align*}

\item Sea ahora $Z$ la variable que mide el número de meses al año en que un empleado falta alguna vez. Entonces, como la probabilidad de
que un empleado falte alguna vez en un mes, según el apartado anterior es $0.3437$, tenemos que $Z\sim B(12\,,\,0.3437)$. Así pues, la
probabilidad que nos piden  es
\begin{align*}
P(Z>2)&=1-P(Z\leq 2)=1-P(Z=0)-P(Z=1)-P(Z=2)=\\
&= 1-\binom{12}{0}0.3437^0 0.6563^{12}-\binom{12}{1}0.3437^1 0.6563^{11}-
\binom{12}{2}0.3437^2 0.6563^{10}=\\
&=1-0,0064-0,0401-0,1156=0.8379.
\end{align*}
\end{enumerate}
}


\newproblem{vad-37}{med}{*}
%ENUNCIADO
{Sabiendo que la prevalencia de la isquemia cardíaca es del 1\%, y que la aplicación de un test diagnóstico para detectar la isquemia
cardíaca tiene una sensibilidad del 90\%, y una especificidad del 95\%. Calcular:
\begin{enumerate}
\item Los valores predictivos, tanto el positivo como el negativo.
\item La probabilidad de diagnóstico acertado.
\item Si tenemos un grupo de 10 enfermos de isquemia cardíaca, ¿cuál es la probabilidad de que diagnostiquemos la enfermedad a
menos de 8?
\end{enumerate}
} 
%SOLUCIÓN
{
}
%RESOLUCIÓN
{
}


\newproblem{vad-38}{med}{*}
%ENUNCIADO
{Un test diagnóstico para una enfermedad devuelve un 1\% de resultados positivos, y sus valores predictivos positivo y negativo valen, respectivamente, $0.95$ y $0.98$. Se pide:
\begin{enumerate}
\item ¿Cuál es la prevalencia de la enfermedad?
\item ¿Cuánto valen la sensibilidad y la especificidad del test?
\item Si aplicamos el test a 12 individuos enfermos, ¿qué probabilidad hay de que se equivoque en alguno de ellos?
\item Si aplicamos el test a 12 individuos, ¿que probabilidad hay de que acierte en todos?
\end{enumerate}
} 
%SOLUCIÓN
{
\begin{enumerate}
\item $P(E)=0.0293$.
\item Sensibilidad $P(+|E)=0.3242$ y especificidad $P(-|\bar E)=0.9995$. 
\item Llamando $X$ al número de diagnósticos erróneos en 12 individuos enfermos, $P(X\geq 1)=1$. 
\item Llamando $Y$ al número de diagnósticos acertados en 12 individuos, $P(X=12)=0.7818$. 
\end{enumerate}
}
%RESOLUCIÓN
{
}


\newproblem{vad-39}{med}{*}
%ENUNCIADO
{La probabilidad de que en un grupo de 5 individuos mayores de 70 años todos padezcan arterioesclerosis cerebral es de $12.5$ por mil.
\begin{enumerate}
\item ¿Cuál es la probabilidad de padecer la enfermedad entre los mayores de 70 años?
\item En un grupo de 1000 personas, ¿cuál es la probabilidad de que padezcan la enfermedad más de 450?
\end{enumerate}
} 
%SOLUCIÓN
{
}
%RESOLUCIÓN
{
}


\newproblem{vad-40}{med}{*}
%ENUNCIADO
{Si sabemos, por estudios previos, que las cepas que provocarán la gripe del siguiente otoño-invierno afectarán a un 20\% de la
población:
\begin{enumerate}
\item ¿Cuál es la probabilidad de que en una población de 10000 habitantes queden infectados menos de 1900?
\item Suponiendo que se vacunan los 10000 habitantes y sabiendo, por estudios previos, que la vacuna inmuniza al 98\% de los vacunados,
¿Cuál es la probabilidad de que queden sin inmunizar menos de 180?
\item De nuevo, suponiendo que se han vacunado los 10000 habitantes y teniendo en cuenta que, por estudios previos, la vacuna produce
reacciones alérgicas en uno de cada 5000 casos, ¿cuál es la probabilidad de que se produzca alguna reacción alérgica en dicha población?
\end{enumerate}
} 
%SOLUCIÓN
{
}
%RESOLUCIÓN
{
}

