\documentclass[aspectratio=149,10pt,xcolor=dvipsnames,t]{beamer}
%---------------------------------------------------------------------------
% GENERAL PACKAGES
%---------------------------------------------------------------------------
\usepackage[utf8x]{inputenc} % Sets UTF8 codification
\usepackage[T1]{fontenc}
\usepackage[spanish]{babel} % Sets spanish language
\usepackage{amsmath} % Math symbols and environments
\usepackage{amsfonts}
\usepackage{amssymb}

\usepackage{array}
\usepackage{multirow}
\usepackage{graphicx}
%\usepackage{url}
\usepackage{textcomp}

%---------------------------------------------------------------------------
% FONTS
%---------------------------------------------------------------------------
\usepackage{mathpazo} % Palatino

%---------------------------------------------------------------------------
% CONFIGURATION
%---------------------------------------------------------------------------
\setbeamersize{text margin left=.5cm, text margin right=.5cm} % Defines margin sizes 
\beamertemplatenavigationsymbolsempty % Hide navitation bar
\usefonttheme[onlymath]{serif} % Math text in serif
\setbeamertemplate{blocks}[rounded] % Blocks with rounded corners
%\setbeamercolor{block title}{bg=RoyalBlue!10} % Color of block title
%\setbeamercolor{block body}{bg=RoyalBlue!10} % Color of block body

\begin{document}
%---------------------------------------------------------------------SLIDE----
\begin{frame}[c]
\vspace{2cm}

\begin{center}
\structure{\LARGE {\textbf{Ejercicios de Estadística}}}
\bigskip

\large
\begin{tabular}{rl}
Temas: & \structure{Estadística descriptiva}\\
Titulaciones: & \structure{Ciencias de la Salud}
\end{tabular}

\bigskip
Alfredo Sánchez Alberca (\texttt{asalber@ceu.es})

\includegraphics[scale=0.2]{img/logo_uspceu}

\biskip
\includegraphics[scale=0.07]{img/cc-logo}
\includegraphics[scale=0.2]{img/cc-by}
\includegraphics[scale=0.2]{img/cc-e}
\includegraphics[scale=0.2]{img/cc-c}
\end{center}
\end{frame}

%---------------------------------------------------------------------SLIDE----
\begin{frame}[c]
\large
Se realiza un estudio para determinar la efectividad de un medicamento para controlar la hipertensión a
180 pacientes hipertensos.
Para ello se les suministra una cantidad determinada del mismo obteniéndose la siguiente tabla de frecuencias:
\[
\begin{array}{|c|c|c|c|c|}
\hline
\textrm{Dosis (mg)} & n_i & f_i & N_i & F_i\\
\hline\hline
(100,400] & 15 & & & \\
\hline
(400,700] & & & & 0.2167\\
\hline
(700,800] & 36 & & & \\
\hline
(800,900] & & 0.3333 & &\\
\hline
(900,1000] & & & & \\
\hline
\end{array}
\]
Se pide:
\begin{enumerate}
\item Completar la tabla. 
\item ¿Cuál ha sido la dosis media de medicamento administrado? ¿Es representativa?
\item ¿Cuál fue la cantidad mínima de medicamento suministrada al 40\% de los pacientes más medicados?
\end{enumerate}
\end{frame}


%---------------------------------------------------------------------SLIDE----
\begin{frame}[c]
\large
\begin{enumerate}
\setcounter{enumi}{3}
\item Si se considera que a partir de una administración de 725mg hay que hacer un seguimiento para controlar posibles
hipotensiones, ¿qué porcentaje de pacientes necesitan ese seguimiento?
\item Si a un paciente se le suministró una cantidad de medicamento de 725 mg y a otro una cantidad tipificada de
$0.95$, ¿A cuál se le administró una cantidad mayor? Justificar la respuesta. 
\item Calcular el coeficiente de asimetría e interpretarlo. 
\item Dibujar el diagrama de cajas e interpretarlo. 
\item Si se cambia de medicamento por otro cuya cantidad a administrar viene dada en función del anterior medicamento
mediante la relación $Y=100+0.7X$, siendo $X$ la cantidad de medicamento original e $Y$ la cantidad de medicamento
nuevo, ¿cuál será la media de la cantidad administrada del nuevo medicamento? ¿Es más representativa que en el
medicamento original? Justificar la respuesta.
\end{enumerate}
\noindent Nota: Para facilitar los cálculos se dan las siguientes sumas:
\[
\sum x_in_i=137700 \quad \sum x_i^2n_i=112410000 \quad \sum (x_i-\bar x)^3n_i= -1965735000 \quad \sum (x_i-\bar x)^4n_i=1162291162500
\]
\end{frame}


%------------------------------------------------------------------SLIDE----
\begin{frame}
\begin{columns}
\begin{column}[T]{0.65\textwidth}
\begin{enumerate}
\item Completar la tabla de frecuencias.
\[
\begin{array}{|c|c|c|c|c|}
\hline
\textrm{Dosis (mg)} & n_i & f_i & N_i & F_i\\
\hline\hline
(100,400] & 15 & & & \\
\hline
(400,700] & & & & 0.2167\\
\hline
(700,800] & 36 & & & \\
\hline
(800,900] & & 0.3333 & &\\
\hline
(900,1000] & & & & \\
\hline
\end{array}
\]
\end{enumerate}
\end{column}
\begin{column}[T]{0.35\textwidth}
\structure{Datos}\\
$X=$ Dosis de medicamento en mg\\
$n=180$ pacientes
\end{column}
\end{columns}
\end{frame}


%------------------------------------------------------------------SLIDE----
\begin{frame}
\begin{columns}
\begin{column}[T]{0.65\textwidth}
\begin{enumerate}
\item[2.] ¿Cuál ha sido la dosis media de medicamento administrado? ¿Es representativa?
\end{enumerate}
\end{column}
\begin{column}[T]{0.35\textwidth}
\structure{Datos}\\
$X=$ Dosis de medicamento en mg\\
$n=180$ pacientes
$\sum x_in_i=137700$ mg\\
$\sum x_i^2n_i=112410000$ mg$^2$
\end{column}
\end{columns}
\end{frame}


%------------------------------------------------------------------SLIDE----
\begin{frame}
\begin{columns}
\begin{column}[T]{0.65\textwidth}
\begin{enumerate}
\item[3.] ¿Cuál fue la cantidad mínima de medicamento suministrada al 40\% de los pacientes más medicados?
\end{enumerate}
\end{column}
\begin{column}[T]{0.35\textwidth}
\structure{Datos}\\
$X=$ Dosis de medicamento en mg\\
$n=180$ pacientes\\
\[
\begin{array}{|c|c|}
\hline
\textrm{Dosis (mg)} & F_i\\
\hline\hline
(100,400] & 0.0833 \\
\hline
(400,700] & 0.2167\\
\hline
(700,800] & 0.4167 \\
\hline
(800,900] & 0.7500 \\
\hline
(900,1000] & 1 \\
\hline
\end{array}
\]
\end{column}
\end{columns}
\end{frame}


%------------------------------------------------------------------SLIDE----
\begin{frame}
\begin{columns}
\begin{column}[T]{0.65\textwidth}
\begin{enumerate}
\item[4.] Si se considera que a partir de una administración de 725 mg hay que hacer un seguimiento para controlar posibles
hipotensiones, ¿qué porcentaje de pacientes necesitan ese seguimiento?
\end{enumerate}
\end{column}
\begin{column}[T]{0.35\textwidth}
\structure{Datos}\\
$X=$ Dosis de medicamento en mg\\
$n=180$ pacientes\\
\[
\begin{array}{|c|c|}
\hline
\textrm{Dosis (mg)} & F_i\\
\hline\hline
(100,400] & 0.0833 \\
\hline
(400,700] & 0.2167\\
\hline
(700,800] & 0.4167 \\
\hline
(800,900] & 0.7500 \\
\hline
(900,1000] & 1 \\
\hline
\end{array}
\]
\end{column}
\end{columns}
\end{frame}


%------------------------------------------------------------------SLIDE----
\begin{frame}
\begin{columns}
\begin{column}[T]{0.65\textwidth}
\begin{enumerate}
\item[5.] Si a un paciente se le suministró una cantidad de medicamento de 725 mg y a otro una cantidad tipificada de
$0.95$, ¿A cuál se le administró una cantidad mayor? Justificar la respuesta. 
\end{enumerate}
\end{column}
\begin{column}[T]{0.35\textwidth}
\structure{Datos}\\
$X=$ Dosis de medicamento en mg\\
$n=180$ pacientes\\
$\bar x=765$ mg\\
$s=198.1792$ mg
\end{column}
\end{columns}
\end{frame}


%------------------------------------------------------------------SLIDE----
\begin{frame}
\begin{columns}
\begin{column}[T]{0.65\textwidth}
\begin{enumerate}
\item[6.] Calcular el coeficiente de asimetría e interpretarlo. 
\end{enumerate}
\end{column}
\begin{column}[T]{0.35\textwidth}
\structure{Datos}\\
$X=$ Dosis de medicamento en mg\\
$n=180$ pacientes\\
$\bar x=765$ mg\\
$s=198.1792$ mg\\
$\sum (x_i-\bar x)^3n_i= -1965735000$ mg$^3$
\end{column}
\end{columns}
\end{frame}


%------------------------------------------------------------------SLIDE----
\begin{frame}
\begin{columns}
\begin{column}[T]{0.65\textwidth}
\begin{enumerate}
\item[7.] Dibujar el diagrama de cajas e interpretarlo. 
\end{enumerate}
\end{column}
\begin{column}[T]{0.35\textwidth}
\structure{Datos}\\
$X=$ Dosis de medicamento en mg\\
$n=180$ pacientes\\
\[
\begin{array}{|c|c|}
\hline
\textrm{Dosis (mg)} & F_i\\
\hline\hline
(100,400] & 0.0833 \\
\hline
(400,700] & 0.2167\\
\hline
(700,800] & 0.4167 \\
\hline
(800,900] & 0.75 \\
\hline
(900,1000] & 1 \\
\hline
\end{array}
\]
\end{column}
\end{columns}
\end{frame}


%------------------------------------------------------------------SLIDE----
\begin{frame}
\begin{columns}
\begin{column}[T]{0.65\textwidth}
\begin{enumerate}
\item[7.] Dibujar el diagrama de cajas e interpretarlo. 
\end{enumerate}
\end{column}
\begin{column}[T]{0.35\textwidth}
\structure{Datos}\\
$X=$ Dosis de medicamento en mg\\
$n=180$ pacientes\\
$C_1=716.65$ mg\\
$C_2=824.9925$ mg\\
$C_3=900$ mg
\end{column}
\end{columns}
\end{frame}


%------------------------------------------------------------------SLIDE----
\begin{frame}
\begin{columns}
\begin{column}[T]{0.65\textwidth}
\begin{enumerate}
\item[8.] Si se cambia de medicamento por otro cuya cantidad a administrar viene dada en función del anterior medicamento
mediante la relación $Y=100+0.7X$, siendo $X$ la cantidad de medicamento original e $Y$ la cantidad de medicamento
nuevo, ¿cuál será la media de la cantidad administrada del nuevo medicamento? ¿Es más representativa que en el
medicamento original? Justificar la respuesta.
\end{enumerate}
\end{column}
\begin{column}[T]{0.35\textwidth}
\structure{Datos}\\
$X=$ Dosis de medicamento en mg\\
$n=180$ pacientes\\
$\bar x=765$ mg\\
$s=198.1792$ mg
\end{column}
\end{columns}
\end{frame}
\end{document}