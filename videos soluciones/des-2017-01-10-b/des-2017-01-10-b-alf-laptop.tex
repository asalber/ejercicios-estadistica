% Author: Alfredo Sánchez Alberca (asalber@ceu.es}
% !TEX program = xelatex
\documentclass[aspectratio=149,10pt,t]{beamer}
%-------------------------------------------------------------------------------
% GENERAL PACKAGES
%-------------------------------------------------------------------------------
% Language
\usepackage{polyglossia}
\setmainlanguage{spanish}
% Maths
\usepackage{amsmath} % Math symbols and environments
\usepackage{amsfonts}
\usepackage{amssymb}
% Tables
\usepackage{array}
\usepackage{multirow}
\usepackage{booktabs}
% Graphics
\usepackage{graphicx}
\usepackage{tikz}
\usetikzlibrary{positioning}


% Colors
\definecolor{blueceu}{RGB}{0,164,227}
\definecolor{greenceu}{RGB}{194,205,24}
\definecolor{redceu}{RGB}{238,50,36}
\definecolor{purpleceu}{RGB}{169,78,145}
\definecolor{greyceu}{RGB}{117,117,97}
\definecolor{darkgrey}{RGB}{40,40,50}
\definecolor{softblueceu}{RGB}{193,225,246}
\setbeamercolor{structure}{fg=blueceu}
\setbeamercolor{normal text}{fg=darkgrey}
\hypersetup{colorlinks, urlcolor=purpleceu}

%-------------------------------------------------------------------------------
% FONTS
%-------------------------------------------------------------------------------
\usepackage{fontspec}
\setmainfont[Ligatures=TeX]{TeX Gyre Pagella}
\usepackage{unicode-math}
\setmathfont[math-style=ISO,bold-style=ISO,vargreek-shape=TeX]{TeX Gyre Pagella Math}
% Creative common icons
\usepackage[scale=1.5]{ccicons}

%-------------------------------------------------------------------------------
% CONFIGURATION
%-------------------------------------------------------------------------------
\setbeamersize{text margin left=.5cm, text margin right=.5cm} % Defines margin sizes
\beamertemplatenavigationsymbolsempty % Hide navitation bar
\usefonttheme[onlymath]{serif} % Math text in serif
\setbeamertemplate{blocks}[rounded] % Blocks with rounded corners
%\setbeamercolor{block title}{bg=RoyalBlue!10} % Color of block title
%\setbeamercolor{block body}{bg=RoyalBlue!10} % Color of block body


%-------------------------------------------------------------------------------
% DOCUMENT
%-------------------------------------------------------------------------------
\begin{document}
%---------------------------------------------------------------------SLIDE----
\begin{frame}[c]
\vspace{1.5cm}

\begin{center}
\structure{\LARGE {\textbf{Ejercicios de Estadística}}}
\bigskip

\large
\begin{tabular}{rl}
Temas: & \structure{Estadística Descriptiva}\\
Titulaciones: & \structure{Medicina}
\end{tabular}

\bigskip
Alfredo Sánchez Alberca\\
\url{asalber@ceu.es}\\
\url{http://aprendeconalf.es}\\

\includegraphics[scale=0.2]{../img/logo_uspceu}

\bigskip
{\color{darkgrey}\ccbyncsaeu}
\end{center}
\end{frame}


%----------------------------------------------------------------------SLIDE----
\begin{frame}[c]
	\large
	En dos poblaciones de mujeres $A$ y $B$ se ha tomado una muestra y se ha medido el número de embarazos de cada mujer durante su vida fértil obteniéndo los siguientes resultados:

	\[
	\begin{array}{ccccccccccccccccc}
	\toprule
	A & 2 & 3 & 4 & 4 & 3 & 2 & 6 & 1 & 5 & 3 & 4 & 4 & 3 & 2 & 5 & 0\\
	B & 1 & 0 & 2 & 1 & 0 & 2 & 0 & 3 & 0 & 1 & 0 & 2 & 5 & 1 & 1 & 1\\
	\bottomrule
	\end{array}
	\]

	\begin{enumerate}
	  \item Construir los diagramas de caja de ambas muestras y compararlos.
	  \item ¿En qué muestra es más representativa la media? Justificar la respuesta.
	  \item Calcular el coeficiente de asimetría de ambas distribuciones.
	  ¿Qué distribución es más asimétrica?
	  \item ¿Qué número de embarazos es relativamente mayor, 5 embarazos en la población $A$ o 3 en la $B$?
	\end{enumerate}

	Utilizar las siguientes sumas para los cálculos:\\
	$\sum a_i=51$, $\sum a_i^2=199$, $\sum (a_i-\bar a)^3=-11.6016$, $\sum (a_i-\bar a)^4=217.9954$,\\
	$\sum b_i=20$, $\sum b_i^2=52$, $\sum (b_i-\bar b)^3=49.5$, $\sum (b_i-\bar b)^4=220.3125$.
\end{frame}


%----------------------------------------------------------------------SLIDE----
\begin{frame}
	\begin{columns}
		\begin{column}[T]{0.7\textwidth}
			En dos poblaciones de mujeres $A$ y $B$ se ha tomado una muestra y se ha medido el número de embarazos de cada mujer durante su vida fértil obteniéndo los siguientes resultados:
			\[
			\begin{array}{ccccccccccccccccc}
			\toprule
			A & 2 & 3 & 4 & 4 & 3 & 2 & 6 & 1 & 5 & 3 & 4 & 4 & 3 & 2 & 5 & 0\\
			B & 1 & 0 & 2 & 1 & 0 & 2 & 0 & 3 & 0 & 1 & 0 & 2 & 5 & 1 & 1 & 1\\
			\bottomrule
			\end{array}
			\]
		\end{column}
		\begin{column}[T]{0.3\textwidth}
			\structure{Datos}\\
			$A\equiv$ Número de embarazos una mujer de la población $A$\\
			$B\equiv$ Número de embarazos una mujer de la población $B$\\
		\end{column}
	\end{columns}
\end{frame}


%----------------------------------------------------------------------SLIDE----
\begin{frame}
	\begin{columns}
		\begin{column}[T]{0.6\textwidth}
			\begin{enumerate}
			  \item Construir los diagramas de caja de ambas muestras y compararlos.
			\end{enumerate}
		\end{column}
		\begin{column}[T]{0.4\textwidth}
			\structure{Datos}\\
			$A\equiv$ Número de embarazos una mujer de la población $A$\\
			$B\equiv$ Número de embarazos una mujer de la población $B$
			\[
			\begin{array}{lll}
				\begin{array}[t]{cr}
					A & F_i\\
					\midrule
					0 & 0.0625\\
					1 & 0.125\\
					2 & 0.3125\\
					3 & 0.5625\\
					4 & 0.8125\\
					5 & 0.9375\\
					6 & 1\\
					\bottomrule
					\end{array} &
					\quad &
					\begin{array}[t]{cr}
						B & F_i\\
						\midrule
						0 & 0.3125\\
						1 & 0.6875\\
						2 & 0.875\\
						3 & 0.9375\\
						5 & 1\\
						\bottomrule
					\end{array}
				\end{array}
			\]
		\end{column}
	\end{columns}
\end{frame}


%----------------------------------------------------------------------SLIDE----
\begin{frame}
	\begin{columns}
		\begin{column}[T]{0.65\textwidth}
			\begin{enumerate}
			  \item[2.] ¿En qué muestra es más representativa la media? Justificar la respuesta.
			\end{enumerate}
		\end{column}
		\begin{column}[T]{0.35\textwidth}
			\structure{Datos}\\
			$A\equiv$ Número de embarazos una mujer de la población $A$\\
			$B\equiv$ Número de embarazos una mujer de la población $B$\\
			$\sum a_i=51$ hijos\\
			$\sum a_i^2=199$ hijos$^2$\\
			$\sum (a_i-\bar a)^3=-11.6016$ hijos$^3$\\
			$\sum (a_i-\bar a)^4=217.9954$ hijos$^4$\\
			$\sum b_i=20$ hijos\\
			$\sum b_i^2=52$ hijos$^2$\\
			$\sum (b_i-\bar b)^3=49.5$ hijos$^3$\\
			$\sum (b_i-\bar b)^4=220.3125$ hijos$^4$
			\end{column}
	\end{columns}
\end{frame}


%----------------------------------------------------------------------SLIDE----
\begin{frame}
	\begin{columns}
		\begin{column}[T]{0.65\textwidth}
			\begin{enumerate}
			  \item[3.] Calcular el coeficiente de asimetría de ambas distribuciones.
			  ¿Qué distribución es más asimétrica?
			\end{enumerate}
		\end{column}
		\begin{column}[T]{0.35\textwidth}
			\structure{Datos}\\
			$A\equiv$ Número de embarazos una mujer de la población $A$\\
			$B\equiv$ Número de embarazos una mujer de la población $B$\\
			$\sum a_i=51$ hijos\\
			$\sum a_i^2=199$ hijos$^2$\\
			$\sum (a_i-\bar a)^3=-11.6016$ hijos$^3$\\
			$\sum (a_i-\bar a)^4=217.9954$ hijos$^4$\\
			$\sum b_i=20$ hijos\\
			$\sum b_i^2=52$ hijos$^2$\\
			$\sum (b_i-\bar b)^3=49.5$ hijos$^3$\\
			$\sum (b_i-\bar b)^4=220.3125$ hijos$^4$\\
			$\bar a=3.1875$ hijos\\
			$s_a = 1.5091$ hijos\\
			$\bar b=1.25$ hijos\\
			$s_b = 1.299$ hijos
			\end{column}
	\end{columns}
\end{frame}


%----------------------------------------------------------------------SLIDE----
\begin{frame}
	\begin{columns}
		\begin{column}[T]{0.65\textwidth}
			\begin{enumerate}
			  \item[4.] ¿Qué número de embarazos es relativamente mayor, 5 embarazos en la población $A$ o 3 en la $B$?
			\end{enumerate}
		\end{column}
		\begin{column}[T]{0.35\textwidth}
			\structure{Datos}\\
			$A\equiv$ Número de embarazos una mujer de la población $A$\\
			$B\equiv$ Número de embarazos una mujer de la población $B$\\
			$\sum a_i=51$ hijos\\
			$\sum a_i^2=199$ hijos$^2$\\
			$\sum (a_i-\bar a)^3=-11.6016$ hijos$^3$\\
			$\sum (a_i-\bar a)^4=217.9954$ hijos$^4$\\
			$\sum b_i=20$ hijos\\
			$\sum b_i^2=52$ hijos$^2$\\
			$\sum (b_i-\bar b)^3=49.5$ hijos$^3$\\
			$\sum (b_i-\bar b)^4=220.3125$ hijos$^4$\\
			$\bar a=3.1875$ hijos\\
			$s_a = 1.5091$ hijos\\
			$\bar b=1.25$ hijos\\
			$s_b = 1.299$ hijos
			\end{column}
	\end{columns}
\end{frame}

\end{document}
