% Author: Alfredo Sánchez Alberca (asalber@ceu.es}
% !TEX program = xelatex
\documentclass[aspectratio=149,10pt,t]{beamer}
%-------------------------------------------------------------------------------
% GENERAL PACKAGES
%-------------------------------------------------------------------------------
% Language
\usepackage{polyglossia}
\setmainlanguage{spanish}
% Maths
\usepackage{amsmath} % Math symbols and environments
\usepackage{amsfonts}
\usepackage{amssymb}
% Tables
\usepackage{array}
\usepackage{multirow}
\usepackage{booktabs}
% Graphics
\usepackage{graphicx}
\usepackage{tikz}
\usetikzlibrary{positioning}


% Colors
\definecolor{blueceu}{RGB}{0,164,227}
\definecolor{greenceu}{RGB}{194,205,24}
\definecolor{redceu}{RGB}{238,50,36}
\definecolor{purpleceu}{RGB}{169,78,145}
\definecolor{greyceu}{RGB}{117,117,97}
\definecolor{darkgrey}{RGB}{40,40,50}
\definecolor{softblueceu}{RGB}{193,225,246}
\setbeamercolor{structure}{fg=blueceu}
\setbeamercolor{normal text}{fg=darkgrey}
\hypersetup{colorlinks, urlcolor=purpleceu}

%-------------------------------------------------------------------------------
% FONTS
%-------------------------------------------------------------------------------
\usepackage{fontspec}
\setmainfont[Ligatures=TeX]{TeX Gyre Pagella}
\usepackage{unicode-math}
\setmathfont[math-style=ISO,bold-style=ISO,vargreek-shape=TeX]{TeX Gyre Pagella Math}
% Creative common icons
\usepackage[scale=1.5]{ccicons}

%-------------------------------------------------------------------------------
% CONFIGURATION
%-------------------------------------------------------------------------------
\setbeamersize{text margin left=.5cm, text margin right=.5cm} % Defines margin sizes
\beamertemplatenavigationsymbolsempty % Hide navitation bar
\usefonttheme[onlymath]{serif} % Math text in serif
\setbeamertemplate{blocks}[rounded] % Blocks with rounded corners
%\setbeamercolor{block title}{bg=RoyalBlue!10} % Color of block title
%\setbeamercolor{block body}{bg=RoyalBlue!10} % Color of block body


%-------------------------------------------------------------------------------
% DOCUMENT
%-------------------------------------------------------------------------------
\begin{document}
%---------------------------------------------------------------------SLIDE----
\begin{frame}[c]
\vspace{1.5cm}

\begin{center}
\structure{\LARGE {\textbf{Ejercicios de Estadística}}}
\bigskip

\large
\begin{tabular}{rl}
Temas: & \structure{Estadística Descriptiva}\\
Titulaciones: & \structure{Medicina}
\end{tabular}

\bigskip
Alfredo Sánchez Alberca\\
\url{asalber@ceu.es}\\
\url{http://aprendeconalf.es}\\

\includegraphics[scale=0.2]{../img/logo_uspceu}

\bigskip
{\color{darkgrey}\ccbyncsaeu}
\end{center}
\end{frame}


%----------------------------------------------------------------------SLIDE----
\begin{frame}[c]
	\large
	La siguiente tabla recoge la distribución de frecuencias del tiempo de atención en un servicio de urgencias de una muestra de pacientes.

	\[
	\begin{array}{cr}
	\toprule
	\mbox{Tiempo} & \mbox{Pacientes}\\
	(0,10] & 22\\
	(10,20] & 43\\
	(20,30] & 33\\
	(30,40] & 12\\
	(40,50] & 6\\
	(50,60] & 4\\
	\bottomrule
	\end{array}
	\]

	Se pide:
	\begin{enumerate}
		\item Dibujar el polígono de frecuencias relativas acumuladas.
	  \item Calcular la mediana del tiempo de atención e interpretarla.
	  \item Qué porcentaje de pacientes han requerido un tiempo de atención superior a 38 minutos?
	\end{enumerate}
\end{frame}


%----------------------------------------------------------------------SLIDE----
\begin{frame}
	\begin{columns}
		\begin{column}[T]{0.7\textwidth}
			\begin{enumerate}
			  \item Dibujar el polígono de frecuencias relativas acumuladas.
			\end{enumerate}
			\[
			\begin{array}{cr}
				X & n_i\\
				\midrule
				(0,10] & 22\\
				(10,20] & 43\\
				(20,30] & 33\\
				(30,40] & 12\\
				(40,50] & 6\\
				(50,60] & 4\\
				\bottomrule
			\end{array}
			\]
		\end{column}
		\begin{column}[T]{0.3\textwidth}
			\structure{Datos}\\
			$X\equiv$ Tiempo de atención\\
		\end{column}
	\end{columns}
\end{frame}


%----------------------------------------------------------------------SLIDE----
\begin{frame}
	\begin{columns}
		\begin{column}[T]{0.7\textwidth}
			\begin{enumerate}
			  \item[2.] Calcular la mediana del tiempo de atención e interpretarla.
			\end{enumerate}
		\end{column}
		\begin{column}[T]{0.3\textwidth}
			\structure{Datos}\\
			$X\equiv$ Tiempo de atención\\
			\[
			\begin{array}{cr}
				X & F_i\\
				\midrule
				(0,10] & 0.18\\
				(10,20] & 0.54\\
				(20,30] & 0.82\\
				(30,40] & 0.92\\
				(40,50] & 0.97\\
				(50,60] & 1\\
				\bottomrule
			\end{array}
			\]
		\end{column}
	\end{columns}
\end{frame}


%----------------------------------------------------------------------SLIDE----
\begin{frame}
	\begin{columns}
		\begin{column}[T]{0.7\textwidth}
			\begin{enumerate}
			  \item[3.] Qué porcentaje de pacientes han requerido un tiempo de atención superior a 38 minutos?
			\end{enumerate}
		\end{column}
		\begin{column}[T]{0.3\textwidth}
			\structure{Datos}\\
			$X\equiv$ Tiempo de atención\\
			\[
			\begin{array}{cr}
				X & F_i\\
				\midrule
				(0,10] & 0.18\\
				(10,20] & 0.54\\
				(20,30] & 0.82\\
				(30,40] & 0.92\\
				(40,50] & 0.97\\
				(50,60] & 1\\
				\bottomrule
			\end{array}
			\]
		\end{column}
	\end{columns}
\end{frame}

\end{document}
