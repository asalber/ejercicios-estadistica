\documentclass[aspectratio=149,10pt,xcolor=dvipsnames,t]{beamer}
%---------------------------------------------------------------------------
% GENERAL PACKAGES
%---------------------------------------------------------------------------
\usepackage[utf8x]{inputenc} % Sets UTF8 codification
\usepackage[T1]{fontenc}
\usepackage[spanish]{babel} % Sets spanish language
\usepackage{amsmath} % Math symbols and environments
\usepackage{amsfonts}
\usepackage{amssymb}

\usepackage{array}
\usepackage{multirow}
\usepackage{graphicx}
%\usepackage{url}
\usepackage{textcomp}

%---------------------------------------------------------------------------
% FONTS
%---------------------------------------------------------------------------
\usepackage{mathpazo} % Palatino

%---------------------------------------------------------------------------
% CONFIGURATION
%---------------------------------------------------------------------------
\setbeamersize{text margin left=.5cm, text margin right=.5cm} % Defines margin sizes 
\beamertemplatenavigationsymbolsempty % Hide navitation bar
\usefonttheme[onlymath]{serif} % Math text in serif
\setbeamertemplate{blocks}[rounded] % Blocks with rounded corners
%\setbeamercolor{block title}{bg=RoyalBlue!10} % Color of block title
%\setbeamercolor{block body}{bg=RoyalBlue!10} % Color of block body

\begin{document}
%---------------------------------------------------------------------SLIDE----
\begin{frame}[c]
\vspace{2cm}

\begin{center}
\structure{\LARGE {\textbf{Ejercicios de Estadística}}}
\bigskip

\large
\begin{tabular}{rl}
Temas: & \structure{Probabilidad}\\
Titulaciones: & \structure{Todas}
\end{tabular}

\bigskip
Alfredo Sánchez Alberca (\texttt{asalber@ceu.es})

\includegraphics[scale=0.2]{img/logo_uspceu}

\biskip
\includegraphics[scale=0.07]{img/cc-logo}
\includegraphics[scale=0.2]{img/cc-by}
\includegraphics[scale=0.2]{img/cc-e}
\includegraphics[scale=0.2]{img/cc-c}\end{center}
\end{frame}

%---------------------------------------------------------------------SLIDE----
\begin{frame}[c]
\large
Un grupo de 50 alumnos tiene 3 asignaturas $A$, $B$ y $C$ en un curso. 
El número de aprobados en la convocatoria ordinaria y extraordinaria aparece en la siguiente tabla (se supone que quien
ha aprobado en la convocatoria ordinaria no se presenta a la extraordinaria):
\begin{center}
\begin{tabular}{lrr}
\hline
Asignatura & Ordinaria & Extraordinaria\\
A & 25 & 12\\
B & 14 & 10\\
C & 32 & 8\\
\hline
\end{tabular}
\end{center}
Suponiendo que el aprobado en cada asignatura es independiente de las demás, se pide:
\begin{enumerate}
\item ¿Cuál es la probabilidad de aprobar alguna asignatura en la convocatoria ordinaria?
\item ¿Cuál es la probabilidad de aprobar las tres asignaturas?
\item Si un alumno ha aprobado la asignatura $A$, ¿cuál es la probabilidad de que hubiese aprobado en la convocatoria ordinaria?
\end{enumerate}
\end{frame}


%------------------------------------------------------------------SLIDE----
\begin{frame}
\begin{columns}
\begin{column}[T]{0.65\textwidth}
\begin{enumerate}
\item ¿Cuál es la probabilidad de aprobar alguna asignatura en la convocatoria ordinaria?
\end{enumerate}
\end{column}
\begin{column}[T]{0.35\textwidth}
\structure{Datos}\\

\medskip
\begin{tabular}{lrr}
\hline
Asignatura & Ord & Ext\\
A & 25 & 12\\
B & 14 & 10\\
C & 32 & 8\\
\hline
\end{tabular}
\end{column}
\end{columns}
\end{frame}


%------------------------------------------------------------------SLIDE----
\begin{frame}
\begin{columns}
\begin{column}[T]{0.6\textwidth}
\begin{enumerate}
\setcounter{enumi}{1}
\item ¿Cuál es la probabilidad de aprobar las tres asignaturas?
\end{enumerate}
\end{column}
\begin{column}[T]{0.4\textwidth}
\structure{Datos}\\

\medskip
\begin{tabular}{lrr}
\hline
Asignatura & Ord & Ext\\
A & 25 & 12\\
B & 14 & 10\\
C & 32 & 8\\
\hline
\end{tabular}
\end{column}
\end{columns}
\end{frame}


%------------------------------------------------------------------SLIDE----
\begin{frame}
\begin{columns}
\begin{column}[T]{0.65\textwidth}
\begin{enumerate}
\setcounter{enumi}{2}
\item Si un alumno ha aprobado la asignatura $A$, ¿cuál es la probabilidad de que hubiese aprobado en la convocatoria ordinaria?
\end{enumerate}
\end{column}
\begin{column}[T]{0.35\textwidth}
\structure{Datos}\\

\medskip
\begin{tabular}{lrr}
\hline
Asignatura & Ord & Ext\\
A & 25 & 12\\
B & 14 & 10\\
C & 32 & 8\\
\hline
\end{tabular}
\end{column}
\end{columns}
\end{frame}

\end{document}