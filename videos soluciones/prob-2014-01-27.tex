\documentclass[aspectratio=149,10pt,xcolor=dvipsnames,t]{beamer}
%---------------------------------------------------------------------------
% GENERAL PACKAGES
%---------------------------------------------------------------------------
\usepackage[utf8x]{inputenc} % Sets UTF8 codification
\usepackage[T1]{fontenc}
\usepackage[spanish]{babel} % Sets spanish language
\usepackage{amsmath} % Math symbols and environments
\usepackage{amsfonts}
\usepackage{amssymb}

\usepackage{array}
\usepackage{multirow}
\usepackage{graphicx}
%\usepackage{url}
\usepackage{textcomp}

%---------------------------------------------------------------------------
% FONTS
%---------------------------------------------------------------------------
\usepackage{mathpazo} % Palatino

%---------------------------------------------------------------------------
% CONFIGURATION
%---------------------------------------------------------------------------
\setbeamersize{text margin left=.5cm, text margin right=.5cm} % Defines margin sizes 
\beamertemplatenavigationsymbolsempty % Hide navitation bar
\usefonttheme[onlymath]{serif} % Math text in serif
\setbeamertemplate{blocks}[rounded] % Blocks with rounded corners
%\setbeamercolor{block title}{bg=RoyalBlue!10} % Color of block title
%\setbeamercolor{block body}{bg=RoyalBlue!10} % Color of block body

\begin{document}
%---------------------------------------------------------------------SLIDE----
\begin{frame}[c]
\vspace{2cm}

\begin{center}
\structure{\LARGE {\textbf{Ejercicios de Estadística}}}
\bigskip

\large
\begin{tabular}{rl}
Temas: & \structure{Probabilidad: Tests diagnósticos}\\
Titulaciones: & \structure{Medicina}
\end{tabular}

\bigskip
Alfredo Sánchez Alberca (\texttt{asalber@ceu.es})

\includegraphics[scale=0.2]{img/logo_uspceu}

\biskip
\includegraphics[scale=0.07]{img/cc-logo}
\includegraphics[scale=0.2]{img/cc-by}
\includegraphics[scale=0.2]{img/cc-e}
\includegraphics[scale=0.2]{img/cc-c}
\end{center}
\end{frame}

%---------------------------------------------------------------------SLIDE----
\begin{frame}[c]
\large
Un test para la detección precoz de una enfermedad da positivo en el 90\% de los casos en los que existe
la enfermedad y también da positivo en el 2\% de los casos en los que no existe. 
Suponiendo que la prevalencia de la enfermedad es del 5\%, ¿cuál es la probabilidad de que un individuo en el que test
da positivo realmente padezca la enfermedad? ¿Diagnosticarías la enfermedad?

Si pasados unos días se le repite el test al mismo individuo y vuelve a dar positivo, ¿qué diagnosticarías ahora?
Justificar la respuesta. 
Supóngase que el segundo test es independiente del primero. 
\end{frame}


%------------------------------------------------------------------SLIDE----
\begin{frame}
\begin{columns}
\begin{column}[T]{0.65\textwidth}
Un test para la detección precoz de una enfermedad da positivo en el 90\% de los casos en los que existe
la enfermedad y también da positivo en el 2\% de los casos en los que no existe. 
Suponiendo que la prevalencia de la enfermedad es del 5\%, ¿cuál es la probabilidad de que un individuo en el que test
da positivo realmente padezca la enfermedad? ¿Diagnosticarías la enfermedad?
\end{column}
\begin{column}[T]{0.35\textwidth}
\structure{Datos}\\
$E=$ Tener la enfermedad\\
$+=$ Resultado del test positivo\\
$90\%$ de + en los enfermos\\[5mm]
$2\%$ de + en los sanos \\[5mm]
$5\%$ de prevalencia de la enfermedad
\end{column}
\end{columns}
\end{frame}


%------------------------------------------------------------------SLIDE----
\begin{frame}
\begin{columns}
\begin{column}[T]{0.65\textwidth}
Si pasados unos días se le repite el test al mismo individuo y vuelve a dar positivo, ¿qué diagnosticarías ahora?
Justificar la respuesta.\\ 
Supóngase que el segundo test es independiente del primero. 
\end{column}
\begin{column}[T]{0.35\textwidth}
\structure{Datos}\\
$E=$ Tener la enfermedad\\
$+=$ Resultado del test positivo\\
$P(+/E)=0.9$\\
$P(+/\overline E)=0.02$\\
$P(E/+)=0.7031$
\end{column}
\end{columns}
\end{frame}

\end{document}