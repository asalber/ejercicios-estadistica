% Author: Alfredo Sánchez Alberca (asalber@ceu.es}
% !TEX program = xelatex
\documentclass[aspectratio=149,10pt,t]{beamer}
%-------------------------------------------------------------------------------
% GENERAL PACKAGES
%-------------------------------------------------------------------------------
% Language
\usepackage{polyglossia}
\setmainlanguage{spanish}
% Maths
\usepackage{amsmath} % Math symbols and environments
\usepackage{amsfonts}
\usepackage{amssymb}
% Tables
\usepackage{array}
\usepackage{multirow}
\usepackage{booktabs}
% Graphics
\usepackage{graphicx}
\usepackage{tikz}
\usetikzlibrary{positioning}


% Colors
\definecolor{blueceu}{RGB}{0,164,227}
\definecolor{greenceu}{RGB}{194,205,24}
\definecolor{redceu}{RGB}{238,50,36}
\definecolor{purpleceu}{RGB}{169,78,145}
\definecolor{greyceu}{RGB}{117,117,97}
\definecolor{darkgrey}{RGB}{40,40,50}
\definecolor{softblueceu}{RGB}{193,225,246}
\setbeamercolor{structure}{fg=blueceu}
\setbeamercolor{normal text}{fg=darkgrey}
\hypersetup{colorlinks, urlcolor=purpleceu}

%-------------------------------------------------------------------------------
% FONTS
%-------------------------------------------------------------------------------
\usepackage{fontspec}
\setmainfont[Ligatures=TeX]{TeX Gyre Pagella}
\usepackage{unicode-math}
\setmathfont[math-style=ISO,bold-style=ISO,vargreek-shape=TeX]{TeX Gyre Pagella Math}
% Creative common icons
\usepackage[scale=1.5]{ccicons}

%-------------------------------------------------------------------------------
% CONFIGURATION
%-------------------------------------------------------------------------------
\setbeamersize{text margin left=.5cm, text margin right=.5cm} % Defines margin sizes
\beamertemplatenavigationsymbolsempty % Hide navitation bar
\usefonttheme[onlymath]{serif} % Math text in serif
\setbeamertemplate{blocks}[rounded] % Blocks with rounded corners
%\setbeamercolor{block title}{bg=RoyalBlue!10} % Color of block title
%\setbeamercolor{block body}{bg=RoyalBlue!10} % Color of block body


%-------------------------------------------------------------------------------
% DOCUMENT
%-------------------------------------------------------------------------------
\begin{document}
%---------------------------------------------------------------------SLIDE----
\begin{frame}[c]
\vspace{1.5cm}

\begin{center}
\structure{\LARGE {\textbf{Ejercicios de Estadística}}}
\bigskip

\large
\begin{tabular}{rl}
Temas: & \structure{Regresión lineal y no lineal}\\
Titulaciones: & \structure{Medicina}
\end{tabular}

\bigskip
Alfredo Sánchez Alberca\\
\url{asalber@ceu.es}\\
\url{http://aprendeconalf.es}\\

\includegraphics[scale=0.2]{../img/logo_uspceu}

\bigskip
{\color{darkgrey}\ccbyncsaeu}
\end{center}
\end{frame}


%----------------------------------------------------------------------SLIDE----
\begin{frame}[c]
	En un estudio se ha medido la reducción en el nivel de colesterol de un grupo de personas hipertensas tras un programa de ejercicios.
	Los resultados aparecen en la siguiente tabla.

	\medskip

	\resizebox{\textwidth}{!}{
	$
	\begin{array}{lrrrrrrrrrr}
		\toprule
		\mbox{Minutos de ejercicio} & 96 & 106 & 163 & 207 & 227 & 244 & 261 & 271 & 272 & 301\\
		\mbox{Reducción de colesterol (mg/dl)} & 4 & 5 & 8 & 13 & 15 & 17 & 22 & 39 & 31 & 45\\
		\bottomrule
	\end{array}
	$
	}

	\begin{enumerate}
	  \item ¿Qué modelo de regresión explica mejor la reducción de colesterol en función de los minutos de ejercicio, el lineal o el exponencial? Justificar la respuesta.
	  \item Según el modelo de regresión lineal, ¿cuánto disminuirá el colesterol por cada minuto más de ejercicio?
	  \item Según el modelo logarítmico, ¿cuántos minutos de ejercicio se necesitan para reducir el colesterol 100 mg/dl?
	  ¿Es fiable la predicción?
	  Justificar la respuesta.
	\end{enumerate}

	Utilizar las siguientes sumas ($X$=Minutos de ejercicio e $Y$=Reducción de colesterol):\\
	 $\sum x_i=2148$ min, $\sum x_i^2=507082$ min$^2$,\\
	 $\sum \log(x_i)=53.0559$ $\log(\mbox{min})$, $\sum \log(x_i)^2=282.9578$ $\log(\mbox{min})^2$, \\
	 $\sum y_j=199$ mg/dl, $\sum y_j^2=5779$ (mg/dl)$^2$,\\
	 $\sum \log(y_j)=27.1766$ $\log(\mbox{mg/dl})$, $\sum \log(y_j)^2=80.035$ $\log(\mbox{mg/dl})^2$,\\
	 $\sum x_iy_j=50750$ min$\cdot$mg/dl, $\sum x_i\log(y_j)=6359.0468$ min$\cdot\log(\mbox{mg/dl})$, $\sum \log(x_i)y_j=1097.978$ $\log(\mbox{min})$mg/dl, $\sum \log(x_i)\log(y_j)=147.0682$ $\log(\mbox{min})\log(\mbox{mg/dl})$.
\end{frame}


%----------------------------------------------------------------------SLIDE----
\begin{frame}
	\begin{columns}
		\begin{column}[T]{0.6\textwidth}
			\begin{enumerate}
				\item ¿Qué modelo de regresión explica mejor la reducción de colesterol en función de los minutos de ejercicio, el lineal o el exponencial?
			\end{enumerate}
		\end{column}
		\begin{column}[T]{0.4\textwidth}
			\structure{Datos}\\
			$X\equiv$ Minutos de ejercicio\\
			$Y\equiv$ Reducción del colesterol\\
			$\sum x_i=2148$ min\\
			$\sum x_i^2=507082$ min$^2$\\
	 	 	$\sum \log(x_i)=53.0559$ $\log(\mbox{min})$\\
			$\sum \log(x_i)^2=282.9578$ $\log(\mbox{min})^2$\\
	 	 	$\sum y_j=199$ mg/dl\\
			$\sum y_j^2=5779$ (mg/dl)$^2$\\
	 	 	$\sum \log(y_j)=27.1766$ $\log(\mbox{mg/dl})$\\
			$\sum \log(y_j)^2=80.035$ $\log(\mbox{mg/dl})^2$\\
	 	 	$\sum x_iy_j=50750$ min$\cdot$mg/dl\\
			$\sum x_i\log(y_j)=6359.0468$ min$\cdot\log(\mbox{mg/dl})$\\
			$\sum \log(x_i)y_j=1097.978$ $\log(\mbox{min})$mg/dl\\
			$\sum \log(x_i)\log(y_j)=147.0682$ $\log(\mbox{min})\log(\mbox{mg/dl})$.
		\end{column}
	\end{columns}
\end{frame}


%----------------------------------------------------------------------SLIDE----
\begin{frame}
	\begin{columns}
		\begin{column}[T]{0.6\textwidth}
			\begin{enumerate}
				\item[2.] Según el modelo de regresión lineal, ¿cuánto disminuirá el colesterol por cada minuto más de ejercicio?
			\end{enumerate}
		\end{column}
		\begin{column}[T]{0.4\textwidth}
			\structure{Datos}\\
			$X\equiv$ Minutos de ejercicio\\
			$Y\equiv$ Reducción del colesterol\\
			$\bar x=214.8$ min\\
			$s_x^2=4569.16$ min$^2$\\
			$\bar y=19.9$ mmHg\\
			$s_y^2=181.89$ mmHg$^2$\\
			$s_{xy}=800.48$ min$\cdot$mmHg
		\end{column}
	\end{columns}
\end{frame}




%----------------------------------------------------------------------SLIDE----
\begin{frame}
	\begin{columns}
		\begin{column}[T]{0.6\textwidth}
			\begin{enumerate}
				\item[3.] Según el modelo logarítmico, ¿cuántos minutos de ejercicio se necesitan para reducir el colesterol 100 mg/dl?
			  ¿Es fiable la predicción?
			\end{enumerate}
		\end{column}
		\begin{column}[T]{0.4\textwidth}
			\structure{Datos}\\
			$X\equiv$ Minutos de ejercicio\\
			$Y\equiv$ Reducción del colesterol\\
			$\bar x=214.8$ min\\
			$s_x^2=4569.16$ min$^2$\\
			$\overline {\log(y)}=2.7177$ $\log(\mbox{mmHg})$\\
			$s_{\log(y)}^2=0.6178$ $\log(\mbox{mmHg})^2$\\
			$s_{x\log(y)}=52.1504$ min$\cdot\log(\mbox{mmHg})$
		\end{column}
	\end{columns}
\end{frame}

\end{document}
