% Author: Alfredo Sánchez Alberca (asalber@ceu.es}
% !TEX program = xelatex
\documentclass[aspectratio=149,10pt,t]{beamer}
%-------------------------------------------------------------------------------
% GENERAL PACKAGES
%-------------------------------------------------------------------------------
% Language
\usepackage{polyglossia}
\setmainlanguage{spanish}
% Maths
\usepackage{amsmath} % Math symbols and environments
\usepackage{amsfonts}
\usepackage{amssymb}
% Tables
\usepackage{array}
\usepackage{multirow}
\usepackage{booktabs}
% Graphics
\usepackage{graphicx}
\usepackage{tikz}
\usetikzlibrary{positioning}

% Colors
\definecolor{blueceu}{RGB}{0,164,227}
\definecolor{greenceu}{RGB}{194,205,24}
\definecolor{redceu}{RGB}{238,50,36}
\definecolor{purpleceu}{RGB}{169,78,145}
\definecolor{greyceu}{RGB}{117,117,97}
\definecolor{darkgrey}{RGB}{40,40,50}
\definecolor{softblueceu}{RGB}{193,225,246}
\setbeamercolor{structure}{fg=blueceu}
\setbeamercolor{normal text}{fg=darkgrey}
\hypersetup{colorlinks, urlcolor=purpleceu}

% Boxes
\usepackage[most]{tcolorbox}
\usepackage{setspace}
\newtcolorbox{datos}{
  enhanced,
  colback=blueceu!10, 
  colframe=blueceu, 
  fonttitle=\bfseries, 
  left=3pt, 
  right=3pt, 
  boxrule=0.5pt,
  %halign=left,
  code={\setstretch{1.2}},
  title={Datos},
}

%-------------------------------------------------------------------------------
% FONTS
%-------------------------------------------------------------------------------
\usepackage{fontspec}
\setmainfont[Ligatures=TeX]{TeX Gyre Pagella}
\usepackage{unicode-math}
\setmathfont[math-style=ISO,bold-style=ISO,vargreek-shape=TeX]{TeX Gyre Pagella Math}
% Creative common icons
\usepackage[scale=1.5]{ccicons}

%-------------------------------------------------------------------------------
% CONFIGURATION
%-------------------------------------------------------------------------------
\setbeamersize{text margin left=.5cm, text margin right=.5cm} % Defines margin sizes
\beamertemplatenavigationsymbolsempty % Hide navitation bar
\usefonttheme[onlymath]{serif} % Math text in serif
\setbeamertemplate{blocks}[rounded] % Blocks with rounded corners
%\setbeamercolor{block title}{bg=RoyalBlue!10} % Color of block title
%\setbeamercolor{block body}{bg=RoyalBlue!10} % Color of block body


%-------------------------------------------------------------------------------
% DOCUMENT
%-------------------------------------------------------------------------------
\begin{document}
%---------------------------------------------------------------------SLIDE----
\begin{frame}[c]
\vspace{1.5cm}

\begin{center}
\structure{\LARGE {\textbf{Ejercicios de Estadística}}}
\bigskip

\large
\begin{tabular}{rl}
Temas: & \structure{Regresión lineal y no lineal}\\
Titulaciones: & \structure{Medicina, Farmacia}
\end{tabular}

\bigskip
Alfredo Sánchez Alberca\\
\url{asalber@ceu.es}\\
\url{http://aprendeconalf.es}\\

\includegraphics[scale=0.2]{../img/logo_uspceu}

\bigskip
{\color{darkgrey}\ccbyncsaeu}
\end{center}
\end{frame}


%----------------------------------------------------------------------SLIDE----
\begin{frame}[c]
	%\large
	La siguiente tabla muestra las tasas de incidencia de gripe por cada 100.000 habitantes registradas al cabo de un número de días desde el comienzo de el estudio.

\[
\begin{array}{lrrrrrrrr}
  \toprule
  \mbox{Días} & 1 & 5 & 8 & 12 & 20 & 26 & 38 & 44\\
  \mbox{Tasa gripe} & 60 & 66 & 71 & 80 & 106 & 132 & 194 & 235\\
  \bottomrule
\end{array}
\]

Se pide:

\begin{enumerate}
  \item Calcular la tasa de incidencia de gripe a los 50 días desde el comienzo del estudio mediante un modelo de regresión lineal.
  \item ¿Cuánto varía la tasa de incidencia de gripe cada día según el modelo lineal?
  \item Calcular la tasa de incidencia de gripe a los 50 días desde el comienzo del estudio mediante un modelo de regresión exponencial.
  \item ¿Cuál de las predicciones anteriores es más fiable?
  Razonar la respuesta.
\end{enumerate}
Utilizar las siguientes sumas para los cálculos ($X=$Días e $Y=$Tasa de gripe):\\
$\sum x_i=154$, $\sum \log(x_i)=19.8494$, $\sum y_j=944$, $\sum \log(y_j)=37.2024$,\\
$\sum x_i^2=4690$, $\sum \log(x_i)^2=60.2309$, $\sum y_j^2=140918$, $\sum \log(y_j)^2=174.8363$,\\
$\sum x_iy_j=25182$, $\sum \log(x_i)y_j=2795.2484$, $\sum x_i\log(y_j)=772.3504$, $\sum \log(x_i)\log(y_j)=96.1974$.
\end{frame}


%----------------------------------------------------------------------SLIDE----
\begin{frame}
	\begin{columns}
		\begin{column}[T]{0.64\textwidth}
			\begin{enumerate}
				\item Calcular la tasa de incidencia de gripe a los 50 días desde el comienzo del estudio mediante un modelo de regresión lineal.
			\end{enumerate}
		\end{column}
		\begin{column}[T]{0.36\textwidth}
			\begin{datos}
				$X\equiv$ Días\\
				$Y\equiv$ Tasa de gripe\\
				$\sum x_i=154$\\
				$\sum \log(x_i)=19.8494$\\ 
				$\sum y_j=944$\\ 
				$\sum \log(y_j)=37.2024$\\
				$\sum x_i^2=4690$\\
				$\sum \log(x_i)^2=60.2309$\\
				$\sum y_j^2=140918$\\
				$\sum \log(y_j)^2=174.8363$\\
				$\sum x_iy_j=25182$\\
				$\sum \log(x_i)y_j=2795.2484$\\
				$\sum x_i\log(y_j)=772.3504$\\
				$\sum \log(x_i)\log(y_j)=96.1974$
			\end{datos}
		\end{column}
	\end{columns}
\end{frame}


%----------------------------------------------------------------------SLIDE----
\begin{frame}
	\begin{columns}
		\begin{column}[T]{0.55\textwidth}
			\begin{enumerate}
				\item[2.] ¿Cuánto varía la tasa de incidencia de gripe cada día según el modelo lineal?
			\end{enumerate}
		\end{column}
		\begin{column}[T]{0.45\textwidth}
			\begin{datos}
				$X\equiv$ Días\\
				$Y\equiv$ Tasa de gripe\\
				Recta de regresión de $Y$ sobre $X$:\\
				$y=39.7951+4.0626x$ 
			\end{datos}
		\end{column}
	\end{columns}
\end{frame}


%----------------------------------------------------------------------SLIDE----
\begin{frame}
	\begin{columns}
		\begin{column}[T]{0.64\textwidth}
			\begin{enumerate}
				\item[3.] Calcular la tasa de incidencia de gripe a los 50 días desde el comienzo del estudio mediante un modelo de regresión exponencial.
			\end{enumerate}
		\end{column}
		\begin{column}[T]{0.36\textwidth}
			\begin{datos}
				$X\equiv$ Días\\
				$Y\equiv$ Tasa de gripe\\
				$\bar x=19.25$\\ 
				$s_x^2=215.6875$\\ 
				$\bar y=118$\\
				$s_y^2=3690.75$\\
				$s_{xy}=876.25$\\ 
				$\sum \log(x_i)=19.8494$\\ 
				$\sum \log(y_j)=37.2024$\\
				$\sum \log(x_i)^2=60.2309$\\
				$\sum \log(y_j)^2=174.8363$\\
				$\sum \log(x_i)y_j=2795.2484$\\
				$\sum x_i\log(y_j)=772.3504$\\
				$\sum \log(x_i)\log(y_j)=96.1974$
			\end{datos}
		\end{column}
	\end{columns}
\end{frame}


%----------------------------------------------------------------------SLIDE----
\begin{frame}
	\begin{columns}
		\begin{column}[T]{0.64\textwidth}
			\begin{enumerate}
				\item[4.] ¿Cuál de las predicciones anteriores es más fiable?
			Razonar la respuesta.			\end{enumerate}
		\end{column}
		\begin{column}[T]{0.36\textwidth}
			\begin{datos}
				$X\equiv$ Días\\
				$Y\equiv$ Tasa de gripe\\
				$\bar x=19.25$\\ 
				$s_x^2=215.6875$\\ 
				$\bar y=118$\\
				$s_y^2=3690.75$\\
				$s_{xy}=876.25$\\
				$\overline{\log(y)}=4.6503$\\
				$s_{\log(y)}^2=0.2293$\\
				$s_{x\log(y)}=7.0255$
			\end{datos}
		\end{column}
	\end{columns}
\end{frame}
\end{document}
