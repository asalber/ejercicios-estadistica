\documentclass[aspectratio=149,10pt,xcolor=dvipsnames,t]{beamer}
%---------------------------------------------------------------------------
% GENERAL PACKAGES
%---------------------------------------------------------------------------
\usepackage[utf8x]{inputenc} % Sets UTF8 codification
\usepackage[T1]{fontenc}
\usepackage[spanish]{babel} % Sets spanish language
\usepackage{amsmath} % Math symbols and environments
\usepackage{amsfonts}
\usepackage{amssymb}

\usepackage{array}
\usepackage{multirow}
\usepackage{graphicx}
%\usepackage{url}
\usepackage{textcomp}

%---------------------------------------------------------------------------
% FONTS
%---------------------------------------------------------------------------
\usepackage{mathpazo} % Palatino

%---------------------------------------------------------------------------
% CONFIGURATION
%---------------------------------------------------------------------------
\setbeamersize{text margin left=.5cm, text margin right=.5cm} % Defines margin sizes 
\beamertemplatenavigationsymbolsempty % Hide navitation bar
\usefonttheme[onlymath]{serif} % Math text in serif
\setbeamertemplate{blocks}[rounded] % Blocks with rounded corners
%\setbeamercolor{block title}{bg=RoyalBlue!10} % Color of block title
%\setbeamercolor{block body}{bg=RoyalBlue!10} % Color of block body

\begin{document}
%---------------------------------------------------------------------SLIDE----
\begin{frame}[c]
\vspace{2cm}

\begin{center}
\structure{\LARGE {\textbf{Ejercicios de Estadística}}}
\bigskip

\large
\begin{tabular}{rl}
Temas: & \structure{Regresión no lineal}\\
Titulaciones: & \structure{Química}
\end{tabular}

\bigskip
Alfredo Sánchez Alberca (\texttt{asalber@ceu.es})

\includegraphics[scale=0.2]{img/logo_uspceu}

\biskip
\includegraphics[scale=0.07]{img/cc-logo}
\includegraphics[scale=0.2]{img/cc-by}
\includegraphics[scale=0.2]{img/cc-e}
\includegraphics[scale=0.2]{img/cc-c}
\end{center}
\end{frame}

%---------------------------------------------------------------------SLIDE----
\begin{frame}[c]
\large
En un estudio se ha medido el calor liberado en una reacción química en distintos instantes
desde el comienzo de la reacción, obteniendo los siguientes datos:
\[
\begin{array}{lrrrrr}
\hline
\textrm{Tiempo en minutos} & 2.5 & 3.7 & 4.1 & 5.3 & 6.2 \\
\textrm{Calor en calorías} & 15.9 & 44.5 & 65.6 & 206.5 & 498.7\\
\hline
\end{array}
\]
Se pide:
\begin{enumerate}
\item Calcular el modelo de regresión lineal del calor sobre el tiempo. 
Según este modelo ¿Cuándo cambiarán las calorías por cada minuto que pase?
\item Calcular el modelo de regresión exponencial del calor sobre el tiempo.
\item Utilizando el mejor de los dos modelos anteriores, predecir el calor generado a los 5 minutos de la reacción. ¿Es
fiable la predicción? Justificar la respuesta. 
\end{enumerate}
\end{frame}


%------------------------------------------------------------------SLIDE----
\begin{frame}
\begin{columns}
\begin{column}[T]{0.65\textwidth}
\begin{enumerate}
\item Calcular el modelo de regresión lineal del calor sobre el tiempo. 
Según este modelo ¿Cuándo cambiarán las calorías por cada minuto que pase?
\[
\begin{array}{lrrrrr}
\hline
X & 2.5 & 3.7 & 4.1 & 5.3 & 6.2 \\
Y & 15.9 & 44.5 & 65.6 & 206.5 & 498.7\\
\hline
\end{array}
\]
\end{enumerate}
\end{column}
\begin{column}[T]{0.35\textwidth}
\structure{Datos}\\
$X=$ Tiempo en minutos\\
$Y=$ Calor liberado en calorías
\end{column}
\end{columns}
\end{frame}


%------------------------------------------------------------------SLIDE----
\begin{frame}
\begin{columns}
\begin{column}[T]{0.65\textwidth}
\begin{enumerate}
\item[2.] Calcular el modelo de regresión exponencial del calor sobre el tiempo.
\[
\begin{array}{lrrrrr}
\hline
X & 2.5 & 3.7 & 4.1 & 5.3 & 6.2 \\
Y & 15.9 & 44.5 & 65.6 & 206.5 & 498.7\\
\hline
\end{array}
\]
\end{enumerate}
\end{column}
\begin{column}[T]{0.35\textwidth}
\structure{Datos}\\
$X=$ Tiempo en minutos\\
$Y=$ Calor liberado en calorías\\
$\bar x = 4.36$ min\\
$s_x^2=1.6464$ min$^2$
\end{column}
\end{columns}
\end{frame}


%------------------------------------------------------------------SLIDE----
\begin{frame}
\begin{columns}
\begin{column}[T]{0.65\textwidth}
\begin{enumerate}
\item[3.] Utilizando el mejor de los dos modelos anteriores, predecir el calor generado a los 5 minutos de la reacción. ¿Es
fiable la predicción? Justificar la respuesta. 
\end{enumerate}
\end{column}
\begin{column}[T]{0.35\textwidth}
\structure{Datos}\\
$X=$ Tiempo en minutos\\
$Y=$ Calor liberado en calorías\\
$Z=\log(Y)$\\
$s_x^2=1.6464$ min$^2$\\
$s_y^2=31940.3344$ cal$^2$\\
$s_z^2=1.4427 \log^2(\mbox{cal})$\\
$s_{xy}=207.1436$ min$\cdot$cal\\
$s_{xz}=1.5405$ min$\cdot\log(\mbox{cal})$
\end{column}
\end{columns}
\end{frame}

\end{document}