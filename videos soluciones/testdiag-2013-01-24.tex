\documentclass[aspectratio=149,10pt,xcolor=dvipsnames,t]{beamer}
%---------------------------------------------------------------------------
% GENERAL PACKAGES
%---------------------------------------------------------------------------
\usepackage[utf8x]{inputenc} % Sets UTF8 codification
\usepackage[T1]{fontenc}
\usepackage[spanish]{babel} % Sets spanish language
\usepackage{amsmath} % Math symbols and environments
\usepackage{amsfonts}
\usepackage{amssymb}

\usepackage{array}
\usepackage{multirow}
\usepackage{graphicx}
%\usepackage{url}
\usepackage{textcomp}

%---------------------------------------------------------------------------
% FONTS
%---------------------------------------------------------------------------
\usepackage{mathpazo} % Palatino

%---------------------------------------------------------------------------
% CONFIGURATION
%---------------------------------------------------------------------------
\setbeamersize{text margin left=.5cm, text margin right=.5cm} % Defines margin sizes 
\beamertemplatenavigationsymbolsempty % Hide navitation bar
\usefonttheme[onlymath]{serif} % Math text in serif
\setbeamertemplate{blocks}[rounded] % Blocks with rounded corners
%\setbeamercolor{block title}{bg=RoyalBlue!10} % Color of block title
%\setbeamercolor{block body}{bg=RoyalBlue!10} % Color of block body

\begin{document}
%---------------------------------------------------------------------SLIDE----
\begin{frame}[c]
\vspace{2cm}

\begin{center}
\structure{\LARGE {\textbf{Ejercicios de Estadística}}}
\bigskip

\large
\begin{tabular}{rl}
Temas: & \structure{Test diagnósticos}\\
Titulaciones: & \structure{Ciencias de la Salud}
\end{tabular}

\bigskip
Alfredo Sánchez Alberca (\texttt{asalber@ceu.es})

\includegraphics[scale=0.2]{img/logo_uspceu}

\biskip
\includegraphics[scale=0.07]{img/cc-logo}
\includegraphics[scale=0.2]{img/cc-by}
\includegraphics[scale=0.2]{img/cc-e}
\includegraphics[scale=0.2]{img/cc-c}
\end{center}
\end{frame}

%---------------------------------------------------------------------SLIDE----
\begin{frame}[c]
\large
Para detectar el parásito del paludismo existe un test de respuesta inmediata que produce un 2\% de falsos
positivos y un 4\% de falsos negativos. 
En una determinada región de África se sabe que hay un 32\% de personas con paludismo. 
Se pide:
\begin{enumerate}
\item ¿Cuál es la probabilidad de que el test de un diagnóstico acertado?
\item ¿Cuál es el poder predictivo negativo del test?
\item ¿Cuánto debería valer la sensibilidad del test para que el poder predictivo negativo fuese de al menos el 99\%?
\end{enumerate} 
Nota: Un falso positivo se produce cuando el test da positivo al aplicarlo a una persona sana. Un falso negativo se
produce cuando el test da negativo al aplicarlo a una persona enferma. El poder predictivo negativo es el porcentaje de
personas sanas entre las que han dado negativo en el test.
\end{frame}


%------------------------------------------------------------------SLIDE----
\begin{frame}
\begin{columns}
\begin{column}[T]{0.6\textwidth}
\begin{enumerate}
\item ¿Cuál es la probabilidad de que el test de un diagnóstico acertado?
\end{enumerate}
\end{column}
\begin{column}[T]{0.4\textwidth}
\structure{Datos}\\
$E=$ Tener paludismo\\
$+=$ Resultado del test positivo\\
$-=$ Resultado del test negativo\\
2\% de falsos positivos\\[7mm]
4\% de falsos negativos\\[7mm]
32\% de personas con paludismo
\end{column}
\end{columns}
\end{frame}


%------------------------------------------------------------------SLIDE----
\begin{frame}
\begin{columns}
\begin{column}[T]{0.6\textwidth}
\begin{enumerate}
\setcounter{enumi}{1}
\item ¿Cuál es el poder predictivo negativo del test?
\end{enumerate}
\end{column}
\begin{column}[T]{0.4\textwidth}
\structure{Datos}\\
$E=$ Tener paludismo\\
$+=$ Resultado del test positivo\\
$-=$ Resultado del test negativo\\
$P(+|\overline{E})=0.02$\\
$P(-|E) = 0.04$\\
$P(E)=0.32$
\end{column}
\end{columns}
\end{frame}


%------------------------------------------------------------------SLIDE----
\begin{frame}
\begin{columns}
\begin{column}[T]{0.6\textwidth}
\begin{enumerate}
\setcounter{enumi}{2}
\item ¿Cuánto debería valer la sensibilidad del test para que el poder predictivo negativo fuese de al menos el 99\%?
\end{enumerate}
\end{column}
\begin{column}[T]{0.4\textwidth}
\structure{Datos}\\
$E=$ Tener paludismo\\
$+=$ Resultado del test positivo\\
$-=$ Resultado del test negativo\\
$P(+|\overline{E})=0.02$\\
$P(-|E) = 0.04$\\
$P(E)=0.32$
\end{column}
\end{columns}
\end{frame}

\end{document}