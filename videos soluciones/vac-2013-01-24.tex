\documentclass[aspectratio=149,10pt,xcolor=dvipsnames,t]{beamer}
%---------------------------------------------------------------------------
% GENERAL PACKAGES
%---------------------------------------------------------------------------
\usepackage[utf8x]{inputenc} % Sets UTF8 codification
\usepackage[T1]{fontenc}
\usepackage[spanish]{babel} % Sets spanish language
\usepackage{amsmath} % Math symbols and environments
\usepackage{amsfonts}
\usepackage{amssymb}

\usepackage{array}
\usepackage{multirow}
\usepackage{graphicx}
%\usepackage{url}
\usepackage{textcomp}

%---------------------------------------------------------------------------
% FONTS
%---------------------------------------------------------------------------
\usepackage{mathpazo} % Palatino

%---------------------------------------------------------------------------
% CONFIGURATION
%---------------------------------------------------------------------------
\setbeamersize{text margin left=.5cm, text margin right=.5cm} % Defines margin sizes 
\beamertemplatenavigationsymbolsempty % Hide navitation bar
\usefonttheme[onlymath]{serif} % Math text in serif
\setbeamertemplate{blocks}[rounded] % Blocks with rounded corners
%\setbeamercolor{block title}{bg=RoyalBlue!10} % Color of block title
%\setbeamercolor{block body}{bg=RoyalBlue!10} % Color of block body

\begin{document}
%---------------------------------------------------------------------SLIDE----
\begin{frame}[c]
\vspace{2cm}

\begin{center}
\structure{\LARGE {\textbf{Ejercicios de Estadística}}}
\bigskip

\large
\begin{tabular}{rl}
Temas: & \structure{Variables Aleatorias Continuas}\\
Titulaciones: & \structure{Todas}
\end{tabular}

\bigskip
Alfredo Sánchez Alberca (\texttt{asalber@ceu.es})

\includegraphics[scale=0.2]{img/logo_uspceu}

\biskip
\includegraphics[scale=0.07]{img/cc-logo}
\includegraphics[scale=0.2]{img/cc-by}
\includegraphics[scale=0.2]{img/cc-e}
\includegraphics[scale=0.2]{img/cc-c}
\end{center}
\end{frame}

%---------------------------------------------------------------------SLIDE----
\begin{frame}[c]
\large
En una población se sabe que las estaturas de los hombres y de las mujeres siguen una distribución
normal con la misma desviación típica y que la media de los hombres es 5 cm mayor que la de las mujeres. 
También se sabe que el 75\% de los hombres miden menos de 178 cm y que el 10\% de las mujeres miden más de $176.8$ cm. 
Se pide:
\begin{enumerate}
\item Calcular las medias y las desviaciones típicas de las distribuciones de estaturas de los hombres y de las mujeres.
\item Calcular la probabilidad de que un hombre mida entre 170 y 180 cm.
\item Calcular el percentil 90 de la estatura de los hombres. 
\end{enumerate}
\end{frame}


%------------------------------------------------------------------SLIDE----
\begin{frame}
\begin{columns}
\begin{column}[T]{0.55\textwidth}
\begin{enumerate}
\item Calcular las medias y las desviaciones típicas de las distribuciones de estaturas de los hombres y de las mujeres.
\end{enumerate}
\end{column}
\begin{column}[T]{0.45\textwidth}
\structure{Datos}\\
$H=$ Estatura hombres $\sim N(\mu_h,\sigma_h)$\\
$M=$ Estatura mujeres $\sim N(\mu_m,\sigma_m)$\\
La estatura media de los hombres es $5$ cm mayor que la de las mujeres\\[5mm]
La estatura de los hombres y las mujeres tienen la misma desviación típica\\[5mm]
El $75\%$ de los hombres miden menos de $178$ cm\\[5mm]
El $10\%$ de las mujeres miden más de $176.8$ cm
\end{column}
\end{columns}
\end{frame}


%------------------------------------------------------------------SLIDE----
\begin{frame}
\begin{columns}
\begin{column}[T]{0.6\textwidth}
\begin{enumerate}
\setcounter{enumi}{1}
\item Calcular la probabilidad de que un hombre mida entre 170 y 180 cm.
\end{enumerate}
\end{column}
\begin{column}[T]{0.4\textwidth}
\structure{Datos}\\
$H=$ Estatura hombres \\
$H \sim N(173.83,\,6.23)$
\end{column}
\end{columns}
\end{frame}


%------------------------------------------------------------------SLIDE----
\begin{frame}
\begin{columns}
\begin{column}[T]{0.6\textwidth}
\begin{enumerate}
\setcounter{enumi}{2}
\item Calcular el percentil 90 de la estatura de los hombres. 
\end{enumerate}
\end{column}
\begin{column}[T]{0.4\textwidth}
\structure{Datos}\\
$H=$ Estatura hombres\\
$H\sim N(173.83,\,6.23)$
\end{column}
\end{columns}
\end{frame}

\end{document}