\documentclass[aspectratio=149,10pt,xcolor=dvipsnames,t]{beamer}
%---------------------------------------------------------------------------
% GENERAL PACKAGES
%---------------------------------------------------------------------------
\usepackage[utf8x]{inputenc} % Sets UTF8 codification
\usepackage[T1]{fontenc}
\usepackage[spanish]{babel} % Sets spanish language
\usepackage{amsmath} % Math symbols and environments
\usepackage{amsfonts}
\usepackage{amssymb}

\usepackage{array}
\usepackage{multirow}
\usepackage{graphicx}
%\usepackage{url}
\usepackage{textcomp}

%---------------------------------------------------------------------------
% FONTS
%---------------------------------------------------------------------------
\usepackage{mathpazo} % Palatino

%---------------------------------------------------------------------------
% CONFIGURATION
%---------------------------------------------------------------------------
\setbeamersize{text margin left=.5cm, text margin right=.5cm} % Defines margin sizes 
\beamertemplatenavigationsymbolsempty % Hide navitation bar
\usefonttheme[onlymath]{serif} % Math text in serif
\setbeamertemplate{blocks}[rounded] % Blocks with rounded corners
%\setbeamercolor{block title}{bg=RoyalBlue!10} % Color of block title
%\setbeamercolor{block body}{bg=RoyalBlue!10} % Color of block body


\begin{document}
%---------------------------------------------------------------------SLIDE----
\begin{frame}[c]
\vspace{2cm}

\begin{center}
\structure{\LARGE {\textbf{Ejercicios de Estadística}}}
\bigskip

\large
\begin{tabular}{rl}
Temas: & \structure{Variables aleatorias}\\
Titulaciones: & \structure{Farmacia, Biotecnología}
\end{tabular}

\bigskip
Alfredo Sánchez Alberca (\texttt{asalber@ceu.es})

\includegraphics[scale=0.2]{img/logo_uspceu}

\biskip
\includegraphics[scale=0.07]{img/cc-logo}
\includegraphics[scale=0.2]{img/cc-by}
\includegraphics[scale=0.2]{img/cc-e}
\includegraphics[scale=0.2]{img/cc-c}
\end{center}
\end{frame}

%---------------------------------------------------------------------SLIDE----
\begin{frame}[c]
\large
Cierta normativa obliga a que los envases de un fármaco no tengan menos de 120 mg  de principio activo.
La máquina dosificadora de un laboratorio produce estos envases siguiendo una distribución normal de mediana 123 mg y
desviación típica 2 mg.
Se pide:
\begin{enumerate}
\item Calcular la probabilidad de que un envase del fármaco elegido al azar cumpla la normativa.
\item ¿Entre qué cantidades estará el contenido del 90\% central de los envases producidos en el laboratorio? 
\item ¿Qué cantidad debiera imponer otra normativa para que resulte ser cumplida por el 95\% de los envases producidos
en este laboratorio? 
\end{enumerate} 
\end{frame}

\newgeometry{top=.0cm, left=.0cm, right=.1cm}

%------------------------------------------------------------------SLIDE----
\begin{frame}
\begin{columns}
\begin{column}[T]{0.6\textwidth}
\begin{enumerate}
\item Calcular la probabilidad de que un envase del fármaco elegido al azar cumpla la normativa.
\end{enumerate}
\end{column}
\begin{column}[T]{0.4\textwidth}
\structure{Datos}\\
$X=$ Cantidad de principio activo de un envase $\sim N(123,2)$\\
Contenido mínimo de 120 mg según normativa
\end{column}
\end{columns}
\end{frame}


%------------------------------------------------------------------SLIDE----
\begin{frame}
\begin{columns}
\begin{column}[T]{0.6\textwidth}
\begin{enumerate}
\item[2.] ¿Entre qué cantidades estará el contenido del 90\% central de los envases producidos en el laboratorio? 
\end{enumerate}
\end{column}
\begin{column}[T]{0.4\textwidth}
\structure{Datos}\\
$X=$ Cantidad de principio activo de un envase $\sim N(123,2)$
\end{column}
\end{columns}
\end{frame}


%------------------------------------------------------------------SLIDE----
\begin{frame}
\begin{columns}
\begin{column}[T]{0.6\textwidth}
\begin{enumerate}
\item[3.]¿Qué cantidad debiera imponer otra normativa para que resulte ser cumplida por el 95\% de los envases producidos
en este laboratorio? 
\end{enumerate}
\end{column}
\begin{column}[T]{0.4\textwidth}
\structure{Datos}\\
$X=$ Cantidad de principio activo de un envase $\sim N(123,2)$
\end{column}
\end{columns}
\end{frame}

\end{document}