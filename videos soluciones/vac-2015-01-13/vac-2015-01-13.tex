% Author: Alfredo Sánchez Alberca (asalber@ceu.es}
\documentclass[aspectratio=149,10pt,xcolor=dvipsnames,t]{beamer}

% GENERAL PACKAGES
\usepackage[utf8x]{inputenc} % Sets UTF8 codification \usepackage[T1]{fontenc} \usepackage[spanish]{babel} % Sets spanish language
\usepackage{amsmath} % Math symbols and environments
\usepackage{amsfonts}
\usepackage{amssymb}

\usepackage{array}
\usepackage{multirow}
\usepackage{graphicx}
\usepackage[usenames,dvipsnames]{pstricks}
\usepackage{pst-all,pst-math,pst-plot,pst-infixplot,pst-xkey,pstricks-add}
% \usepackage{url}
\usepackage{textcomp}
\usepackage[spanish]{babel}

% COLORS
\definecolor{blueceu}{RGB}{0,164,227} 
\definecolor{greenceu}{RGB}{194,205,24} 
\definecolor{redceu}{RGB}{238,50,36} 
\definecolor{purpleceu}{RGB}{169,78,145} 
\definecolor{greyceu}{RGB}{117,117,97} 
\definecolor{darkgrey}{RGB}{40,40,50}
\definecolor{softblueceu}{RGB}{193,225,246} 
\setbeamercolor{structure}{fg=blueceu}
\setbeamercolor{normal text}{fg=darkgrey}
\hypersetup{colorlinks, urlcolor=purpleceu}

% FONTS
\usepackage{mathpazo} % Palatino

% CONFIGURATION
\setbeamersize{text margin left=.5cm, text margin right=.5cm} % Defines margin sizes 
\beamertemplatenavigationsymbolsempty % Hide navitation bar
\usefonttheme[onlymath]{serif} % Math text in serif
\setbeamertemplate{blocks}[rounded] % Blocks with rounded corners
%\setbeamercolor{block title}{bg=RoyalBlue!10} % Color of block title
%\setbeamercolor{block body}{bg=RoyalBlue!10} % Color of block body

\begin{document}
% ---------------------------------------------------------------------SLIDE----
\begin{frame}[c]
	\vspace{1.5cm}
	
	\begin{center}
		\structure{\LARGE {\textbf{Ejercicios de Estadística}}}
		\bigskip
		
		\large
		\begin{tabular}{rl}
			Temas: & \structure{Variables Aleatorias Continuas: Distribución Normal} \\
			Titulaciones: & \structure{Ciencias biosanitarias}
		\end{tabular}
		
		\bigskip
		Alfredo Sánchez Alberca\\
		\url{asalber@ceu.es}\\
		\url{http://aprendeconalf.es}\\
		
		\includegraphics[scale=0.2]{../img/logo_uspceu}
		
		\biskip
		\includegraphics[scale=0.07]{../img/cc-logo}
		\includegraphics[scale=0.2]{../img/cc-by}
		\includegraphics[scale=0.2]{../img/cc-e}
		\includegraphics[scale=0.2]{../img/cc-c}
	\end{center}
\end{frame}
	
%---------------------------------------------------------------------SLIDE----
\begin{frame}[c]
\large
En un estudio realizado a 720 niños de 8 años se observó que 324 tenían un peso superior a $27.2$ kg y que 216
tenían un peso entre $24.6$ y $27.2$ kg.
Suponiendo que el peso de los niños de 8 años sigue una distribución normal, se pide:
\begin{enumerate}
  \item Calcular la media y la desviación típica del peso de los niños de 8 años.
  \item ¿Cuántos niños de 8 años tendrán un peso comprendido entre 24 y 28 kg?
  \item Si un niño de 8 años pesa $28.5$ kg, ¿cuánto debe adelgazar para situarse por debajo del percentil 60 del peso?
\end{enumerate}
\end{frame}
	
	
%------------------------------------------------------------------SLIDE----
\begin{frame}
\begin{columns}
\begin{column}[T]{0.75\textwidth}
En un estudio realizado a 720 niños de 8 años se observó que 324 tenían un peso superior a $27.2$ kg y que 216
tenían un peso entre $24.6$ y $27.2$ kg.
Suponiendo que el peso de los niños de 8 años sigue una distribución normal, se pide:
\begin{enumerate}
  \item Calcular la media y la desviación típica del peso de los niños de 8 años.
\end{enumerate}
\end{column}
\begin{column}[T]{0.25\textwidth}
\structure{Datos}\\
$X\equiv$ Peso del niño\\
$n=720$ niños\\
$X\sim N(\mu,\sigma)$
\end{column}
\end{columns} 
\end{frame}
	
	
%------------------------------------------------------------------SLIDE----
\begin{frame}
\begin{columns}
\begin{column}[T]{0.75\textwidth}
\begin{enumerate}
\setcounter{enumi}{1}
\item ¿Cuántos niños de 8 años tendrán un peso comprendido entre 24 y 28 kg?
\end{enumerate}
\end{column}
\begin{column}[T]{0.25\textwidth}
\structure{Datos}\\
$X\equiv$ Peso del niño\\
$n=720$ niños\\
$X\sim N(26.78\,,\,3.25)$
\end{column}
\end{columns} 
\end{frame}	
	
	
%------------------------------------------------------------------SLIDE----
\begin{frame}
\begin{columns}
\begin{column}[T]{0.75\textwidth}
\begin{enumerate}
\setcounter{enumi}{2}
\item Si un niño de 8 años pesa $28.5$ kg, ¿cuánto debe adelgazar para situarse por debajo del percentil 60 del peso?
\end{enumerate}
\end{column}
\begin{column}[T]{0.25\textwidth}
\structure{Datos}\\
$X\equiv$ Peso del niño\\
$n=720$ niños\\
$X\sim N(26.78\,,\,3.25)$
\end{column}
\end{columns} 
\end{frame}	
	
	
\end{document}