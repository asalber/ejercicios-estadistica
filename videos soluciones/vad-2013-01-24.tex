\documentclass[aspectratio=149,10pt,xcolor=dvipsnames,t]{beamer}
%---------------------------------------------------------------------------
% GENERAL PACKAGES
%---------------------------------------------------------------------------
\usepackage[utf8x]{inputenc} % Sets UTF8 codification
\usepackage[T1]{fontenc}
\usepackage[spanish]{babel} % Sets spanish language
\usepackage{amsmath} % Math symbols and environments
\usepackage{amsfonts}
\usepackage{amssymb}

\usepackage{array}
\usepackage{multirow}
\usepackage{graphicx}
%\usepackage{url}
\usepackage{textcomp}

%---------------------------------------------------------------------------
% FONTS
%---------------------------------------------------------------------------
\usepackage{mathpazo} % Palatino

%---------------------------------------------------------------------------
% CONFIGURATION
%---------------------------------------------------------------------------
\setbeamersize{text margin left=.5cm, text margin right=.5cm} % Defines margin sizes 
\beamertemplatenavigationsymbolsempty % Hide navitation bar
\usefonttheme[onlymath]{serif} % Math text in serif
\setbeamertemplate{blocks}[rounded] % Blocks with rounded corners
%\setbeamercolor{block title}{bg=RoyalBlue!10} % Color of block title
%\setbeamercolor{block body}{bg=RoyalBlue!10} % Color of block body

\begin{document}
%---------------------------------------------------------------------SLIDE----
\begin{frame}[c]
\vspace{2cm}

\begin{center}
\structure{\LARGE {\textbf{Ejercicios de Estadística}}}
\bigskip

\large
\begin{tabular}{rl}
Temas: & \structure{Variables Aleatorias Discretas}\\
Titulaciones: & \structure{Farmacia, Biotecnología}
\end{tabular}

\bigskip
Alfredo Sánchez Alberca (\texttt{asalber@ceu.es})

\includegraphics[scale=0.2]{img/logo_uspceu}

\biskip
\includegraphics[scale=0.07]{img/cc-logo}
\includegraphics[scale=0.2]{img/cc-by}
\includegraphics[scale=0.2]{img/cc-e}
\includegraphics[scale=0.2]{img/cc-c}
\end{center}
\end{frame}

%---------------------------------------------------------------------SLIDE----
\begin{frame}[c]
\large
Se sabe que el $0.1\%$ de los comprimidos fabricados en un laboratorio no supera los controles de calidad. 
Se pide:
\begin{enumerate}
\item Calcular la probabilidad de que en un envase de 500 comprimidos haya más de 2 comprimidos que no superan los controles de calidad.
\item Calcular la probabilidad de que en un lote de 10 envases haya más de 7 envases en los que todos sus comprimidos superen los controles de calidad. 
\end{enumerate}
\end{frame}


%------------------------------------------------------------------SLIDE----
\begin{frame}
\begin{columns}
\begin{column}[T]{0.6\textwidth}
\begin{enumerate}
\item Calcular la probabilidad de que en un envase de 500 comprimidos haya más de 2 comprimidos que no superan los controles de calidad.
\end{enumerate}
\end{column}
\begin{column}[T]{0.4\textwidth}
\structure{Datos}\\
$0.1\%$ de los comprimidos no pasan los controles de calidad
\end{column}
\end{columns}
\end{frame}


%------------------------------------------------------------------SLIDE----
\begin{frame}
\begin{columns}
\begin{column}[T]{0.6\textwidth}
\begin{enumerate}
\setcounter{enumi}{1}
\item Calcular la probabilidad de que en un lote de 10 envases haya más de 7 envases en los que todos sus comprimidos superen los controles de calidad.
\end{enumerate}
\end{column}
\begin{column}[T]{0.4\textwidth}
\structure{Datos}\\
$X=$ Número de comprimidos defectuosos en un envase $\sim P(0.5)$\\
\end{column}
\end{columns}
\end{frame}

\end{document}