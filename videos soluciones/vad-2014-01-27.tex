\documentclass[aspectratio=149,10pt,xcolor=dvipsnames,t]{beamer}
%---------------------------------------------------------------------------
% GENERAL PACKAGES
%---------------------------------------------------------------------------
\usepackage[utf8x]{inputenc} % Sets UTF8 codification
\usepackage[T1]{fontenc}
\usepackage[spanish]{babel} % Sets spanish language
\usepackage{amsmath} % Math symbols and environments
\usepackage{amsfonts}
\usepackage{amssymb}

\usepackage{array}
\usepackage{multirow}
\usepackage{graphicx}
%\usepackage{url}
\usepackage{textcomp}

%---------------------------------------------------------------------------
% FONTS
%---------------------------------------------------------------------------
\usepackage{mathpazo} % Palatino

%---------------------------------------------------------------------------
% CONFIGURATION
%---------------------------------------------------------------------------
\setbeamersize{text margin left=0.5cm, text margin right=0.5cm} % Defines margin sizes 
\beamertemplatenavigationsymbolsempty % Hide navitation bar
\usefonttheme[onlymath]{serif} % Math text in serif
\setbeamertemplate{blocks}[rounded] % Blocks with rounded corners
%\setbeamercolor{block title}{bg=RoyalBlue!10} % Color of block title
%\setbeamercolor{block body}{bg=RoyalBlue!10} % Color of block body

\begin{document}
%---------------------------------------------------------------------SLIDE----
\begin{frame}[c]
\vspace{2cm}

\begin{center}
\structure{\LARGE {\textbf{Ejercicios de Estadística}}}
\bigskip

\large
\begin{tabular}{rl}
Temas: & \structure{Variables aleatorias}\\
Titulaciones: & \structure{Farmacia}
\end{tabular}

\bigskip
Alfredo Sánchez Alberca (\texttt{asalber@ceu.es})

\includegraphics[scale=0.2]{img/logo_uspceu}

\biskip
\includegraphics[scale=0.07]{img/cc-logo}
\includegraphics[scale=0.2]{img/cc-by}
\includegraphics[scale=0.2]{img/cc-e}
\includegraphics[scale=0.2]{img/cc-c}
\end{center}
\end{frame}

%---------------------------------------------------------------------SLIDE----
\begin{frame}[c]
\large
Una farmacia recibe una media de 4 visitas por guardia nocturna que realiza. Se pide:
\begin{enumerate}
\item Calcular la probabilidad de que en una guardia nocturna reciba más de 3 visitas.
\item Si la farmacia tiene que hacer guardias nocturnas de un fin de semana semana completo (sábado y domingo), ¿qué
probabilidad hay de que reciba al menos 6 visitas a lo largo del fin de semana?
\item ¿Qué probabilidad hay de que en 10 guardias nocturnas que realice haya al menos 7 noches en las que reciba más de
3 visitas? ¿Cuántas noches se espera que tenga más de 3 visitas?
\end{enumerate} 
\end{frame}


\newgeometry{top=0cm, left=.0cm, right=.15cm}


%------------------------------------------------------------------SLIDE----
\begin{frame}
\begin{columns}
\begin{column}[T]{0.6\textwidth}
\begin{enumerate}
\item Calcular la probabilidad de que en una guardia nocturna reciba más de 3 visitas.
\end{enumerate}
\end{column}
\begin{column}[T]{0.4\textwidth}
\structure{Datos}\\
Media de 4 visitas por guardia\\
$X$= Número de visitas en una guardia

\end{column}
\end{columns}
\end{frame}


%------------------------------------------------------------------SLIDE----
\begin{frame}
\begin{columns}
\begin{column}[T]{0.6\textwidth}
\begin{enumerate}
\item[2.] Si la farmacia tiene que hacer guardias nocturnas de un fin de semana semana completo (sábado y domingo), ¿qué
probabilidad hay de que reciba al menos 6 visitas a lo largo del fin de semana?
\end{enumerate}

\end{column}
\begin{column}[T]{0.4\textwidth}
\structure{Datos}\\
Media de 4 visitas por guardia\\
$Y$= Número de visitas en un fin de semana
\end{column}
\end{columns}
\end{frame}


%------------------------------------------------------------------SLIDE----
\begin{frame}
\begin{columns}
\begin{column}[T]{0.59\textwidth}
\begin{enumerate}
\item[3.]¿Qué probabilidad hay de que en 10 guardias nocturnas que realice haya al menos 7 noches en las que reciba más de
3 visitas? ¿Cuántas noches se espera que tenga más de 3 visitas? 
\end{enumerate}
\end{column}
\begin{column}[T]{0.4\textwidth}
\structure{Datos}\\
$X$= Número de visitas en una guardia $\sim P(4)$\\
$P(X>3)=0.5665$\\
$Z$= Número de guardias con más de 3 visitas en una muestra de 10 guardias
\end{column}
\end{columns}
\end{frame}

\end{document}