% Author: Alfredo Sánchez Alberca (asalber@ceu.es}
% !TEX program = xelatex
\documentclass[aspectratio=149,10pt,t]{beamer}
%-------------------------------------------------------------------------------
% GENERAL PACKAGES
%-------------------------------------------------------------------------------
% Language
\usepackage{polyglossia}
\setdefaultlanguage{spanish}
% Maths
\usepackage{amsmath} % Math symbols and environments
\usepackage{amsfonts}
\usepackage{amssymb}
% Tables
\usepackage{array}
\usepackage{multirow}
% Graphics
\usepackage{graphicx}
\usepackage{tikz}
\usetikzlibrary{positioning}


% Colors
\definecolor{blueceu}{RGB}{0,164,227}
\definecolor{greenceu}{RGB}{194,205,24}
\definecolor{redceu}{RGB}{238,50,36}
\definecolor{purpleceu}{RGB}{169,78,145}
\definecolor{greyceu}{RGB}{117,117,97}
\definecolor{darkgrey}{RGB}{40,40,50}
\definecolor{softblueceu}{RGB}{193,225,246}
\setbeamercolor{structure}{fg=blueceu}
\setbeamercolor{normal text}{fg=darkgrey}
\hypersetup{colorlinks, urlcolor=purpleceu}

%-------------------------------------------------------------------------------
% FONTS
%-------------------------------------------------------------------------------
\usepackage{fontspec}
\setmainfont[Ligatures=TeX]{TeX Gyre Pagella}
\usepackage{unicode-math}
\setmathfont[math-style=ISO,bold-style=ISO,vargreek-shape=TeX]{TeX Gyre Pagella Math}
% Creative common icons
\usepackage[scale=1.5]{ccicons}

%-------------------------------------------------------------------------------
% CONFIGURATION
%-------------------------------------------------------------------------------
\setbeamersize{text margin left=.5cm, text margin right=.5cm} % Defines margin sizes
\beamertemplatenavigationsymbolsempty % Hide navitation bar
\usefonttheme[onlymath]{serif} % Math text in serif
\setbeamertemplate{blocks}[rounded] % Blocks with rounded corners
%\setbeamercolor{block title}{bg=RoyalBlue!10} % Color of block title
%\setbeamercolor{block body}{bg=RoyalBlue!10} % Color of block body


%-------------------------------------------------------------------------------
% DOCUMENT
%-------------------------------------------------------------------------------
\begin{document}
% ---------------------------------------------------------------------SLIDE----
\begin{frame}[c]
	\vspace{1.5cm}

	\begin{center}
		\structure{\LARGE {\textbf{Ejercicios de Estadística}}}
		\bigskip

		\large
		\begin{tabular}{rl}
			Temas: & \structure{Variables Aleatorias Discretas y Tests diagnósticos} \\
			Titulaciones: & \structure{Medicina y Farmacia}
		\end{tabular}

		\bigskip
		Alfredo Sánchez Alberca\\
		\url{asalber@ceu.es}\\
		\url{http://aprendeconalf.es}\\

		\includegraphics[scale=0.2]{../img/logo_uspceu}

		\bigskip
		{\color{darkgrey}\ccbyncsaeu}
	\end{center}
\end{frame}

%---------------------------------------------------------------------SLIDE----
\begin{frame}[c]
\large
Se sabe que en personas con infección urinaria el número medio de bacterias por mm$^3$ de orina es 5, mientras que en personas
sanas la media es de 2 bacterias por mm$^3$.
Se pide:
\begin{enumerate}
\item Calcular la probabilidad de que en una muestra de medio mm$^3$ de orina de un individuo con infección haya alguna bacteria.
\item Calcular la probabilidad de que en una muestra de dos mm$^3$ de orina de un individuo sano haya menos de 3 bacterias.
\item Si un test diagnóstico para detectar la infección urinaria da positivo cuando en un mm$^3$ de orina hay más de 6 bacterias,
¿cuál es la sensibilidad del test? ¿Y cuál es su especificidad?
\item Si la prevalencia de la infección urinaria en la población es del 5\%, ¿cuál es el valor predictivo positivo del test
diagnóstico del apartado anterior? ¿Y su valor predictivo negativo?
\item Si se toman 5 muestras de un mm$^3$ de una persona con infección de orina, ¿cuál es la probabilidad de que se produzca
algún falso negativo?
\end{enumerate}
\end{frame}


%------------------------------------------------------------------SLIDE----
\begin{frame}
\begin{columns}
\begin{column}[T]{0.7\textwidth}
\begin{enumerate}
\item Calcular la probabilidad de que en una muestra de medio mm$^3$ de orina de un individuo con infección haya alguna bacteria.
\end{enumerate}
\end{column}
\begin{column}[T]{0.3\textwidth}
\structure{Datos}\\
$\mu_I = 5$ bacterias / mm$^3$\\
$\mu_S = 2$ bacterias / mm$^3$\\
\end{column}
\end{columns}
\end{frame}


%------------------------------------------------------------------SLIDE----
\begin{frame}
\begin{columns}
\begin{column}[T]{0.7\textwidth}
\begin{enumerate}
\setcounter{enumi}{1}
\item Calcular la probabilidad de que en una muestra de dos mm$^3$ de orina de un individuo sano haya menos de 3 bacterias.
\end{enumerate}
\end{column}
\begin{column}[T]{0.3\textwidth}
\structure{Datos}\\
$\mu_I = 5$ bacterias / mm$^3$\\
$\mu_S = 2$ bacterias / mm$^3$\\
\end{column}
\end{columns}
\end{frame}


%------------------------------------------------------------------SLIDE----
\begin{frame}
\begin{columns}
\begin{column}[T]{0.7\textwidth}
\begin{enumerate}
\setcounter{enumi}{2}
\item Si un test diagnóstico para detectar la infección urinaria da positivo cuando en un mm$^3$ de orina hay más de 6 bacterias,
¿cuál es la sensibilidad del test? ¿Y cuál es su especificidad?
\end{enumerate}
\end{column}
\begin{column}[T]{0.3\textwidth}
\structure{Datos}\\
$\mu_I = 5$ bacterias / mm$^3$\\
$\mu_S = 2$ bacterias / mm$^3$\\
\end{column}
\end{columns}
\end{frame}


%------------------------------------------------------------------SLIDE----
\begin{frame}
\begin{columns}
\begin{column}[T]{0.7\textwidth}
\begin{enumerate}
\setcounter{enumi}{3}
\item Si la prevalencia de la infección urinaria en la población es del 5\%, ¿cuál es el valor predictivo positivo del test
diagnóstico del apartado anterior? ¿Y su valor predictivo negativo?
\end{enumerate}
\end{column}
\begin{column}[T]{0.3\textwidth}
\structure{Datos}\\
$\mu_I = 5$ bacterias / mm$^3$\\
$\mu_S = 2$ bacterias / mm$^3$\\
\end{column}
\end{columns}
\end{frame}


%------------------------------------------------------------------SLIDE----
\begin{frame}
\begin{columns}
\begin{column}[T]{0.7\textwidth}
\begin{enumerate}
\setcounter{enumi}{4}
\item Si se toman 5 muestras de un mm$^3$ de una persona con infección de orina, ¿cuál es la probabilidad de que se produzca
algún falso negativo?
\end{enumerate}
\end{column}
\begin{column}[T]{0.3\textwidth}
\structure{Datos}\\
$\mu_I = 5$ bacterias / mm$^3$\\
$\mu_S = 2$ bacterias / mm$^3$\\
\end{column}
\end{columns}
\end{frame}	\end{document}
