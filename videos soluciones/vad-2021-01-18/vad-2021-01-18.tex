% Author: Alfredo Sánchez Alberca (asalber@ceu.es}
% !TEX program = xelatex
\documentclass[aspectratio=169,10pt,t]{beamer}
%-------------------------------------------------------------------------------
% GENERAL PACKAGES
%-------------------------------------------------------------------------------
% Language
\usepackage{polyglossia}
\setmainlanguage{spanish}
% Maths
\usepackage{amsmath} % Math symbols and environments
\usepackage{amsfonts}
\usepackage{amssymb}
% Tables
\usepackage{array}
\usepackage{multirow}
\usepackage{booktabs}

% Graphics
\usepackage{graphicx}
\usepackage{tikz}
\usetikzlibrary{positioning}

% Colors
\definecolor{blueceu}{RGB}{0,164,227}
\definecolor{greenceu}{RGB}{194,205,24}
\definecolor{redceu}{RGB}{238,50,36}
\definecolor{purpleceu}{RGB}{169,78,145}
\definecolor{greyceu}{RGB}{117,117,97}
\definecolor{darkgrey}{RGB}{40,40,50}
\definecolor{softblueceu}{RGB}{193,225,246}
\setbeamercolor{structure}{fg=blueceu}
\setbeamercolor{normal text}{fg=darkgrey}
\hypersetup{colorlinks, urlcolor=purpleceu}

% Boxes
\usepackage[most]{tcolorbox}
\usepackage{setspace}
\newtcolorbox{datos}{
  enhanced,
  colback=blueceu!10, 
  colframe=blueceu, 
  fonttitle=\bfseries, 
  left=3pt, 
  right=3pt, 
  boxrule=0.5pt,
  code={\setstretch{1.2}},
  title={Datos},
}

%-------------------------------------------------------------------------------
% FONTS
%-------------------------------------------------------------------------------
\usepackage{fontspec}
\setmainfont[Ligatures=TeX]{TeX Gyre Pagella}
\usepackage{unicode-math}
\setmathfont[math-style=ISO, bold-style=ISO]{TeX Gyre Pagella Math}
% Creative common icons
\usepackage[
    type={CC},
    modifier={by-nc-sa},
    version={3.0},
    imagemodifier={-eu}
]{doclicense}

%-------------------------------------------------------------------------------
% CONFIGURATION
%-------------------------------------------------------------------------------
\setbeamersize{text margin left=.5cm, text margin right=.5cm} % Defines margin sizes
\beamertemplatenavigationsymbolsempty % Hide navigation bar
\usefonttheme[onlymath]{serif} % Math text in serif
\setbeamertemplate{blocks}[rounded] % Blocks with rounded corners
%\setbeamercolor{block title}{bg=RoyalBlue!10} % Color of block title
%\setbeamercolor{block body}{bg=RoyalBlue!10} % Color of block body

%-------------------------------------------------------------------------------
% COMMANDS
%-------------------------------------------------------------------------------
% \newcommand{\sen}{\operatorname{sen}}
% \newcommand{\tg}{\operatorname{tg}}
% \newcommand{\arcsen}{\operatorname{arcsen}}
% \newcommand{\arctg}{\operatorname{arctg}}

%-------------------------------------------------------------------------------
% DOCUMENT
%-------------------------------------------------------------------------------
\begin{document}
%---------------------------------------------------------------------SLIDE----
\begin{frame}[c]
\vspace{1.5cm}

\begin{center}
\structure{\LARGE {\textbf{Ejercicios de Estadística}}}
\bigskip

\large
\begin{tabular}{rl}
Temas: & \structure{Variables aleatorias discretas}\\
Titulaciones: & \structure{Óptica}
\end{tabular}

\bigskip
Alfredo Sánchez Alberca\\
\url{asalber@ceu.es}\\
\url{https://aprendeconalf.es}\\

\includegraphics[scale=0.2]{../img/logo_uspceu}

\bigskip
\doclicenseIcon
\end{center}
\end{frame}

%---------------------------------------------------------------------SLIDE----
\begin{frame}[c]
\Large
La probabilidad de que un hijo de una madre con el gen del daltonismo y un padre sin el gen del daltonismo sea un varón daltónico es $0.25$.
\begin{enumerate}
\item Si esta pareja tiene 5 hijos, ¿cuál es la probabilidad de que a lo sumo 2 sean varones daltónicos?
\item Si esta pareja tiene 5 hijos, y el sexo de los hijos es equiprobable, ¿cuál es la probabilidad de que 3 o más sean mujeres?
\item Si se toma una muestra aleatoria de 10000 hombres de una población en la que hay un varón daltónico por cada 5000 hombres, ¿cuál es la probabilidad de que haya más de 3 varones daltónicos?
\end{enumerate}
\end{frame}


%------------------------------------------------------------------SLIDE----
\begin{frame}
\begin{columns}
\begin{column}[T]{0.7\textwidth}
La probabilidad de que un hijo de una madre con el gen del daltonismo y un padre sin el gen del daltonismo sea un varón daltónico es $0.25$.
\begin{enumerate}
\item Si esta pareja tiene 5 hijos, ¿cuál es la probabilidad de que a lo sumo 2 sean varones daltónicos?
\end{enumerate}
\end{column}
\quad
\begin{column}[T]{0.3\textwidth}
\begin{datos}
$n = 5$\\
$P(\mbox{Varón daltónico})=0.25$
\end{datos}
\end{column}
\end{columns}
\end{frame}


%------------------------------------------------------------------SLIDE----
\begin{frame}
\begin{columns}
\begin{column}[T]{0.7\textwidth}
\begin{enumerate}
\item[2.] Si esta pareja tiene 5 hijos, y el sexo de los hijos es equiprobable, ¿cuál es la probabilidad de que 3 o más sean mujeres?
\end{enumerate}
\end{column}
\begin{column}[T]{0.3\textwidth}
\begin{datos}
$n=5$\\
$P(\mbox{mujer})=0.5$
\end{datos}
\end{column}
\end{columns}
\end{frame}

%------------------------------------------------------------------SLIDE----
\begin{frame}
\begin{columns}
\begin{column}[T]{0.7\textwidth}
\begin{enumerate}
\item[3.] Si se toma una muestra aleatoria de 10000 hombres de una población en la que hay un varón daltónico por cada 5000 hombres, ¿cuál es la probabilidad de que haya más de 3 varones daltónicos?
\end{enumerate}
\end{column}
\begin{column}[T]{0.32\textwidth}
\begin{datos}
$n=10000$\\
$P(\mbox{Varón daltónico})=0.0002$
\end{datos}
\end{column}
\end{columns}
\end{frame}

\end{document}