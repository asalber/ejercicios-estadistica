\documentclass[aspectratio=149,10pt,xcolor=dvipsnames,t]{beamer}
%---------------------------------------------------------------------------
% GENERAL PACKAGES
%---------------------------------------------------------------------------
\usepackage[utf8x]{inputenc} % Sets UTF8 codification
\usepackage[T1]{fontenc}
\usepackage[spanish]{babel} % Sets spanish language
\usepackage{amsmath} % Math symbols and environments
\usepackage{amsfonts}
\usepackage{amssymb}

\usepackage{array}
\usepackage{multirow}
\usepackage{graphicx}
%\usepackage{url}
\usepackage{textcomp}

%---------------------------------------------------------------------------
% FONTS
%---------------------------------------------------------------------------
\usepackage{mathpazo} % Palatino

%---------------------------------------------------------------------------
% CONFIGURATION
%---------------------------------------------------------------------------
\setbeamersize{text margin left=.5cm, text margin right=.5cm} % Defines margin sizes 
\beamertemplatenavigationsymbolsempty % Hide navitation bar
\usefonttheme[onlymath]{serif} % Math text in serif
\setbeamertemplate{blocks}[rounded] % Blocks with rounded corners
%\setbeamercolor{block title}{bg=RoyalBlue!10} % Color of block title
%\setbeamercolor{block body}{bg=RoyalBlue!10} % Color of block body

\begin{document}
%---------------------------------------------------------------------SLIDE----
\begin{frame}[c]
\vspace{2cm}

\begin{center}
\structure{\LARGE {\textbf{Ejercicios de Estadística}}}
\bigskip

\large
\begin{tabular}{rl}
Temas: & \structure{Estadística descriptiva}\\
Titulaciones: & \structure{Ciencias de la Salud}
\end{tabular}

\bigskip
Alfredo Sánchez Alberca (\texttt{asalber@ceu.es})

\includegraphics[scale=0.2]{img/logo_uspceu}

\biskip
\includegraphics[scale=0.07]{img/cc-logo}
\includegraphics[scale=0.2]{img/cc-by}
\includegraphics[scale=0.2]{img/cc-e}
\includegraphics[scale=0.2]{img/cc-c}\end{center}
\end{frame}

%---------------------------------------------------------------------SLIDE----
\begin{frame}[c]
En un grupo de personas sometidas a una anestesia general se ha medido
la dosis de sustancia anestésica recibida ($X$) en mg  y el tiempo que estuvieron dormidas ($Y$) en horas.
Las frecuencias observadas aparecen en la siguiente tabla:
\[
\begin{array}{|l|rrr|r|}
\hline 
X\backslash Y & [1,2) & [2,3) & [3,4) & n_x\\
\hline
(20,30] & 14 & 10 & 0 & 24\\
(30,40] & 12 & 26 & 7 & 45\\
(40,50] & 2 & 12 & 17 & 31\\
\hline
n_y & 28 & 48 & 24 & 100\\
\hline
\end{array}
\]
Se pide:
\begin{enumerate}
\item ¿En qué variable es más representativa la media? Justificar la respuesta.
\item ¿Por encima de cuánto tiempo estarán dormidas el 10\% de las personas que reciben una dosis entre 30 y 40 mg?
\item ¿En qué variable hay más asimetría? Justificar la respuesta.
\item Según el modelo de regresión lineal, ¿cuánta sustancia anestésica será necesaria para domir a alguien durante al menos dos horas?
¿Es fiable la predicción? 
Justificar la respuesta.
\end{enumerate}
\end{frame}


%------------------------------------------------------------------SLIDE----
\begin{frame}
\begin{columns}
\begin{column}[T]{0.55\textwidth}
\begin{enumerate}
\item ¿En qué variable es más representativa la media? Justificar la respuesta
\end{enumerate}
\end{column}
\begin{column}[T]{0.45\textwidth}
\structure{Datos}\\
$X$=Dosis de anestesia en mg\\
$Y$=Tiempo dormidas en horas
\[
\begin{array}{|l|rrr|r|}
\hline 
X\backslash Y & [1,2) & [2,3) & [3,4) & n_x\\
\hline
(20,30] & 14 & 10 & 0 & 24\\
(30,40] & 12 & 26 & 7 & 45\\
(40,50] & 2 & 12 & 17 & 31\\
\hline
n_y & 28 & 48 & 24 & 100\\
\hline
\end{array}
\]
\end{column}
\end{columns}
\end{frame}


%------------------------------------------------------------------SLIDE----
\begin{frame}
\begin{columns}
\begin{column}[T]{0.55\textwidth}
\begin{enumerate}
\setcounter{enumi}{1}
\item ¿Por encima de cuánto tiempo estarán dormidas el 10\% de las personas que reciben una dosis entre 30 y 40 mg?
\end{enumerate}
\end{column}
\begin{column}[T]{0.45\textwidth}
\structure{Datos}\\
$X$=Dosis de anestesia en mg\\
$Y$=Tiempo dormidas en horas
\[
\begin{array}{|l|rrr|r|}
\hline 
X\backslash Y & [1,2) & [2,3) & [3,4) & n_x\\
\hline
(20,30] & 14 & 10 & 0 & 24\\
(30,40] & 12 & 26 & 7 & 45\\
(40,50] & 2 & 12 & 17 & 31\\
\hline
n_y & 28 & 48 & 24 & 100\\
\hline
\end{array}
\]
\end{column}
\end{columns}
\end{frame}


%------------------------------------------------------------------SLIDE----
\begin{frame}
\begin{columns}
\begin{column}[T]{0.55\textwidth}
\begin{enumerate}
\setcounter{enumi}{2}
\item ¿En qué variable hay más asimetría? Justificar la respuesta.
\end{enumerate}
\end{column}
\begin{column}[T]{0.45\textwidth}
\structure{Datos}\\
$X$=Dosis de anestesia en mg\\
$Y$=Tiempo dormidas en horas
\[
\begin{array}{|l|rrr|r|}
\hline 
X\backslash Y & [1,2) & [2,3) & [3,4) & n_x\\
\hline
(20,30] & 14 & 10 & 0 & 24\\
(30,40] & 12 & 26 & 7 & 45\\
(40,50] & 2 & 12 & 17 & 31\\
\hline
n_y & 28 & 48 & 24 & 100\\
\hline
\end{array}
\]
$\bar x =35.7$ mg, $s_x=7.3831$ mg\\
$\bar y =2.46$ horas, $s_y=0.72$ horas
\end{column}
\end{columns}
\end{frame}


%------------------------------------------------------------------SLIDE----
\begin{frame}
\begin{columns}
\begin{column}[T]{0.55\textwidth}
\begin{enumerate}
\setcounter{enumi}{3}
\item Según el modelo de regresión lineal, ¿cuánta sustancia anestésica será necesaria para domir a alguien durante al menos dos horas?
¿Es fiable la predicción? 
Justificar la respuesta.
\end{enumerate}
\end{column}
\begin{column}[T]{0.45\textwidth}
\structure{Datos}\\
$X$=Dosis de anestesia en mg\\
$Y$=Tiempo dormidas en horas
\[
\begin{array}{|l|rrr|r|}
\hline 
X\backslash Y & [1,2) & [2,3) & [3,4) & n_x\\
\hline
(20,30] & 14 & 10 & 0 & 24\\
(30,40] & 12 & 26 & 7 & 45\\
(40,50] & 2 & 12 & 17 & 31\\
\hline
n_y & 28 & 48 & 24 & 100\\
\hline
\end{array}
\]
$\bar x =35.7$ mg, $s^2_x=54.51$ mg$^2$\\
$\bar y =2.46$ horas, $s^2_y=0.5184$ horas$^2$
\end{column}
\end{columns}
\end{frame}

\end{document}