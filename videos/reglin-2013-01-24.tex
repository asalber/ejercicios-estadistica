\documentclass[aspectratio=149,10pt,xcolor=dvipsnames,t]{beamer}
%---------------------------------------------------------------------------
% GENERAL PACKAGES
%---------------------------------------------------------------------------
\usepackage[utf8x]{inputenc} % Sets UTF8 codification
\usepackage[T1]{fontenc}
\usepackage[spanish]{babel} % Sets spanish language
\usepackage{amsmath} % Math symbols and environments
\usepackage{amsfonts}
\usepackage{amssymb}

\usepackage{array}
\usepackage{multirow}
\usepackage{graphicx}
%\usepackage{url}
\usepackage{textcomp}

%---------------------------------------------------------------------------
% FONTS
%---------------------------------------------------------------------------
\usepackage{mathpazo} % Palatino

%---------------------------------------------------------------------------
% CONFIGURATION
%---------------------------------------------------------------------------
\setbeamersize{text margin left=.5cm, text margin right=.5cm} % Defines margin sizes 
\beamertemplatenavigationsymbolsempty % Hide navitation bar
\usefonttheme[onlymath]{serif} % Math text in serif
\setbeamertemplate{blocks}[rounded] % Blocks with rounded corners
%\setbeamercolor{block title}{bg=RoyalBlue!10} % Color of block title
%\setbeamercolor{block body}{bg=RoyalBlue!10} % Color of block body

\begin{document}
%---------------------------------------------------------------------SLIDE----
\begin{frame}[c]
\vspace{2cm}

\begin{center}
\structure{\LARGE {\textbf{Ejercicios de Estadística}}}
\bigskip

\large
\begin{tabular}{rl}
Temas: & \structure{Regresión lineal}\\
Titulaciones: & \structure{Todas}
\end{tabular}

\bigskip
Alfredo Sánchez Alberca (\texttt{asalber@ceu.es})

\includegraphics[scale=0.2]{img/logo_uspceu}

\biskip
\includegraphics[scale=0.07]{img/cc-logo}
\includegraphics[scale=0.2]{img/cc-by}
\includegraphics[scale=0.2]{img/cc-e}
\includegraphics[scale=0.2]{img/cc-c}\end{center}
\end{frame}

%---------------------------------------------------------------------SLIDE----
\begin{frame}[c]
Al realizar un estudio de regresión lineal de dos variables $X$ e $Y$, se sabe que las rectas de
regresión se cortan en el punto $(5,15)$, que el coeficiente de correlación lineal es $-0.85$ y que la pendiente de la
recta de regresión de $X$ sobre $Y$ es el doble que la de la recta de $Y$ sobre $X$. Se pide:
\begin{enumerate}
\item Calcular las ecuaciones de las rectas de regresión de $Y$ sobre $X$ y de $X$ sobre $Y$.
\item ¿Qué porcentaje de la variabilidad de $Y$ queda explicado por el modelo lineal?
\end{enumerate}
\end{frame}


%------------------------------------------------------------------SLIDE----
\begin{frame}
\begin{columns}
\begin{column}[T]{0.55\textwidth}
\begin{enumerate}
\item Calcular las ecuaciones de las rectas de regresión de $Y$ sobre $X$ y de $X$ sobre $Y$.
\end{enumerate}
\end{column}
\begin{column}[T]{0.45\textwidth}
\structure{Datos}\\
Punto de corte de las rectas $(5,15)$\\
$r=-0.85$\\
$b_{xy}=2b_{yx}$
\end{column}
\end{columns}
\end{frame}


%------------------------------------------------------------------SLIDE----
\begin{frame}
\begin{columns}
\begin{column}[T]{0.55\textwidth}
\begin{enumerate}
\setcounter{enumi}{1}
\item ¿Qué porcentaje de la variabilidad de $Y$ queda explicado por el modelo lineal?
\end{enumerate}
\end{column}
\begin{column}[T]{0.45\textwidth}
\structure{Datos}\\
$r=-0.85$
\end{column}
\end{columns}
\end{frame}


\end{document}