\documentclass[aspectratio=149,10pt,xcolor=dvipsnames,t]{beamer}
%---------------------------------------------------------------------------
% GENERAL PACKAGES
%---------------------------------------------------------------------------
\usepackage[utf8x]{inputenc} % Sets UTF8 codification
\usepackage[T1]{fontenc}
\usepackage[spanish]{babel} % Sets spanish language
\usepackage{amsmath} % Math symbols and environments
\usepackage{amsfonts}
\usepackage{amssymb}

\usepackage{array}
\usepackage{multirow}
\usepackage{graphicx}
%\usepackage{url}
\usepackage{textcomp}

%---------------------------------------------------------------------------
% FONTS
%---------------------------------------------------------------------------
\usepackage{mathpazo} % Palatino

%---------------------------------------------------------------------------
% CONFIGURATION
%---------------------------------------------------------------------------
\setbeamersize{text margin left=.5cm, text margin right=.5cm} % Defines margin sizes 
\beamertemplatenavigationsymbolsempty % Hide navitation bar
\usefonttheme[onlymath]{serif} % Math text in serif
\setbeamertemplate{blocks}[rounded] % Blocks with rounded corners
%\setbeamercolor{block title}{bg=RoyalBlue!10} % Color of block title
%\setbeamercolor{block body}{bg=RoyalBlue!10} % Color of block body

\begin{document}
%---------------------------------------------------------------------SLIDE----
\begin{frame}[c]
\vspace{2cm}

\begin{center}
\structure{\LARGE {\textbf{Ejercicios de Estadística}}}
\bigskip

\large
\begin{tabular}{rl}
Temas: & \structure{Regresión no lineal}\\
Titulaciones: & \structure{Química, Biotecnología}
\end{tabular}

\bigskip
Alfredo Sánchez Alberca (\texttt{asalber@ceu.es})

\includegraphics[scale=0.2]{img/logo_uspceu}

\biskip
\includegraphics[scale=0.07]{img/cc-logo}
\includegraphics[scale=0.2]{img/cc-by}
\includegraphics[scale=0.2]{img/cc-e}
\includegraphics[scale=0.2]{img/cc-c}\end{center}
\end{frame}

%---------------------------------------------------------------------SLIDE----
\begin{frame}[c]
Se sometió a una persona a unas sesiones de entrenamiento para el manejo de una máquina de análisis químicos y se valoró la destreza en el manejo en diversas ocasiones, valorandola en una escala de 0 a 100. 
Los resultados obtenidos aparecen en la siguiente tabla
\begin{center}
\begin{tabular}{lrrrrrr}
\hline
Sesiones & 2 & 5 & 7 & 10 & 12 & 16\\
Destreza & 15 & 40 & 62 & 86 & 92 & 95\\
\hline
\end{tabular}
\end{center} 
Se pide:
\begin{enumerate}
\item Calcular la destreza alcanzada al cabo de 8 sesiones empleando el modelo logarítmico. 
\item Calcular el número de sesiones necesarias para alcanzar una destreza de 80 empleando el modelo exponencial.
\item Justificar razonadamente cuál de las predicciones anteriores es más fiable. 
\end{enumerate}
\end{frame}


%------------------------------------------------------------------SLIDE----
\begin{frame}
\begin{columns}
\begin{column}[T]{0.5\textwidth}
\begin{enumerate}
\item Calcular la destreza alcanzada al cabo de 8 sesiones empleando el modelo logarítmico. 
\end{enumerate}
\end{column}
\begin{column}[T]{0.5\textwidth}
\structure{Datos}\\
$X$=Sesiones de entrenamiento\\
$Y$=Destreza alcanzada\\[.5cm]
\begin{tabular}{lrrrrrr}
\hline
Sesiones & 2 & 5 & 7 & 10 & 12 & 16\\
Destreza & 15 & 40 & 62 & 86 & 92 & 95\\
\hline
\end{tabular}
\end{column}
\end{columns}
\end{frame}


%------------------------------------------------------------------SLIDE----
\begin{frame}
\begin{columns}
\begin{column}[T]{0.5\textwidth}
\begin{enumerate}
\setcounter{enumi}{1}
\item Calcular el número de sesiones necesarias para alcanzar una destreza de 80 empleando el modelo exponencial.\end{enumerate}
\end{column}
\begin{column}[T]{0.5\textwidth}
\structure{Datos}\\
$X$=Sesiones de entrenamiento\\
$Y$=Destreza alcanzada\\
$Z=\log(X)$\\
\medskip
\begin{tabular}{lrrrrrr}
\hline
Sesiones & 2 & 5 & 7 & 10 & 12 & 16\\
Destreza & 15 & 40 & 62 & 86 & 92 & 95\\
\hline
\end{tabular} 

\medskip
$\bar z = 1.9681$ $\log(\text{sesiones})$\\
$s_z^2=0.4635$ $\log^2(\text{sesiones})$\\
$\bar y = 65$ puntos\\
$s_{zy} = 19.6489$ $\log(\text{sesiones}\cdot\text{puntos})$
\end{column}
\end{columns}
\end{frame}


%------------------------------------------------------------------SLIDE----
\begin{frame}
\begin{columns}
\begin{column}[T]{0.5\textwidth}
\begin{enumerate}
\setcounter{enumi}{2}
\item Justificar razonadamente cuál de las predicciones anteriores es más fiable. 
\end{enumerate}
\end{column}
\begin{column}[T]{0.5\textwidth}
\structure{Datos}\\
$X$=Sesiones de entrenamiento\\
$Y$=Destreza alcanzada\\
$Z=\log(X)$\\
\medskip
\begin{tabular}{lrrrrrr}
\hline
Sesiones & 2 & 5 & 7 & 10 & 12 & 16\\
Destreza & 15 & 40 & 62 & 86 & 92 & 95\\
\hline
\end{tabular}

\medskip
$s_z^2=0.4635$ $\log^2(\text{sesiones})$\\
$s_y^2 = 867.3333$ puntos$^2$\\
$s_{zy} = 19.6489$ $\log(\text{sesiones}\cdot\text{puntos})$
\end{column}
\end{columns}
\end{frame}


\end{document}